%% -*- coding: utf-8 -*-
%% Timestamp: "2025-05-05 12:25:00 (ywatanabe)"
%% File: "/home/ywatanabe/proj/SciTex/manuscript/src/discussion.tex"

\section{Discussion}
\label{sec:discussion}

% This section demonstrates effective discussion structure and LaTeX features

SciTex demonstrates how structured LaTeX templates can improve the organization and consistency of scientific manuscripts. The modular design provides clear separation between content and formatting, allowing for more efficient document preparation.

% This demonstrates how to effectively discuss results and compare to previous work

\subsection{Benefits of Structured Templates}
\label{subsec:benefits}

The organization of content into modular files offers several benefits for scientific writing. First, it enables focused editing of individual sections without navigating through the entire document. Second, it provides consistent formatting across the manuscript while allowing authors to concentrate on content development.

As shown in Section~\ref{sec:results}, the template includes features for figure management, table formatting, and consistent referencing. These elements help maintain document consistency and reduce manual formatting tasks.

% This demonstrates how to reference previous sections

\subsection{Areas for Customization}
\label{subsec:customization}

While the current template provides a solid foundation, users may want to customize it for specific purposes. Users can modify the template structure based on their specific requirements without disrupting the overall system.

The figure and table management systems are designed with flexibility in mind, allowing adaptation to different journal or conference formatting requirements. Users can adjust settings in the configuration files to meet specific publication guidelines.

% This demonstrates how to create a numbered list in LaTeX

Potential customizations include:

\begin{enumerate}
    \item \textbf{Journal-specific formatting} - Adjusting margins, fonts, and layout for different publishers
    \item \textbf{Citation style adaptation} - Modifying bibliography formatting for different fields or journals
    \item \textbf{Custom section organization} - Adding, removing, or reordering document sections
    \item \textbf{Figure formatting options} - Adjusting how figures are presented and captioned
\end{enumerate}

\subsection{Application Areas}
\label{subsec:applications}

The SciTex template can be applied to various document types beyond traditional research papers. With minimal adaptation, it can be used for:

\begin{itemize}
    \item Research proposals and grant applications
    \item Technical reports and white papers
    \item Conference proceedings and extended abstracts
    \item Academic theses and dissertations
\end{itemize}

The structured approach to document preparation is particularly beneficial for documents with complex elements like figures, tables, equations, and citations. The consistency in formatting across different document types helps establish a recognizable style for research groups or organizations.

\subsection{Future Work}
\label{subsec:future-work}

Several planned enhancements for the SciTex system will further improve its capabilities:

\begin{itemize}
    \item \textbf{Literature Review Assistant} - An integrated tool to help researchers organize, summarize, and cite relevant literature. This feature will:
    \begin{itemize}
        \item Import citation data from reference managers and databases
        \item Generate structured literature review templates with key information fields
        \item Provide automated citation clustering based on topic and relevance
        \item Create citation network visualizations to identify research gaps
    \end{itemize}
    
    \item \textbf{Mermaid Diagram Support} - Integration of Mermaid syntax for creating flowcharts, sequence diagrams, and other visualizations directly in the manuscript:
    \begin{itemize}
        \item Automatic conversion of Mermaid code to publication-quality vector graphics
        \item Template library for common scientific diagram types
        \item Visual editor for diagram creation and modification
        \item Version control for diagram evolution
    \end{itemize}
    
    \item \textbf{Advanced Statistical Integration} - Direct integration with statistical packages for result generation and visualization
    
    \item \textbf{Journal-Specific Template Repository} - A collection of pre-configured templates for major scientific journals
    
    \item \textbf{Collaboration Enhancement} - Real-time collaborative editing with change tracking and comment system
\end{itemize}

The literature review feature, in particular, addresses a critical pain point in scientific writing by helping researchers maintain organized connections between their work and existing literature. By structuring literature data into a standardized format, the system will facilitate evidence synthesis and gap identification, making it easier for researchers to position their work within the broader scientific context.

In conclusion, SciTex demonstrates how structured LaTeX templates can simplify the preparation of scientific documents. The modular organization, automated figure and table management, and consistent referencing system help researchers focus on content while maintaining professional document formatting. Future enhancements will continue to address the challenges researchers face in document preparation and collaboration.

%%%% EOF