\section{Results}

\subsection{Dataset Characteristics}
The NeuroVista dataset provided extensive long-term iEEG recordings from 9 patients with drug-resistant focal epilepsy. After excluding the initial 100-day post-implantation period, we analyzed a total of XX patient-years of continuous recording (mean: XX months per patient, range: 6-14 months). Across all patients, we identified and analyzed:
- Lead seizures (>5h inter-seizure interval): 342 events
- All seizures (types 1 and 2): 1,044 events  
- Interictal control segments: 1,044 matched segments

The 16-channel electrode arrays provided comprehensive spatial coverage of the seizure onset zones, with an average of XX ± XX channels showing significant PAC modulation during seizure events.

\subsection{Phase-Amplitude Coupling Patterns}

\subsubsection{Baseline PAC Characteristics}
During interictal periods (>4 hours from any seizure), we observed stable baseline PAC patterns with the following characteristics:
- Dominant coupling between theta/alpha phase (4-13 Hz) and gamma amplitude (30-100 Hz)
- Mean modulation index (MI): 0.XX ± 0.XX (z-score: 0 ± 1 by design)
- Spatial heterogeneity: XX\% of channels showed significant PAC (z > 2)
- Temporal stability: coefficient of variation < XX\% over 1-hour windows

\subsubsection{Pre-ictal PAC Evolution}
Analysis of the 60-minute pre-seizure period revealed progressive changes in PAC:
- Significant increase in theta-gamma coupling starting XX ± XX minutes before seizure onset (p < 0.001, Wilcoxon signed-rank test)
- Peak PAC z-scores: XX ± XX in the final 5 minutes before onset
- Spatial spread: PAC elevation initially focal (XX ± XX channels) expanding to XX ± XX channels at onset
- Frequency shift: Progressive increase in the optimal phase frequency from XX Hz to XX Hz approaching seizure

\subsubsection{Ictal PAC Dynamics}
During seizures, PAC showed dramatic alterations:
- Maximum coupling strength: z-score = XX ± XX (p < 0.001 vs baseline)
- Dominant frequency pairs: Phase XX-XX Hz coupled with amplitude XX-XX Hz
- Spatial synchronization: XX\% increase in inter-channel PAC correlation
- Temporal evolution: Initial spike within XX seconds of electrical onset, followed by sustained elevation

\subsubsection{Post-ictal Recovery}
Following seizure termination, PAC exhibited gradual normalization:
- Recovery time to baseline: XX ± XX minutes
- Transient suppression period: XX\% of seizures showed below-baseline PAC for XX ± XX minutes
- Asymmetric recovery: Phase frequencies normalized faster (XX min) than amplitude frequencies (XX min)

\subsection{Spatial Distribution and Channel Consistency}

Analysis of PAC spatial patterns across the 16-channel arrays revealed:
- Consistent "driver" channels: XX ± XX channels per patient showed reliable pre-ictal PAC elevation
- Spatial gradient: PAC strength decreased with distance from seizure onset zone (correlation r = -0.XX, p < 0.01)
- Channel stability: XX\% of high-PAC channels remained consistent across multiple seizures
- Network effects: Increased inter-channel coupling preceded clinical manifestations by XX ± XX seconds

\subsection{Seizure Prediction Performance}

\subsubsection{Classification Accuracy}
Using PAC features extracted from 10-second windows, we trained patient-specific classifiers to discriminate pre-ictal from interictal states:
- Mean AUC across patients: 0.XX ± 0.XX (range: 0.XX - 0.XX)
- Optimal prediction horizon: XX minutes (sensitivity = XX\%, specificity = XX\%)
- Feature importance: Theta-gamma PAC contributed XX\% of predictive power

\subsubsection{Pseudo-prospective Evaluation}
Testing on held-out data (post day 200) demonstrated:
- Sensitivity for lead seizures: XX\% (XX/XX seizures detected)
- False positive rate: XX per day
- Time in warning: XX\% ± XX\%
- Improvement over chance: XX\% (p < 0.001, permutation test)

\subsubsection{Patient-specific Variability}
Performance varied across patients, correlating with:
- Seizure frequency (r = 0.XX, p < 0.05)
- Electrode coverage of onset zone (r = 0.XX, p < 0.05)
- Baseline PAC stability (r = -0.XX, p < 0.05)

Best performing patients (n=3) achieved AUC > 0.XX with sensitivity > XX\% at <XX\% time in warning.

\subsection{Temporal Patterns and Circadian Rhythms}

Long-term PAC analysis revealed multi-scale temporal patterns:
- Circadian modulation: Peak PAC at XX:XX hours (XX\% increase vs nadir)
- Weekly patterns: XX\% of patients showed 7-day periodicity
- Relationship to seizure timing: XX\% of seizures occurred within 2 hours of daily PAC maximum
- Pre-seizure buildup: Gradual PAC increase over XX ± XX hours before seizure clusters

\subsection{Comparison with Traditional Features}

PAC-based prediction outperformed conventional EEG features:
- vs. Spectral power: +XX\% AUC improvement (p < 0.01)
- vs. Line length: +XX\% AUC improvement (p < 0.05)
- vs. Multi-feature combination: Comparable performance with XX\% fewer features

Combined models incorporating PAC with traditional features achieved marginal improvement (AUC increase: 0.XX, p = 0.XX).

\label{sec:results}