%% -*- coding: utf-8 -*-
%% Timestamp: "$(date +"%Y-%m-%d %H:%M:%S") (ywatanabe)"
%% File: _Figure_ID_XX_svg.tex
%%
%% SVG FIGURE TEMPLATE
%% ==================
%% Instructions: 
%% 1. Copy this template to the src/figures/src/ directory
%% 2. Rename to match your SVG file: Figure_ID_NN_description.tex
%%    - NN should be a two-digit number (01, 02, etc.)
%%    - description should be a short, descriptive name
%% 3. Complete the figure title and legend below
%% 4. Adjust figure width as needed (default is 0.8\textwidth)
%%
%% Reference in text with: Figure~\ref{fig:NN}
%% See FIGURE_TABLE_GUIDE.md for more details on SVG figures
%%
%% Notes on SVG support:
%% - SVG files are automatically converted to JPG for inclusion in the PDF
%% - For best results, create your SVG at the intended display size
%% - Text in the SVG will be rasterized - use captions for important text

\caption{\textbf{
SVG FIGURE TITLE HERE
}
\smallskip
\\
FIGURE LEGEND HERE. Provide a detailed description of this vector graphic figure. SVG files are excellent for diagrams, flowcharts, and illustrations that require clean lines and sharp edges. The original SVG file maintains perfect quality at any scale, but will be converted to a high-resolution raster image for inclusion in the final document.

For diagrams with labeled sections, you can reference them specifically:
The process begins with step 1 (left), continues through steps 2-3 (middle), and concludes with step 4 (right).
}
% FIGURE WIDTH CONTROL (uncomment/modify one of the following)
% width=0.8\textwidth  % 80% of page width (good default for SVG diagrams)
% width=0.9\textwidth  % 90% of page width
% width=1\textwidth    % Full page width (for complex diagrams)
% width=0.6\textwidth  % 60% of page width (for simple diagrams)
% format=vector        % Indicates this is a vector graphic (informational only)

%%%% EOF