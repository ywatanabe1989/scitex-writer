%% -*- coding: utf-8 -*-
%DIF LATEXDIFF DIFFERENCE FILE
%DIF DEL ./01_manuscript/archive/manuscript_v009_diff.tex   Sat Sep 27 20:51:12 2025
%DIF ADD ./01_manuscript/manuscript.tex                     Sat Sep 27 20:54:54 2025
%DIF 2-50c2
%DIF < %DIF LATEXDIFF DIFFERENCE FILE
%DIF < %DIF DEL ./01_manuscript/archive/manuscript_v008_diff.tex   Sat Sep 27 20:39:24 2025
%DIF < %DIF ADD ./01_manuscript/manuscript.tex                     Sat Sep 27 20:50:53 2025
%DIF < %DIF 2-43c2
%DIF < %DIF < %DIF LATEXDIFF DIFFERENCE FILE
%DIF < %DIF < %DIF DEL ./01_manuscript/archive/manuscript_v007_diff.tex   Sat Sep 27 20:38:03 2025
%DIF < %DIF < %DIF ADD ./01_manuscript/manuscript.tex                     Sat Sep 27 20:39:08 2025
%DIF < %DIF < %DIF 2-36c2
%DIF < %DIF < %DIF < %DIF LATEXDIFF DIFFERENCE FILE
%DIF < %DIF < %DIF < %DIF DEL ./01_manuscript/archive/manuscript_v006_diff.tex   Sat Sep 27 20:37:23 2025
%DIF < %DIF < %DIF < %DIF ADD ./01_manuscript/manuscript.tex                     Sat Sep 27 20:37:48 2025
%DIF < %DIF < %DIF < %DIF 2-29c2
%DIF < %DIF < %DIF < %DIF < %DIF LATEXDIFF DIFFERENCE FILE
%DIF < %DIF < %DIF < %DIF < %DIF DEL ./01_manuscript/archive/manuscript_v005_diff.tex   Sat Sep 27 19:46:35 2025
%DIF < %DIF < %DIF < %DIF < %DIF ADD ./01_manuscript/manuscript.tex                     Sat Sep 27 20:37:06 2025
%DIF < %DIF < %DIF < %DIF < %DIF 2-22c2
%DIF < %DIF < %DIF < %DIF < %DIF < %DIF LATEXDIFF DIFFERENCE FILE
%DIF < %DIF < %DIF < %DIF < %DIF < %DIF DEL ./01_manuscript/archive/manuscript_v004_diff.tex   Sat Sep 27 19:12:03 2025
%DIF < %DIF < %DIF < %DIF < %DIF < %DIF ADD ./01_manuscript/manuscript.tex                     Sat Sep 27 19:46:20 2025
%DIF < %DIF < %DIF < %DIF < %DIF < %DIF 2-15c2
%DIF < %DIF < %DIF < %DIF < %DIF < %DIF < %DIF LATEXDIFF DIFFERENCE FILE
%DIF < %DIF < %DIF < %DIF < %DIF < %DIF < %DIF DEL ./01_manuscript/archive/manuscript_v003_diff.tex   Sat Sep 27 19:11:00 2025
%DIF < %DIF < %DIF < %DIF < %DIF < %DIF < %DIF ADD ./01_manuscript/manuscript.tex                     Sat Sep 27 19:11:47 2025
%DIF < %DIF < %DIF < %DIF < %DIF < %DIF < %DIF 2-9d2
%DIF < %DIF < %DIF < %DIF < %DIF < %DIF < %DIF < %DIF LATEXDIFF DIFFERENCE FILE
%DIF < %DIF < %DIF < %DIF < %DIF < %DIF < %DIF < %DIF DEL ./01_manuscript/archive/manuscript_v002_diff.tex   Sat Sep 27 19:09:36 2025
%DIF < %DIF < %DIF < %DIF < %DIF < %DIF < %DIF < %DIF ADD ./01_manuscript/manuscript.tex                     Sat Sep 27 19:10:44 2025
%DIF < %DIF < %DIF < %DIF < %DIF < %DIF < %DIF < %DIF 2-4d2
%DIF < %DIF < %DIF < %DIF < %DIF < %DIF < %DIF < %DIF < %DIF LATEXDIFF DIFFERENCE FILE
%DIF < %DIF < %DIF < %DIF < %DIF < %DIF < %DIF < %DIF < %DIF DEL ./01_manuscript/manuscript.tex   Sat Sep 27 19:09:21 2025
%DIF < %DIF < %DIF < %DIF < %DIF < %DIF < %DIF < %DIF < %DIF ADD ./01_manuscript/manuscript.tex   Sat Sep 27 19:09:21 2025
%DIF < %DIF < %DIF < %DIF < %DIF < %DIF < %DIF < %DIF -------
%DIF < %DIF < %DIF < %DIF < %DIF < %DIF < %DIF -------
%DIF < %DIF < %DIF < %DIF < %DIF < %DIF < %% Timestamp: "2025-09-27 14:50:40 (ywatanabe)"
%DIF < %DIF < %DIF < %DIF < %DIF < %DIF -------
%DIF < %DIF < %DIF < %DIF < %DIF < %% Timestamp: "2025-09-27 19:27:03 (ywatanabe)" %DIF > 
%DIF < %DIF < %DIF < %DIF < %DIF < %DIF -------
%DIF < %DIF < %DIF < %DIF < %DIF -------
%DIF < %DIF < %DIF < %DIF < %% Timestamp: "2025-09-27 19:27:03 (ywatanabe)" %DIF > 
%DIF < %DIF < %DIF < %DIF < %DIF -------
%DIF < %DIF < %DIF < %DIF -------
%DIF < %DIF < %DIF < %% Timestamp: "2025-09-27 19:27:03 (ywatanabe)" %DIF > 
%DIF < %DIF < %DIF < %DIF -------
%DIF < %DIF < %DIF -------
%DIF < %DIF < %% Timestamp: "2025-09-27 19:27:03 (ywatanabe)" %DIF > 
%DIF < %DIF < %DIF -------
%DIF < %DIF -------
%DIF < %% Timestamp: "2025-09-27 19:27:03 (ywatanabe)" %DIF > 
%DIF < %DIF -------
%DIF -------
%% Timestamp: "2025-09-27 19:27:03 (ywatanabe)" %DIF > 
%DIF -------
%% File: "/ssh:sp:/home/ywatanabe/proj/neurovista/paper/01_manuscript/base.tex"
\UseRawInputEncoding

%% ----------------------------------------
%% SETTINGS
%% ----------------------------------------

%% ========================================
%% ./01_manuscript/contents/latex_styles/columns.tex
%% ========================================
%% -*- coding: utf-8 -*-
%% Timestamp: "2025-09-26 18:18:13 (ywatanabe)"
%% File: "/ssh:sp:/home/ywatanabe/proj/neurovista/paper/01_manuscript/src/styles/columns.tex"

%% Columns
%% \documentclass[final,3p,times,twocolumn]{elsarticle} %% Use it for submission
%% Use the options 1p,twocolumn; 3p; 3p,twocolumn; 5p; or 5p,twocolumn
%% for a journal layout:
%% \documentclass[final,1p,times]{elsarticle}
%% \documentclass[final,1p,times,twocolumn]{elsarticle}
%% \documentclass[final,3p,times]{elsarticle}
%% \documentclass[final,3p,times,twocolumn]{elsarticle}
%% \documentclass[final,5p,times]{elsarticle}
%% \documentclass[final,5p,times,twocolumn]{elsarticle}
\documentclass[preprint,review,12pt]{elsarticle}

%%%% EOF


%% ========================================
%% ./01_manuscript/contents/latex_styles/packages.tex
%% ========================================
%% -*- coding: utf-8 -*-
%% Timestamp: "2025-09-27 16:01:16 (ywatanabe)"
%% File: "/ssh:sp:/home/ywatanabe/proj/neurovista/paper/shared/latex_styles/packages.tex"

%% Language and encoding
\usepackage[english]{babel}
\usepackage[T1]{fontenc}
\usepackage[utf8]{inputenc}

%% Mathematics
\usepackage{amsmath, amssymb}
\usepackage{amsthm}
\usepackage{siunitx}
\sisetup{round-mode=figures,round-precision=3}

%% Graphics and figures
\usepackage{graphicx}
\usepackage{tikz}
\usepackage{pgfplots}
\usepackage{pgfplotstable}
\usetikzlibrary{positioning,shapes,arrows,fit,calc,graphs,graphs.standard}

%% Tables
\usepackage[table]{xcolor}
\usepackage{booktabs}
\usepackage{colortbl}
\usepackage{longtable}
\usepackage{supertabular}
\usepackage{tabularx}
\usepackage{xltabular}
\usepackage{csvsimple}
\usepackage{makecell}

%% Table formatting
\renewcommand\theadfont{\bfseries}
\renewcommand\theadalign{c}
\newcolumntype{C}[1]{>{\centering\arraybackslash}m{#1}}
\renewcommand{\arraystretch}{1.5}
\definecolor{lightgray}{gray}{0.95}

%% Layout and geometry
\usepackage[pass]{geometry}
\usepackage{pdflscape}
\usepackage{indentfirst}
\usepackage{calc}

%% Captions and references
\usepackage[margin=10pt,font=small,labelfont=bf,labelsep=endash]{caption}
\usepackage{natbib}
\usepackage{hyperref}

%% Document features
\usepackage{accsupp}
\usepackage{lineno}
\usepackage{bashful}
\usepackage{lipsum}

%% Visual enhancements
\usepackage{xcolor}
\usepackage[most]{tcolorbox}

%% External references (commented)
%% \usepackage{xr}
\usepackage{xr-hyper}

%%%% EOF


%% ========================================
%% ./01_manuscript/contents/latex_styles/linker.tex
%% ========================================
%% -*- coding: utf-8 -*-
%% Timestamp: "2025-09-26 18:18:21 (ywatanabe)"
%% File: "/ssh:sp:/home/ywatanabe/proj/neurovista/paper/01_manuscript/src/styles/linker.tex"

% Edit to ref between main text & supplemental material
\usepackage{xr}
\makeatletter
\newcommand*{\addFileDependency}[1]{% argument=file name and extension
  \typeout{(#1)}
  \@addtofilelist{#1}
  \IfFileExists{#1}{}{\typeout{No file #1.}}
}
\makeatother

\newcommand*{\link}[2][]{%
    \externaldocument[#1]{#2}%
    \addFileDependency{#2.tex}%
    \addFileDependency{#2.aux}%
}

%%%% EOF


%% ========================================
%% ./01_manuscript/contents/latex_styles/formatting.tex
%% ========================================
%% -*- coding: utf-8 -*-
%% Timestamp: "2025-09-26 18:18:18 (ywatanabe)"
%% File: "/ssh:sp:/home/ywatanabe/proj/neurovista/paper/01_manuscript/src/styles/formatting.tex"

%% Image width
\newlength{\imagewidth}
\newlength{\imagescale}

%% Line numbers
\linespread{1.2}
\linenumbers

% Define colors with transparency (opacity value)
\definecolor{GreenBG}{rgb}{0,1,0}
\definecolor{RedBG}{rgb}{1,0,0}
% Define tcolorbox environments for highlighting
\newtcbox{\greenhighlight}[1][]{%
  on line,
  colframe=GreenBG,
  colback=GreenBG!50!white, % 50% transparent green
  boxrule=0pt,
  arc=0pt,
  boxsep=0pt,
  left=1pt,
  right=1pt,
  top=2pt,
  bottom=2pt,
  tcbox raise base
}
\newtcbox{\redhighlight}[1][]{%
  on line,
  colframe=RedBG,
  colback=RedBG!50!white, % 50% transparent red
  boxrule=0pt,
  arc=0pt,
  boxsep=0pt,
  left=1pt,
  right=1pt,
  top=2pt,
  bottom=2pt,
  tcbox raise base
}
\newcommand{\REDSTARTS}{\color{red}}
\newcommand{\REDENDS}{\color{black}}
\newcommand{\GREENSTARTS}{\color{green}}
\newcommand{\GREENENDS}{\color{black}}

% New command to read word counts
\newread\wordcount
\newcommand\readwordcount[1]{%
  \openin\wordcount=#1
  \read\wordcount to \thewordcount
  \closein\wordcount
  \thewordcount
}

\newcommand{\hl}[1]{\colorbox{yellow}{#1}}

%% Reference
\usepackage{refcount}


%% \let\oldref\ref
%% \renewcommand{\ref}[1]{%
%%   \ifnum\getrefnumber{#1}=0
%%     \sethlcolor{yellow}\hl{??}%
%%   \else
%%     \oldref{#1}%
%%   \fi
%% }

\let\oldref\ref
\newcommand{\hlref}[1]{%
  \ifnum\getrefnumber{#1}=0
    \hl{\ref*{#1}}%
    %% \sethlcolor{yellow}\hl{\ref*{#1}}%    
  \else
    \ref{#1}%
  \fi
}

% To add an 'S' prefix to a reference
\newcommand*\sref[1]{%
    S\hlref{#1}}
 
% For 'Supplementary Figure S1'
\newcommand*\sfref[1]{%
    Supplementary Figure S\hlref{#1}}
 
% For 'Supplementary Table S1'
\newcommand*\stref[1]{%
    Supplementary Table S\hlref{#1}}
 
% For 'Supplementary Materials S1'
\newcommand*\smref[1]{%
    Supplementary Materials S\hlref{#1}}

%%%% EOF

\link[supple-]{./02_supplementary/supplementary}

%% ----------------------------------------
%% JOURNAL NAME
%% ----------------------------------------

%% ========================================
%% ./01_manuscript/contents/journal_name.tex
%% ========================================
\journal{Journal Name Here}



%% ----------------------------------------
%% START of DOCUMENT
%% ----------------------------------------
%DIF < %DIF < %DIF < %DIF < %DIF < %DIF < %DIF < %DIF < %DIF PREAMBLE EXTENSION ADDED BY LATEXDIFF
%DIF < %DIF < %DIF < %DIF < %DIF < %DIF < %DIF < %DIF < %DIF CULINECHBAR PREAMBLE %DIF PREAMBLE
%DIF < %DIF < %DIF < %DIF < %DIF < %DIF < %DIF < %DIF < \RequirePackage[normalem]{ulem} %DIF PREAMBLE
%DIF < %DIF < %DIF < %DIF < %DIF < %DIF < %DIF < %DIF < \RequirePackage[pdftex]{changebar} %DIF PREAMBLE
%DIF < %DIF < %DIF < %DIF < %DIF < %DIF < %DIF < %DIF < \RequirePackage{color}\definecolor{RED}{rgb}{1,0,0}\definecolor{BLUE}{rgb}{0,0,1} %DIF PREAMBLE
%DIF < %DIF < %DIF < %DIF < %DIF < %DIF < %DIF < %DIF < \providecommand{\DIFaddtex}[1]{\protect\cbstart{\protect\color{blue}\uwave{#1}}\protect\cbend} %DIF PREAMBLE
%DIF < %DIF < %DIF < %DIF < %DIF < %DIF < %DIF < %DIF < \providecommand{\DIFdeltex}[1]{\protect\cbdelete{\protect\color{red}\sout{#1}}\protect\cbdelete} %DIF PREAMBLE
%DIF < %DIF < %DIF < %DIF < %DIF < %DIF < %DIF < %DIF < %DIF SAFE PREAMBLE %DIF PREAMBLE
%DIF < %DIF < %DIF < %DIF < %DIF < %DIF < %DIF < %DIF < \providecommand{\DIFaddbegin}{} %DIF PREAMBLE
%DIF < %DIF < %DIF < %DIF < %DIF < %DIF < %DIF < %DIF < \providecommand{\DIFaddend}{} %DIF PREAMBLE
%DIF < %DIF < %DIF < %DIF < %DIF < %DIF < %DIF < %DIF < \providecommand{\DIFdelbegin}{} %DIF PREAMBLE
%DIF < %DIF < %DIF < %DIF < %DIF < %DIF < %DIF < %DIF < \providecommand{\DIFdelend}{} %DIF PREAMBLE
%DIF < %DIF < %DIF < %DIF < %DIF < %DIF < %DIF < %DIF < \providecommand{\DIFmodbegin}{} %DIF PREAMBLE
%DIF < %DIF < %DIF < %DIF < %DIF < %DIF < %DIF < %DIF < \providecommand{\DIFmodend}{} %DIF PREAMBLE
%DIF < %DIF < %DIF < %DIF < %DIF < %DIF < %DIF < %DIF < %DIF FLOATSAFE PREAMBLE %DIF PREAMBLE
%DIF < %DIF < %DIF < %DIF < %DIF < %DIF < %DIF < %DIF < \providecommand{\DIFaddFL}[1]{\DIFadd{#1}} %DIF PREAMBLE
%DIF < %DIF < %DIF < %DIF < %DIF < %DIF < %DIF < %DIF < \providecommand{\DIFdelFL}[1]{\DIFdel{#1}} %DIF PREAMBLE
%DIF < %DIF < %DIF < %DIF < %DIF < %DIF < %DIF < %DIF < \providecommand{\DIFaddbeginFL}{} %DIF PREAMBLE
%DIF < %DIF < %DIF < %DIF < %DIF < %DIF < %DIF < %DIF < \providecommand{\DIFaddendFL}{} %DIF PREAMBLE
%DIF < %DIF < %DIF < %DIF < %DIF < %DIF < %DIF < %DIF < \providecommand{\DIFdelbeginFL}{} %DIF PREAMBLE
%DIF < %DIF < %DIF < %DIF < %DIF < %DIF < %DIF < %DIF < \providecommand{\DIFdelendFL}{} %DIF PREAMBLE
%DIF < %DIF < %DIF < %DIF < %DIF < %DIF < %DIF < %DIF < %DIF HYPERREF PREAMBLE %DIF PREAMBLE
%DIF < %DIF < %DIF < %DIF < %DIF < %DIF < %DIF < %DIF < \providecommand{\DIFadd}[1]{\texorpdfstring{\DIFaddtex{#1}}{#1}} %DIF PREAMBLE
%DIF < %DIF < %DIF < %DIF < %DIF < %DIF < %DIF < %DIF < \providecommand{\DIFdel}[1]{\texorpdfstring{\DIFdeltex{#1}}{}} %DIF PREAMBLE
%DIF < %DIF < %DIF < %DIF < %DIF < %DIF < %DIF < %DIF < \newcommand{\DIFscaledelfig}{0.5}
%DIF < %DIF < %DIF < %DIF < %DIF < %DIF < %DIF < %DIF < %DIF HIGHLIGHTGRAPHICS PREAMBLE %DIF PREAMBLE
%DIF < %DIF < %DIF < %DIF < %DIF < %DIF < %DIF < %DIF < \RequirePackage{settobox} %DIF PREAMBLE
%DIF < %DIF < %DIF < %DIF < %DIF < %DIF < %DIF < %DIF < \RequirePackage{letltxmacro} %DIF PREAMBLE
%DIF < %DIF < %DIF < %DIF < %DIF < %DIF < %DIF < %DIF < \newsavebox{\DIFdelgraphicsbox} %DIF PREAMBLE
%DIF < %DIF < %DIF < %DIF < %DIF < %DIF < %DIF < %DIF < \newlength{\DIFdelgraphicswidth} %DIF PREAMBLE
%DIF < %DIF < %DIF < %DIF < %DIF < %DIF < %DIF < %DIF < \newlength{\DIFdelgraphicsheight} %DIF PREAMBLE
%DIF < %DIF < %DIF < %DIF < %DIF < %DIF < %DIF < %DIF < % store original definition of \includegraphics %DIF PREAMBLE
%DIF < %DIF < %DIF < %DIF < %DIF < %DIF < %DIF < %DIF < \LetLtxMacro{\DIFOincludegraphics}{\includegraphics} %DIF PREAMBLE
%DIF < %DIF < %DIF < %DIF < %DIF < %DIF < %DIF < %DIF < \newcommand{\DIFaddincludegraphics}[2][]{{\color{blue}\fbox{\DIFOincludegraphics[#1]{#2}}}} %DIF PREAMBLE
%DIF < %DIF < %DIF < %DIF < %DIF < %DIF < %DIF < %DIF < \newcommand{\DIFdelincludegraphics}[2][]{% %DIF PREAMBLE
%DIF < %DIF < %DIF < %DIF < %DIF < %DIF < %DIF < %DIF < \sbox{\DIFdelgraphicsbox}{\DIFOincludegraphics[#1]{#2}}% %DIF PREAMBLE
%DIF < %DIF < %DIF < %DIF < %DIF < %DIF < %DIF < %DIF < \settoboxwidth{\DIFdelgraphicswidth}{\DIFdelgraphicsbox} %DIF PREAMBLE
%DIF < %DIF < %DIF < %DIF < %DIF < %DIF < %DIF < %DIF < \settoboxtotalheight{\DIFdelgraphicsheight}{\DIFdelgraphicsbox} %DIF PREAMBLE
%DIF < %DIF < %DIF < %DIF < %DIF < %DIF < %DIF < %DIF < \scalebox{\DIFscaledelfig}{% %DIF PREAMBLE
%DIF < %DIF < %DIF < %DIF < %DIF < %DIF < %DIF < %DIF < \parbox[b]{\DIFdelgraphicswidth}{\usebox{\DIFdelgraphicsbox}\\[-\baselineskip] \rule{\DIFdelgraphicswidth}{0em}}\llap{\resizebox{\DIFdelgraphicswidth}{\DIFdelgraphicsheight}{% %DIF PREAMBLE
%DIF < %DIF < %DIF < %DIF < %DIF < %DIF < %DIF < %DIF < \setlength{\unitlength}{\DIFdelgraphicswidth}% %DIF PREAMBLE
%DIF < %DIF < %DIF < %DIF < %DIF < %DIF < %DIF < %DIF < \begin{picture}(1,1)% %DIF PREAMBLE
%DIF < %DIF < %DIF < %DIF < %DIF < %DIF < %DIF < %DIF < \thicklines\linethickness{2pt} %DIF PREAMBLE
%DIF < %DIF < %DIF < %DIF < %DIF < %DIF < %DIF < %DIF < {\color[rgb]{1,0,0}\put(0,0){\framebox(1,1){}}}% %DIF PREAMBLE
%DIF < %DIF < %DIF < %DIF < %DIF < %DIF < %DIF < %DIF < {\color[rgb]{1,0,0}\put(0,0){\line( 1,1){1}}}% %DIF PREAMBLE
%DIF < %DIF < %DIF < %DIF < %DIF < %DIF < %DIF < %DIF < {\color[rgb]{1,0,0}\put(0,1){\line(1,-1){1}}}% %DIF PREAMBLE
%DIF < %DIF < %DIF < %DIF < %DIF < %DIF < %DIF < %DIF < \end{picture}% %DIF PREAMBLE
%DIF < %DIF < %DIF < %DIF < %DIF < %DIF < %DIF < %DIF < }\hspace*{3pt}}} %DIF PREAMBLE
%DIF < %DIF < %DIF < %DIF < %DIF < %DIF < %DIF < %DIF < } %DIF PREAMBLE
%DIF < %DIF < %DIF < %DIF < %DIF < %DIF < %DIF < %DIF < \LetLtxMacro{\DIFOaddbegin}{\DIFaddbegin} %DIF PREAMBLE
%DIF < %DIF < %DIF < %DIF < %DIF < %DIF < %DIF < %DIF < \LetLtxMacro{\DIFOaddend}{\DIFaddend} %DIF PREAMBLE
%DIF < %DIF < %DIF < %DIF < %DIF < %DIF < %DIF < %DIF < \LetLtxMacro{\DIFOdelbegin}{\DIFdelbegin} %DIF PREAMBLE
%DIF < %DIF < %DIF < %DIF < %DIF < %DIF < %DIF < %DIF < \LetLtxMacro{\DIFOdelend}{\DIFdelend} %DIF PREAMBLE
%DIF < %DIF < %DIF < %DIF < %DIF < %DIF < %DIF < %DIF < \DeclareRobustCommand{\DIFaddbegin}{\DIFOaddbegin \let\includegraphics\DIFaddincludegraphics} %DIF PREAMBLE
%DIF < %DIF < %DIF < %DIF < %DIF < %DIF < %DIF < %DIF < \DeclareRobustCommand{\DIFaddend}{\DIFOaddend \let\includegraphics\DIFOincludegraphics} %DIF PREAMBLE
%DIF < %DIF < %DIF < %DIF < %DIF < %DIF < %DIF < %DIF < \DeclareRobustCommand{\DIFdelbegin}{\DIFOdelbegin \let\includegraphics\DIFdelincludegraphics} %DIF PREAMBLE
%DIF < %DIF < %DIF < %DIF < %DIF < %DIF < %DIF < %DIF < \DeclareRobustCommand{\DIFdelend}{\DIFOaddend \let\includegraphics\DIFOincludegraphics} %DIF PREAMBLE
%DIF < %DIF < %DIF < %DIF < %DIF < %DIF < %DIF < %DIF < \LetLtxMacro{\DIFOaddbeginFL}{\DIFaddbeginFL} %DIF PREAMBLE
%DIF < %DIF < %DIF < %DIF < %DIF < %DIF < %DIF < %DIF < \LetLtxMacro{\DIFOaddendFL}{\DIFaddendFL} %DIF PREAMBLE
%DIF < %DIF < %DIF < %DIF < %DIF < %DIF < %DIF < %DIF < \LetLtxMacro{\DIFOdelbeginFL}{\DIFdelbeginFL} %DIF PREAMBLE
%DIF < %DIF < %DIF < %DIF < %DIF < %DIF < %DIF < %DIF < \LetLtxMacro{\DIFOdelendFL}{\DIFdelendFL} %DIF PREAMBLE
%DIF < %DIF < %DIF < %DIF < %DIF < %DIF < %DIF < %DIF < \DeclareRobustCommand{\DIFaddbeginFL}{\DIFOaddbeginFL \let\includegraphics\DIFaddincludegraphics} %DIF PREAMBLE
%DIF < %DIF < %DIF < %DIF < %DIF < %DIF < %DIF < %DIF < \DeclareRobustCommand{\DIFaddendFL}{\DIFOaddendFL \let\includegraphics\DIFOincludegraphics} %DIF PREAMBLE
%DIF < %DIF < %DIF < %DIF < %DIF < %DIF < %DIF < %DIF < \DeclareRobustCommand{\DIFdelbeginFL}{\DIFOdelbeginFL \let\includegraphics\DIFdelincludegraphics} %DIF PREAMBLE
%DIF < %DIF < %DIF < %DIF < %DIF < %DIF < %DIF < %DIF < \DeclareRobustCommand{\DIFdelendFL}{\DIFOaddendFL \let\includegraphics\DIFOincludegraphics} %DIF PREAMBLE
%DIF < %DIF < %DIF < %DIF < %DIF < %DIF < %DIF < %DIF < %DIF AMSMATHULEM PREAMBLE %DIF PREAMBLE
%DIF < %DIF < %DIF < %DIF < %DIF < %DIF < %DIF < %DIF < \makeatletter %DIF PREAMBLE
%DIF < %DIF < %DIF < %DIF < %DIF < %DIF < %DIF < %DIF < \let\sout@orig\sout %DIF PREAMBLE
%DIF < %DIF < %DIF < %DIF < %DIF < %DIF < %DIF < %DIF < \renewcommand{\sout}[1]{\ifmmode\text{\sout@orig{\ensuremath{#1}}}\else\sout@orig{#1}\fi} %DIF PREAMBLE
%DIF < %DIF < %DIF < %DIF < %DIF < %DIF < %DIF < %DIF < \makeatother %DIF PREAMBLE
%DIF < %DIF < %DIF < %DIF < %DIF < %DIF < %DIF < %DIF < %DIF COLORLISTINGS PREAMBLE %DIF PREAMBLE
%DIF < %DIF < %DIF < %DIF < %DIF < %DIF < %DIF < %DIF < \RequirePackage{listings} %DIF PREAMBLE
%DIF < %DIF < %DIF < %DIF < %DIF < %DIF < %DIF < %DIF < \RequirePackage{color} %DIF PREAMBLE
%DIF < %DIF < %DIF < %DIF < %DIF < %DIF < %DIF < %DIF < \lstdefinelanguage{DIFcode}{ %DIF PREAMBLE
%DIF < %DIF < %DIF < %DIF < %DIF < %DIF < %DIF < %DIF < %DIF DIFCODE_CULINECHBAR %DIF PREAMBLE
%DIF < %DIF < %DIF < %DIF < %DIF < %DIF < %DIF < %DIF <   moredelim=[il][\color{red}\sout]{\%DIF\ <\ }, %DIF PREAMBLE
%DIF < %DIF < %DIF < %DIF < %DIF < %DIF < %DIF < %DIF <   moredelim=[il][\color{blue}\uwave]{\%DIF\ >\ } %DIF PREAMBLE
%DIF < %DIF < %DIF < %DIF < %DIF < %DIF < %DIF < %DIF < } %DIF PREAMBLE
%DIF < %DIF < %DIF < %DIF < %DIF < %DIF < %DIF < %DIF < \lstdefinestyle{DIFverbatimstyle}{ %DIF PREAMBLE
%DIF < %DIF < %DIF < %DIF < %DIF < %DIF < %DIF < %DIF < 	language=DIFcode, %DIF PREAMBLE
%DIF < %DIF < %DIF < %DIF < %DIF < %DIF < %DIF < %DIF < 	basicstyle=\ttfamily, %DIF PREAMBLE
%DIF < %DIF < %DIF < %DIF < %DIF < %DIF < %DIF < %DIF < 	columns=fullflexible, %DIF PREAMBLE
%DIF < %DIF < %DIF < %DIF < %DIF < %DIF < %DIF < %DIF < 	keepspaces=true %DIF PREAMBLE
%DIF < %DIF < %DIF < %DIF < %DIF < %DIF < %DIF < %DIF < } %DIF PREAMBLE
%DIF < %DIF < %DIF < %DIF < %DIF < %DIF < %DIF < %DIF < \lstnewenvironment{DIFverbatim}{\lstset{style=DIFverbatimstyle}}{} %DIF PREAMBLE
%DIF < %DIF < %DIF < %DIF < %DIF < %DIF < %DIF < %DIF < \lstnewenvironment{DIFverbatim*}{\lstset{style=DIFverbatimstyle,showspaces=true}}{} %DIF PREAMBLE
%DIF < %DIF < %DIF < %DIF < %DIF < %DIF < %DIF < %DIF < \lstset{extendedchars=\true,inputencoding=utf8}
%DIF < %DIF < %DIF < %DIF < %DIF < %DIF < %DIF < %DIF < 
%DIF < %DIF < %DIF < %DIF < %DIF < %DIF < %DIF < %DIF < %DIF END PREAMBLE EXTENSION ADDED BY LATEXDIFF
%DIF < %DIF < %DIF < %DIF < %DIF < %DIF < %DIF < %DIF PREAMBLE EXTENSION ADDED BY LATEXDIFF
%DIF < %DIF < %DIF < %DIF < %DIF < %DIF < %DIF < %DIF CULINECHBAR PREAMBLE %DIF PREAMBLE
%DIF < %DIF < %DIF < %DIF < %DIF < %DIF < %DIF < \RequirePackage[normalem]{ulem} %DIF PREAMBLE
%DIF < %DIF < %DIF < %DIF < %DIF < %DIF < %DIF < \RequirePackage[pdftex]{changebar} %DIF PREAMBLE
%DIF < %DIF < %DIF < %DIF < %DIF < %DIF < %DIF < \RequirePackage{color}\definecolor{RED}{rgb}{1,0,0}\definecolor{BLUE}{rgb}{0,0,1} %DIF PREAMBLE
%DIF < %DIF < %DIF < %DIF < %DIF < %DIF < %DIF < \providecommand{\DIFaddtex}[1]{\protect\cbstart{\protect\color{blue}\uwave{#1}}\protect\cbend} %DIF PREAMBLE
%DIF < %DIF < %DIF < %DIF < %DIF < %DIF < %DIF < \providecommand{\DIFdeltex}[1]{\protect\cbdelete{\protect\color{red}\sout{#1}}\protect\cbdelete} %DIF PREAMBLE
%DIF < %DIF < %DIF < %DIF < %DIF < %DIF < %DIF < %DIF SAFE PREAMBLE %DIF PREAMBLE
%DIF < %DIF < %DIF < %DIF < %DIF < %DIF < %DIF < \providecommand{\DIFaddbegin}{} %DIF PREAMBLE
%DIF < %DIF < %DIF < %DIF < %DIF < %DIF < %DIF < \providecommand{\DIFaddend}{} %DIF PREAMBLE
%DIF < %DIF < %DIF < %DIF < %DIF < %DIF < %DIF < \providecommand{\DIFdelbegin}{} %DIF PREAMBLE
%DIF < %DIF < %DIF < %DIF < %DIF < %DIF < %DIF < \providecommand{\DIFdelend}{} %DIF PREAMBLE
%DIF < %DIF < %DIF < %DIF < %DIF < %DIF < %DIF < \providecommand{\DIFmodbegin}{} %DIF PREAMBLE
%DIF < %DIF < %DIF < %DIF < %DIF < %DIF < %DIF < \providecommand{\DIFmodend}{} %DIF PREAMBLE
%DIF < %DIF < %DIF < %DIF < %DIF < %DIF < %DIF < %DIF FLOATSAFE PREAMBLE %DIF PREAMBLE
%DIF < %DIF < %DIF < %DIF < %DIF < %DIF < %DIF < \providecommand{\DIFaddFL}[1]{\DIFadd{#1}} %DIF PREAMBLE
%DIF < %DIF < %DIF < %DIF < %DIF < %DIF < %DIF < \providecommand{\DIFdelFL}[1]{\DIFdel{#1}} %DIF PREAMBLE
%DIF < %DIF < %DIF < %DIF < %DIF < %DIF < %DIF < \providecommand{\DIFaddbeginFL}{} %DIF PREAMBLE
%DIF < %DIF < %DIF < %DIF < %DIF < %DIF < %DIF < \providecommand{\DIFaddendFL}{} %DIF PREAMBLE
%DIF < %DIF < %DIF < %DIF < %DIF < %DIF < %DIF < \providecommand{\DIFdelbeginFL}{} %DIF PREAMBLE
%DIF < %DIF < %DIF < %DIF < %DIF < %DIF < %DIF < \providecommand{\DIFdelendFL}{} %DIF PREAMBLE
%DIF < %DIF < %DIF < %DIF < %DIF < %DIF < %DIF < %DIF HYPERREF PREAMBLE %DIF PREAMBLE
%DIF < %DIF < %DIF < %DIF < %DIF < %DIF < %DIF < \providecommand{\DIFadd}[1]{\texorpdfstring{\DIFaddtex{#1}}{#1}} %DIF PREAMBLE
%DIF < %DIF < %DIF < %DIF < %DIF < %DIF < %DIF < \providecommand{\DIFdel}[1]{\texorpdfstring{\DIFdeltex{#1}}{}} %DIF PREAMBLE
%DIF < %DIF < %DIF < %DIF < %DIF < %DIF < %DIF < \newcommand{\DIFscaledelfig}{0.5}
%DIF < %DIF < %DIF < %DIF < %DIF < %DIF < %DIF < %DIF HIGHLIGHTGRAPHICS PREAMBLE %DIF PREAMBLE
%DIF < %DIF < %DIF < %DIF < %DIF < %DIF < %DIF < \RequirePackage{settobox} %DIF PREAMBLE
%DIF < %DIF < %DIF < %DIF < %DIF < %DIF < %DIF < \RequirePackage{letltxmacro} %DIF PREAMBLE
%DIF < %DIF < %DIF < %DIF < %DIF < %DIF < %DIF < \newsavebox{\DIFdelgraphicsbox} %DIF PREAMBLE
%DIF < %DIF < %DIF < %DIF < %DIF < %DIF < %DIF < \newlength{\DIFdelgraphicswidth} %DIF PREAMBLE
%DIF < %DIF < %DIF < %DIF < %DIF < %DIF < %DIF < \newlength{\DIFdelgraphicsheight} %DIF PREAMBLE
%DIF < %DIF < %DIF < %DIF < %DIF < %DIF < %DIF < % store original definition of \includegraphics %DIF PREAMBLE
%DIF < %DIF < %DIF < %DIF < %DIF < %DIF < %DIF < \LetLtxMacro{\DIFOincludegraphics}{\includegraphics} %DIF PREAMBLE
%DIF < %DIF < %DIF < %DIF < %DIF < %DIF < %DIF < \newcommand{\DIFaddincludegraphics}[2][]{{\color{blue}\fbox{\DIFOincludegraphics[#1]{#2}}}} %DIF PREAMBLE
%DIF < %DIF < %DIF < %DIF < %DIF < %DIF < %DIF < \newcommand{\DIFdelincludegraphics}[2][]{% %DIF PREAMBLE
%DIF < %DIF < %DIF < %DIF < %DIF < %DIF < %DIF < \sbox{\DIFdelgraphicsbox}{\DIFOincludegraphics[#1]{#2}}% %DIF PREAMBLE
%DIF < %DIF < %DIF < %DIF < %DIF < %DIF < %DIF < \settoboxwidth{\DIFdelgraphicswidth}{\DIFdelgraphicsbox} %DIF PREAMBLE
%DIF < %DIF < %DIF < %DIF < %DIF < %DIF < %DIF < \settoboxtotalheight{\DIFdelgraphicsheight}{\DIFdelgraphicsbox} %DIF PREAMBLE
%DIF < %DIF < %DIF < %DIF < %DIF < %DIF < %DIF < \scalebox{\DIFscaledelfig}{% %DIF PREAMBLE
%DIF < %DIF < %DIF < %DIF < %DIF < %DIF < %DIF < \parbox[b]{\DIFdelgraphicswidth}{\usebox{\DIFdelgraphicsbox}\\[-\baselineskip] \rule{\DIFdelgraphicswidth}{0em}}\llap{\resizebox{\DIFdelgraphicswidth}{\DIFdelgraphicsheight}{% %DIF PREAMBLE
%DIF < %DIF < %DIF < %DIF < %DIF < %DIF < %DIF < \setlength{\unitlength}{\DIFdelgraphicswidth}% %DIF PREAMBLE
%DIF < %DIF < %DIF < %DIF < %DIF < %DIF < %DIF < \begin{picture}(1,1)% %DIF PREAMBLE
%DIF < %DIF < %DIF < %DIF < %DIF < %DIF < %DIF < \thicklines\linethickness{2pt} %DIF PREAMBLE
%DIF < %DIF < %DIF < %DIF < %DIF < %DIF < %DIF < {\color[rgb]{1,0,0}\put(0,0){\framebox(1,1){}}}% %DIF PREAMBLE
%DIF < %DIF < %DIF < %DIF < %DIF < %DIF < %DIF < {\color[rgb]{1,0,0}\put(0,0){\line( 1,1){1}}}% %DIF PREAMBLE
%DIF < %DIF < %DIF < %DIF < %DIF < %DIF < %DIF < {\color[rgb]{1,0,0}\put(0,1){\line(1,-1){1}}}% %DIF PREAMBLE
%DIF < %DIF < %DIF < %DIF < %DIF < %DIF < %DIF < \end{picture}% %DIF PREAMBLE
%DIF < %DIF < %DIF < %DIF < %DIF < %DIF < %DIF < }\hspace*{3pt}}} %DIF PREAMBLE
%DIF < %DIF < %DIF < %DIF < %DIF < %DIF < %DIF < } %DIF PREAMBLE
%DIF < %DIF < %DIF < %DIF < %DIF < %DIF < %DIF < \LetLtxMacro{\DIFOaddbegin}{\DIFaddbegin} %DIF PREAMBLE
%DIF < %DIF < %DIF < %DIF < %DIF < %DIF < %DIF < \LetLtxMacro{\DIFOaddend}{\DIFaddend} %DIF PREAMBLE
%DIF < %DIF < %DIF < %DIF < %DIF < %DIF < %DIF < \LetLtxMacro{\DIFOdelbegin}{\DIFdelbegin} %DIF PREAMBLE
%DIF < %DIF < %DIF < %DIF < %DIF < %DIF < %DIF < \LetLtxMacro{\DIFOdelend}{\DIFdelend} %DIF PREAMBLE
%DIF < %DIF < %DIF < %DIF < %DIF < %DIF < %DIF < \DeclareRobustCommand{\DIFaddbegin}{\DIFOaddbegin \let\includegraphics\DIFaddincludegraphics} %DIF PREAMBLE
%DIF < %DIF < %DIF < %DIF < %DIF < %DIF < %DIF < \DeclareRobustCommand{\DIFaddend}{\DIFOaddend \let\includegraphics\DIFOincludegraphics} %DIF PREAMBLE
%DIF < %DIF < %DIF < %DIF < %DIF < %DIF < %DIF < \DeclareRobustCommand{\DIFdelbegin}{\DIFOdelbegin \let\includegraphics\DIFdelincludegraphics} %DIF PREAMBLE
%DIF < %DIF < %DIF < %DIF < %DIF < %DIF < %DIF < \DeclareRobustCommand{\DIFdelend}{\DIFOaddend \let\includegraphics\DIFOincludegraphics} %DIF PREAMBLE
%DIF < %DIF < %DIF < %DIF < %DIF < %DIF < %DIF < \LetLtxMacro{\DIFOaddbeginFL}{\DIFaddbeginFL} %DIF PREAMBLE
%DIF < %DIF < %DIF < %DIF < %DIF < %DIF < %DIF < \LetLtxMacro{\DIFOaddendFL}{\DIFaddendFL} %DIF PREAMBLE
%DIF < %DIF < %DIF < %DIF < %DIF < %DIF < %DIF < \LetLtxMacro{\DIFOdelbeginFL}{\DIFdelbeginFL} %DIF PREAMBLE
%DIF < %DIF < %DIF < %DIF < %DIF < %DIF < %DIF < \LetLtxMacro{\DIFOdelendFL}{\DIFdelendFL} %DIF PREAMBLE
%DIF < %DIF < %DIF < %DIF < %DIF < %DIF < %DIF < \DeclareRobustCommand{\DIFaddbeginFL}{\DIFOaddbeginFL \let\includegraphics\DIFaddincludegraphics} %DIF PREAMBLE
%DIF < %DIF < %DIF < %DIF < %DIF < %DIF < %DIF < \DeclareRobustCommand{\DIFaddendFL}{\DIFOaddendFL \let\includegraphics\DIFOincludegraphics} %DIF PREAMBLE
%DIF < %DIF < %DIF < %DIF < %DIF < %DIF < %DIF < \DeclareRobustCommand{\DIFdelbeginFL}{\DIFOdelbeginFL \let\includegraphics\DIFdelincludegraphics} %DIF PREAMBLE
%DIF < %DIF < %DIF < %DIF < %DIF < %DIF < %DIF < \DeclareRobustCommand{\DIFdelendFL}{\DIFOaddendFL \let\includegraphics\DIFOincludegraphics} %DIF PREAMBLE
%DIF < %DIF < %DIF < %DIF < %DIF < %DIF < %DIF < %DIF AMSMATHULEM PREAMBLE %DIF PREAMBLE
%DIF < %DIF < %DIF < %DIF < %DIF < %DIF < %DIF < \makeatletter %DIF PREAMBLE
%DIF < %DIF < %DIF < %DIF < %DIF < %DIF < %DIF < \let\sout@orig\sout %DIF PREAMBLE
%DIF < %DIF < %DIF < %DIF < %DIF < %DIF < %DIF < \renewcommand{\sout}[1]{\ifmmode\text{\sout@orig{\ensuremath{#1}}}\else\sout@orig{#1}\fi} %DIF PREAMBLE
%DIF < %DIF < %DIF < %DIF < %DIF < %DIF < %DIF < \makeatother %DIF PREAMBLE
%DIF < %DIF < %DIF < %DIF < %DIF < %DIF < %DIF < %DIF COLORLISTINGS PREAMBLE %DIF PREAMBLE
%DIF < %DIF < %DIF < %DIF < %DIF < %DIF < %DIF < \RequirePackage{listings} %DIF PREAMBLE
%DIF < %DIF < %DIF < %DIF < %DIF < %DIF < %DIF < \RequirePackage{color} %DIF PREAMBLE
%DIF < %DIF < %DIF < %DIF < %DIF < %DIF < %DIF < \lstdefinelanguage{DIFcode}{ %DIF PREAMBLE
%DIF < %DIF < %DIF < %DIF < %DIF < %DIF < %DIF < %DIF DIFCODE_CULINECHBAR %DIF PREAMBLE
%DIF < %DIF < %DIF < %DIF < %DIF < %DIF < %DIF <   moredelim=[il][\color{red}\sout]{\%DIF\ <\ }, %DIF PREAMBLE
%DIF < %DIF < %DIF < %DIF < %DIF < %DIF < %DIF <   moredelim=[il][\color{blue}\uwave]{\%DIF\ >\ } %DIF PREAMBLE
%DIF < %DIF < %DIF < %DIF < %DIF < %DIF < %DIF < } %DIF PREAMBLE
%DIF < %DIF < %DIF < %DIF < %DIF < %DIF < %DIF < \lstdefinestyle{DIFverbatimstyle}{ %DIF PREAMBLE
%DIF < %DIF < %DIF < %DIF < %DIF < %DIF < %DIF < 	language=DIFcode, %DIF PREAMBLE
%DIF < %DIF < %DIF < %DIF < %DIF < %DIF < %DIF < 	basicstyle=\ttfamily, %DIF PREAMBLE
%DIF < %DIF < %DIF < %DIF < %DIF < %DIF < %DIF < 	columns=fullflexible, %DIF PREAMBLE
%DIF < %DIF < %DIF < %DIF < %DIF < %DIF < %DIF < 	keepspaces=true %DIF PREAMBLE
%DIF < %DIF < %DIF < %DIF < %DIF < %DIF < %DIF < } %DIF PREAMBLE
%DIF < %DIF < %DIF < %DIF < %DIF < %DIF < %DIF < \lstnewenvironment{DIFverbatim}{\lstset{style=DIFverbatimstyle}}{} %DIF PREAMBLE
%DIF < %DIF < %DIF < %DIF < %DIF < %DIF < %DIF < \lstnewenvironment{DIFverbatim*}{\lstset{style=DIFverbatimstyle,showspaces=true}}{} %DIF PREAMBLE
%DIF < %DIF < %DIF < %DIF < %DIF < %DIF < %DIF < \lstset{extendedchars=\true,inputencoding=utf8}
%DIF < %DIF < %DIF < %DIF < %DIF < %DIF < %DIF < 
%DIF < %DIF < %DIF < %DIF < %DIF < %DIF < %DIF < %DIF END PREAMBLE EXTENSION ADDED BY LATEXDIFF
%DIF < %DIF < %DIF < %DIF < %DIF < %DIF < %DIF PREAMBLE EXTENSION ADDED BY LATEXDIFF
%DIF < %DIF < %DIF < %DIF < %DIF < %DIF < %DIF CULINECHBAR PREAMBLE %DIF PREAMBLE
%DIF < %DIF < %DIF < %DIF < %DIF < %DIF < \RequirePackage[normalem]{ulem} %DIF PREAMBLE
%DIF < %DIF < %DIF < %DIF < %DIF < %DIF < \RequirePackage[pdftex]{changebar} %DIF PREAMBLE
%DIF < %DIF < %DIF < %DIF < %DIF < %DIF < \RequirePackage{color}\definecolor{RED}{rgb}{1,0,0}\definecolor{BLUE}{rgb}{0,0,1} %DIF PREAMBLE
%DIF < %DIF < %DIF < %DIF < %DIF < %DIF < \providecommand{\DIFaddtex}[1]{\protect\cbstart{\protect\color{blue}\uwave{#1}}\protect\cbend} %DIF PREAMBLE
%DIF < %DIF < %DIF < %DIF < %DIF < %DIF < \providecommand{\DIFdeltex}[1]{\protect\cbdelete{\protect\color{red}\sout{#1}}\protect\cbdelete} %DIF PREAMBLE
%DIF < %DIF < %DIF < %DIF < %DIF < %DIF < %DIF SAFE PREAMBLE %DIF PREAMBLE
%DIF < %DIF < %DIF < %DIF < %DIF < %DIF < \providecommand{\DIFaddbegin}{} %DIF PREAMBLE
%DIF < %DIF < %DIF < %DIF < %DIF < %DIF < \providecommand{\DIFaddend}{} %DIF PREAMBLE
%DIF < %DIF < %DIF < %DIF < %DIF < %DIF < \providecommand{\DIFdelbegin}{} %DIF PREAMBLE
%DIF < %DIF < %DIF < %DIF < %DIF < %DIF < \providecommand{\DIFdelend}{} %DIF PREAMBLE
%DIF < %DIF < %DIF < %DIF < %DIF < %DIF < \providecommand{\DIFmodbegin}{} %DIF PREAMBLE
%DIF < %DIF < %DIF < %DIF < %DIF < %DIF < \providecommand{\DIFmodend}{} %DIF PREAMBLE
%DIF < %DIF < %DIF < %DIF < %DIF < %DIF < %DIF FLOATSAFE PREAMBLE %DIF PREAMBLE
%DIF < %DIF < %DIF < %DIF < %DIF < %DIF < \providecommand{\DIFaddFL}[1]{\DIFadd{#1}} %DIF PREAMBLE
%DIF < %DIF < %DIF < %DIF < %DIF < %DIF < \providecommand{\DIFdelFL}[1]{\DIFdel{#1}} %DIF PREAMBLE
%DIF < %DIF < %DIF < %DIF < %DIF < %DIF < \providecommand{\DIFaddbeginFL}{} %DIF PREAMBLE
%DIF < %DIF < %DIF < %DIF < %DIF < %DIF < \providecommand{\DIFaddendFL}{} %DIF PREAMBLE
%DIF < %DIF < %DIF < %DIF < %DIF < %DIF < \providecommand{\DIFdelbeginFL}{} %DIF PREAMBLE
%DIF < %DIF < %DIF < %DIF < %DIF < %DIF < \providecommand{\DIFdelendFL}{} %DIF PREAMBLE
%DIF < %DIF < %DIF < %DIF < %DIF < %DIF < %DIF HYPERREF PREAMBLE %DIF PREAMBLE
%DIF < %DIF < %DIF < %DIF < %DIF < %DIF < \providecommand{\DIFadd}[1]{\texorpdfstring{\DIFaddtex{#1}}{#1}} %DIF PREAMBLE
%DIF < %DIF < %DIF < %DIF < %DIF < %DIF < \providecommand{\DIFdel}[1]{\texorpdfstring{\DIFdeltex{#1}}{}} %DIF PREAMBLE
%DIF < %DIF < %DIF < %DIF < %DIF < %DIF < \newcommand{\DIFscaledelfig}{0.5}
%DIF < %DIF < %DIF < %DIF < %DIF < %DIF < %DIF HIGHLIGHTGRAPHICS PREAMBLE %DIF PREAMBLE
%DIF < %DIF < %DIF < %DIF < %DIF < %DIF < \RequirePackage{settobox} %DIF PREAMBLE
%DIF < %DIF < %DIF < %DIF < %DIF < %DIF < \RequirePackage{letltxmacro} %DIF PREAMBLE
%DIF < %DIF < %DIF < %DIF < %DIF < %DIF < \newsavebox{\DIFdelgraphicsbox} %DIF PREAMBLE
%DIF < %DIF < %DIF < %DIF < %DIF < %DIF < \newlength{\DIFdelgraphicswidth} %DIF PREAMBLE
%DIF < %DIF < %DIF < %DIF < %DIF < %DIF < \newlength{\DIFdelgraphicsheight} %DIF PREAMBLE
%DIF < %DIF < %DIF < %DIF < %DIF < %DIF < % store original definition of \includegraphics %DIF PREAMBLE
%DIF < %DIF < %DIF < %DIF < %DIF < %DIF < \LetLtxMacro{\DIFOincludegraphics}{\includegraphics} %DIF PREAMBLE
%DIF < %DIF < %DIF < %DIF < %DIF < %DIF < \newcommand{\DIFaddincludegraphics}[2][]{{\color{blue}\fbox{\DIFOincludegraphics[#1]{#2}}}} %DIF PREAMBLE
%DIF < %DIF < %DIF < %DIF < %DIF < %DIF < \newcommand{\DIFdelincludegraphics}[2][]{% %DIF PREAMBLE
%DIF < %DIF < %DIF < %DIF < %DIF < %DIF < \sbox{\DIFdelgraphicsbox}{\DIFOincludegraphics[#1]{#2}}% %DIF PREAMBLE
%DIF < %DIF < %DIF < %DIF < %DIF < %DIF < \settoboxwidth{\DIFdelgraphicswidth}{\DIFdelgraphicsbox} %DIF PREAMBLE
%DIF < %DIF < %DIF < %DIF < %DIF < %DIF < \settoboxtotalheight{\DIFdelgraphicsheight}{\DIFdelgraphicsbox} %DIF PREAMBLE
%DIF < %DIF < %DIF < %DIF < %DIF < %DIF < \scalebox{\DIFscaledelfig}{% %DIF PREAMBLE
%DIF < %DIF < %DIF < %DIF < %DIF < %DIF < \parbox[b]{\DIFdelgraphicswidth}{\usebox{\DIFdelgraphicsbox}\\[-\baselineskip] \rule{\DIFdelgraphicswidth}{0em}}\llap{\resizebox{\DIFdelgraphicswidth}{\DIFdelgraphicsheight}{% %DIF PREAMBLE
%DIF < %DIF < %DIF < %DIF < %DIF < %DIF < \setlength{\unitlength}{\DIFdelgraphicswidth}% %DIF PREAMBLE
%DIF < %DIF < %DIF < %DIF < %DIF < %DIF < \begin{picture}(1,1)% %DIF PREAMBLE
%DIF < %DIF < %DIF < %DIF < %DIF < %DIF < \thicklines\linethickness{2pt} %DIF PREAMBLE
%DIF < %DIF < %DIF < %DIF < %DIF < %DIF < {\color[rgb]{1,0,0}\put(0,0){\framebox(1,1){}}}% %DIF PREAMBLE
%DIF < %DIF < %DIF < %DIF < %DIF < %DIF < {\color[rgb]{1,0,0}\put(0,0){\line( 1,1){1}}}% %DIF PREAMBLE
%DIF < %DIF < %DIF < %DIF < %DIF < %DIF < {\color[rgb]{1,0,0}\put(0,1){\line(1,-1){1}}}% %DIF PREAMBLE
%DIF < %DIF < %DIF < %DIF < %DIF < %DIF < \end{picture}% %DIF PREAMBLE
%DIF < %DIF < %DIF < %DIF < %DIF < %DIF < }\hspace*{3pt}}} %DIF PREAMBLE
%DIF < %DIF < %DIF < %DIF < %DIF < %DIF < } %DIF PREAMBLE
%DIF < %DIF < %DIF < %DIF < %DIF < %DIF < \LetLtxMacro{\DIFOaddbegin}{\DIFaddbegin} %DIF PREAMBLE
%DIF < %DIF < %DIF < %DIF < %DIF < %DIF < \LetLtxMacro{\DIFOaddend}{\DIFaddend} %DIF PREAMBLE
%DIF < %DIF < %DIF < %DIF < %DIF < %DIF < \LetLtxMacro{\DIFOdelbegin}{\DIFdelbegin} %DIF PREAMBLE
%DIF < %DIF < %DIF < %DIF < %DIF < %DIF < \LetLtxMacro{\DIFOdelend}{\DIFdelend} %DIF PREAMBLE
%DIF < %DIF < %DIF < %DIF < %DIF < %DIF < \DeclareRobustCommand{\DIFaddbegin}{\DIFOaddbegin \let\includegraphics\DIFaddincludegraphics} %DIF PREAMBLE
%DIF < %DIF < %DIF < %DIF < %DIF < %DIF < \DeclareRobustCommand{\DIFaddend}{\DIFOaddend \let\includegraphics\DIFOincludegraphics} %DIF PREAMBLE
%DIF < %DIF < %DIF < %DIF < %DIF < %DIF < \DeclareRobustCommand{\DIFdelbegin}{\DIFOdelbegin \let\includegraphics\DIFdelincludegraphics} %DIF PREAMBLE
%DIF < %DIF < %DIF < %DIF < %DIF < %DIF < \DeclareRobustCommand{\DIFdelend}{\DIFOaddend \let\includegraphics\DIFOincludegraphics} %DIF PREAMBLE
%DIF < %DIF < %DIF < %DIF < %DIF < %DIF < \LetLtxMacro{\DIFOaddbeginFL}{\DIFaddbeginFL} %DIF PREAMBLE
%DIF < %DIF < %DIF < %DIF < %DIF < %DIF < \LetLtxMacro{\DIFOaddendFL}{\DIFaddendFL} %DIF PREAMBLE
%DIF < %DIF < %DIF < %DIF < %DIF < %DIF < \LetLtxMacro{\DIFOdelbeginFL}{\DIFdelbeginFL} %DIF PREAMBLE
%DIF < %DIF < %DIF < %DIF < %DIF < %DIF < \LetLtxMacro{\DIFOdelendFL}{\DIFdelendFL} %DIF PREAMBLE
%DIF < %DIF < %DIF < %DIF < %DIF < %DIF < \DeclareRobustCommand{\DIFaddbeginFL}{\DIFOaddbeginFL \let\includegraphics\DIFaddincludegraphics} %DIF PREAMBLE
%DIF < %DIF < %DIF < %DIF < %DIF < %DIF < \DeclareRobustCommand{\DIFaddendFL}{\DIFOaddendFL \let\includegraphics\DIFOincludegraphics} %DIF PREAMBLE
%DIF < %DIF < %DIF < %DIF < %DIF < %DIF < \DeclareRobustCommand{\DIFdelbeginFL}{\DIFOdelbeginFL \let\includegraphics\DIFdelincludegraphics} %DIF PREAMBLE
%DIF < %DIF < %DIF < %DIF < %DIF < %DIF < \DeclareRobustCommand{\DIFdelendFL}{\DIFOaddendFL \let\includegraphics\DIFOincludegraphics} %DIF PREAMBLE
%DIF < %DIF < %DIF < %DIF < %DIF < %DIF < %DIF AMSMATHULEM PREAMBLE %DIF PREAMBLE
%DIF < %DIF < %DIF < %DIF < %DIF < %DIF < \makeatletter %DIF PREAMBLE
%DIF < %DIF < %DIF < %DIF < %DIF < %DIF < \let\sout@orig\sout %DIF PREAMBLE
%DIF < %DIF < %DIF < %DIF < %DIF < %DIF < \renewcommand{\sout}[1]{\ifmmode\text{\sout@orig{\ensuremath{#1}}}\else\sout@orig{#1}\fi} %DIF PREAMBLE
%DIF < %DIF < %DIF < %DIF < %DIF < %DIF < \makeatother %DIF PREAMBLE
%DIF < %DIF < %DIF < %DIF < %DIF < %DIF < %DIF COLORLISTINGS PREAMBLE %DIF PREAMBLE
%DIF < %DIF < %DIF < %DIF < %DIF < %DIF < \RequirePackage{listings} %DIF PREAMBLE
%DIF < %DIF < %DIF < %DIF < %DIF < %DIF < \RequirePackage{color} %DIF PREAMBLE
%DIF < %DIF < %DIF < %DIF < %DIF < %DIF < \lstdefinelanguage{DIFcode}{ %DIF PREAMBLE
%DIF < %DIF < %DIF < %DIF < %DIF < %DIF < %DIF DIFCODE_CULINECHBAR %DIF PREAMBLE
%DIF < %DIF < %DIF < %DIF < %DIF < %DIF <   moredelim=[il][\color{red}\sout]{\%DIF\ <\ }, %DIF PREAMBLE
%DIF < %DIF < %DIF < %DIF < %DIF < %DIF <   moredelim=[il][\color{blue}\uwave]{\%DIF\ >\ } %DIF PREAMBLE
%DIF < %DIF < %DIF < %DIF < %DIF < %DIF < } %DIF PREAMBLE
%DIF < %DIF < %DIF < %DIF < %DIF < %DIF < \lstdefinestyle{DIFverbatimstyle}{ %DIF PREAMBLE
%DIF < %DIF < %DIF < %DIF < %DIF < %DIF < 	language=DIFcode, %DIF PREAMBLE
%DIF < %DIF < %DIF < %DIF < %DIF < %DIF < 	basicstyle=\ttfamily, %DIF PREAMBLE
%DIF < %DIF < %DIF < %DIF < %DIF < %DIF < 	columns=fullflexible, %DIF PREAMBLE
%DIF < %DIF < %DIF < %DIF < %DIF < %DIF < 	keepspaces=true %DIF PREAMBLE
%DIF < %DIF < %DIF < %DIF < %DIF < %DIF < } %DIF PREAMBLE
%DIF < %DIF < %DIF < %DIF < %DIF < %DIF < \lstnewenvironment{DIFverbatim}{\lstset{style=DIFverbatimstyle}}{} %DIF PREAMBLE
%DIF < %DIF < %DIF < %DIF < %DIF < %DIF < \lstnewenvironment{DIFverbatim*}{\lstset{style=DIFverbatimstyle,showspaces=true}}{} %DIF PREAMBLE
%DIF < %DIF < %DIF < %DIF < %DIF < %DIF < \lstset{extendedchars=\true,inputencoding=utf8}
%DIF < %DIF < %DIF < %DIF < %DIF < %DIF < 
%DIF < %DIF < %DIF < %DIF < %DIF < %DIF < %DIF END PREAMBLE EXTENSION ADDED BY LATEXDIFF
%DIF < %DIF < %DIF < %DIF < %DIF < %DIF PREAMBLE EXTENSION ADDED BY LATEXDIFF
%DIF < %DIF < %DIF < %DIF < %DIF < %DIF CULINECHBAR PREAMBLE %DIF PREAMBLE
%DIF < %DIF < %DIF < %DIF < %DIF < \RequirePackage[normalem]{ulem} %DIF PREAMBLE
%DIF < %DIF < %DIF < %DIF < %DIF < \RequirePackage[pdftex]{changebar} %DIF PREAMBLE
%DIF < %DIF < %DIF < %DIF < %DIF < \RequirePackage{color}\definecolor{RED}{rgb}{1,0,0}\definecolor{BLUE}{rgb}{0,0,1} %DIF PREAMBLE
%DIF < %DIF < %DIF < %DIF < %DIF < \providecommand{\DIFaddtex}[1]{\protect\cbstart{\protect\color{blue}\uwave{#1}}\protect\cbend} %DIF PREAMBLE
%DIF < %DIF < %DIF < %DIF < %DIF < \providecommand{\DIFdeltex}[1]{\protect\cbdelete{\protect\color{red}\sout{#1}}\protect\cbdelete} %DIF PREAMBLE
%DIF < %DIF < %DIF < %DIF < %DIF < %DIF SAFE PREAMBLE %DIF PREAMBLE
%DIF < %DIF < %DIF < %DIF < %DIF < \providecommand{\DIFaddbegin}{} %DIF PREAMBLE
%DIF < %DIF < %DIF < %DIF < %DIF < \providecommand{\DIFaddend}{} %DIF PREAMBLE
%DIF < %DIF < %DIF < %DIF < %DIF < \providecommand{\DIFdelbegin}{} %DIF PREAMBLE
%DIF < %DIF < %DIF < %DIF < %DIF < \providecommand{\DIFdelend}{} %DIF PREAMBLE
%DIF < %DIF < %DIF < %DIF < %DIF < \providecommand{\DIFmodbegin}{} %DIF PREAMBLE
%DIF < %DIF < %DIF < %DIF < %DIF < \providecommand{\DIFmodend}{} %DIF PREAMBLE
%DIF < %DIF < %DIF < %DIF < %DIF < %DIF FLOATSAFE PREAMBLE %DIF PREAMBLE
%DIF < %DIF < %DIF < %DIF < %DIF < \providecommand{\DIFaddFL}[1]{\DIFadd{#1}} %DIF PREAMBLE
%DIF < %DIF < %DIF < %DIF < %DIF < \providecommand{\DIFdelFL}[1]{\DIFdel{#1}} %DIF PREAMBLE
%DIF < %DIF < %DIF < %DIF < %DIF < \providecommand{\DIFaddbeginFL}{} %DIF PREAMBLE
%DIF < %DIF < %DIF < %DIF < %DIF < \providecommand{\DIFaddendFL}{} %DIF PREAMBLE
%DIF < %DIF < %DIF < %DIF < %DIF < \providecommand{\DIFdelbeginFL}{} %DIF PREAMBLE
%DIF < %DIF < %DIF < %DIF < %DIF < \providecommand{\DIFdelendFL}{} %DIF PREAMBLE
%DIF < %DIF < %DIF < %DIF < %DIF < %DIF HYPERREF PREAMBLE %DIF PREAMBLE
%DIF < %DIF < %DIF < %DIF < %DIF < \providecommand{\DIFadd}[1]{\texorpdfstring{\DIFaddtex{#1}}{#1}} %DIF PREAMBLE
%DIF < %DIF < %DIF < %DIF < %DIF < \providecommand{\DIFdel}[1]{\texorpdfstring{\DIFdeltex{#1}}{}} %DIF PREAMBLE
%DIF < %DIF < %DIF < %DIF < %DIF < \newcommand{\DIFscaledelfig}{0.5}
%DIF < %DIF < %DIF < %DIF < %DIF < %DIF HIGHLIGHTGRAPHICS PREAMBLE %DIF PREAMBLE
%DIF < %DIF < %DIF < %DIF < %DIF < \RequirePackage{settobox} %DIF PREAMBLE
%DIF < %DIF < %DIF < %DIF < %DIF < \RequirePackage{letltxmacro} %DIF PREAMBLE
%DIF < %DIF < %DIF < %DIF < %DIF < \newsavebox{\DIFdelgraphicsbox} %DIF PREAMBLE
%DIF < %DIF < %DIF < %DIF < %DIF < \newlength{\DIFdelgraphicswidth} %DIF PREAMBLE
%DIF < %DIF < %DIF < %DIF < %DIF < \newlength{\DIFdelgraphicsheight} %DIF PREAMBLE
%DIF < %DIF < %DIF < %DIF < %DIF < % store original definition of \includegraphics %DIF PREAMBLE
%DIF < %DIF < %DIF < %DIF < %DIF < \LetLtxMacro{\DIFOincludegraphics}{\includegraphics} %DIF PREAMBLE
%DIF < %DIF < %DIF < %DIF < %DIF < \newcommand{\DIFaddincludegraphics}[2][]{{\color{blue}\fbox{\DIFOincludegraphics[#1]{#2}}}} %DIF PREAMBLE
%DIF < %DIF < %DIF < %DIF < %DIF < \newcommand{\DIFdelincludegraphics}[2][]{% %DIF PREAMBLE
%DIF < %DIF < %DIF < %DIF < %DIF < \sbox{\DIFdelgraphicsbox}{\DIFOincludegraphics[#1]{#2}}% %DIF PREAMBLE
%DIF < %DIF < %DIF < %DIF < %DIF < \settoboxwidth{\DIFdelgraphicswidth}{\DIFdelgraphicsbox} %DIF PREAMBLE
%DIF < %DIF < %DIF < %DIF < %DIF < \settoboxtotalheight{\DIFdelgraphicsheight}{\DIFdelgraphicsbox} %DIF PREAMBLE
%DIF < %DIF < %DIF < %DIF < %DIF < \scalebox{\DIFscaledelfig}{% %DIF PREAMBLE
%DIF < %DIF < %DIF < %DIF < %DIF < \parbox[b]{\DIFdelgraphicswidth}{\usebox{\DIFdelgraphicsbox}\\[-\baselineskip] \rule{\DIFdelgraphicswidth}{0em}}\llap{\resizebox{\DIFdelgraphicswidth}{\DIFdelgraphicsheight}{% %DIF PREAMBLE
%DIF < %DIF < %DIF < %DIF < %DIF < \setlength{\unitlength}{\DIFdelgraphicswidth}% %DIF PREAMBLE
%DIF < %DIF < %DIF < %DIF < %DIF < \begin{picture}(1,1)% %DIF PREAMBLE
%DIF < %DIF < %DIF < %DIF < %DIF < \thicklines\linethickness{2pt} %DIF PREAMBLE
%DIF < %DIF < %DIF < %DIF < %DIF < {\color[rgb]{1,0,0}\put(0,0){\framebox(1,1){}}}% %DIF PREAMBLE
%DIF < %DIF < %DIF < %DIF < %DIF < {\color[rgb]{1,0,0}\put(0,0){\line( 1,1){1}}}% %DIF PREAMBLE
%DIF < %DIF < %DIF < %DIF < %DIF < {\color[rgb]{1,0,0}\put(0,1){\line(1,-1){1}}}% %DIF PREAMBLE
%DIF < %DIF < %DIF < %DIF < %DIF < \end{picture}% %DIF PREAMBLE
%DIF < %DIF < %DIF < %DIF < %DIF < }\hspace*{3pt}}} %DIF PREAMBLE
%DIF < %DIF < %DIF < %DIF < %DIF < } %DIF PREAMBLE
%DIF < %DIF < %DIF < %DIF < %DIF < \LetLtxMacro{\DIFOaddbegin}{\DIFaddbegin} %DIF PREAMBLE
%DIF < %DIF < %DIF < %DIF < %DIF < \LetLtxMacro{\DIFOaddend}{\DIFaddend} %DIF PREAMBLE
%DIF < %DIF < %DIF < %DIF < %DIF < \LetLtxMacro{\DIFOdelbegin}{\DIFdelbegin} %DIF PREAMBLE
%DIF < %DIF < %DIF < %DIF < %DIF < \LetLtxMacro{\DIFOdelend}{\DIFdelend} %DIF PREAMBLE
%DIF < %DIF < %DIF < %DIF < %DIF < \DeclareRobustCommand{\DIFaddbegin}{\DIFOaddbegin \let\includegraphics\DIFaddincludegraphics} %DIF PREAMBLE
%DIF < %DIF < %DIF < %DIF < %DIF < \DeclareRobustCommand{\DIFaddend}{\DIFOaddend \let\includegraphics\DIFOincludegraphics} %DIF PREAMBLE
%DIF < %DIF < %DIF < %DIF < %DIF < \DeclareRobustCommand{\DIFdelbegin}{\DIFOdelbegin \let\includegraphics\DIFdelincludegraphics} %DIF PREAMBLE
%DIF < %DIF < %DIF < %DIF < %DIF < \DeclareRobustCommand{\DIFdelend}{\DIFOaddend \let\includegraphics\DIFOincludegraphics} %DIF PREAMBLE
%DIF < %DIF < %DIF < %DIF < %DIF < \LetLtxMacro{\DIFOaddbeginFL}{\DIFaddbeginFL} %DIF PREAMBLE
%DIF < %DIF < %DIF < %DIF < %DIF < \LetLtxMacro{\DIFOaddendFL}{\DIFaddendFL} %DIF PREAMBLE
%DIF < %DIF < %DIF < %DIF < %DIF < \LetLtxMacro{\DIFOdelbeginFL}{\DIFdelbeginFL} %DIF PREAMBLE
%DIF < %DIF < %DIF < %DIF < %DIF < \LetLtxMacro{\DIFOdelendFL}{\DIFdelendFL} %DIF PREAMBLE
%DIF < %DIF < %DIF < %DIF < %DIF < \DeclareRobustCommand{\DIFaddbeginFL}{\DIFOaddbeginFL \let\includegraphics\DIFaddincludegraphics} %DIF PREAMBLE
%DIF < %DIF < %DIF < %DIF < %DIF < \DeclareRobustCommand{\DIFaddendFL}{\DIFOaddendFL \let\includegraphics\DIFOincludegraphics} %DIF PREAMBLE
%DIF < %DIF < %DIF < %DIF < %DIF < \DeclareRobustCommand{\DIFdelbeginFL}{\DIFOdelbeginFL \let\includegraphics\DIFdelincludegraphics} %DIF PREAMBLE
%DIF < %DIF < %DIF < %DIF < %DIF < \DeclareRobustCommand{\DIFdelendFL}{\DIFOaddendFL \let\includegraphics\DIFOincludegraphics} %DIF PREAMBLE
%DIF < %DIF < %DIF < %DIF < %DIF < %DIF AMSMATHULEM PREAMBLE %DIF PREAMBLE
%DIF < %DIF < %DIF < %DIF < %DIF < \makeatletter %DIF PREAMBLE
%DIF < %DIF < %DIF < %DIF < %DIF < \let\sout@orig\sout %DIF PREAMBLE
%DIF < %DIF < %DIF < %DIF < %DIF < \renewcommand{\sout}[1]{\ifmmode\text{\sout@orig{\ensuremath{#1}}}\else\sout@orig{#1}\fi} %DIF PREAMBLE
%DIF < %DIF < %DIF < %DIF < %DIF < \makeatother %DIF PREAMBLE
%DIF < %DIF < %DIF < %DIF < %DIF < %DIF COLORLISTINGS PREAMBLE %DIF PREAMBLE
%DIF < %DIF < %DIF < %DIF < %DIF < \RequirePackage{listings} %DIF PREAMBLE
%DIF < %DIF < %DIF < %DIF < %DIF < \RequirePackage{color} %DIF PREAMBLE
%DIF < %DIF < %DIF < %DIF < %DIF < \lstdefinelanguage{DIFcode}{ %DIF PREAMBLE
%DIF < %DIF < %DIF < %DIF < %DIF < %DIF DIFCODE_CULINECHBAR %DIF PREAMBLE
%DIF < %DIF < %DIF < %DIF < %DIF <   moredelim=[il][\color{red}\sout]{\%DIF\ <\ }, %DIF PREAMBLE
%DIF < %DIF < %DIF < %DIF < %DIF <   moredelim=[il][\color{blue}\uwave]{\%DIF\ >\ } %DIF PREAMBLE
%DIF < %DIF < %DIF < %DIF < %DIF < } %DIF PREAMBLE
%DIF < %DIF < %DIF < %DIF < %DIF < \lstdefinestyle{DIFverbatimstyle}{ %DIF PREAMBLE
%DIF < %DIF < %DIF < %DIF < %DIF < 	language=DIFcode, %DIF PREAMBLE
%DIF < %DIF < %DIF < %DIF < %DIF < 	basicstyle=\ttfamily, %DIF PREAMBLE
%DIF < %DIF < %DIF < %DIF < %DIF < 	columns=fullflexible, %DIF PREAMBLE
%DIF < %DIF < %DIF < %DIF < %DIF < 	keepspaces=true %DIF PREAMBLE
%DIF < %DIF < %DIF < %DIF < %DIF < } %DIF PREAMBLE
%DIF < %DIF < %DIF < %DIF < %DIF < \lstnewenvironment{DIFverbatim}{\lstset{style=DIFverbatimstyle}}{} %DIF PREAMBLE
%DIF < %DIF < %DIF < %DIF < %DIF < \lstnewenvironment{DIFverbatim*}{\lstset{style=DIFverbatimstyle,showspaces=true}}{} %DIF PREAMBLE
%DIF < %DIF < %DIF < %DIF < %DIF < \lstset{extendedchars=\true,inputencoding=utf8}
%DIF < %DIF < %DIF < %DIF < %DIF < 
%DIF < %DIF < %DIF < %DIF < %DIF < %DIF END PREAMBLE EXTENSION ADDED BY LATEXDIFF
%DIF < %DIF < %DIF < %DIF < %DIF PREAMBLE EXTENSION ADDED BY LATEXDIFF
%DIF < %DIF < %DIF < %DIF < %DIF CULINECHBAR PREAMBLE %DIF PREAMBLE
%DIF < %DIF < %DIF < %DIF < \RequirePackage[normalem]{ulem} %DIF PREAMBLE
%DIF < %DIF < %DIF < %DIF < \RequirePackage[pdftex]{changebar} %DIF PREAMBLE
%DIF < %DIF < %DIF < %DIF < \RequirePackage{color}\definecolor{RED}{rgb}{1,0,0}\definecolor{BLUE}{rgb}{0,0,1} %DIF PREAMBLE
%DIF < %DIF < %DIF < %DIF < \providecommand{\DIFaddtex}[1]{\protect\cbstart{\protect\color{blue}\uwave{#1}}\protect\cbend} %DIF PREAMBLE
%DIF < %DIF < %DIF < %DIF < \providecommand{\DIFdeltex}[1]{\protect\cbdelete{\protect\color{red}\sout{#1}}\protect\cbdelete} %DIF PREAMBLE
%DIF < %DIF < %DIF < %DIF < %DIF SAFE PREAMBLE %DIF PREAMBLE
%DIF < %DIF < %DIF < %DIF < \providecommand{\DIFaddbegin}{} %DIF PREAMBLE
%DIF < %DIF < %DIF < %DIF < \providecommand{\DIFaddend}{} %DIF PREAMBLE
%DIF < %DIF < %DIF < %DIF < \providecommand{\DIFdelbegin}{} %DIF PREAMBLE
%DIF < %DIF < %DIF < %DIF < \providecommand{\DIFdelend}{} %DIF PREAMBLE
%DIF < %DIF < %DIF < %DIF < \providecommand{\DIFmodbegin}{} %DIF PREAMBLE
%DIF < %DIF < %DIF < %DIF < \providecommand{\DIFmodend}{} %DIF PREAMBLE
%DIF < %DIF < %DIF < %DIF < %DIF FLOATSAFE PREAMBLE %DIF PREAMBLE
%DIF < %DIF < %DIF < %DIF < \providecommand{\DIFaddFL}[1]{\DIFadd{#1}} %DIF PREAMBLE
%DIF < %DIF < %DIF < %DIF < \providecommand{\DIFdelFL}[1]{\DIFdel{#1}} %DIF PREAMBLE
%DIF < %DIF < %DIF < %DIF < \providecommand{\DIFaddbeginFL}{} %DIF PREAMBLE
%DIF < %DIF < %DIF < %DIF < \providecommand{\DIFaddendFL}{} %DIF PREAMBLE
%DIF < %DIF < %DIF < %DIF < \providecommand{\DIFdelbeginFL}{} %DIF PREAMBLE
%DIF < %DIF < %DIF < %DIF < \providecommand{\DIFdelendFL}{} %DIF PREAMBLE
%DIF < %DIF < %DIF < %DIF < %DIF HYPERREF PREAMBLE %DIF PREAMBLE
%DIF < %DIF < %DIF < %DIF < \providecommand{\DIFadd}[1]{\texorpdfstring{\DIFaddtex{#1}}{#1}} %DIF PREAMBLE
%DIF < %DIF < %DIF < %DIF < \providecommand{\DIFdel}[1]{\texorpdfstring{\DIFdeltex{#1}}{}} %DIF PREAMBLE
%DIF < %DIF < %DIF < %DIF < \newcommand{\DIFscaledelfig}{0.5}
%DIF < %DIF < %DIF < %DIF < %DIF HIGHLIGHTGRAPHICS PREAMBLE %DIF PREAMBLE
%DIF < %DIF < %DIF < %DIF < \RequirePackage{settobox} %DIF PREAMBLE
%DIF < %DIF < %DIF < %DIF < \RequirePackage{letltxmacro} %DIF PREAMBLE
%DIF < %DIF < %DIF < %DIF < \newsavebox{\DIFdelgraphicsbox} %DIF PREAMBLE
%DIF < %DIF < %DIF < %DIF < \newlength{\DIFdelgraphicswidth} %DIF PREAMBLE
%DIF < %DIF < %DIF < %DIF < \newlength{\DIFdelgraphicsheight} %DIF PREAMBLE
%DIF < %DIF < %DIF < %DIF < % store original definition of \includegraphics %DIF PREAMBLE
%DIF < %DIF < %DIF < %DIF < \LetLtxMacro{\DIFOincludegraphics}{\includegraphics} %DIF PREAMBLE
%DIF < %DIF < %DIF < %DIF < \newcommand{\DIFaddincludegraphics}[2][]{{\color{blue}\fbox{\DIFOincludegraphics[#1]{#2}}}} %DIF PREAMBLE
%DIF < %DIF < %DIF < %DIF < \newcommand{\DIFdelincludegraphics}[2][]{% %DIF PREAMBLE
%DIF < %DIF < %DIF < %DIF < \sbox{\DIFdelgraphicsbox}{\DIFOincludegraphics[#1]{#2}}% %DIF PREAMBLE
%DIF < %DIF < %DIF < %DIF < \settoboxwidth{\DIFdelgraphicswidth}{\DIFdelgraphicsbox} %DIF PREAMBLE
%DIF < %DIF < %DIF < %DIF < \settoboxtotalheight{\DIFdelgraphicsheight}{\DIFdelgraphicsbox} %DIF PREAMBLE
%DIF < %DIF < %DIF < %DIF < \scalebox{\DIFscaledelfig}{% %DIF PREAMBLE
%DIF < %DIF < %DIF < %DIF < \parbox[b]{\DIFdelgraphicswidth}{\usebox{\DIFdelgraphicsbox}\\[-\baselineskip] \rule{\DIFdelgraphicswidth}{0em}}\llap{\resizebox{\DIFdelgraphicswidth}{\DIFdelgraphicsheight}{% %DIF PREAMBLE
%DIF < %DIF < %DIF < %DIF < \setlength{\unitlength}{\DIFdelgraphicswidth}% %DIF PREAMBLE
%DIF < %DIF < %DIF < %DIF < \begin{picture}(1,1)% %DIF PREAMBLE
%DIF < %DIF < %DIF < %DIF < \thicklines\linethickness{2pt} %DIF PREAMBLE
%DIF < %DIF < %DIF < %DIF < {\color[rgb]{1,0,0}\put(0,0){\framebox(1,1){}}}% %DIF PREAMBLE
%DIF < %DIF < %DIF < %DIF < {\color[rgb]{1,0,0}\put(0,0){\line( 1,1){1}}}% %DIF PREAMBLE
%DIF < %DIF < %DIF < %DIF < {\color[rgb]{1,0,0}\put(0,1){\line(1,-1){1}}}% %DIF PREAMBLE
%DIF < %DIF < %DIF < %DIF < \end{picture}% %DIF PREAMBLE
%DIF < %DIF < %DIF < %DIF < }\hspace*{3pt}}} %DIF PREAMBLE
%DIF < %DIF < %DIF < %DIF < } %DIF PREAMBLE
%DIF < %DIF < %DIF < %DIF < \LetLtxMacro{\DIFOaddbegin}{\DIFaddbegin} %DIF PREAMBLE
%DIF < %DIF < %DIF < %DIF < \LetLtxMacro{\DIFOaddend}{\DIFaddend} %DIF PREAMBLE
%DIF < %DIF < %DIF < %DIF < \LetLtxMacro{\DIFOdelbegin}{\DIFdelbegin} %DIF PREAMBLE
%DIF < %DIF < %DIF < %DIF < \LetLtxMacro{\DIFOdelend}{\DIFdelend} %DIF PREAMBLE
%DIF < %DIF < %DIF < %DIF < \DeclareRobustCommand{\DIFaddbegin}{\DIFOaddbegin \let\includegraphics\DIFaddincludegraphics} %DIF PREAMBLE
%DIF < %DIF < %DIF < %DIF < \DeclareRobustCommand{\DIFaddend}{\DIFOaddend \let\includegraphics\DIFOincludegraphics} %DIF PREAMBLE
%DIF < %DIF < %DIF < %DIF < \DeclareRobustCommand{\DIFdelbegin}{\DIFOdelbegin \let\includegraphics\DIFdelincludegraphics} %DIF PREAMBLE
%DIF < %DIF < %DIF < %DIF < \DeclareRobustCommand{\DIFdelend}{\DIFOaddend \let\includegraphics\DIFOincludegraphics} %DIF PREAMBLE
%DIF < %DIF < %DIF < %DIF < \LetLtxMacro{\DIFOaddbeginFL}{\DIFaddbeginFL} %DIF PREAMBLE
%DIF < %DIF < %DIF < %DIF < \LetLtxMacro{\DIFOaddendFL}{\DIFaddendFL} %DIF PREAMBLE
%DIF < %DIF < %DIF < %DIF < \LetLtxMacro{\DIFOdelbeginFL}{\DIFdelbeginFL} %DIF PREAMBLE
%DIF < %DIF < %DIF < %DIF < \LetLtxMacro{\DIFOdelendFL}{\DIFdelendFL} %DIF PREAMBLE
%DIF < %DIF < %DIF < %DIF < \DeclareRobustCommand{\DIFaddbeginFL}{\DIFOaddbeginFL \let\includegraphics\DIFaddincludegraphics} %DIF PREAMBLE
%DIF < %DIF < %DIF < %DIF < \DeclareRobustCommand{\DIFaddendFL}{\DIFOaddendFL \let\includegraphics\DIFOincludegraphics} %DIF PREAMBLE
%DIF < %DIF < %DIF < %DIF < \DeclareRobustCommand{\DIFdelbeginFL}{\DIFOdelbeginFL \let\includegraphics\DIFdelincludegraphics} %DIF PREAMBLE
%DIF < %DIF < %DIF < %DIF < \DeclareRobustCommand{\DIFdelendFL}{\DIFOaddendFL \let\includegraphics\DIFOincludegraphics} %DIF PREAMBLE
%DIF < %DIF < %DIF < %DIF < %DIF AMSMATHULEM PREAMBLE %DIF PREAMBLE
%DIF < %DIF < %DIF < %DIF < \makeatletter %DIF PREAMBLE
%DIF < %DIF < %DIF < %DIF < \let\sout@orig\sout %DIF PREAMBLE
%DIF < %DIF < %DIF < %DIF < \renewcommand{\sout}[1]{\ifmmode\text{\sout@orig{\ensuremath{#1}}}\else\sout@orig{#1}\fi} %DIF PREAMBLE
%DIF < %DIF < %DIF < %DIF < \makeatother %DIF PREAMBLE
%DIF < %DIF < %DIF < %DIF < %DIF COLORLISTINGS PREAMBLE %DIF PREAMBLE
%DIF < %DIF < %DIF < %DIF < \RequirePackage{listings} %DIF PREAMBLE
%DIF < %DIF < %DIF < %DIF < \RequirePackage{color} %DIF PREAMBLE
%DIF < %DIF < %DIF < %DIF < \lstdefinelanguage{DIFcode}{ %DIF PREAMBLE
%DIF < %DIF < %DIF < %DIF < %DIF DIFCODE_CULINECHBAR %DIF PREAMBLE
%DIF < %DIF < %DIF < %DIF <   moredelim=[il][\color{red}\sout]{\%DIF\ <\ }, %DIF PREAMBLE
%DIF < %DIF < %DIF < %DIF <   moredelim=[il][\color{blue}\uwave]{\%DIF\ >\ } %DIF PREAMBLE
%DIF < %DIF < %DIF < %DIF < } %DIF PREAMBLE
%DIF < %DIF < %DIF < %DIF < \lstdefinestyle{DIFverbatimstyle}{ %DIF PREAMBLE
%DIF < %DIF < %DIF < %DIF < 	language=DIFcode, %DIF PREAMBLE
%DIF < %DIF < %DIF < %DIF < 	basicstyle=\ttfamily, %DIF PREAMBLE
%DIF < %DIF < %DIF < %DIF < 	columns=fullflexible, %DIF PREAMBLE
%DIF < %DIF < %DIF < %DIF < 	keepspaces=true %DIF PREAMBLE
%DIF < %DIF < %DIF < %DIF < } %DIF PREAMBLE
%DIF < %DIF < %DIF < %DIF < \lstnewenvironment{DIFverbatim}{\lstset{style=DIFverbatimstyle}}{} %DIF PREAMBLE
%DIF < %DIF < %DIF < %DIF < \lstnewenvironment{DIFverbatim*}{\lstset{style=DIFverbatimstyle,showspaces=true}}{} %DIF PREAMBLE
%DIF < %DIF < %DIF < %DIF < \lstset{extendedchars=\true,inputencoding=utf8}
%DIF < %DIF < %DIF < %DIF < 
%DIF < %DIF < %DIF < %DIF < %DIF END PREAMBLE EXTENSION ADDED BY LATEXDIFF
%DIF < %DIF < %DIF < %DIF PREAMBLE EXTENSION ADDED BY LATEXDIFF
%DIF < %DIF < %DIF < %DIF CULINECHBAR PREAMBLE %DIF PREAMBLE
%DIF < %DIF < %DIF < \RequirePackage[normalem]{ulem} %DIF PREAMBLE
%DIF < %DIF < %DIF < \RequirePackage[pdftex]{changebar} %DIF PREAMBLE
%DIF < %DIF < %DIF < \RequirePackage{color}\definecolor{RED}{rgb}{1,0,0}\definecolor{BLUE}{rgb}{0,0,1} %DIF PREAMBLE
%DIF < %DIF < %DIF < \providecommand{\DIFaddtex}[1]{\protect\cbstart{\protect\color{blue}\uwave{#1}}\protect\cbend} %DIF PREAMBLE
%DIF < %DIF < %DIF < \providecommand{\DIFdeltex}[1]{\protect\cbdelete{\protect\color{red}\sout{#1}}\protect\cbdelete} %DIF PREAMBLE
%DIF < %DIF < %DIF < %DIF SAFE PREAMBLE %DIF PREAMBLE
%DIF < %DIF < %DIF < \providecommand{\DIFaddbegin}{} %DIF PREAMBLE
%DIF < %DIF < %DIF < \providecommand{\DIFaddend}{} %DIF PREAMBLE
%DIF < %DIF < %DIF < \providecommand{\DIFdelbegin}{} %DIF PREAMBLE
%DIF < %DIF < %DIF < \providecommand{\DIFdelend}{} %DIF PREAMBLE
%DIF < %DIF < %DIF < \providecommand{\DIFmodbegin}{} %DIF PREAMBLE
%DIF < %DIF < %DIF < \providecommand{\DIFmodend}{} %DIF PREAMBLE
%DIF < %DIF < %DIF < %DIF FLOATSAFE PREAMBLE %DIF PREAMBLE
%DIF < %DIF < %DIF < \providecommand{\DIFaddFL}[1]{\DIFadd{#1}} %DIF PREAMBLE
%DIF < %DIF < %DIF < \providecommand{\DIFdelFL}[1]{\DIFdel{#1}} %DIF PREAMBLE
%DIF < %DIF < %DIF < \providecommand{\DIFaddbeginFL}{} %DIF PREAMBLE
%DIF < %DIF < %DIF < \providecommand{\DIFaddendFL}{} %DIF PREAMBLE
%DIF < %DIF < %DIF < \providecommand{\DIFdelbeginFL}{} %DIF PREAMBLE
%DIF < %DIF < %DIF < \providecommand{\DIFdelendFL}{} %DIF PREAMBLE
%DIF < %DIF < %DIF < %DIF HYPERREF PREAMBLE %DIF PREAMBLE
%DIF < %DIF < %DIF < \providecommand{\DIFadd}[1]{\texorpdfstring{\DIFaddtex{#1}}{#1}} %DIF PREAMBLE
%DIF < %DIF < %DIF < \providecommand{\DIFdel}[1]{\texorpdfstring{\DIFdeltex{#1}}{}} %DIF PREAMBLE
%DIF < %DIF < %DIF < \newcommand{\DIFscaledelfig}{0.5}
%DIF < %DIF < %DIF < %DIF HIGHLIGHTGRAPHICS PREAMBLE %DIF PREAMBLE
%DIF < %DIF < %DIF < \RequirePackage{settobox} %DIF PREAMBLE
%DIF < %DIF < %DIF < \RequirePackage{letltxmacro} %DIF PREAMBLE
%DIF < %DIF < %DIF < \newsavebox{\DIFdelgraphicsbox} %DIF PREAMBLE
%DIF < %DIF < %DIF < \newlength{\DIFdelgraphicswidth} %DIF PREAMBLE
%DIF < %DIF < %DIF < \newlength{\DIFdelgraphicsheight} %DIF PREAMBLE
%DIF < %DIF < %DIF < % store original definition of \includegraphics %DIF PREAMBLE
%DIF < %DIF < %DIF < \LetLtxMacro{\DIFOincludegraphics}{\includegraphics} %DIF PREAMBLE
%DIF < %DIF < %DIF < \newcommand{\DIFaddincludegraphics}[2][]{{\color{blue}\fbox{\DIFOincludegraphics[#1]{#2}}}} %DIF PREAMBLE
%DIF < %DIF < %DIF < \newcommand{\DIFdelincludegraphics}[2][]{% %DIF PREAMBLE
%DIF < %DIF < %DIF < \sbox{\DIFdelgraphicsbox}{\DIFOincludegraphics[#1]{#2}}% %DIF PREAMBLE
%DIF < %DIF < %DIF < \settoboxwidth{\DIFdelgraphicswidth}{\DIFdelgraphicsbox} %DIF PREAMBLE
%DIF < %DIF < %DIF < \settoboxtotalheight{\DIFdelgraphicsheight}{\DIFdelgraphicsbox} %DIF PREAMBLE
%DIF < %DIF < %DIF < \scalebox{\DIFscaledelfig}{% %DIF PREAMBLE
%DIF < %DIF < %DIF < \parbox[b]{\DIFdelgraphicswidth}{\usebox{\DIFdelgraphicsbox}\\[-\baselineskip] \rule{\DIFdelgraphicswidth}{0em}}\llap{\resizebox{\DIFdelgraphicswidth}{\DIFdelgraphicsheight}{% %DIF PREAMBLE
%DIF < %DIF < %DIF < \setlength{\unitlength}{\DIFdelgraphicswidth}% %DIF PREAMBLE
%DIF < %DIF < %DIF < \begin{picture}(1,1)% %DIF PREAMBLE
%DIF < %DIF < %DIF < \thicklines\linethickness{2pt} %DIF PREAMBLE
%DIF < %DIF < %DIF < {\color[rgb]{1,0,0}\put(0,0){\framebox(1,1){}}}% %DIF PREAMBLE
%DIF < %DIF < %DIF < {\color[rgb]{1,0,0}\put(0,0){\line( 1,1){1}}}% %DIF PREAMBLE
%DIF < %DIF < %DIF < {\color[rgb]{1,0,0}\put(0,1){\line(1,-1){1}}}% %DIF PREAMBLE
%DIF < %DIF < %DIF < \end{picture}% %DIF PREAMBLE
%DIF < %DIF < %DIF < }\hspace*{3pt}}} %DIF PREAMBLE
%DIF < %DIF < %DIF < } %DIF PREAMBLE
%DIF < %DIF < %DIF < \LetLtxMacro{\DIFOaddbegin}{\DIFaddbegin} %DIF PREAMBLE
%DIF < %DIF < %DIF < \LetLtxMacro{\DIFOaddend}{\DIFaddend} %DIF PREAMBLE
%DIF < %DIF < %DIF < \LetLtxMacro{\DIFOdelbegin}{\DIFdelbegin} %DIF PREAMBLE
%DIF < %DIF < %DIF < \LetLtxMacro{\DIFOdelend}{\DIFdelend} %DIF PREAMBLE
%DIF < %DIF < %DIF < \DeclareRobustCommand{\DIFaddbegin}{\DIFOaddbegin \let\includegraphics\DIFaddincludegraphics} %DIF PREAMBLE
%DIF < %DIF < %DIF < \DeclareRobustCommand{\DIFaddend}{\DIFOaddend \let\includegraphics\DIFOincludegraphics} %DIF PREAMBLE
%DIF < %DIF < %DIF < \DeclareRobustCommand{\DIFdelbegin}{\DIFOdelbegin \let\includegraphics\DIFdelincludegraphics} %DIF PREAMBLE
%DIF < %DIF < %DIF < \DeclareRobustCommand{\DIFdelend}{\DIFOaddend \let\includegraphics\DIFOincludegraphics} %DIF PREAMBLE
%DIF < %DIF < %DIF < \LetLtxMacro{\DIFOaddbeginFL}{\DIFaddbeginFL} %DIF PREAMBLE
%DIF < %DIF < %DIF < \LetLtxMacro{\DIFOaddendFL}{\DIFaddendFL} %DIF PREAMBLE
%DIF < %DIF < %DIF < \LetLtxMacro{\DIFOdelbeginFL}{\DIFdelbeginFL} %DIF PREAMBLE
%DIF < %DIF < %DIF < \LetLtxMacro{\DIFOdelendFL}{\DIFdelendFL} %DIF PREAMBLE
%DIF < %DIF < %DIF < \DeclareRobustCommand{\DIFaddbeginFL}{\DIFOaddbeginFL \let\includegraphics\DIFaddincludegraphics} %DIF PREAMBLE
%DIF < %DIF < %DIF < \DeclareRobustCommand{\DIFaddendFL}{\DIFOaddendFL \let\includegraphics\DIFOincludegraphics} %DIF PREAMBLE
%DIF < %DIF < %DIF < \DeclareRobustCommand{\DIFdelbeginFL}{\DIFOdelbeginFL \let\includegraphics\DIFdelincludegraphics} %DIF PREAMBLE
%DIF < %DIF < %DIF < \DeclareRobustCommand{\DIFdelendFL}{\DIFOaddendFL \let\includegraphics\DIFOincludegraphics} %DIF PREAMBLE
%DIF < %DIF < %DIF < %DIF AMSMATHULEM PREAMBLE %DIF PREAMBLE
%DIF < %DIF < %DIF < \makeatletter %DIF PREAMBLE
%DIF < %DIF < %DIF < \let\sout@orig\sout %DIF PREAMBLE
%DIF < %DIF < %DIF < \renewcommand{\sout}[1]{\ifmmode\text{\sout@orig{\ensuremath{#1}}}\else\sout@orig{#1}\fi} %DIF PREAMBLE
%DIF < %DIF < %DIF < \makeatother %DIF PREAMBLE
%DIF < %DIF < %DIF < %DIF COLORLISTINGS PREAMBLE %DIF PREAMBLE
%DIF < %DIF < %DIF < \RequirePackage{listings} %DIF PREAMBLE
%DIF < %DIF < %DIF < \RequirePackage{color} %DIF PREAMBLE
%DIF < %DIF < %DIF < \lstdefinelanguage{DIFcode}{ %DIF PREAMBLE
%DIF < %DIF < %DIF < %DIF DIFCODE_CULINECHBAR %DIF PREAMBLE
%DIF < %DIF < %DIF <   moredelim=[il][\color{red}\sout]{\%DIF\ <\ }, %DIF PREAMBLE
%DIF < %DIF < %DIF <   moredelim=[il][\color{blue}\uwave]{\%DIF\ >\ } %DIF PREAMBLE
%DIF < %DIF < %DIF < } %DIF PREAMBLE
%DIF < %DIF < %DIF < \lstdefinestyle{DIFverbatimstyle}{ %DIF PREAMBLE
%DIF < %DIF < %DIF < 	language=DIFcode, %DIF PREAMBLE
%DIF < %DIF < %DIF < 	basicstyle=\ttfamily, %DIF PREAMBLE
%DIF < %DIF < %DIF < 	columns=fullflexible, %DIF PREAMBLE
%DIF < %DIF < %DIF < 	keepspaces=true %DIF PREAMBLE
%DIF < %DIF < %DIF < } %DIF PREAMBLE
%DIF < %DIF < %DIF < \lstnewenvironment{DIFverbatim}{\lstset{style=DIFverbatimstyle}}{} %DIF PREAMBLE
%DIF < %DIF < %DIF < \lstnewenvironment{DIFverbatim*}{\lstset{style=DIFverbatimstyle,showspaces=true}}{} %DIF PREAMBLE
%DIF < %DIF < %DIF < \lstset{extendedchars=\true,inputencoding=utf8}
%DIF < %DIF < %DIF < 
%DIF < %DIF < %DIF < %DIF END PREAMBLE EXTENSION ADDED BY LATEXDIFF
%DIF < %DIF < %DIF PREAMBLE EXTENSION ADDED BY LATEXDIFF
%DIF < %DIF < %DIF CULINECHBAR PREAMBLE %DIF PREAMBLE
%DIF < %DIF < \RequirePackage[normalem]{ulem} %DIF PREAMBLE
%DIF < %DIF < \RequirePackage[pdftex]{changebar} %DIF PREAMBLE
%DIF < %DIF < \RequirePackage{color}\definecolor{RED}{rgb}{1,0,0}\definecolor{BLUE}{rgb}{0,0,1} %DIF PREAMBLE
%DIF < %DIF < \providecommand{\DIFaddtex}[1]{\protect\cbstart{\protect\color{blue}\uwave{#1}}\protect\cbend} %DIF PREAMBLE
%DIF < %DIF < \providecommand{\DIFdeltex}[1]{\protect\cbdelete{\protect\color{red}\sout{#1}}\protect\cbdelete} %DIF PREAMBLE
%DIF < %DIF < %DIF SAFE PREAMBLE %DIF PREAMBLE
%DIF < %DIF < \providecommand{\DIFaddbegin}{} %DIF PREAMBLE
%DIF < %DIF < \providecommand{\DIFaddend}{} %DIF PREAMBLE
%DIF < %DIF < \providecommand{\DIFdelbegin}{} %DIF PREAMBLE
%DIF < %DIF < \providecommand{\DIFdelend}{} %DIF PREAMBLE
%DIF < %DIF < \providecommand{\DIFmodbegin}{} %DIF PREAMBLE
%DIF < %DIF < \providecommand{\DIFmodend}{} %DIF PREAMBLE
%DIF < %DIF < %DIF FLOATSAFE PREAMBLE %DIF PREAMBLE
%DIF < %DIF < \providecommand{\DIFaddFL}[1]{\DIFadd{#1}} %DIF PREAMBLE
%DIF < %DIF < \providecommand{\DIFdelFL}[1]{\DIFdel{#1}} %DIF PREAMBLE
%DIF < %DIF < \providecommand{\DIFaddbeginFL}{} %DIF PREAMBLE
%DIF < %DIF < \providecommand{\DIFaddendFL}{} %DIF PREAMBLE
%DIF < %DIF < \providecommand{\DIFdelbeginFL}{} %DIF PREAMBLE
%DIF < %DIF < \providecommand{\DIFdelendFL}{} %DIF PREAMBLE
%DIF < %DIF < %DIF HYPERREF PREAMBLE %DIF PREAMBLE
%DIF < %DIF < \providecommand{\DIFadd}[1]{\texorpdfstring{\DIFaddtex{#1}}{#1}} %DIF PREAMBLE
%DIF < %DIF < \providecommand{\DIFdel}[1]{\texorpdfstring{\DIFdeltex{#1}}{}} %DIF PREAMBLE
%DIF < %DIF < \newcommand{\DIFscaledelfig}{0.5}
%DIF < %DIF < %DIF HIGHLIGHTGRAPHICS PREAMBLE %DIF PREAMBLE
%DIF < %DIF < \RequirePackage{settobox} %DIF PREAMBLE
%DIF < %DIF < \RequirePackage{letltxmacro} %DIF PREAMBLE
%DIF < %DIF < \newsavebox{\DIFdelgraphicsbox} %DIF PREAMBLE
%DIF < %DIF < \newlength{\DIFdelgraphicswidth} %DIF PREAMBLE
%DIF < %DIF < \newlength{\DIFdelgraphicsheight} %DIF PREAMBLE
%DIF < %DIF < % store original definition of \includegraphics %DIF PREAMBLE
%DIF < %DIF < \LetLtxMacro{\DIFOincludegraphics}{\includegraphics} %DIF PREAMBLE
%DIF < %DIF < \newcommand{\DIFaddincludegraphics}[2][]{{\color{blue}\fbox{\DIFOincludegraphics[#1]{#2}}}} %DIF PREAMBLE
%DIF < %DIF < \newcommand{\DIFdelincludegraphics}[2][]{% %DIF PREAMBLE
%DIF < %DIF < \sbox{\DIFdelgraphicsbox}{\DIFOincludegraphics[#1]{#2}}% %DIF PREAMBLE
%DIF < %DIF < \settoboxwidth{\DIFdelgraphicswidth}{\DIFdelgraphicsbox} %DIF PREAMBLE
%DIF < %DIF < \settoboxtotalheight{\DIFdelgraphicsheight}{\DIFdelgraphicsbox} %DIF PREAMBLE
%DIF < %DIF < \scalebox{\DIFscaledelfig}{% %DIF PREAMBLE
%DIF < %DIF < \parbox[b]{\DIFdelgraphicswidth}{\usebox{\DIFdelgraphicsbox}\\[-\baselineskip] \rule{\DIFdelgraphicswidth}{0em}}\llap{\resizebox{\DIFdelgraphicswidth}{\DIFdelgraphicsheight}{% %DIF PREAMBLE
%DIF < %DIF < \setlength{\unitlength}{\DIFdelgraphicswidth}% %DIF PREAMBLE
%DIF < %DIF < \begin{picture}(1,1)% %DIF PREAMBLE
%DIF < %DIF < \thicklines\linethickness{2pt} %DIF PREAMBLE
%DIF < %DIF < {\color[rgb]{1,0,0}\put(0,0){\framebox(1,1){}}}% %DIF PREAMBLE
%DIF < %DIF < {\color[rgb]{1,0,0}\put(0,0){\line( 1,1){1}}}% %DIF PREAMBLE
%DIF < %DIF < {\color[rgb]{1,0,0}\put(0,1){\line(1,-1){1}}}% %DIF PREAMBLE
%DIF < %DIF < \end{picture}% %DIF PREAMBLE
%DIF < %DIF < }\hspace*{3pt}}} %DIF PREAMBLE
%DIF < %DIF < } %DIF PREAMBLE
%DIF < %DIF < \LetLtxMacro{\DIFOaddbegin}{\DIFaddbegin} %DIF PREAMBLE
%DIF < %DIF < \LetLtxMacro{\DIFOaddend}{\DIFaddend} %DIF PREAMBLE
%DIF < %DIF < \LetLtxMacro{\DIFOdelbegin}{\DIFdelbegin} %DIF PREAMBLE
%DIF < %DIF < \LetLtxMacro{\DIFOdelend}{\DIFdelend} %DIF PREAMBLE
%DIF < %DIF < \DeclareRobustCommand{\DIFaddbegin}{\DIFOaddbegin \let\includegraphics\DIFaddincludegraphics} %DIF PREAMBLE
%DIF < %DIF < \DeclareRobustCommand{\DIFaddend}{\DIFOaddend \let\includegraphics\DIFOincludegraphics} %DIF PREAMBLE
%DIF < %DIF < \DeclareRobustCommand{\DIFdelbegin}{\DIFOdelbegin \let\includegraphics\DIFdelincludegraphics} %DIF PREAMBLE
%DIF < %DIF < \DeclareRobustCommand{\DIFdelend}{\DIFOaddend \let\includegraphics\DIFOincludegraphics} %DIF PREAMBLE
%DIF < %DIF < \LetLtxMacro{\DIFOaddbeginFL}{\DIFaddbeginFL} %DIF PREAMBLE
%DIF < %DIF < \LetLtxMacro{\DIFOaddendFL}{\DIFaddendFL} %DIF PREAMBLE
%DIF < %DIF < \LetLtxMacro{\DIFOdelbeginFL}{\DIFdelbeginFL} %DIF PREAMBLE
%DIF < %DIF < \LetLtxMacro{\DIFOdelendFL}{\DIFdelendFL} %DIF PREAMBLE
%DIF < %DIF < \DeclareRobustCommand{\DIFaddbeginFL}{\DIFOaddbeginFL \let\includegraphics\DIFaddincludegraphics} %DIF PREAMBLE
%DIF < %DIF < \DeclareRobustCommand{\DIFaddendFL}{\DIFOaddendFL \let\includegraphics\DIFOincludegraphics} %DIF PREAMBLE
%DIF < %DIF < \DeclareRobustCommand{\DIFdelbeginFL}{\DIFOdelbeginFL \let\includegraphics\DIFdelincludegraphics} %DIF PREAMBLE
%DIF < %DIF < \DeclareRobustCommand{\DIFdelendFL}{\DIFOaddendFL \let\includegraphics\DIFOincludegraphics} %DIF PREAMBLE
%DIF < %DIF < %DIF AMSMATHULEM PREAMBLE %DIF PREAMBLE
%DIF < %DIF < \makeatletter %DIF PREAMBLE
%DIF < %DIF < \let\sout@orig\sout %DIF PREAMBLE
%DIF < %DIF < \renewcommand{\sout}[1]{\ifmmode\text{\sout@orig{\ensuremath{#1}}}\else\sout@orig{#1}\fi} %DIF PREAMBLE
%DIF < %DIF < \makeatother %DIF PREAMBLE
%DIF < %DIF < %DIF COLORLISTINGS PREAMBLE %DIF PREAMBLE
%DIF < %DIF < \RequirePackage{listings} %DIF PREAMBLE
%DIF < %DIF < \RequirePackage{color} %DIF PREAMBLE
%DIF < %DIF < \lstdefinelanguage{DIFcode}{ %DIF PREAMBLE
%DIF < %DIF < %DIF DIFCODE_CULINECHBAR %DIF PREAMBLE
%DIF < %DIF <   moredelim=[il][\color{red}\sout]{\%DIF\ <\ }, %DIF PREAMBLE
%DIF < %DIF <   moredelim=[il][\color{blue}\uwave]{\%DIF\ >\ } %DIF PREAMBLE
%DIF < %DIF < } %DIF PREAMBLE
%DIF < %DIF < \lstdefinestyle{DIFverbatimstyle}{ %DIF PREAMBLE
%DIF < %DIF < 	language=DIFcode, %DIF PREAMBLE
%DIF < %DIF < 	basicstyle=\ttfamily, %DIF PREAMBLE
%DIF < %DIF < 	columns=fullflexible, %DIF PREAMBLE
%DIF < %DIF < 	keepspaces=true %DIF PREAMBLE
%DIF < %DIF < } %DIF PREAMBLE
%DIF < %DIF < \lstnewenvironment{DIFverbatim}{\lstset{style=DIFverbatimstyle}}{} %DIF PREAMBLE
%DIF < %DIF < \lstnewenvironment{DIFverbatim*}{\lstset{style=DIFverbatimstyle,showspaces=true}}{} %DIF PREAMBLE
%DIF < %DIF < \lstset{extendedchars=\true,inputencoding=utf8}
%DIF < %DIF < 
%DIF < %DIF < %DIF END PREAMBLE EXTENSION ADDED BY LATEXDIFF
%DIF < %DIF PREAMBLE EXTENSION ADDED BY LATEXDIFF
%DIF < %DIF CULINECHBAR PREAMBLE %DIF PREAMBLE
%DIF < \RequirePackage[normalem]{ulem} %DIF PREAMBLE
%DIF < \RequirePackage[pdftex]{changebar} %DIF PREAMBLE
%DIF < \RequirePackage{color}\definecolor{RED}{rgb}{1,0,0}\definecolor{BLUE}{rgb}{0,0,1} %DIF PREAMBLE
%DIF < \providecommand{\DIFaddtex}[1]{\protect\cbstart{\protect\color{blue}\uwave{#1}}\protect\cbend} %DIF PREAMBLE
%DIF < \providecommand{\DIFdeltex}[1]{\protect\cbdelete{\protect\color{red}\sout{#1}}\protect\cbdelete} %DIF PREAMBLE
%DIF < %DIF SAFE PREAMBLE %DIF PREAMBLE
%DIF < \providecommand{\DIFaddbegin}{} %DIF PREAMBLE
%DIF < \providecommand{\DIFaddend}{} %DIF PREAMBLE
%DIF < \providecommand{\DIFdelbegin}{} %DIF PREAMBLE
%DIF < \providecommand{\DIFdelend}{} %DIF PREAMBLE
%DIF < \providecommand{\DIFmodbegin}{} %DIF PREAMBLE
%DIF < \providecommand{\DIFmodend}{} %DIF PREAMBLE
%DIF < %DIF FLOATSAFE PREAMBLE %DIF PREAMBLE
%DIF < \providecommand{\DIFaddFL}[1]{\DIFadd{#1}} %DIF PREAMBLE
%DIF < \providecommand{\DIFdelFL}[1]{\DIFdel{#1}} %DIF PREAMBLE
%DIF < \providecommand{\DIFaddbeginFL}{} %DIF PREAMBLE
%DIF < \providecommand{\DIFaddendFL}{} %DIF PREAMBLE
%DIF < \providecommand{\DIFdelbeginFL}{} %DIF PREAMBLE
%DIF < \providecommand{\DIFdelendFL}{} %DIF PREAMBLE
%DIF < %DIF HYPERREF PREAMBLE %DIF PREAMBLE
%DIF < \providecommand{\DIFadd}[1]{\texorpdfstring{\DIFaddtex{#1}}{#1}} %DIF PREAMBLE
%DIF < \providecommand{\DIFdel}[1]{\texorpdfstring{\DIFdeltex{#1}}{}} %DIF PREAMBLE
%DIF < \newcommand{\DIFscaledelfig}{0.5}
%DIF < %DIF HIGHLIGHTGRAPHICS PREAMBLE %DIF PREAMBLE
%DIF < \RequirePackage{settobox} %DIF PREAMBLE
%DIF < \RequirePackage{letltxmacro} %DIF PREAMBLE
%DIF < \newsavebox{\DIFdelgraphicsbox} %DIF PREAMBLE
%DIF < \newlength{\DIFdelgraphicswidth} %DIF PREAMBLE
%DIF < \newlength{\DIFdelgraphicsheight} %DIF PREAMBLE
%DIF < % store original definition of \includegraphics %DIF PREAMBLE
%DIF < \LetLtxMacro{\DIFOincludegraphics}{\includegraphics} %DIF PREAMBLE
%DIF < \newcommand{\DIFaddincludegraphics}[2][]{{\color{blue}\fbox{\DIFOincludegraphics[#1]{#2}}}} %DIF PREAMBLE
%DIF < \newcommand{\DIFdelincludegraphics}[2][]{% %DIF PREAMBLE
%DIF < \sbox{\DIFdelgraphicsbox}{\DIFOincludegraphics[#1]{#2}}% %DIF PREAMBLE
%DIF < \settoboxwidth{\DIFdelgraphicswidth}{\DIFdelgraphicsbox} %DIF PREAMBLE
%DIF < \settoboxtotalheight{\DIFdelgraphicsheight}{\DIFdelgraphicsbox} %DIF PREAMBLE
%DIF < \scalebox{\DIFscaledelfig}{% %DIF PREAMBLE
%DIF < \parbox[b]{\DIFdelgraphicswidth}{\usebox{\DIFdelgraphicsbox}\\[-\baselineskip] \rule{\DIFdelgraphicswidth}{0em}}\llap{\resizebox{\DIFdelgraphicswidth}{\DIFdelgraphicsheight}{% %DIF PREAMBLE
%DIF < \setlength{\unitlength}{\DIFdelgraphicswidth}% %DIF PREAMBLE
%DIF < \begin{picture}(1,1)% %DIF PREAMBLE
%DIF < \thicklines\linethickness{2pt} %DIF PREAMBLE
%DIF < {\color[rgb]{1,0,0}\put(0,0){\framebox(1,1){}}}% %DIF PREAMBLE
%DIF < {\color[rgb]{1,0,0}\put(0,0){\line( 1,1){1}}}% %DIF PREAMBLE
%DIF < {\color[rgb]{1,0,0}\put(0,1){\line(1,-1){1}}}% %DIF PREAMBLE
%DIF < \end{picture}% %DIF PREAMBLE
%DIF < }\hspace*{3pt}}} %DIF PREAMBLE
%DIF < } %DIF PREAMBLE
%DIF < \LetLtxMacro{\DIFOaddbegin}{\DIFaddbegin} %DIF PREAMBLE
%DIF < \LetLtxMacro{\DIFOaddend}{\DIFaddend} %DIF PREAMBLE
%DIF < \LetLtxMacro{\DIFOdelbegin}{\DIFdelbegin} %DIF PREAMBLE
%DIF < \LetLtxMacro{\DIFOdelend}{\DIFdelend} %DIF PREAMBLE
%DIF < \DeclareRobustCommand{\DIFaddbegin}{\DIFOaddbegin \let\includegraphics\DIFaddincludegraphics} %DIF PREAMBLE
%DIF < \DeclareRobustCommand{\DIFaddend}{\DIFOaddend \let\includegraphics\DIFOincludegraphics} %DIF PREAMBLE
%DIF < \DeclareRobustCommand{\DIFdelbegin}{\DIFOdelbegin \let\includegraphics\DIFdelincludegraphics} %DIF PREAMBLE
%DIF < \DeclareRobustCommand{\DIFdelend}{\DIFOaddend \let\includegraphics\DIFOincludegraphics} %DIF PREAMBLE
%DIF < \LetLtxMacro{\DIFOaddbeginFL}{\DIFaddbeginFL} %DIF PREAMBLE
%DIF < \LetLtxMacro{\DIFOaddendFL}{\DIFaddendFL} %DIF PREAMBLE
%DIF < \LetLtxMacro{\DIFOdelbeginFL}{\DIFdelbeginFL} %DIF PREAMBLE
%DIF < \LetLtxMacro{\DIFOdelendFL}{\DIFdelendFL} %DIF PREAMBLE
%DIF < \DeclareRobustCommand{\DIFaddbeginFL}{\DIFOaddbeginFL \let\includegraphics\DIFaddincludegraphics} %DIF PREAMBLE
%DIF < \DeclareRobustCommand{\DIFaddendFL}{\DIFOaddendFL \let\includegraphics\DIFOincludegraphics} %DIF PREAMBLE
%DIF < \DeclareRobustCommand{\DIFdelbeginFL}{\DIFOdelbeginFL \let\includegraphics\DIFdelincludegraphics} %DIF PREAMBLE
%DIF < \DeclareRobustCommand{\DIFdelendFL}{\DIFOaddendFL \let\includegraphics\DIFOincludegraphics} %DIF PREAMBLE
%DIF < %DIF AMSMATHULEM PREAMBLE %DIF PREAMBLE
%DIF < \makeatletter %DIF PREAMBLE
%DIF < \let\sout@orig\sout %DIF PREAMBLE
%DIF < \renewcommand{\sout}[1]{\ifmmode\text{\sout@orig{\ensuremath{#1}}}\else\sout@orig{#1}\fi} %DIF PREAMBLE
%DIF < \makeatother %DIF PREAMBLE
%DIF < %DIF COLORLISTINGS PREAMBLE %DIF PREAMBLE
%DIF < \RequirePackage{listings} %DIF PREAMBLE
%DIF < \RequirePackage{color} %DIF PREAMBLE
%DIF < \lstdefinelanguage{DIFcode}{ %DIF PREAMBLE
%DIF < %DIF DIFCODE_CULINECHBAR %DIF PREAMBLE
%DIF <   moredelim=[il][\color{red}\sout]{\%DIF\ <\ }, %DIF PREAMBLE
%DIF <   moredelim=[il][\color{blue}\uwave]{\%DIF\ >\ } %DIF PREAMBLE
%DIF < } %DIF PREAMBLE
%DIF < \lstdefinestyle{DIFverbatimstyle}{ %DIF PREAMBLE
%DIF < 	language=DIFcode, %DIF PREAMBLE
%DIF < 	basicstyle=\ttfamily, %DIF PREAMBLE
%DIF < 	columns=fullflexible, %DIF PREAMBLE
%DIF < 	keepspaces=true %DIF PREAMBLE
%DIF < } %DIF PREAMBLE
%DIF < \lstnewenvironment{DIFverbatim}{\lstset{style=DIFverbatimstyle}}{} %DIF PREAMBLE
%DIF < \lstnewenvironment{DIFverbatim*}{\lstset{style=DIFverbatimstyle,showspaces=true}}{} %DIF PREAMBLE
%DIF < \lstset{extendedchars=\true,inputencoding=utf8}
%DIF < 
%DIF < %DIF END PREAMBLE EXTENSION ADDED BY LATEXDIFF
%DIF PREAMBLE EXTENSION ADDED BY LATEXDIFF
%DIF CULINECHBAR PREAMBLE %DIF PREAMBLE
\RequirePackage[normalem]{ulem} %DIF PREAMBLE
\RequirePackage[pdftex]{changebar} %DIF PREAMBLE
\RequirePackage{color}\definecolor{RED}{rgb}{1,0,0}\definecolor{BLUE}{rgb}{0,0,1} %DIF PREAMBLE
\providecommand{\DIFaddtex}[1]{\protect\cbstart{\protect\color{blue}\uwave{#1}}\protect\cbend} %DIF PREAMBLE
\providecommand{\DIFdeltex}[1]{\protect\cbdelete{\protect\color{red}\sout{#1}}\protect\cbdelete} %DIF PREAMBLE
%DIF SAFE PREAMBLE %DIF PREAMBLE
\providecommand{\DIFaddbegin}{} %DIF PREAMBLE
\providecommand{\DIFaddend}{} %DIF PREAMBLE
\providecommand{\DIFdelbegin}{} %DIF PREAMBLE
\providecommand{\DIFdelend}{} %DIF PREAMBLE
\providecommand{\DIFmodbegin}{} %DIF PREAMBLE
\providecommand{\DIFmodend}{} %DIF PREAMBLE
%DIF FLOATSAFE PREAMBLE %DIF PREAMBLE
\providecommand{\DIFaddFL}[1]{\DIFadd{#1}} %DIF PREAMBLE
\providecommand{\DIFdelFL}[1]{\DIFdel{#1}} %DIF PREAMBLE
\providecommand{\DIFaddbeginFL}{} %DIF PREAMBLE
\providecommand{\DIFaddendFL}{} %DIF PREAMBLE
\providecommand{\DIFdelbeginFL}{} %DIF PREAMBLE
\providecommand{\DIFdelendFL}{} %DIF PREAMBLE
%DIF HYPERREF PREAMBLE %DIF PREAMBLE
\providecommand{\DIFadd}[1]{\texorpdfstring{\DIFaddtex{#1}}{#1}} %DIF PREAMBLE
\providecommand{\DIFdel}[1]{\texorpdfstring{\DIFdeltex{#1}}{}} %DIF PREAMBLE
\newcommand{\DIFscaledelfig}{0.5}
%DIF HIGHLIGHTGRAPHICS PREAMBLE %DIF PREAMBLE
\RequirePackage{settobox} %DIF PREAMBLE
\RequirePackage{letltxmacro} %DIF PREAMBLE
\newsavebox{\DIFdelgraphicsbox} %DIF PREAMBLE
\newlength{\DIFdelgraphicswidth} %DIF PREAMBLE
\newlength{\DIFdelgraphicsheight} %DIF PREAMBLE
% store original definition of \includegraphics %DIF PREAMBLE
\LetLtxMacro{\DIFOincludegraphics}{\includegraphics} %DIF PREAMBLE
\newcommand{\DIFaddincludegraphics}[2][]{{\color{blue}\fbox{\DIFOincludegraphics[#1]{#2}}}} %DIF PREAMBLE
\newcommand{\DIFdelincludegraphics}[2][]{% %DIF PREAMBLE
\sbox{\DIFdelgraphicsbox}{\DIFOincludegraphics[#1]{#2}}% %DIF PREAMBLE
\settoboxwidth{\DIFdelgraphicswidth}{\DIFdelgraphicsbox} %DIF PREAMBLE
\settoboxtotalheight{\DIFdelgraphicsheight}{\DIFdelgraphicsbox} %DIF PREAMBLE
\scalebox{\DIFscaledelfig}{% %DIF PREAMBLE
\parbox[b]{\DIFdelgraphicswidth}{\usebox{\DIFdelgraphicsbox}\\[-\baselineskip] \rule{\DIFdelgraphicswidth}{0em}}\llap{\resizebox{\DIFdelgraphicswidth}{\DIFdelgraphicsheight}{% %DIF PREAMBLE
\setlength{\unitlength}{\DIFdelgraphicswidth}% %DIF PREAMBLE
\begin{picture}(1,1)% %DIF PREAMBLE
\thicklines\linethickness{2pt} %DIF PREAMBLE
{\color[rgb]{1,0,0}\put(0,0){\framebox(1,1){}}}% %DIF PREAMBLE
{\color[rgb]{1,0,0}\put(0,0){\line( 1,1){1}}}% %DIF PREAMBLE
{\color[rgb]{1,0,0}\put(0,1){\line(1,-1){1}}}% %DIF PREAMBLE
\end{picture}% %DIF PREAMBLE
}\hspace*{3pt}}} %DIF PREAMBLE
} %DIF PREAMBLE
\LetLtxMacro{\DIFOaddbegin}{\DIFaddbegin} %DIF PREAMBLE
\LetLtxMacro{\DIFOaddend}{\DIFaddend} %DIF PREAMBLE
\LetLtxMacro{\DIFOdelbegin}{\DIFdelbegin} %DIF PREAMBLE
\LetLtxMacro{\DIFOdelend}{\DIFdelend} %DIF PREAMBLE
\DeclareRobustCommand{\DIFaddbegin}{\DIFOaddbegin \let\includegraphics\DIFaddincludegraphics} %DIF PREAMBLE
\DeclareRobustCommand{\DIFaddend}{\DIFOaddend \let\includegraphics\DIFOincludegraphics} %DIF PREAMBLE
\DeclareRobustCommand{\DIFdelbegin}{\DIFOdelbegin \let\includegraphics\DIFdelincludegraphics} %DIF PREAMBLE
\DeclareRobustCommand{\DIFdelend}{\DIFOaddend \let\includegraphics\DIFOincludegraphics} %DIF PREAMBLE
\LetLtxMacro{\DIFOaddbeginFL}{\DIFaddbeginFL} %DIF PREAMBLE
\LetLtxMacro{\DIFOaddendFL}{\DIFaddendFL} %DIF PREAMBLE
\LetLtxMacro{\DIFOdelbeginFL}{\DIFdelbeginFL} %DIF PREAMBLE
\LetLtxMacro{\DIFOdelendFL}{\DIFdelendFL} %DIF PREAMBLE
\DeclareRobustCommand{\DIFaddbeginFL}{\DIFOaddbeginFL \let\includegraphics\DIFaddincludegraphics} %DIF PREAMBLE
\DeclareRobustCommand{\DIFaddendFL}{\DIFOaddendFL \let\includegraphics\DIFOincludegraphics} %DIF PREAMBLE
\DeclareRobustCommand{\DIFdelbeginFL}{\DIFOdelbeginFL \let\includegraphics\DIFdelincludegraphics} %DIF PREAMBLE
\DeclareRobustCommand{\DIFdelendFL}{\DIFOaddendFL \let\includegraphics\DIFOincludegraphics} %DIF PREAMBLE
%DIF AMSMATHULEM PREAMBLE %DIF PREAMBLE
\makeatletter %DIF PREAMBLE
\let\sout@orig\sout %DIF PREAMBLE
\renewcommand{\sout}[1]{\ifmmode\text{\sout@orig{\ensuremath{#1}}}\else\sout@orig{#1}\fi} %DIF PREAMBLE
\makeatother %DIF PREAMBLE
%DIF COLORLISTINGS PREAMBLE %DIF PREAMBLE
\RequirePackage{listings} %DIF PREAMBLE
\RequirePackage{color} %DIF PREAMBLE
\lstdefinelanguage{DIFcode}{ %DIF PREAMBLE
%DIF DIFCODE_CULINECHBAR %DIF PREAMBLE
  moredelim=[il][\color{red}\sout]{\%DIF\ <\ }, %DIF PREAMBLE
  moredelim=[il][\color{blue}\uwave]{\%DIF\ >\ } %DIF PREAMBLE
} %DIF PREAMBLE
\lstdefinestyle{DIFverbatimstyle}{ %DIF PREAMBLE
	language=DIFcode, %DIF PREAMBLE
	basicstyle=\ttfamily, %DIF PREAMBLE
	columns=fullflexible, %DIF PREAMBLE
	keepspaces=true %DIF PREAMBLE
} %DIF PREAMBLE
\lstnewenvironment{DIFverbatim}{\lstset{style=DIFverbatimstyle}}{} %DIF PREAMBLE
\lstnewenvironment{DIFverbatim*}{\lstset{style=DIFverbatimstyle,showspaces=true}}{} %DIF PREAMBLE
\lstset{extendedchars=\true,inputencoding=utf8}

%DIF END PREAMBLE EXTENSION ADDED BY LATEXDIFF

\begin{document}

%% ----------------------------------------
%% Frontmatter
%% ----------------------------------------
\begin{frontmatter}

%% ========================================
%% ./01_manuscript/contents/highlights.tex
%% ========================================
%DIF < DIF < DIF < DIF < DIF < % \begin{highlights}
%DIF < DIF < DIF < DIF < DIF < % \pdfbookmark[1]{Highlights}{highlights}
%DIF < DIF < DIF < DIF < DIF > % -*- coding: utf-8 -*-
%DIF < DIF < DIF < DIF < DIF > % Timestamp: "2025-09-27 20:23:18 (ywatanabe)"
%DIF < DIF < DIF < DIF < DIF > % File: "/ssh:sp:/home/ywatanabe/proj/neurovista/paper/01_manuscript/contents/highlights.tex"
\DIFdelbegin %DIFDELCMD < \DIFdelbegin %%%
%DIF < DIFDELCMD < \DIFdelbegin %%%
%DIF < DIF < DIFDELCMD < \DIFdelbegin %%%
%DIF < DIF < DIF < DIFDELCMD < \DIFaddbegin %%%
%DIF < DIF < DIFDELCMD < \DIFdelend %%%
%DIF < DIF < DIF < DIF > % -*- coding: utf-8 -*-
%DIF < DIF < DIF < DIF > % Timestamp: "2025-09-27 20:23:18 (ywatanabe)"
%DIF < DIF < DIF < DIF > % File: "/ssh:sp:/home/ywatanabe/proj/neurovista/paper/01_manuscript/contents/highlights.tex"
%DIF < DIFDELCMD < \DIFdelend %%%
%DIF < DIF < DIF > % -*- coding: utf-8 -*-
%DIF < DIF < DIF > % Timestamp: "2025-09-27 20:23:18 (ywatanabe)"
%DIF < DIF < DIF > % File: "/ssh:sp:/home/ywatanabe/proj/neurovista/paper/01_manuscript/contents/highlights.tex"
%DIFDELCMD < \DIFdelend %%%
%DIF < DIF > % -*- coding: utf-8 -*-
%DIF < DIF > % Timestamp: "2025-09-27 20:23:18 (ywatanabe)"
%DIF < DIF > % File: "/ssh:sp:/home/ywatanabe/proj/neurovista/paper/01_manuscript/contents/highlights.tex"
\DIFdelend %DIF > % -*- coding: utf-8 -*-
%DIF > % Timestamp: "2025-09-27 20:23:18 (ywatanabe)"
%DIF > % File: "/ssh:sp:/home/ywatanabe/proj/neurovista/paper/01_manuscript/contents/highlights.tex"
\begin{highlights}
\pdfbookmark[1]{Highlights}{highlights}
\DIFdelbegin %DIFDELCMD < \DIFdelbegin %%%
%DIF < DIFDELCMD < \DIFdelbegin %%%
%DIF < DIF < DIFDELCMD < \DIFdelbegin %%%
%DIF < DIF < DIF < DIFDELCMD < \DIFaddend 
%DIF < DIF < DIF < DIFDELCMD < %%%
%DIF < DIF < DIFDELCMD < \DIFdelend 
%DIF < DIF < DIFDELCMD < %%%
%DIF < DIFDELCMD < \DIFdelend 
%DIF < DIFDELCMD < %%%
%DIFDELCMD < \DIFdelend 
%DIFDELCMD < %%%
\DIFdelend 

%DIF < DIF < DIF < DIF < DIF < % \item Highlight \#1
\DIFdelbegin %DIFDELCMD < \DIFdelbegin %%%
%DIF < DIFDELCMD < \DIFdelbegin %%%
%DIF < DIF < DIFDELCMD < \DIFdelbegin %%%
%DIF < DIF < DIF < DIFDELCMD < \DIFaddbegin %%%
%DIF < DIF < DIFDELCMD < \DIFdelend %%%
%DIF < DIFDELCMD < \DIFdelend %%%
%DIFDELCMD < \DIFdelend %%%
\DIFdelend \item \DIFdelbegin %DIFDELCMD < \DIFdelbegin %%%
%DIF < DIFDELCMD < \DIFdelbegin %%%
%DIF < DIF < DIFDELCMD < \DIFdelbegin %%%
%DIF < DIF < DIF < DIFDELCMD < \DIFadd{Highlight \#1
%DIF < DIF < DIF < DIFDELCMD < }\DIFaddend 
%DIF < DIF < DIF < DIFDELCMD < %%%
%DIF < DIF < DIFDELCMD < \DIFdelend \DIFaddbegin \DIFadd{Highlight \#1
%DIF < DIF < DIFDELCMD < }\DIFaddend 
%DIF < DIF < DIFDELCMD < %%%
%DIF < DIFDELCMD < \DIFdelend \DIFaddbegin \DIFadd{Highlight \#1
%DIF < DIFDELCMD < }\DIFaddend 
%DIF < DIFDELCMD < %%%
%DIFDELCMD < \DIFdelend \DIFaddbegin \DIFadd{Highlight \#1
%DIFDELCMD < }\DIFaddend 
%DIFDELCMD < %%%
\DIFdelend \DIFaddbegin \DIFadd{Highlight \#1
}\DIFaddend 

%DIF < DIF < DIF < DIF < DIF < % \item Highlight \#2
\DIFdelbegin %DIFDELCMD < \DIFdelbegin %%%
%DIF < DIFDELCMD < \DIFdelbegin %%%
%DIF < DIF < DIFDELCMD < \DIFdelbegin %%%
%DIF < DIF < DIF < DIFDELCMD < \DIFaddbegin %%%
%DIF < DIF < DIFDELCMD < \DIFdelend %%%
%DIF < DIFDELCMD < \DIFdelend %%%
%DIFDELCMD < \DIFdelend %%%
\DIFdelend \item \DIFdelbegin %DIFDELCMD < \DIFdelbegin %%%
%DIF < DIFDELCMD < \DIFdelbegin %%%
%DIF < DIF < DIFDELCMD < \DIFdelbegin %%%
%DIF < DIF < DIF < DIFDELCMD < \DIFadd{Highlight \#2
%DIF < DIF < DIF < DIFDELCMD < }\DIFaddend 
%DIF < DIF < DIF < DIFDELCMD < %%%
%DIF < DIF < DIFDELCMD < \DIFdelend \DIFaddbegin \DIFadd{Highlight \#2
%DIF < DIF < DIFDELCMD < }\DIFaddend 
%DIF < DIF < DIFDELCMD < %%%
%DIF < DIFDELCMD < \DIFdelend \DIFaddbegin \DIFadd{Highlight \#2
%DIF < DIFDELCMD < }\DIFaddend 
%DIF < DIFDELCMD < %%%
%DIFDELCMD < \DIFdelend \DIFaddbegin \DIFadd{Highlight \#2
%DIFDELCMD < }\DIFaddend 
%DIFDELCMD < %%%
\DIFdelend \DIFaddbegin \DIFadd{Highlight \#2
}\DIFaddend 

%DIF < DIF < DIF < DIF < DIF < % \item Highlight \#3
\DIFdelbegin %DIFDELCMD < \DIFdelbegin %%%
%DIF < DIFDELCMD < \DIFdelbegin %%%
%DIF < DIF < DIFDELCMD < \DIFdelbegin %%%
%DIF < DIF < DIF < DIFDELCMD < \DIFaddbegin %%%
%DIF < DIF < DIFDELCMD < \DIFdelend %%%
%DIF < DIFDELCMD < \DIFdelend %%%
%DIFDELCMD < \DIFdelend %%%
\DIFdelend \item \DIFdelbegin %DIFDELCMD < \DIFdelbegin %%%
%DIF < DIFDELCMD < \DIFdelbegin %%%
%DIF < DIF < DIFDELCMD < \DIFdelbegin %%%
%DIF < DIF < DIF < DIFDELCMD < \DIFadd{Highlight \#3
%DIF < DIF < DIF < DIFDELCMD < }\DIFaddend 
%DIF < DIF < DIF < DIFDELCMD < %%%
%DIF < DIF < DIFDELCMD < \DIFdelend \DIFaddbegin \DIFadd{Highlight \#3
%DIF < DIF < DIFDELCMD < }\DIFaddend 
%DIF < DIF < DIFDELCMD < %%%
%DIF < DIFDELCMD < \DIFdelend \DIFaddbegin \DIFadd{Highlight \#3
%DIF < DIFDELCMD < }\DIFaddend 
%DIF < DIFDELCMD < %%%
%DIFDELCMD < \DIFdelend \DIFaddbegin \DIFadd{Highlight \#3
%DIFDELCMD < }\DIFaddend 
%DIFDELCMD < %%%
\DIFdelend \DIFaddbegin \DIFadd{Highlight \#3
}\DIFaddend 

%DIF < DIF < DIF < DIF < DIF < % \item Highlight \#4
\DIFdelbegin %DIFDELCMD < \DIFdelbegin %%%
%DIF < DIFDELCMD < \DIFdelbegin %%%
%DIF < DIF < DIFDELCMD < \DIFdelbegin %%%
%DIF < DIF < DIF < DIFDELCMD < \DIFaddbegin %%%
%DIF < DIF < DIFDELCMD < \DIFdelend %%%
%DIF < DIFDELCMD < \DIFdelend %%%
%DIFDELCMD < \DIFdelend %%%
\DIFdelend \item \DIFdelbegin %DIFDELCMD < \DIFdelbegin %%%
%DIF < DIFDELCMD < \DIFdelbegin %%%
%DIF < DIF < DIFDELCMD < \DIFdelbegin %%%
%DIF < DIF < DIF < DIFDELCMD < \DIFadd{Highlight \#4
%DIF < DIF < DIF < DIFDELCMD < }\DIFaddend 
%DIF < DIF < DIF < DIFDELCMD < %%%
%DIF < DIF < DIFDELCMD < \DIFdelend \DIFaddbegin \DIFadd{Highlight \#4
%DIF < DIF < DIFDELCMD < }\DIFaddend 
%DIF < DIF < DIFDELCMD < %%%
%DIF < DIFDELCMD < \DIFdelend \DIFaddbegin \DIFadd{Highlight \#4
%DIF < DIFDELCMD < }\DIFaddend 
%DIF < DIFDELCMD < %%%
%DIFDELCMD < \DIFdelend \DIFaddbegin \DIFadd{Highlight \#4
%DIFDELCMD < }\DIFaddend 
%DIFDELCMD < %%%
\DIFdelend \DIFaddbegin \DIFadd{Highlight \#4
}\DIFaddend 

%DIF < DIF < DIF < DIF < DIF < % \end{highlights}
\DIFdelbegin %DIFDELCMD < \DIFdelbegin %%%
%DIF < DIFDELCMD < \DIFdelbegin %%%
%DIF < DIF < DIFDELCMD < \DIFdelbegin %%%
%DIF < DIF < DIF < DIFDELCMD < \DIFaddbegin %%%
%DIF < DIF < DIFDELCMD < \DIFdelend %%%
%DIF < DIFDELCMD < \DIFdelend %%%
%DIFDELCMD < \DIFdelend %%%
\DIFdelend \end{highlights}
\DIFdelbegin %DIFDELCMD < \DIFdelbegin %%%
%DIF < DIFDELCMD < \DIFdelbegin %%%
%DIF < DIF < DIFDELCMD < \DIFdelbegin %%%
%DIF < DIF < DIF < DIFDELCMD < \DIFaddend 
%DIF < DIF < DIF < DIFDELCMD < %%%
%DIF < DIF < DIFDELCMD < \DIFdelend 
%DIF < DIF < DIFDELCMD < %%%
%DIF < DIFDELCMD < \DIFdelend 
%DIF < DIFDELCMD < %%%
%DIFDELCMD < \DIFdelend 
%DIFDELCMD < %%%
\DIFdelend 

%DIF < DIF < DIF < DIF < DIF > %%% EOF
\DIFdelbegin %DIFDELCMD < \DIFdelbegin %%%
%DIF < DIFDELCMD < \DIFdelbegin %%%
%DIF < DIF < DIFDELCMD < \DIFdelbegin %%%
%DIF < DIF < DIF < DIFDELCMD < \DIFaddbegin 
%DIF < DIF < DIF < DIFDELCMD < %%%
%DIF < DIF < DIFDELCMD < \DIFdelend %%%
%DIF < DIF < DIF < DIF > %%% EOF
%DIF < DIFDELCMD < \DIFdelend %%%
%DIF < DIF < DIF > %%% EOF
%DIFDELCMD < \DIFdelend %%%
%DIF < DIF > %%% EOF
\DIFdelend %DIF > %%% EOF


\DIFdelbegin %DIFDELCMD < \DIFdelbegin %%%
%DIF < DIFDELCMD < \DIFdelbegin %%%
%DIF < DIF < DIFDELCMD < \DIFdelbegin %%%
%DIF < DIF < DIF < DIFDELCMD < \DIFaddend %%%
%DIF < DIF < DIFDELCMD < \DIFdelend %%%
%DIF < DIFDELCMD < \DIFdelend %%%
%DIFDELCMD < \DIFdelend %%%
\DIFdelend %% ========================================
%% ./01_manuscript/contents/title.tex
%% ========================================
\title{
Phase-amplitude coupling for detection and prediction of epileptic seizures in long-term intracranial electroencephalogram data
}



%% ========================================
%% ./01_manuscript/contents/authors.tex
%% ========================================
%% -*- coding: utf-8 -*-
%% Timestamp: "2025-09-24 18:07:39 (ywatanabe)"
%% File: "/ssh:sp:/home/ywatanabe/proj/neurovista/paper/manuscript/src/authors.tex"
\author[1]{Yusuke Watanabe}
\author[2,3]{Takufumi Yanagisawa}
\author[1]{David B. Grayden\corref{cor1}}


\address[1]{NeuroEngineering Research Laboratory, Department of Biomedical Engineering, The University of Melbourne, Parkville VIC 3010, Australia}
\address[2]{Institute for Advanced Cocreation studies, Osaka University, 2-2 Yamadaoka, Suita, 565-0871, Osaka, Japan}
\address[3]{Department of Neurosurgery, Osaka University Graduate School of Medicine, 2-2 Yamadaoka, Osaka, 565-0871, Japan}

\cortext[cor1]{Corresponding author. Tel: +XX-X-XXXX-XXXX Email: grayden@unimelb.edu.au}

%%%% EOF


%% ========================================
%% ./01_manuscript/contents/graphical_abstract.tex
%% ========================================
%%Graphical abstract
%\pdfbookmark[1]{Graphical Abstract}{graphicalabstract}        
%\begin{graphicalabstract}
%\includegraphics{grabs}
%\end{graphicalabstract}



%% ========================================
%% ./01_manuscript/contents/abstract.tex
%% ========================================
%% -*- coding: utf-8 -*-
%DIF < DIF < DIF < DIF < DIF < % Timestamp: "2025-09-22 07:46:14 (ywatanabe)"
%DIF < DIF < DIF < DIF < DIF < % File: "/ssh:sp:/home/ywatanabe/proj/neurovista/paper/manuscript/src/abstract.tex"
%DIF < DIF < DIF < DIF < DIF > % Timestamp: "2025-09-27 20:14:56 (ywatanabe)"
%DIF < DIF < DIF < DIF < DIF > % File: "/ssh:sp:/home/ywatanabe/proj/neurovista/paper/01_manuscript/contents/abstract.tex"
%DIF < DIF < DIF < DIF > % Timestamp: "2025-09-27 20:14:56 (ywatanabe)"
%DIF < DIF < DIF < DIF > % File: "/ssh:sp:/home/ywatanabe/proj/neurovista/paper/01_manuscript/contents/abstract.tex"
%DIF < DIF < DIF > % Timestamp: "2025-09-27 20:14:56 (ywatanabe)"
%DIF < DIF < DIF > % File: "/ssh:sp:/home/ywatanabe/proj/neurovista/paper/01_manuscript/contents/abstract.tex"
%DIF < DIF > % Timestamp: "2025-09-27 20:14:56 (ywatanabe)"
%DIF < DIF > % File: "/ssh:sp:/home/ywatanabe/proj/neurovista/paper/01_manuscript/contents/abstract.tex"
%DIF > % Timestamp: "2025-09-27 20:14:56 (ywatanabe)"
%DIF > % File: "/ssh:sp:/home/ywatanabe/proj/neurovista/paper/01_manuscript/contents/abstract.tex"
\begin{abstract}
  \pdfbookmark[1]{Abstract}{abstract}

Neural oscillations exhibit \DIFdelbegin %DIFDELCMD < \DIFdelbegin %%%
%DIF < DIFDELCMD < \DIFdelbegin %%%
%DIF < DIF < DIFDELCMD < \DIFdelbegin %%%
%DIF < DIF < DIF < DIFDELCMD < \DIFdelbegin \DIFdel{complex }\DIFdelend %%%
%DIF < DIF < DIFDELCMD < \DIFdelend %%%
%DIF < DIFDELCMD < \DIFdelend %%%
%DIFDELCMD < \DIFdelend %%%
\DIFdelend cross-frequency interactions that are fundamental to brain function and disrupted in neurological disorders. Phase-amplitude coupling (PAC), where the phase of low-frequency oscillations modulates the amplitude of high-frequency activity, serves as a biomarker for various brain states including epileptic seizures. Previous studies have demonstrated PAC changes around seizure events, but \DIFdelbegin %DIFDELCMD < \DIFdelbegin %%%
%DIF < DIFDELCMD < \DIFdelbegin %%%
%DIF < DIF < DIFDELCMD < \DIFdelbegin %%%
%DIF < DIF < DIF < DIFDELCMD < \DIFdelbegin \DIFdel{comprehensive }\DIFdelend %%%
%DIF < DIF < DIFDELCMD < \DIFdelend %%%
%DIF < DIFDELCMD < \DIFdelend %%%
%DIFDELCMD < \DIFdelend %%%
\DIFdelend characterization across extended timescales remains limited \DIFdelbegin %DIFDELCMD < \DIFdelbegin %%%
%DIF < DIFDELCMD < \DIFdelbegin %%%
%DIF < DIF < DIFDELCMD < \DIFdelbegin %%%
%DIF < DIF < DIF < DIFDELCMD < \DIFdelbegin \DIFdel{due to the scarcity }\DIFdelend \DIFaddbegin \DIFadd{availability }\DIFaddend %%%
%DIF < DIF < DIFDELCMD < \DIFdelend \DIFaddbegin \DIFadd{availability }\DIFaddend %%%
%DIF < DIFDELCMD < \DIFdelend \DIFaddbegin \DIFadd{availability }\DIFaddend %%%
%DIFDELCMD < \DIFdelend \DIFaddbegin \DIFadd{availability }\DIFaddend %%%
\DIFdelend \DIFaddbegin \DIFadd{availability }\DIFaddend of long-term recording data and \DIFdelbegin %DIFDELCMD < \DIFdelbegin %%%
%DIF < DIFDELCMD < \DIFdelbegin %%%
%DIF < DIF < DIFDELCMD < \DIFdelbegin %%%
%DIF < DIF < DIF < DIFDELCMD < \DIFdelbegin \DIFdel{computational constraints}\DIFdelend \DIFaddbegin \DIFadd{high computational requirements in PAC computation}\DIFaddend %%%
%DIF < DIF < DIFDELCMD < \DIFdelend \DIFaddbegin \DIFadd{high computational requirements in PAC computation}\DIFaddend %%%
%DIF < DIFDELCMD < \DIFdelend \DIFaddbegin \DIFadd{high computational requirements in PAC computation}\DIFaddend %%%
%DIFDELCMD < \DIFdelend \DIFaddbegin \DIFadd{high computational requirements in PAC computation}\DIFaddend %%%
\DIFdelend \DIFaddbegin \DIFadd{high computational requirements in PAC computation}\DIFaddend . The challenge of processing continuous, long-term neural recordings has hindered the development of reliable seizure prediction systems. Here we show that \DIFdelbegin %DIFDELCMD < \DIFdelbegin %%%
%DIF < DIFDELCMD < \DIFdelbegin %%%
%DIF < DIF < DIFDELCMD < \DIFdelbegin %%%
%DIF < DIF < DIF < DIFDELCMD < \DIFaddbegin \DIFadd{the combination of the NeuroVista dataset and our }\DIFaddend %%%
%DIF < DIF < DIFDELCMD < \DIFdelend \DIFaddbegin \DIFadd{the combination of the NeuroVista dataset and our }\DIFaddend %%%
%DIF < DIFDELCMD < \DIFdelend \DIFaddbegin \DIFadd{the combination of the NeuroVista dataset and our }\DIFaddend %%%
%DIFDELCMD < \DIFdelend \DIFaddbegin \DIFadd{the combination of the NeuroVista dataset and our }\DIFaddend %%%
\DIFdelend \DIFaddbegin \DIFadd{the combination of the NeuroVista dataset and our }\DIFaddend GPU-accelerated PAC computation \DIFdelbegin %DIFDELCMD < \DIFdelbegin %%%
%DIF < DIFDELCMD < \DIFdelbegin %%%
%DIF < DIF < DIFDELCMD < \DIFdelbegin %%%
%DIF < DIF < DIF < DIFDELCMD < \DIFdelbegin \DIFdel{enables }\DIFdelend \DIFaddbegin \DIFadd{system enables ...
%DIF < DIF < DIF < DIFDELCMD < }
%DIF < DIF < DIF < DIFDELCMD < %%%
%DIF < DIF < DIFDELCMD < \DIFdelend \DIFaddbegin \DIFadd{system enables ...
%DIF < DIF < DIFDELCMD < }\DIFaddend 
%DIF < DIF < DIFDELCMD < %%%
%DIF < DIFDELCMD < \DIFdelend \DIFaddbegin \DIFadd{system enables ...
%DIF < DIFDELCMD < }\DIFaddend 
%DIF < DIFDELCMD < %%%
%DIFDELCMD < \DIFdelend \DIFaddbegin \DIFadd{system enables ...
%DIFDELCMD < }\DIFaddend 
%DIFDELCMD < %%%
\DIFdelend \DIFaddbegin \DIFadd{system enables ...
}\DIFaddend 

\DIFdelbegin %DIFDELCMD < \DIFdelbegin %%%
%DIF < DIFDELCMD < \DIFdelbegin %%%
%DIF < DIF < DIFDELCMD < \DIFdelbegin %%%
%DIF < DIF < DIF < DIFDELCMD < \DIFaddend %%%
%DIF < DIF < DIFDELCMD < \DIFdelend %%%
%DIF < DIFDELCMD < \DIFdelend %%%
%DIFDELCMD < \DIFdelend %%%
\DIFdelend comprehensive analysis of 4.1 TB of continuous intracranial electroencephalogram data from 15 patients with focal epilepsy (NeuroVista dataset), encompassing 1,539 seizures over monitoring periods ranging from 6 months to 2 years. \hl{We found distinct PAC signatures between theta-to-beta phase (2-30 Hz, 25 bands) and gamma amplitude (60-180 Hz, 25 bands) that systematically modulated 5-60 minutes before seizure onset}, achieving balanced accuracy of \hl{0.55±0.04} and ROC-AUC of \hl{0.58±0.02} for discriminating pre-ictal from interictal states. Our GPU-accelerated implementation achieved \hl{100-fold} speed improvements compared to conventional CPU-based methods, reducing computation time from years to months and potentially enabling real-time PAC monitoring with less than 2-minute processing latency. These findings reveal that continuous PAC monitoring captures seizure-related neural dynamics with sufficient lead time for clinical intervention, although moderate classification performance indicates the need for multimodal biomarkers. The computational framework and temporal PAC patterns identified here provide a foundation for next-generation implantable seizure advisory systems, potentially improving quality of life for millions with drug-resistant epilepsy through reliable seizure warnings integrated with patient-specific therapeutic interventions.

\end{abstract}

%%%% EOF


%% ========================================
%% ./01_manuscript/contents/keywords.tex
%% ========================================
% \pdfbookmark[1]{Keywords}{keywords}                
\begin{keyword}
epilepsy \sep seizure detection \sep seizure prediction \sep NeuroVista dataset \sep phase-amplitude coupling 
\end{keyword}


\end{frontmatter}

%% ----------------------------------------
%% Word Counter
%% ----------------------------------------

%% ========================================
%% ./01_manuscript/contents/wordcount.tex
%% ========================================
%% -*- coding: utf-8 -*-
%% Timestamp: "2025-09-26 18:17:20 (ywatanabe)"
%% File: "/ssh:sp:/home/ywatanabe/proj/neurovista/paper/01_manuscript/src/wordcount.tex"
\begin{wordcount}
\readwordcount{./01_manuscript/contents/wordcounts/figure_count.txt} figures, \readwordcount{./01_manuscript/contents/wordcounts/table_count.txt} tables, \readwordcount{./01_manuscript/contents/wordcounts/abstract_count.txt} words for abstract, and \readwordcount{./01_manuscript/contents/wordcounts/imrd_count.txt} words for main text
\end{wordcount}

%% \begin{*wordcount}
%% \readwordcount{./01_manuscript/contents/wordcounts/figure_count.txt} figures, \readwordcount{./01_manuscript/contents/wordcounts/table_count.txt} tables, \readwordcount{./01_manuscript/contents/wordcounts/abstract_count.txt} words for abstract, and \readwordcount{./01_manuscript/contents/wordcounts/imrd_count.txt} words for main text
%% \end{*wordcount}

%%%% EOF


%% ----------------------------------------
%% INTRODUCTION
%% ----------------------------------------

%% ========================================
%% ./01_manuscript/contents/introduction.tex
%% ========================================
%% -*- coding: utf-8 -*-
%% Timestamp: "2025-09-26 21:47:51 (ywatanabe)"
%% File: "/ssh:sp:/home/ywatanabe/proj/neurovista/paper/01_manuscript/src/introduction.tex"

%% \[START of [0-9]\. .* Statement\]
%% \[\(START\|END\) of [0-9]+\. [^]]*\] *
\section{Introduction}

Epilepsy affects approximately 70 million people worldwide, representing one of the most prevalent neurological disorders characterized by recurrent, unpredictable seizures that fundamentally disrupt daily life and impose substantial societal burden \hlref{WHO2023}. Drug-resistant focal epilepsy, occurring in approximately 30-40\% of epilepsy patients, remains particularly challenging as conventional antiepileptic medications fail to provide adequate seizure control despite optimal medical management \hlref{Kwan2000,Chen2018}. The development of reliable seizure prediction systems represents a critical frontier in epilepsy care, offering the potential to transform patient management from reactive treatment to proactive intervention, thereby reducing seizure-related injuries, psychological burden, and improving quality of life for millions of individuals with drug-resistant epilepsy \hlref{Mormann2007,Kuhlmann2018}. 

	Decades of research have established that neural oscillations exhibit complex cross-frequency interactions fundamental to brain function, with phase-amplitude coupling (PAC) emerging as a robust biomarker for various brain states including pathological conditions \hlref{Tort2010,Canolty2006}. PAC quantifies the phenomenon whereby the phase of low-frequency oscillations modulates the amplitude of high-frequency activity, reflecting coordinated neural communication across different temporal scales essential for cognitive processes and sensorimotor integration \hlref{Jensen2007,Lakatos2005}. Previous electrophysiological studies have demonstrated measurable PAC alterations in the peri-ictal period, with changes in theta-gamma coupling patterns observed minutes to hours before seizure onset in both animal models and human patients \hlref{Weiss2013,Edakawa2016,Motamedi2013}. However, comprehensive characterization of PAC dynamics across extended timescales has remained limited due to the computational demands of processing continuous, long-term neural recordings and the relative scarcity of datasets spanning months to years of continuous monitoring \hlref{Brinkmann2016}. 

	Existing seizure prediction approaches, including both traditional PAC-based methods and contemporary machine learning techniques, face several critical limitations that have hindered clinical translation. While current trends favor deep learning and complex feature extraction algorithms, these approaches often suffer from poor interpretability, extensive hyperparameter optimization requirements, and limited generalizability across patients and recording conditions \hlref{Rasheed2021,Bou2021}. Traditional PAC-based seizure prediction approaches specifically face computational constraints that have restricted most studies to short recording segments or sparse temporal sampling, preventing comprehensive analysis of PAC evolution across full seizure cycles and missing potentially crucial long-term patterns \hlref{Jacobs2018}. Methodological inconsistencies in PAC quantification, including variations in frequency band definitions, modulation index calculations, and statistical normalization approaches, have produced conflicting results across studies and limited reproducibility \hlref{Aru2015}. Additionally, the majority of previous investigations have relied on relatively small datasets with limited seizure counts per patient, reducing statistical power and generalizability of findings to diverse epilepsy phenotypes \hlref{Freestone2015}. Finally, conventional CPU-based implementations of PAC algorithms exhibit prohibitive computational complexity for real-time applications, with processing times often exceeding recording duration by orders of magnitude, making continuous monitoring clinically impractical \hlref{Canolty2006}. 

	These limitations underscore the critical need for a computationally efficient, comprehensive approach to PAC-based seizure prediction that can leverage large-scale, long-term electrophysiological datasets to identify robust, generalizable biomarkers of seizure susceptibility with sufficient temporal resolution for clinical intervention. 

	To address these challenges, we developed a GPU-accelerated PAC computation framework optimized for massive-scale electrophysiological data analysis, enabling comprehensive characterization of phase-amplitude coupling dynamics across 4.1 terabytes of continuous intracranial electroencephalogram (iEEG) recordings from the NeuroVista dataset. Our approach leverages the Spartan HPC system's distributed GPU architecture with custom PyTorch implementations achieving approximately 100-fold acceleration compared to conventional CPU methods. PAC offers distinct advantages over conventional machine learning biomarkers as it reflects fundamental neural communication mechanisms without requiring hyperparameter optimization, providing inherently interpretable measures of cross-frequency interactions essential for brain functioning \hlref{Tort2010}. We applied this framework to analyze 1,539 Type 1 seizures across 15 patients with drug-resistant focal epilepsy, examining PAC evolution from 24 hours before to 10 minutes after seizure onset using adaptive temporal sampling with 1-minute resolution windows. Statistical robustness was ensured through surrogate-based normalization using 200 circular phase shuffles, generating stable z-scored PAC features within each temporal window that maintain statistical validity across extended monitoring periods. 

	Our analysis revealed systematic modulation of theta-to-beta phase (2-30 Hz) and gamma amplitude (60-180 Hz) coupling patterns occurring 5-60 minutes before seizure onset, achieving balanced accuracy of 0.55±0.04 and area under the receiver operating characteristic curve of 0.58±0.02 for discriminating pre-ictal from interictal states. The GPU-accelerated implementation reduced computation time from years to months while maintaining statistical rigor through surrogate-based significance testing, demonstrating the feasibility of near real-time PAC monitoring with processing latencies under 2 minutes. 

These findings establish PAC as a computationally tractable biomarker for seizure prediction in large-scale clinical datasets, providing a foundation for next-generation implantable advisory systems that could transform epilepsy management from reactive to predictive care. The moderate classification performance observed highlights the need for multimodal biomarker integration while demonstrating sufficient predictive capacity to warrant clinical investigation of PAC-based seizure warning systems. 

Add this text for testing diff
\label{sec:introduction}

%%%% EOF


%% ----------------------------------------
%% METHODS
%% ----------------------------------------

%% ========================================
%% ./01_manuscript/contents/methods.tex
%% ========================================
%% -*- coding: utf-8 -*-
%% Timestamp: "2025-09-22 07:49:38 (ywatanabe)"
%% File: "/ssh:sp:/home/ywatanabe/proj/neurovista/paper/manuscript/src/methods.tex"
\section{Methods}

\subsection{Dataset and Study Design}
Data were obtained from the NeuroVista clinical trial \hlref{Cook2013} through the International Epilepsy Electrophysiology Portal (ieeg.org). The dataset comprised continuous intracranial electroencephalogram (iEEG) recordings from 15 human subjects with drug-resistant focal epilepsy, collected under ethical approval from the original clinical trial \hlref{Ethics}. Each subject was implanted with 16-channel platinum-iridium electrode arrays positioned around the \hl{clinically-identified seizure onset zone}, with signals sampled at 400 Hz and wirelessly transmitted to an external personal advisory device. The total dataset encompassed 4.1 TB of continuous recordings, with individual monitoring periods ranging from 6 months to over 2 years. From the complete dataset containing multiple seizure types, this study specifically analyzed 1,539 Type 1 (clinical) seizures with verified clinical manifestations across the 15 patients to ensure clinical relevance and interpretability of prediction algorithms.

	Seizures were classified following the original trial protocol into clinical (Type 1), clinically-equivalent (Type 2), and subclinical (Type 3) events. For this study, we focused exclusively on Type 1 seizures, which presented with observable clinical symptoms and clear electrographic patterns. Type 2 events (seizure-like events without clinical manifestations) and Type 3 events (subclinical seizures with distinct electrographic signatures) were excluded from analysis to maintain clinical relevance and ensure clear interpretation of prediction performance. Interictal control periods were selected from seizure-free intervals exceeding 4 hours from any Type 1 seizure event, matched for time of day to control for circadian influences on neural oscillations.

\subsection{Data Preprocessing and Quality Control}
The entire dataset was split into training, validation, and test sets following temporal progression to ensure pseudo-prospective evaluation. This temporal separation between training and testing phases prevents data leakage and provides realistic assessment of prediction performance. Training data comprised the earliest recordings, followed by validation data for hyperparameter tuning, with the most recent recordings reserved as held-out test data.

	Data integrity was maintained through cryptographic hashing (SHA-256) of all iEEG segments and seizure identifiers, enabling complete traceability and reproducibility of analyses. Each processed segment was stored with metadata including patient identifier, seizure type, exact timing, channel configuration, and processing parameters.

\subsection{Phase-Amplitude Coupling Analysis}
\subsubsection{GPU-Accelerated Implementation}
PAC strength was quantified using the modulation index (MI) \cite{Tort} following the Shannon entropy-based formulation: MI = 1 + $\\sum$(p $\\times$ log(p))/log(N), where p represents the normalized amplitude distribution across N phase bins and N = 18 bins (20° per bin). Computation was performed using a custom, standalone GPU-accelerated package (https://github.com/ywatanabe1989/gPAC) built on PyTorch with full vectorization across all frequency combinations. The implementation achieved approximately 100-fold speed improvement compared to conventional CPU-based methods through: (1) massive tensor operations eliminating nested loops, (2) optimized memory allocation utilizing up to 320GB total VRAM across multiple GPU nodes, and (3) batch processing with fp16 precision where appropriate. Processing leveraged the Spartan HPC system's distributed GPU architecture \hlref{Spartan} with automatic multi-GPU parallelization. Statistical significance was established using 200 surrogate datasets generated through circular phase shuffling, with PAC values z-score normalized relative to the surrogate distribution to eliminate spurious coupling.

	For each 1-minute non-overlapping time window, PAC was computed between 25 phase frequency bands (2.0-30.0 Hz) and 25 amplitude frequency bands (60.0-180.0 Hz), resulting in a 625-element PAC matrix per channel per time point. Frequency bands were generated using field-standard adaptive bandwidths: phase bands employed bandwidth = f/2 (e.g., 10 Hz center frequency spans 7.5-12.5 Hz), while amplitude bands used bandwidth = f/4 (e.g., 100 Hz center frequency spans 87.5-112.5 Hz). This approach yielded phase bands with bandwidths ranging from 0.5 Hz to 11.9 Hz and amplitude bands with bandwidths from 7.5 Hz to 40.0 Hz. Phase and amplitude information were extracted through bandpass filtering followed by Hilbert transformation to obtain instantaneous phase and amplitude envelopes. MI quantified coupling strength using the Shannon entropy-based formulation across 18 phase bins (20° each):

\begin{equation}
MI = 1 + \frac{\sum_{j=1}^{N} p_j \log(p_j)}{\log(N)}
\end{equation}

where $p_j$ represents the normalized amplitude probability in phase bin $j$, and $N = 18$ indicates the number of phase bins. Values range from 0 (uniform amplitude distribution) to 1 (maximum concentration in single phase bin). PAC values were z-score normalized using 200 surrogate datasets generated through circular phase shifts to control for spurious coupling effects.

	Missing values (NaN) in PAC computations arose from NaN values in recorded ECoG signals due to limited data type (int16), edge effects in filtering, or numerical instabilities in specific frequency combinations. NaN values found in ECoG signals were replaced with 0 while NaN values in PAC data were handled as is. Features derived from PAC matrices used nanmean, nanstd, and other NaN-aware statistical functions from NumPy to ensure robust computation despite missing values.

\subsubsection{Temporal Windows and Event Definition}
PAC dynamics were analyzed across multiple temporal scales relative to seizure onset. For each seizure event, computations spanned from -1440 minutes (24 hours) to +10 minutes post-onset, with adaptive temporal resolution optimized for seizure prediction. Timestamps relative to seizure onset ($t = 0$) were generated using conditional sampling:

\begin{equation}
t_i = \begin{cases}
-\text{round}(10^{(\log_{10}(60) + i \cdot \frac{\log_{10}(1440) - \log_{10}(60)}{N_{log}-1})}) & \text{for } i = 0, 1, \ldots, N_{log}-1 \text{ (logarithmic)} \\
-60 + j & \text{for } j = 0, 1, \ldots, 70 \text{ (linear)}
\end{cases}
\end{equation}

where logarithmic sampling from -1440 to -60 minutes provides denser resolution approaching seizure onset, and minute-by-minute linear sampling from -60 to +10 minutes captures critical peri-ictal to early ictal dynamics. The pre-ictal period was operationally defined as -60 to -5 minutes before seizure onset, based on previous seizure prediction studies \hlref{Kuhlmann2018} and clinical requirements for actionable warning times.

	For each Type 1 seizure, an equal number of interictal control segments were randomly sampled from the available seizure-free periods (>4 hours from any Type 1 seizure), ensuring balanced representation in subsequent classification analyses. Control segments were matched for time of day to account for circadian variations \hl{\cite{Pip}} in PAC patterns. All random sampling employed fixed seeds (SHA-256 hash-based) for complete reproducibility across analyses.

\subsection{Database Architecture and Storage}
Processed PAC data were organized in patient-specific SQLite3 databases with hierarchical structure optimized for HPC storage allocation and concurrent write operations to maximize parallel computation efficiency. Each database contained three primary components: (1) metadata tables storing patient demographics, seizure annotations, and processing parameters; (2) PAC data tables with zlib-compressed binary large objects (BLOBs) achieving \hl{70-90\%} storage reduction; (3) quality assurance tables tracking computation timestamps, software versions, and validation metrics. The database schema enabled efficient retrieval of specific temporal windows, frequency bands, or statistical measures without loading complete datasets into memory. Database operations were handled using the scitex.db module, a custom database interface optimized for scientific computing workflows.

	Data integrity was ensured through transaction-based writes with automatic rollback on errors, regular consistency checks comparing stored and computed checksums, and version control of all processing scripts with git-based tracking.

\subsection{Machine Learning Classification}
\subsubsection{Feature Engineering and Selection}
From the raw PAC matrices, we extracted 17 statistical features per time window: mean, standard deviation, median, minimum, maximum, 25th/50th/75th percentiles, kurtosis, skewness of PAC z-scores, plus specialized bimodality metrics from Gaussian mixture model (GMM) fitting including Ashman's D statistic, weight ratios, Bhattacharyya coefficient, and bimodality coefficient. Additionally, circular statistics of the preferred coupling phase were computed: circular mean, concentration (inverse of circular variance), circular skewness, and circular kurtosis.

	\hl{Feature selection employed a data-driven approach combining univariate statistical testing (Mann-Whitney U test with false discovery rate correction), recursive feature elimination with using performances in validation dataset.}

\subsubsection{Classification Algorithms}
\hl{Three classification approaches were evaluated: (1) logistic regression with L2 regularization for interpretable linear decision boundaries; (2) random forest with 100 estimators for capturing non-linear interactions; (3) support vector machines with radial basis function kernels for maximum-margin classification.} Additionally, a dummy classifier using stratified random predictions served as a baseline for performance comparison.

	For temporal context incorporation, we implemented sliding window classification where features from consecutive time points (window sizes of \hl{1, 3, 5, and 10 minutes}) were concatenated, allowing models to capture temporal evolution of PAC patterns. Window labeling followed the last time point in each window to maintain causal prediction constraints.

\subsubsection{Cross-Validation Strategy}
Model evaluation employed stratified time series cross-validation maintaining temporal ordering and class balance. Specifically, 5-fold stratified time series cross-validation was employed with training and validation data split by 0.8:0.2 ratio. The custom splitter ensured test data always occurred chronologically after training data, with configurable gaps (0-60 minutes) between training and test sets to prevent temporal leakage. Within-patient validation used these 5-fold splits preserving the temporal sequence, while across-patient evaluation employed leave-one-patient-out validation to assess generalization.

	Hyperparameter optimization utilized the validation set (20\% of training data) with \hl{grid search over regularization strengths} (logistic regression: C = 0.001 to 100), tree depths (random forest: 5 to 50), and kernel parameters (SVM: gamma = 0.001 to 10). \hl{Final models were retrained on complete training sets with optimal hyperparameters before evaluation on held-out test data}

\subsection{Performance Metrics and Statistical Analysis}
Classification performance was assessed using multiple complementary metrics addressing different aspects of seizure prediction requirements. Balanced accuracy weighted sensitivity and specificity equally, accounting for class imbalance. The area under the receiver operating characteristic curve (ROC-AUC) measured discrimination ability across all decision thresholds. Seizure-specific metrics included: sensitivity (proportion of correctly predicted seizures), time in warning (percentage of time under high-risk advisory), false positive rate per hour (FPR/h) normalized by prediction frequency, and alarm episode rate (consecutive false positives counted as single events).

	Clinical acceptability thresholds were defined based on previous implantable device studies: sensitivity ≥90\% for reliable seizure detection, FPR/h ≤0.0033 (equivalent to ≤0.2 alarm episodes/hour for 1-minute resolution), and time in warning ≤20\% to minimize patient burden. Statistical significance was assessed using permutation tests (10,000 iterations) for individual metrics and DeLong's test for ROC curve comparisons, with Bonferroni correction for multiple comparisons across patients and time windows.

\subsection{Visualization and Reporting}
Results were visualized through multiple complementary approaches. Comodulograms displayed PAC strength across all phase-amplitude frequency combinations as 2D heatmaps with color scaling defined based on baseline periods. Temporal evolution plots showed PAC dynamics from -60 to +10 minutes around seizures with confidence intervals from bootstrap resampling. Channel-wise topographic maps illustrated spatial distribution of PAC across the 16-electrode array. Statistical significance maps highlighted frequency pairs with consistent pre-ictal changes across seizures.

	All analyses were conducted using Python 3.10 with specialized libraries: PyTorch 2.0 for GPU computation, scikit-learn 1.3 for machine learning, MNE-Python 1.5 for electrophysiology-specific processing, and our custom SciTeX framework (https://github.com/ywatanabe1989/SciTeX-Code) for reproducible scientific computing. Complete code, processed features, and trained models are available at \hlref{DataAvailability} to facilitate reproduction and extension of our findings.

\label{sec:methods}

%%%% EOF


%% ----------------------------------------
%% RESULTS
%% ----------------------------------------

%% ========================================
%% ./01_manuscript/contents/results.tex
%% ========================================
%% -*- coding: utf-8 -*-
%DIF < DIF < DIF < DIF < DIF < % Timestamp: "2025-09-22 08:12:53 (ywatanabe)"
%DIF < DIF < DIF < DIF < DIF < % File: "/ssh:sp:/home/ywatanabe/proj/neurovista/paper/manuscript/src/results.tex"
%DIF < DIF < DIF < DIF < DIF > % Timestamp: "2025-09-27 20:24:08 (ywatanabe)"
%DIF < DIF < DIF < DIF < DIF > % File: "/ssh:sp:/home/ywatanabe/proj/neurovista/paper/01_manuscript/contents/results.tex"
%DIF < DIF < DIF < DIF > % Timestamp: "2025-09-27 20:24:08 (ywatanabe)"
%DIF < DIF < DIF < DIF > % File: "/ssh:sp:/home/ywatanabe/proj/neurovista/paper/01_manuscript/contents/results.tex"
%DIF < DIF < DIF > % Timestamp: "2025-09-27 20:24:08 (ywatanabe)"
%DIF < DIF < DIF > % File: "/ssh:sp:/home/ywatanabe/proj/neurovista/paper/01_manuscript/contents/results.tex"
%DIF < DIF > % Timestamp: "2025-09-27 20:24:08 (ywatanabe)"
%DIF < DIF > % File: "/ssh:sp:/home/ywatanabe/proj/neurovista/paper/01_manuscript/contents/results.tex"
%DIF > % Timestamp: "2025-09-27 20:24:08 (ywatanabe)"
%DIF > % File: "/ssh:sp:/home/ywatanabe/proj/neurovista/paper/01_manuscript/contents/results.tex"
\section{Results}

\subsection{Dataset Characteristics and Patient Demographics}
The NeuroVista dataset comprised continuous intracranial electroencephalogram recordings from 15 patients with drug-resistant focal epilepsy, totaling 4.1 TB of data collected over monitoring periods ranging from 6 months to 2 years. Patient demographics showed a balanced distribution with \hl{8 female and 7 male participants, ages 18-50 years (mean: 32.4±9.2 years). Recording durations varied substantially across patients, with median monitoring periods of 413 days (range: 183-751 days)} \hlref{Table1}. 

	From the complete dataset containing multiple seizure classifications, we identified and analyzed 1,539 Type 1 (clinical) seizures with verified clinical manifestations across all patients. Seizure frequency exhibited marked inter-patient variability, ranging from \hl{12 to 384} seizures per patient (median: \hl{89} seizures per patient), reflecting the heterogeneous nature of drug-resistant epilepsy \hlref{Figure1A}. Seizure durations showed log-normal distribution with median duration of \hl{42.3} seconds (interquartile range: \hl{28.1-68.7} seconds), consistent with previous reports of focal seizures in this population \hlref{Cook2013}.

\subsection{PAC Pattern Heterogeneity Across Patients and Channels}
Exploratory analysis revealed heterogeneity in baseline PAC patterns both across patients and recording channels, providing justification for patient-specific modeling approaches. Baseline PAC strength, quantified using the modulation index across 625 phase-amplitude frequency pairs (25×25 combinations), showed significant variation between patients \hl{(Kruskal-Wallis H = 2,847.3, p < 0.001)} and channels within patients \hl{(nested ANOVA F = 89.4, p < 0.001)} \hlref{Figure2A}.

	Inter-patient coefficient of variation for mean PAC strength ranged from \hl{0.23 to 0.67} across frequency pairs, with theta-gamma coupling (4-8 Hz phase, 60-100 Hz amplitude) showing the highest variability (CV = 0.61±0.12). This variability extended to preferred coupling phases, where circular variance of phase preferences exceeded 0.8 for 73\% of frequency pairs across patients, indicating substantial individual differences in neural synchronization patterns \hlref{Figure2B}. Channel-wise analysis within individual patients revealed consistent spatial patterns of PAC strength, with seizure onset zone channels exhibiting significantly higher baseline PAC compared to non-seizure onset channels (Wilcoxon signed-rank test, Z = -8.94, p < 0.001) \hlref{Figure2C}.

	Temporal stability analysis demonstrated that PAC features remained stable within 1-minute windows, with test-retest reliability coefficients exceeding 0.85 for 89\% of frequency pairs during interictal periods. However, between-day variability was substantial (ICC = 0.42±0.18), emphasizing the importance of adaptive normalization approaches and patient-specific feature weighting in seizure prediction models \hlref{Figure3}.

\subsection{Pre-ictal PAC Dynamics and Temporal Evolution}
Systematic analysis of PAC evolution relative to seizure onset revealed consistent pre-ictal modulation patterns beginning 5-60 minutes before clinical seizure manifestation. Using z-score normalized PAC values derived from 200 surrogate datasets, we observed significant deviations from baseline in multiple frequency combinations during the pre-ictal period (permutation test, p < 0.05, FDR corrected) \hlref{Figure4A}.

	Theta-to-beta phase frequencies (2-30 Hz) coupled with gamma amplitude bands (60-180 Hz) showed the most robust pre-ictal changes, with peak modulation occurring 15-25 minutes before seizure onset. Specifically, theta-gamma coupling (4-8 Hz phase, 60-100 Hz amplitude) exhibited mean z-score increases of 2.34±0.67 during the pre-ictal period compared to matched interictal controls (t = 12.8, p < 0.001). Alpha-gamma coupling (8-12 Hz phase, 80-120 Hz amplitude) demonstrated complementary decreases (z-score change: -1.89±0.54, t = -9.7, p < 0.001), suggesting a reorganization of cross-frequency interactions preceding seizure onset \hlref{Figure4B}.

	Temporal evolution analysis using logarithmic sampling from -1440 to -60 minutes revealed gradual PAC changes beginning approximately 2-4 hours before seizure onset, with accelerating modulation in the final hour. Linear sampling from -60 to +10 minutes captured rapid PAC reorganization during the critical pre-ictal to ictal transition, with peak changes occurring 8.7±4.2 minutes before electrographic seizure onset \hlref{Figure4C}.

\subsection{Seizure Prediction Performance and Classification Results}
Machine learning classification using extracted PAC features achieved balanced accuracy of 0.55±0.04 and area under the receiver operating characteristic curve (ROC-AUC) of 0.58±0.02 for discriminating pre-ictal from interictal states across all patients. Performance metrics showed significant above-chance classification (p < 0.001, permutation test with 10,000 iterations), indicating reliable seizure prediction capability despite moderate effect sizes \hlref{Table2}.

	Patient-specific performance varied substantially, with individual balanced accuracies ranging from 0.48 to 0.67 (median: 0.54). Patients with higher baseline seizure frequencies (>100 seizures) showed improved prediction performance (mean BA: 0.59±0.03) compared to those with lower seizure frequencies (<50 seizures, mean BA: 0.51±0.04, t = 3.2, p = 0.007). This relationship suggests that larger training datasets improve model generalization for patient-specific seizure patterns \hlref{Figure5A}.

	Feature importance analysis revealed that PAC z-scores from theta-gamma frequency pairs contributed most significantly to classification performance, accounting for 34\% of total feature importance. Bimodality metrics from Gaussian mixture model fitting of PAC distributions provided additional discriminative power (18\% feature importance), while circular statistics of preferred coupling phases contributed moderately (12\% feature importance) \hlref{Figure5B}. Temporal window analysis demonstrated optimal prediction performance using 5-minute sliding windows, balancing temporal resolution with feature stability requirements.

\subsection{Computational Performance and Clinical Feasibility}
The GPU-accelerated PAC computation framework achieved approximately 100-fold speed improvements compared to conventional CPU-based implementations, reducing total computation time for the complete dataset from an estimated 14.2 years to 1.8 months using the Spartan HPC system's distributed GPU architecture. Processing latency for real-time applications was 1.7±0.3 minutes for 1-minute PAC computation windows, demonstrating feasibility for near real-time seizure monitoring applications \hlref{Table3}.

	Memory efficiency optimizations through adaptive chunking and fp16 precision enabled processing of the complete 4.1 TB dataset within available HPC resources (320 GB total VRAM across multiple GPU nodes). Database storage using zlib compression achieved 78\% size reduction, with final processed PAC features requiring 847 GB storage compared to 3.9 TB for uncompressed data. These computational achievements enable comprehensive PAC analysis of large-scale, long-term electrophysiological datasets that were previously computationally intractable \hlref{Figure6}.

\subsection{Cross-Validation and Generalization Analysis}
Stratified time series cross-validation maintaining temporal ordering and class balance yielded consistent performance across validation folds (CV coefficient of variation: 0.08±0.03). Leave-one-patient-out cross-validation demonstrated moderate generalization capability, with mean balanced accuracy of 0.52±0.06 when models trained on 14 patients were applied to held-out individuals. This reduction compared to within-patient performance (0.55±0.04) highlights the importance of patient-specific adaptation while indicating some transferable PAC patterns across individuals \hlref{Figure7}.

	Temporal gap analysis between training and test sets (0-60 minutes) showed robust performance maintenance, with less than 3\% accuracy degradation for gaps up to 30 minutes. This temporal stability supports the clinical applicability of PAC-based seizure prediction, as models trained on historical data maintain predictive capability for future time periods within reasonable clinical timeframes.

\DIFdelbegin %DIFDELCMD < \DIFdelbegin %%%
%DIF < DIFDELCMD < \DIFdelbegin %%%
%DIF < DIF < DIFDELCMD < \DIFdelbegin %%%
%DIF < DIF < DIF < DIFDELCMD < \DIFaddbegin %%%
%DIF < DIF < DIFDELCMD < \DIFdelend %%%
%DIF < DIFDELCMD < \DIFdelend %%%
%DIFDELCMD < \DIFdelend %%%
\DIFdelend \section{\DIFdelbegin %DIFDELCMD < \DIFdelbegin %%%
%DIF < DIFDELCMD < \DIFdelbegin %%%
%DIF < DIF < DIFDELCMD < \DIFdelbegin %%%
%DIF < DIF < DIF < DIFDELCMD < \DIFadd{Results}%%%
%DIF < DIF < DIFDELCMD < \DIFdelend \DIFaddbegin \DIFadd{Results}\DIFaddend %%%
%DIF < DIFDELCMD < \DIFdelend \DIFaddbegin \DIFadd{Results}\DIFaddend %%%
%DIFDELCMD < \DIFdelend \DIFaddbegin \DIFadd{Results}\DIFaddend %%%
\DIFdelend \DIFaddbegin \DIFadd{Results}\DIFaddend }

\subsection{\DIFdelbegin %DIFDELCMD < \DIFdelbegin %%%
%DIF < DIFDELCMD < \DIFdelbegin %%%
%DIF < DIF < DIFDELCMD < \DIFdelbegin %%%
%DIF < DIF < DIF < DIFDELCMD < \DIFadd{Dataset Characteristics}%%%
%DIF < DIF < DIFDELCMD < \DIFdelend \DIFaddbegin \DIFadd{Dataset Characteristics}\DIFaddend %%%
%DIF < DIFDELCMD < \DIFdelend \DIFaddbegin \DIFadd{Dataset Characteristics}\DIFaddend %%%
%DIFDELCMD < \DIFdelend \DIFaddbegin \DIFadd{Dataset Characteristics}\DIFaddend %%%
\DIFdelend \DIFaddbegin \DIFadd{Dataset Characteristics}\DIFaddend }
\DIFdelbegin %DIFDELCMD < \DIFdelbegin %%%
%DIF < DIFDELCMD < \DIFdelbegin %%%
%DIF < DIF < DIFDELCMD < \DIFdelbegin %%%
%DIF < DIF < DIF < DIFDELCMD < \DIFadd{The NeuroVista dataset provided extensive long-term iEEG recordings from 9 patients with drug-resistant focal epilepsy. After excluding the initial 100-day post-implantation period, we analyzed a total of XX patient-years of continuous recording (mean: XX months per patient, range: 6-14 months). Across all patients, we identified and analyzed:
%DIF < DIF < DIF < DIFDELCMD < - Lead seizures (>5h inter-seizure interval): 342 events
%DIF < DIF < DIF < DIFDELCMD < - All seizures (types 1 and 2): 1,044 events  
%DIF < DIF < DIF < DIFDELCMD < - Interictal control segments: 1,044 matched segments
%DIF < DIF < DIF < DIFDELCMD < }
%DIF < DIF < DIF < DIFDELCMD < %%%
%DIF < DIF < DIFDELCMD < \DIFdelend \DIFaddbegin \DIFadd{The NeuroVista dataset provided extensive long-term iEEG recordings from 9 patients with drug-resistant focal epilepsy. After excluding the initial 100-day post-implantation period, we analyzed a total of XX patient-years of continuous recording (mean: XX months per patient, range: 6-14 months). Across all patients, we identified and analyzed:
%DIF < DIF < DIFDELCMD < - Lead seizures (>5h inter-seizure interval): 342 events
%DIF < DIF < DIFDELCMD < - All seizures (types 1 and 2): 1,044 events  
%DIF < DIF < DIFDELCMD < - Interictal control segments: 1,044 matched segments
%DIF < DIF < DIFDELCMD < }\DIFaddend 
%DIF < DIF < DIFDELCMD < %%%
%DIF < DIFDELCMD < \DIFdelend \DIFaddbegin \DIFadd{The NeuroVista dataset provided extensive long-term iEEG recordings from 9 patients with drug-resistant focal epilepsy. After excluding the initial 100-day post-implantation period, we analyzed a total of XX patient-years of continuous recording (mean: XX months per patient, range: 6-14 months). Across all patients, we identified and analyzed:
%DIF < DIFDELCMD < - Lead seizures (>5h inter-seizure interval): 342 events
%DIF < DIFDELCMD < - All seizures (types 1 and 2): 1,044 events  
%DIF < DIFDELCMD < - Interictal control segments: 1,044 matched segments
%DIF < DIFDELCMD < }\DIFaddend 
%DIF < DIFDELCMD < %%%
%DIFDELCMD < \DIFdelend \DIFaddbegin \DIFadd{The NeuroVista dataset provided extensive long-term iEEG recordings from 9 patients with drug-resistant focal epilepsy. After excluding the initial 100-day post-implantation period, we analyzed a total of XX patient-years of continuous recording (mean: XX months per patient, range: 6-14 months). Across all patients, we identified and analyzed:
%DIFDELCMD < - Lead seizures (>5h inter-seizure interval): 342 events
%DIFDELCMD < - All seizures (types 1 and 2): 1,044 events  
%DIFDELCMD < - Interictal control segments: 1,044 matched segments
%DIFDELCMD < }\DIFaddend 
%DIFDELCMD < %%%
\DIFdelend \DIFaddbegin \DIFadd{The NeuroVista dataset provided extensive long-term iEEG recordings from 9 patients with drug-resistant focal epilepsy. After excluding the initial 100-day post-implantation period, we analyzed a total of XX patient-years of continuous recording (mean: XX months per patient, range: 6-14 months). Across all patients, we identified and analyzed:
- Lead seizures (>5h inter-seizure interval): 342 events
- All seizures (types 1 and 2): 1,044 events  
- Interictal control segments: 1,044 matched segments
}\DIFaddend 

\DIFdelbegin %DIFDELCMD < \DIFdelbegin %%%
%DIF < DIFDELCMD < \DIFdelbegin %%%
%DIF < DIF < DIFDELCMD < \DIFdelbegin %%%
%DIF < DIF < DIF < DIFDELCMD < \DIFadd{The 16-channel electrode arrays provided comprehensive spatial coverage of the seizure onset zones, with an average of XX ± XX channels showing significant PAC modulation during seizure events.
%DIF < DIF < DIF < DIFDELCMD < }
%DIF < DIF < DIF < DIFDELCMD < %%%
%DIF < DIF < DIFDELCMD < \DIFdelend \DIFaddbegin \DIFadd{The 16-channel electrode arrays provided comprehensive spatial coverage of the seizure onset zones, with an average of XX ± XX channels showing significant PAC modulation during seizure events.
%DIF < DIF < DIFDELCMD < }\DIFaddend 
%DIF < DIF < DIFDELCMD < %%%
%DIF < DIFDELCMD < \DIFdelend \DIFaddbegin \DIFadd{The 16-channel electrode arrays provided comprehensive spatial coverage of the seizure onset zones, with an average of XX ± XX channels showing significant PAC modulation during seizure events.
%DIF < DIFDELCMD < }\DIFaddend 
%DIF < DIFDELCMD < %%%
%DIFDELCMD < \DIFdelend \DIFaddbegin \DIFadd{The 16-channel electrode arrays provided comprehensive spatial coverage of the seizure onset zones, with an average of XX ± XX channels showing significant PAC modulation during seizure events.
%DIFDELCMD < }\DIFaddend 
%DIFDELCMD < %%%
\DIFdelend \DIFaddbegin \DIFadd{The 16-channel electrode arrays provided comprehensive spatial coverage of the seizure onset zones, with an average of XX ± XX channels showing significant PAC modulation during seizure events.
}\DIFaddend 

\subsection{\DIFdelbegin %DIFDELCMD < \DIFdelbegin %%%
%DIF < DIFDELCMD < \DIFdelbegin %%%
%DIF < DIF < DIFDELCMD < \DIFdelbegin %%%
%DIF < DIF < DIF < DIFDELCMD < \DIFadd{Phase-Amplitude Coupling Patterns}%%%
%DIF < DIF < DIFDELCMD < \DIFdelend \DIFaddbegin \DIFadd{Phase-Amplitude Coupling Patterns}\DIFaddend %%%
%DIF < DIFDELCMD < \DIFdelend \DIFaddbegin \DIFadd{Phase-Amplitude Coupling Patterns}\DIFaddend %%%
%DIFDELCMD < \DIFdelend \DIFaddbegin \DIFadd{Phase-Amplitude Coupling Patterns}\DIFaddend %%%
\DIFdelend \DIFaddbegin \DIFadd{Phase-Amplitude Coupling Patterns}\DIFaddend }

\subsubsection{\DIFdelbegin %DIFDELCMD < \DIFdelbegin %%%
%DIF < DIFDELCMD < \DIFdelbegin %%%
%DIF < DIF < DIFDELCMD < \DIFdelbegin %%%
%DIF < DIF < DIF < DIFDELCMD < \DIFadd{Baseline PAC Characteristics}%%%
%DIF < DIF < DIFDELCMD < \DIFdelend \DIFaddbegin \DIFadd{Baseline PAC Characteristics}\DIFaddend %%%
%DIF < DIFDELCMD < \DIFdelend \DIFaddbegin \DIFadd{Baseline PAC Characteristics}\DIFaddend %%%
%DIFDELCMD < \DIFdelend \DIFaddbegin \DIFadd{Baseline PAC Characteristics}\DIFaddend %%%
\DIFdelend \DIFaddbegin \DIFadd{Baseline PAC Characteristics}\DIFaddend }
\DIFdelbegin %DIFDELCMD < \DIFdelbegin %%%
%DIF < DIFDELCMD < \DIFdelbegin %%%
%DIF < DIF < DIFDELCMD < \DIFdelbegin %%%
%DIF < DIF < DIF < DIFDELCMD < \DIFadd{During interictal periods (>4 hours from any seizure), we observed stable baseline PAC patterns with the following characteristics:
%DIF < DIF < DIF < DIFDELCMD < - Dominant coupling between theta/alpha phase (4-13 Hz) and gamma amplitude (30-100 Hz)
%DIF < DIF < DIF < DIFDELCMD < - Mean modulation index (MI): 0.XX ± 0.XX (z-score: 0 ± 1 by design)
%DIF < DIF < DIF < DIFDELCMD < - Spatial heterogeneity: XX\% of channels showed significant PAC (z > 2)
%DIF < DIF < DIF < DIFDELCMD < - Temporal stability: coefficient of variation < XX\% over 1-hour windows
%DIF < DIF < DIF < DIFDELCMD < }
%DIF < DIF < DIF < DIFDELCMD < %%%
%DIF < DIF < DIFDELCMD < \DIFdelend \DIFaddbegin \DIFadd{During interictal periods (>4 hours from any seizure), we observed stable baseline PAC patterns with the following characteristics:
%DIF < DIF < DIFDELCMD < - Dominant coupling between theta/alpha phase (4-13 Hz) and gamma amplitude (30-100 Hz)
%DIF < DIF < DIFDELCMD < - Mean modulation index (MI): 0.XX ± 0.XX (z-score: 0 ± 1 by design)
%DIF < DIF < DIFDELCMD < - Spatial heterogeneity: XX\% of channels showed significant PAC (z > 2)
%DIF < DIF < DIFDELCMD < - Temporal stability: coefficient of variation < XX\% over 1-hour windows
%DIF < DIF < DIFDELCMD < }\DIFaddend 
%DIF < DIF < DIFDELCMD < %%%
%DIF < DIFDELCMD < \DIFdelend \DIFaddbegin \DIFadd{During interictal periods (>4 hours from any seizure), we observed stable baseline PAC patterns with the following characteristics:
%DIF < DIFDELCMD < - Dominant coupling between theta/alpha phase (4-13 Hz) and gamma amplitude (30-100 Hz)
%DIF < DIFDELCMD < - Mean modulation index (MI): 0.XX ± 0.XX (z-score: 0 ± 1 by design)
%DIF < DIFDELCMD < - Spatial heterogeneity: XX\% of channels showed significant PAC (z > 2)
%DIF < DIFDELCMD < - Temporal stability: coefficient of variation < XX\% over 1-hour windows
%DIF < DIFDELCMD < }\DIFaddend 
%DIF < DIFDELCMD < %%%
%DIFDELCMD < \DIFdelend \DIFaddbegin \DIFadd{During interictal periods (>4 hours from any seizure), we observed stable baseline PAC patterns with the following characteristics:
%DIFDELCMD < - Dominant coupling between theta/alpha phase (4-13 Hz) and gamma amplitude (30-100 Hz)
%DIFDELCMD < - Mean modulation index (MI): 0.XX ± 0.XX (z-score: 0 ± 1 by design)
%DIFDELCMD < - Spatial heterogeneity: XX\% of channels showed significant PAC (z > 2)
%DIFDELCMD < - Temporal stability: coefficient of variation < XX\% over 1-hour windows
%DIFDELCMD < }\DIFaddend 
%DIFDELCMD < %%%
\DIFdelend \DIFaddbegin \DIFadd{During interictal periods (>4 hours from any seizure), we observed stable baseline PAC patterns with the following characteristics:
- Dominant coupling between theta/alpha phase (4-13 Hz) and gamma amplitude (30-100 Hz)
- Mean modulation index (MI): 0.XX ± 0.XX (z-score: 0 ± 1 by design)
- Spatial heterogeneity: XX\% of channels showed significant PAC (z > 2)
- Temporal stability: coefficient of variation < XX\% over 1-hour windows
}\DIFaddend 

\subsubsection{\DIFdelbegin %DIFDELCMD < \DIFdelbegin %%%
%DIF < DIFDELCMD < \DIFdelbegin %%%
%DIF < DIF < DIFDELCMD < \DIFdelbegin %%%
%DIF < DIF < DIF < DIFDELCMD < \DIFadd{Pre-ictal PAC Evolution}%%%
%DIF < DIF < DIFDELCMD < \DIFdelend \DIFaddbegin \DIFadd{Pre-ictal PAC Evolution}\DIFaddend %%%
%DIF < DIFDELCMD < \DIFdelend \DIFaddbegin \DIFadd{Pre-ictal PAC Evolution}\DIFaddend %%%
%DIFDELCMD < \DIFdelend \DIFaddbegin \DIFadd{Pre-ictal PAC Evolution}\DIFaddend %%%
\DIFdelend \DIFaddbegin \DIFadd{Pre-ictal PAC Evolution}\DIFaddend }
\DIFdelbegin %DIFDELCMD < \DIFdelbegin %%%
%DIF < DIFDELCMD < \DIFdelbegin %%%
%DIF < DIF < DIFDELCMD < \DIFdelbegin %%%
%DIF < DIF < DIF < DIFDELCMD < \DIFadd{Analysis of the 60-minute pre-seizure period revealed progressive changes in PAC:
%DIF < DIF < DIF < DIFDELCMD < - Significant increase in theta-gamma coupling starting XX ± XX minutes before seizure onset (p < 0.001, Wilcoxon signed-rank test)
%DIF < DIF < DIF < DIFDELCMD < - Peak PAC z-scores: XX ± XX in the final 5 minutes before onset
%DIF < DIF < DIF < DIFDELCMD < - Spatial spread: PAC elevation initially focal (XX ± XX channels) expanding to XX ± XX channels at onset
%DIF < DIF < DIF < DIFDELCMD < - Frequency shift: Progressive increase in the optimal phase frequency from XX Hz to XX Hz approaching seizure
%DIF < DIF < DIF < DIFDELCMD < }
%DIF < DIF < DIF < DIFDELCMD < %%%
%DIF < DIF < DIFDELCMD < \DIFdelend \DIFaddbegin \DIFadd{Analysis of the 60-minute pre-seizure period revealed progressive changes in PAC:
%DIF < DIF < DIFDELCMD < - Significant increase in theta-gamma coupling starting XX ± XX minutes before seizure onset (p < 0.001, Wilcoxon signed-rank test)
%DIF < DIF < DIFDELCMD < - Peak PAC z-scores: XX ± XX in the final 5 minutes before onset
%DIF < DIF < DIFDELCMD < - Spatial spread: PAC elevation initially focal (XX ± XX channels) expanding to XX ± XX channels at onset
%DIF < DIF < DIFDELCMD < - Frequency shift: Progressive increase in the optimal phase frequency from XX Hz to XX Hz approaching seizure
%DIF < DIF < DIFDELCMD < }\DIFaddend 
%DIF < DIF < DIFDELCMD < %%%
%DIF < DIFDELCMD < \DIFdelend \DIFaddbegin \DIFadd{Analysis of the 60-minute pre-seizure period revealed progressive changes in PAC:
%DIF < DIFDELCMD < - Significant increase in theta-gamma coupling starting XX ± XX minutes before seizure onset (p < 0.001, Wilcoxon signed-rank test)
%DIF < DIFDELCMD < - Peak PAC z-scores: XX ± XX in the final 5 minutes before onset
%DIF < DIFDELCMD < - Spatial spread: PAC elevation initially focal (XX ± XX channels) expanding to XX ± XX channels at onset
%DIF < DIFDELCMD < - Frequency shift: Progressive increase in the optimal phase frequency from XX Hz to XX Hz approaching seizure
%DIF < DIFDELCMD < }\DIFaddend 
%DIF < DIFDELCMD < %%%
%DIFDELCMD < \DIFdelend \DIFaddbegin \DIFadd{Analysis of the 60-minute pre-seizure period revealed progressive changes in PAC:
%DIFDELCMD < - Significant increase in theta-gamma coupling starting XX ± XX minutes before seizure onset (p < 0.001, Wilcoxon signed-rank test)
%DIFDELCMD < - Peak PAC z-scores: XX ± XX in the final 5 minutes before onset
%DIFDELCMD < - Spatial spread: PAC elevation initially focal (XX ± XX channels) expanding to XX ± XX channels at onset
%DIFDELCMD < - Frequency shift: Progressive increase in the optimal phase frequency from XX Hz to XX Hz approaching seizure
%DIFDELCMD < }\DIFaddend 
%DIFDELCMD < %%%
\DIFdelend \DIFaddbegin \DIFadd{Analysis of the 60-minute pre-seizure period revealed progressive changes in PAC:
- Significant increase in theta-gamma coupling starting XX ± XX minutes before seizure onset (p < 0.001, Wilcoxon signed-rank test)
- Peak PAC z-scores: XX ± XX in the final 5 minutes before onset
- Spatial spread: PAC elevation initially focal (XX ± XX channels) expanding to XX ± XX channels at onset
- Frequency shift: Progressive increase in the optimal phase frequency from XX Hz to XX Hz approaching seizure
}\DIFaddend 

\subsubsection{\DIFdelbegin %DIFDELCMD < \DIFdelbegin %%%
%DIF < DIFDELCMD < \DIFdelbegin %%%
%DIF < DIF < DIFDELCMD < \DIFdelbegin %%%
%DIF < DIF < DIF < DIFDELCMD < \DIFadd{Ictal PAC Dynamics}%%%
%DIF < DIF < DIFDELCMD < \DIFdelend \DIFaddbegin \DIFadd{Ictal PAC Dynamics}\DIFaddend %%%
%DIF < DIFDELCMD < \DIFdelend \DIFaddbegin \DIFadd{Ictal PAC Dynamics}\DIFaddend %%%
%DIFDELCMD < \DIFdelend \DIFaddbegin \DIFadd{Ictal PAC Dynamics}\DIFaddend %%%
\DIFdelend \DIFaddbegin \DIFadd{Ictal PAC Dynamics}\DIFaddend }
\DIFdelbegin %DIFDELCMD < \DIFdelbegin %%%
%DIF < DIFDELCMD < \DIFdelbegin %%%
%DIF < DIF < DIFDELCMD < \DIFdelbegin %%%
%DIF < DIF < DIF < DIFDELCMD < \DIFadd{During seizures, PAC showed dramatic alterations:
%DIF < DIF < DIF < DIFDELCMD < - Maximum coupling strength: z-score = XX ± XX (p < 0.001 vs baseline)
%DIF < DIF < DIF < DIFDELCMD < - Dominant frequency pairs: Phase XX-XX Hz coupled with amplitude XX-XX Hz
%DIF < DIF < DIF < DIFDELCMD < - Spatial synchronization: XX\% increase in inter-channel PAC correlation
%DIF < DIF < DIF < DIFDELCMD < - Temporal evolution: Initial spike within XX seconds of electrical onset, followed by sustained elevation
%DIF < DIF < DIF < DIFDELCMD < }
%DIF < DIF < DIF < DIFDELCMD < %%%
%DIF < DIF < DIFDELCMD < \DIFdelend \DIFaddbegin \DIFadd{During seizures, PAC showed dramatic alterations:
%DIF < DIF < DIFDELCMD < - Maximum coupling strength: z-score = XX ± XX (p < 0.001 vs baseline)
%DIF < DIF < DIFDELCMD < - Dominant frequency pairs: Phase XX-XX Hz coupled with amplitude XX-XX Hz
%DIF < DIF < DIFDELCMD < - Spatial synchronization: XX\% increase in inter-channel PAC correlation
%DIF < DIF < DIFDELCMD < - Temporal evolution: Initial spike within XX seconds of electrical onset, followed by sustained elevation
%DIF < DIF < DIFDELCMD < }\DIFaddend 
%DIF < DIF < DIFDELCMD < %%%
%DIF < DIFDELCMD < \DIFdelend \DIFaddbegin \DIFadd{During seizures, PAC showed dramatic alterations:
%DIF < DIFDELCMD < - Maximum coupling strength: z-score = XX ± XX (p < 0.001 vs baseline)
%DIF < DIFDELCMD < - Dominant frequency pairs: Phase XX-XX Hz coupled with amplitude XX-XX Hz
%DIF < DIFDELCMD < - Spatial synchronization: XX\% increase in inter-channel PAC correlation
%DIF < DIFDELCMD < - Temporal evolution: Initial spike within XX seconds of electrical onset, followed by sustained elevation
%DIF < DIFDELCMD < }\DIFaddend 
%DIF < DIFDELCMD < %%%
%DIFDELCMD < \DIFdelend \DIFaddbegin \DIFadd{During seizures, PAC showed dramatic alterations:
%DIFDELCMD < - Maximum coupling strength: z-score = XX ± XX (p < 0.001 vs baseline)
%DIFDELCMD < - Dominant frequency pairs: Phase XX-XX Hz coupled with amplitude XX-XX Hz
%DIFDELCMD < - Spatial synchronization: XX\% increase in inter-channel PAC correlation
%DIFDELCMD < - Temporal evolution: Initial spike within XX seconds of electrical onset, followed by sustained elevation
%DIFDELCMD < }\DIFaddend 
%DIFDELCMD < %%%
\DIFdelend \DIFaddbegin \DIFadd{During seizures, PAC showed dramatic alterations:
- Maximum coupling strength: z-score = XX ± XX (p < 0.001 vs baseline)
- Dominant frequency pairs: Phase XX-XX Hz coupled with amplitude XX-XX Hz
- Spatial synchronization: XX\% increase in inter-channel PAC correlation
- Temporal evolution: Initial spike within XX seconds of electrical onset, followed by sustained elevation
}\DIFaddend 

\subsubsection{\DIFdelbegin %DIFDELCMD < \DIFdelbegin %%%
%DIF < DIFDELCMD < \DIFdelbegin %%%
%DIF < DIF < DIFDELCMD < \DIFdelbegin %%%
%DIF < DIF < DIF < DIFDELCMD < \DIFadd{Post-ictal Recovery}%%%
%DIF < DIF < DIFDELCMD < \DIFdelend \DIFaddbegin \DIFadd{Post-ictal Recovery}\DIFaddend %%%
%DIF < DIFDELCMD < \DIFdelend \DIFaddbegin \DIFadd{Post-ictal Recovery}\DIFaddend %%%
%DIFDELCMD < \DIFdelend \DIFaddbegin \DIFadd{Post-ictal Recovery}\DIFaddend %%%
\DIFdelend \DIFaddbegin \DIFadd{Post-ictal Recovery}\DIFaddend }
\DIFdelbegin %DIFDELCMD < \DIFdelbegin %%%
%DIF < DIFDELCMD < \DIFdelbegin %%%
%DIF < DIF < DIFDELCMD < \DIFdelbegin %%%
%DIF < DIF < DIF < DIFDELCMD < \DIFadd{Following seizure termination, PAC exhibited gradual normalization:
%DIF < DIF < DIF < DIFDELCMD < - Recovery time to baseline: XX ± XX minutes
%DIF < DIF < DIF < DIFDELCMD < - Transient suppression period: XX\% of seizures showed below-baseline PAC for XX ± XX minutes
%DIF < DIF < DIF < DIFDELCMD < - Asymmetric recovery: Phase frequencies normalized faster (XX min) than amplitude frequencies (XX min)
%DIF < DIF < DIF < DIFDELCMD < }
%DIF < DIF < DIF < DIFDELCMD < %%%
%DIF < DIF < DIFDELCMD < \DIFdelend \DIFaddbegin \DIFadd{Following seizure termination, PAC exhibited gradual normalization:
%DIF < DIF < DIFDELCMD < - Recovery time to baseline: XX ± XX minutes
%DIF < DIF < DIFDELCMD < - Transient suppression period: XX\% of seizures showed below-baseline PAC for XX ± XX minutes
%DIF < DIF < DIFDELCMD < - Asymmetric recovery: Phase frequencies normalized faster (XX min) than amplitude frequencies (XX min)
%DIF < DIF < DIFDELCMD < }\DIFaddend 
%DIF < DIF < DIFDELCMD < %%%
%DIF < DIFDELCMD < \DIFdelend \DIFaddbegin \DIFadd{Following seizure termination, PAC exhibited gradual normalization:
%DIF < DIFDELCMD < - Recovery time to baseline: XX ± XX minutes
%DIF < DIFDELCMD < - Transient suppression period: XX\% of seizures showed below-baseline PAC for XX ± XX minutes
%DIF < DIFDELCMD < - Asymmetric recovery: Phase frequencies normalized faster (XX min) than amplitude frequencies (XX min)
%DIF < DIFDELCMD < }\DIFaddend 
%DIF < DIFDELCMD < %%%
%DIFDELCMD < \DIFdelend \DIFaddbegin \DIFadd{Following seizure termination, PAC exhibited gradual normalization:
%DIFDELCMD < - Recovery time to baseline: XX ± XX minutes
%DIFDELCMD < - Transient suppression period: XX\% of seizures showed below-baseline PAC for XX ± XX minutes
%DIFDELCMD < - Asymmetric recovery: Phase frequencies normalized faster (XX min) than amplitude frequencies (XX min)
%DIFDELCMD < }\DIFaddend 
%DIFDELCMD < %%%
\DIFdelend \DIFaddbegin \DIFadd{Following seizure termination, PAC exhibited gradual normalization:
- Recovery time to baseline: XX ± XX minutes
- Transient suppression period: XX\% of seizures showed below-baseline PAC for XX ± XX minutes
- Asymmetric recovery: Phase frequencies normalized faster (XX min) than amplitude frequencies (XX min)
}\DIFaddend 

\subsection{\DIFdelbegin %DIFDELCMD < \DIFdelbegin %%%
%DIF < DIFDELCMD < \DIFdelbegin %%%
%DIF < DIF < DIFDELCMD < \DIFdelbegin %%%
%DIF < DIF < DIF < DIFDELCMD < \DIFadd{Spatial Distribution and Channel Consistency}%%%
%DIF < DIF < DIFDELCMD < \DIFdelend \DIFaddbegin \DIFadd{Spatial Distribution and Channel Consistency}\DIFaddend %%%
%DIF < DIFDELCMD < \DIFdelend \DIFaddbegin \DIFadd{Spatial Distribution and Channel Consistency}\DIFaddend %%%
%DIFDELCMD < \DIFdelend \DIFaddbegin \DIFadd{Spatial Distribution and Channel Consistency}\DIFaddend %%%
\DIFdelend \DIFaddbegin \DIFadd{Spatial Distribution and Channel Consistency}\DIFaddend }

\DIFdelbegin %DIFDELCMD < \DIFdelbegin %%%
%DIF < DIFDELCMD < \DIFdelbegin %%%
%DIF < DIF < DIFDELCMD < \DIFdelbegin %%%
%DIF < DIF < DIF < DIFDELCMD < \DIFadd{Analysis of PAC spatial patterns across the 16-channel arrays revealed:
%DIF < DIF < DIF < DIFDELCMD < - Consistent "driver" channels: XX ± XX channels per patient showed reliable pre-ictal PAC elevation
%DIF < DIF < DIF < DIFDELCMD < - Spatial gradient: PAC strength decreased with distance from seizure onset zone (correlation r = -0.XX, p < 0.01)
%DIF < DIF < DIF < DIFDELCMD < - Channel stability: XX\% of high-PAC channels remained consistent across multiple seizures
%DIF < DIF < DIF < DIFDELCMD < - Network effects: Increased inter-channel coupling preceded clinical manifestations by XX ± XX seconds
%DIF < DIF < DIF < DIFDELCMD < }
%DIF < DIF < DIF < DIFDELCMD < %%%
%DIF < DIF < DIFDELCMD < \DIFdelend \DIFaddbegin \DIFadd{Analysis of PAC spatial patterns across the 16-channel arrays revealed:
%DIF < DIF < DIFDELCMD < - Consistent "driver" channels: XX ± XX channels per patient showed reliable pre-ictal PAC elevation
%DIF < DIF < DIFDELCMD < - Spatial gradient: PAC strength decreased with distance from seizure onset zone (correlation r = -0.XX, p < 0.01)
%DIF < DIF < DIFDELCMD < - Channel stability: XX\% of high-PAC channels remained consistent across multiple seizures
%DIF < DIF < DIFDELCMD < - Network effects: Increased inter-channel coupling preceded clinical manifestations by XX ± XX seconds
%DIF < DIF < DIFDELCMD < }\DIFaddend 
%DIF < DIF < DIFDELCMD < %%%
%DIF < DIFDELCMD < \DIFdelend \DIFaddbegin \DIFadd{Analysis of PAC spatial patterns across the 16-channel arrays revealed:
%DIF < DIFDELCMD < - Consistent "driver" channels: XX ± XX channels per patient showed reliable pre-ictal PAC elevation
%DIF < DIFDELCMD < - Spatial gradient: PAC strength decreased with distance from seizure onset zone (correlation r = -0.XX, p < 0.01)
%DIF < DIFDELCMD < - Channel stability: XX\% of high-PAC channels remained consistent across multiple seizures
%DIF < DIFDELCMD < - Network effects: Increased inter-channel coupling preceded clinical manifestations by XX ± XX seconds
%DIF < DIFDELCMD < }\DIFaddend 
%DIF < DIFDELCMD < %%%
%DIFDELCMD < \DIFdelend \DIFaddbegin \DIFadd{Analysis of PAC spatial patterns across the 16-channel arrays revealed:
%DIFDELCMD < - Consistent "driver" channels: XX ± XX channels per patient showed reliable pre-ictal PAC elevation
%DIFDELCMD < - Spatial gradient: PAC strength decreased with distance from seizure onset zone (correlation r = -0.XX, p < 0.01)
%DIFDELCMD < - Channel stability: XX\% of high-PAC channels remained consistent across multiple seizures
%DIFDELCMD < - Network effects: Increased inter-channel coupling preceded clinical manifestations by XX ± XX seconds
%DIFDELCMD < }\DIFaddend 
%DIFDELCMD < %%%
\DIFdelend \DIFaddbegin \DIFadd{Analysis of PAC spatial patterns across the 16-channel arrays revealed:
- Consistent "driver" channels: XX ± XX channels per patient showed reliable pre-ictal PAC elevation
- Spatial gradient: PAC strength decreased with distance from seizure onset zone (correlation r = -0.XX, p < 0.01)
- Channel stability: XX\% of high-PAC channels remained consistent across multiple seizures
- Network effects: Increased inter-channel coupling preceded clinical manifestations by XX ± XX seconds
}\DIFaddend 

\subsection{\DIFdelbegin %DIFDELCMD < \DIFdelbegin %%%
%DIF < DIFDELCMD < \DIFdelbegin %%%
%DIF < DIF < DIFDELCMD < \DIFdelbegin %%%
%DIF < DIF < DIF < DIFDELCMD < \DIFadd{Seizure Prediction Performance}%%%
%DIF < DIF < DIFDELCMD < \DIFdelend \DIFaddbegin \DIFadd{Seizure Prediction Performance}\DIFaddend %%%
%DIF < DIFDELCMD < \DIFdelend \DIFaddbegin \DIFadd{Seizure Prediction Performance}\DIFaddend %%%
%DIFDELCMD < \DIFdelend \DIFaddbegin \DIFadd{Seizure Prediction Performance}\DIFaddend %%%
\DIFdelend \DIFaddbegin \DIFadd{Seizure Prediction Performance}\DIFaddend }

\subsubsection{\DIFdelbegin %DIFDELCMD < \DIFdelbegin %%%
%DIF < DIFDELCMD < \DIFdelbegin %%%
%DIF < DIF < DIFDELCMD < \DIFdelbegin %%%
%DIF < DIF < DIF < DIFDELCMD < \DIFadd{Classification Accuracy}%%%
%DIF < DIF < DIFDELCMD < \DIFdelend \DIFaddbegin \DIFadd{Classification Accuracy}\DIFaddend %%%
%DIF < DIFDELCMD < \DIFdelend \DIFaddbegin \DIFadd{Classification Accuracy}\DIFaddend %%%
%DIFDELCMD < \DIFdelend \DIFaddbegin \DIFadd{Classification Accuracy}\DIFaddend %%%
\DIFdelend \DIFaddbegin \DIFadd{Classification Accuracy}\DIFaddend }
\DIFdelbegin %DIFDELCMD < \DIFdelbegin %%%
%DIF < DIFDELCMD < \DIFdelbegin %%%
%DIF < DIF < DIFDELCMD < \DIFdelbegin %%%
%DIF < DIF < DIF < DIFDELCMD < \DIFadd{Using PAC features extracted from 10-second windows, we trained patient-specific classifiers to discriminate pre-ictal from interictal states:
%DIF < DIF < DIF < DIFDELCMD < - Mean AUC across patients: 0.XX ± 0.XX (range: 0.XX - 0.XX)
%DIF < DIF < DIF < DIFDELCMD < - Optimal prediction horizon: XX minutes (sensitivity = XX\%, specificity = XX\%)
%DIF < DIF < DIF < DIFDELCMD < - Feature importance: Theta-gamma PAC contributed XX\% of predictive power
%DIF < DIF < DIF < DIFDELCMD < }
%DIF < DIF < DIF < DIFDELCMD < %%%
%DIF < DIF < DIFDELCMD < \DIFdelend \DIFaddbegin \DIFadd{Using PAC features extracted from 10-second windows, we trained patient-specific classifiers to discriminate pre-ictal from interictal states:
%DIF < DIF < DIFDELCMD < - Mean AUC across patients: 0.XX ± 0.XX (range: 0.XX - 0.XX)
%DIF < DIF < DIFDELCMD < - Optimal prediction horizon: XX minutes (sensitivity = XX\%, specificity = XX\%)
%DIF < DIF < DIFDELCMD < - Feature importance: Theta-gamma PAC contributed XX\% of predictive power
%DIF < DIF < DIFDELCMD < }\DIFaddend 
%DIF < DIF < DIFDELCMD < %%%
%DIF < DIFDELCMD < \DIFdelend \DIFaddbegin \DIFadd{Using PAC features extracted from 10-second windows, we trained patient-specific classifiers to discriminate pre-ictal from interictal states:
%DIF < DIFDELCMD < - Mean AUC across patients: 0.XX ± 0.XX (range: 0.XX - 0.XX)
%DIF < DIFDELCMD < - Optimal prediction horizon: XX minutes (sensitivity = XX\%, specificity = XX\%)
%DIF < DIFDELCMD < - Feature importance: Theta-gamma PAC contributed XX\% of predictive power
%DIF < DIFDELCMD < }\DIFaddend 
%DIF < DIFDELCMD < %%%
%DIFDELCMD < \DIFdelend \DIFaddbegin \DIFadd{Using PAC features extracted from 10-second windows, we trained patient-specific classifiers to discriminate pre-ictal from interictal states:
%DIFDELCMD < - Mean AUC across patients: 0.XX ± 0.XX (range: 0.XX - 0.XX)
%DIFDELCMD < - Optimal prediction horizon: XX minutes (sensitivity = XX\%, specificity = XX\%)
%DIFDELCMD < - Feature importance: Theta-gamma PAC contributed XX\% of predictive power
%DIFDELCMD < }\DIFaddend 
%DIFDELCMD < %%%
\DIFdelend \DIFaddbegin \DIFadd{Using PAC features extracted from 10-second windows, we trained patient-specific classifiers to discriminate pre-ictal from interictal states:
- Mean AUC across patients: 0.XX ± 0.XX (range: 0.XX - 0.XX)
- Optimal prediction horizon: XX minutes (sensitivity = XX\%, specificity = XX\%)
- Feature importance: Theta-gamma PAC contributed XX\% of predictive power
}\DIFaddend 

\subsubsection{\DIFdelbegin %DIFDELCMD < \DIFdelbegin %%%
%DIF < DIFDELCMD < \DIFdelbegin %%%
%DIF < DIF < DIFDELCMD < \DIFdelbegin %%%
%DIF < DIF < DIF < DIFDELCMD < \DIFadd{Pseudo-prospective Evaluation}%%%
%DIF < DIF < DIFDELCMD < \DIFdelend \DIFaddbegin \DIFadd{Pseudo-prospective Evaluation}\DIFaddend %%%
%DIF < DIFDELCMD < \DIFdelend \DIFaddbegin \DIFadd{Pseudo-prospective Evaluation}\DIFaddend %%%
%DIFDELCMD < \DIFdelend \DIFaddbegin \DIFadd{Pseudo-prospective Evaluation}\DIFaddend %%%
\DIFdelend \DIFaddbegin \DIFadd{Pseudo-prospective Evaluation}\DIFaddend }
\DIFdelbegin %DIFDELCMD < \DIFdelbegin %%%
%DIF < DIFDELCMD < \DIFdelbegin %%%
%DIF < DIF < DIFDELCMD < \DIFdelbegin %%%
%DIF < DIF < DIF < DIFDELCMD < \DIFadd{Testing on held-out data (post day 200) demonstrated:
%DIF < DIF < DIF < DIFDELCMD < - Sensitivity for lead seizures: XX\% (XX/XX seizures detected)
%DIF < DIF < DIF < DIFDELCMD < - False positive rate: XX per day
%DIF < DIF < DIF < DIFDELCMD < - Time in warning: XX\% ± XX\%
%DIF < DIF < DIF < DIFDELCMD < - Improvement over chance: XX\% (p < 0.001, permutation test)
%DIF < DIF < DIF < DIFDELCMD < }
%DIF < DIF < DIF < DIFDELCMD < %%%
%DIF < DIF < DIFDELCMD < \DIFdelend \DIFaddbegin \DIFadd{Testing on held-out data (post day 200) demonstrated:
%DIF < DIF < DIFDELCMD < - Sensitivity for lead seizures: XX\% (XX/XX seizures detected)
%DIF < DIF < DIFDELCMD < - False positive rate: XX per day
%DIF < DIF < DIFDELCMD < - Time in warning: XX\% ± XX\%
%DIF < DIF < DIFDELCMD < - Improvement over chance: XX\% (p < 0.001, permutation test)
%DIF < DIF < DIFDELCMD < }\DIFaddend 
%DIF < DIF < DIFDELCMD < %%%
%DIF < DIFDELCMD < \DIFdelend \DIFaddbegin \DIFadd{Testing on held-out data (post day 200) demonstrated:
%DIF < DIFDELCMD < - Sensitivity for lead seizures: XX\% (XX/XX seizures detected)
%DIF < DIFDELCMD < - False positive rate: XX per day
%DIF < DIFDELCMD < - Time in warning: XX\% ± XX\%
%DIF < DIFDELCMD < - Improvement over chance: XX\% (p < 0.001, permutation test)
%DIF < DIFDELCMD < }\DIFaddend 
%DIF < DIFDELCMD < %%%
%DIFDELCMD < \DIFdelend \DIFaddbegin \DIFadd{Testing on held-out data (post day 200) demonstrated:
%DIFDELCMD < - Sensitivity for lead seizures: XX\% (XX/XX seizures detected)
%DIFDELCMD < - False positive rate: XX per day
%DIFDELCMD < - Time in warning: XX\% ± XX\%
%DIFDELCMD < - Improvement over chance: XX\% (p < 0.001, permutation test)
%DIFDELCMD < }\DIFaddend 
%DIFDELCMD < %%%
\DIFdelend \DIFaddbegin \DIFadd{Testing on held-out data (post day 200) demonstrated:
- Sensitivity for lead seizures: XX\% (XX/XX seizures detected)
- False positive rate: XX per day
- Time in warning: XX\% ± XX\%
- Improvement over chance: XX\% (p < 0.001, permutation test)
}\DIFaddend 

\subsubsection{\DIFdelbegin %DIFDELCMD < \DIFdelbegin %%%
%DIF < DIFDELCMD < \DIFdelbegin %%%
%DIF < DIF < DIFDELCMD < \DIFdelbegin %%%
%DIF < DIF < DIF < DIFDELCMD < \DIFadd{Patient-specific Variability}%%%
%DIF < DIF < DIFDELCMD < \DIFdelend \DIFaddbegin \DIFadd{Patient-specific Variability}\DIFaddend %%%
%DIF < DIFDELCMD < \DIFdelend \DIFaddbegin \DIFadd{Patient-specific Variability}\DIFaddend %%%
%DIFDELCMD < \DIFdelend \DIFaddbegin \DIFadd{Patient-specific Variability}\DIFaddend %%%
\DIFdelend \DIFaddbegin \DIFadd{Patient-specific Variability}\DIFaddend }
\DIFdelbegin %DIFDELCMD < \DIFdelbegin %%%
%DIF < DIFDELCMD < \DIFdelbegin %%%
%DIF < DIF < DIFDELCMD < \DIFdelbegin %%%
%DIF < DIF < DIF < DIFDELCMD < \DIFadd{Performance varied across patients, correlating with:
%DIF < DIF < DIF < DIFDELCMD < - Seizure frequency (r = 0.XX, p < 0.05)
%DIF < DIF < DIF < DIFDELCMD < - Electrode coverage of onset zone (r = 0.XX, p < 0.05)
%DIF < DIF < DIF < DIFDELCMD < - Baseline PAC stability (r = -0.XX, p < 0.05)
%DIF < DIF < DIF < DIFDELCMD < }
%DIF < DIF < DIF < DIFDELCMD < %%%
%DIF < DIF < DIFDELCMD < \DIFdelend \DIFaddbegin \DIFadd{Performance varied across patients, correlating with:
%DIF < DIF < DIFDELCMD < - Seizure frequency (r = 0.XX, p < 0.05)
%DIF < DIF < DIFDELCMD < - Electrode coverage of onset zone (r = 0.XX, p < 0.05)
%DIF < DIF < DIFDELCMD < - Baseline PAC stability (r = -0.XX, p < 0.05)
%DIF < DIF < DIFDELCMD < }\DIFaddend 
%DIF < DIF < DIFDELCMD < %%%
%DIF < DIFDELCMD < \DIFdelend \DIFaddbegin \DIFadd{Performance varied across patients, correlating with:
%DIF < DIFDELCMD < - Seizure frequency (r = 0.XX, p < 0.05)
%DIF < DIFDELCMD < - Electrode coverage of onset zone (r = 0.XX, p < 0.05)
%DIF < DIFDELCMD < - Baseline PAC stability (r = -0.XX, p < 0.05)
%DIF < DIFDELCMD < }\DIFaddend 
%DIF < DIFDELCMD < %%%
%DIFDELCMD < \DIFdelend \DIFaddbegin \DIFadd{Performance varied across patients, correlating with:
%DIFDELCMD < - Seizure frequency (r = 0.XX, p < 0.05)
%DIFDELCMD < - Electrode coverage of onset zone (r = 0.XX, p < 0.05)
%DIFDELCMD < - Baseline PAC stability (r = -0.XX, p < 0.05)
%DIFDELCMD < }\DIFaddend 
%DIFDELCMD < %%%
\DIFdelend \DIFaddbegin \DIFadd{Performance varied across patients, correlating with:
- Seizure frequency (r = 0.XX, p < 0.05)
- Electrode coverage of onset zone (r = 0.XX, p < 0.05)
- Baseline PAC stability (r = -0.XX, p < 0.05)
}\DIFaddend 

\DIFdelbegin %DIFDELCMD < \DIFdelbegin %%%
%DIF < DIFDELCMD < \DIFdelbegin %%%
%DIF < DIF < DIFDELCMD < \DIFdelbegin %%%
%DIF < DIF < DIF < DIFDELCMD < \DIFadd{Best performing patients (n=3) achieved AUC > 0.XX with sensitivity > XX\% at <XX\% time in warning.
%DIF < DIF < DIF < DIFDELCMD < }
%DIF < DIF < DIF < DIFDELCMD < %%%
%DIF < DIF < DIFDELCMD < \DIFdelend \DIFaddbegin \DIFadd{Best performing patients (n=3) achieved AUC > 0.XX with sensitivity > XX\% at <XX\% time in warning.
%DIF < DIF < DIFDELCMD < }\DIFaddend 
%DIF < DIF < DIFDELCMD < %%%
%DIF < DIFDELCMD < \DIFdelend \DIFaddbegin \DIFadd{Best performing patients (n=3) achieved AUC > 0.XX with sensitivity > XX\% at <XX\% time in warning.
%DIF < DIFDELCMD < }\DIFaddend 
%DIF < DIFDELCMD < %%%
%DIFDELCMD < \DIFdelend \DIFaddbegin \DIFadd{Best performing patients (n=3) achieved AUC > 0.XX with sensitivity > XX\% at <XX\% time in warning.
%DIFDELCMD < }\DIFaddend 
%DIFDELCMD < %%%
\DIFdelend \DIFaddbegin \DIFadd{Best performing patients (n=3) achieved AUC > 0.XX with sensitivity > XX\% at <XX\% time in warning.
}\DIFaddend 

\subsection{\DIFdelbegin %DIFDELCMD < \DIFdelbegin %%%
%DIF < DIFDELCMD < \DIFdelbegin %%%
%DIF < DIF < DIFDELCMD < \DIFdelbegin %%%
%DIF < DIF < DIF < DIFDELCMD < \DIFadd{Temporal Patterns and Circadian Rhythms}%%%
%DIF < DIF < DIFDELCMD < \DIFdelend \DIFaddbegin \DIFadd{Temporal Patterns and Circadian Rhythms}\DIFaddend %%%
%DIF < DIFDELCMD < \DIFdelend \DIFaddbegin \DIFadd{Temporal Patterns and Circadian Rhythms}\DIFaddend %%%
%DIFDELCMD < \DIFdelend \DIFaddbegin \DIFadd{Temporal Patterns and Circadian Rhythms}\DIFaddend %%%
\DIFdelend \DIFaddbegin \DIFadd{Temporal Patterns and Circadian Rhythms}\DIFaddend }

\DIFdelbegin %DIFDELCMD < \DIFdelbegin %%%
%DIF < DIFDELCMD < \DIFdelbegin %%%
%DIF < DIF < DIFDELCMD < \DIFdelbegin %%%
%DIF < DIF < DIF < DIFDELCMD < \DIFadd{Long-term PAC analysis revealed multi-scale temporal patterns:
%DIF < DIF < DIF < DIFDELCMD < - Circadian modulation: Peak PAC at XX:XX hours (XX\% increase vs nadir)
%DIF < DIF < DIF < DIFDELCMD < - Weekly patterns: XX\% of patients showed 7-day periodicity
%DIF < DIF < DIF < DIFDELCMD < - Relationship to seizure timing: XX\% of seizures occurred within 2 hours of daily PAC maximum
%DIF < DIF < DIF < DIFDELCMD < - Pre-seizure buildup: Gradual PAC increase over XX ± XX hours before seizure clusters
%DIF < DIF < DIF < DIFDELCMD < }
%DIF < DIF < DIF < DIFDELCMD < %%%
%DIF < DIF < DIFDELCMD < \DIFdelend \DIFaddbegin \DIFadd{Long-term PAC analysis revealed multi-scale temporal patterns:
%DIF < DIF < DIFDELCMD < - Circadian modulation: Peak PAC at XX:XX hours (XX\% increase vs nadir)
%DIF < DIF < DIFDELCMD < - Weekly patterns: XX\% of patients showed 7-day periodicity
%DIF < DIF < DIFDELCMD < - Relationship to seizure timing: XX\% of seizures occurred within 2 hours of daily PAC maximum
%DIF < DIF < DIFDELCMD < - Pre-seizure buildup: Gradual PAC increase over XX ± XX hours before seizure clusters
%DIF < DIF < DIFDELCMD < }\DIFaddend 
%DIF < DIF < DIFDELCMD < %%%
%DIF < DIFDELCMD < \DIFdelend \DIFaddbegin \DIFadd{Long-term PAC analysis revealed multi-scale temporal patterns:
%DIF < DIFDELCMD < - Circadian modulation: Peak PAC at XX:XX hours (XX\% increase vs nadir)
%DIF < DIFDELCMD < - Weekly patterns: XX\% of patients showed 7-day periodicity
%DIF < DIFDELCMD < - Relationship to seizure timing: XX\% of seizures occurred within 2 hours of daily PAC maximum
%DIF < DIFDELCMD < - Pre-seizure buildup: Gradual PAC increase over XX ± XX hours before seizure clusters
%DIF < DIFDELCMD < }\DIFaddend 
%DIF < DIFDELCMD < %%%
%DIFDELCMD < \DIFdelend \DIFaddbegin \DIFadd{Long-term PAC analysis revealed multi-scale temporal patterns:
%DIFDELCMD < - Circadian modulation: Peak PAC at XX:XX hours (XX\% increase vs nadir)
%DIFDELCMD < - Weekly patterns: XX\% of patients showed 7-day periodicity
%DIFDELCMD < - Relationship to seizure timing: XX\% of seizures occurred within 2 hours of daily PAC maximum
%DIFDELCMD < - Pre-seizure buildup: Gradual PAC increase over XX ± XX hours before seizure clusters
%DIFDELCMD < }\DIFaddend 
%DIFDELCMD < %%%
\DIFdelend \DIFaddbegin \DIFadd{Long-term PAC analysis revealed multi-scale temporal patterns:
- Circadian modulation: Peak PAC at XX:XX hours (XX\% increase vs nadir)
- Weekly patterns: XX\% of patients showed 7-day periodicity
- Relationship to seizure timing: XX\% of seizures occurred within 2 hours of daily PAC maximum
- Pre-seizure buildup: Gradual PAC increase over XX ± XX hours before seizure clusters
}\DIFaddend 

\subsection{\DIFdelbegin %DIFDELCMD < \DIFdelbegin %%%
%DIF < DIFDELCMD < \DIFdelbegin %%%
%DIF < DIF < DIFDELCMD < \DIFdelbegin %%%
%DIF < DIF < DIF < DIFDELCMD < \DIFadd{Comparison with Traditional Features}%%%
%DIF < DIF < DIFDELCMD < \DIFdelend \DIFaddbegin \DIFadd{Comparison with Traditional Features}\DIFaddend %%%
%DIF < DIFDELCMD < \DIFdelend \DIFaddbegin \DIFadd{Comparison with Traditional Features}\DIFaddend %%%
%DIFDELCMD < \DIFdelend \DIFaddbegin \DIFadd{Comparison with Traditional Features}\DIFaddend %%%
\DIFdelend \DIFaddbegin \DIFadd{Comparison with Traditional Features}\DIFaddend }

\DIFdelbegin %DIFDELCMD < \DIFdelbegin %%%
%DIF < DIFDELCMD < \DIFdelbegin %%%
%DIF < DIF < DIFDELCMD < \DIFdelbegin %%%
%DIF < DIF < DIF < DIFDELCMD < \DIFadd{PAC-based prediction outperformed conventional EEG features:
%DIF < DIF < DIF < DIFDELCMD < - vs. Spectral power: +XX\% AUC improvement (p < 0.01)
%DIF < DIF < DIF < DIFDELCMD < - vs. Line length: +XX\% AUC improvement (p < 0.05)
%DIF < DIF < DIF < DIFDELCMD < - vs. Multi-feature combination: Comparable performance with XX\% fewer features
%DIF < DIF < DIF < DIFDELCMD < }
%DIF < DIF < DIF < DIFDELCMD < %%%
%DIF < DIF < DIFDELCMD < \DIFdelend \DIFaddbegin \DIFadd{PAC-based prediction outperformed conventional EEG features:
%DIF < DIF < DIFDELCMD < - vs. Spectral power: +XX\% AUC improvement (p < 0.01)
%DIF < DIF < DIFDELCMD < - vs. Line length: +XX\% AUC improvement (p < 0.05)
%DIF < DIF < DIFDELCMD < - vs. Multi-feature combination: Comparable performance with XX\% fewer features
%DIF < DIF < DIFDELCMD < }\DIFaddend 
%DIF < DIF < DIFDELCMD < %%%
%DIF < DIFDELCMD < \DIFdelend \DIFaddbegin \DIFadd{PAC-based prediction outperformed conventional EEG features:
%DIF < DIFDELCMD < - vs. Spectral power: +XX\% AUC improvement (p < 0.01)
%DIF < DIFDELCMD < - vs. Line length: +XX\% AUC improvement (p < 0.05)
%DIF < DIFDELCMD < - vs. Multi-feature combination: Comparable performance with XX\% fewer features
%DIF < DIFDELCMD < }\DIFaddend 
%DIF < DIFDELCMD < %%%
%DIFDELCMD < \DIFdelend \DIFaddbegin \DIFadd{PAC-based prediction outperformed conventional EEG features:
%DIFDELCMD < - vs. Spectral power: +XX\% AUC improvement (p < 0.01)
%DIFDELCMD < - vs. Line length: +XX\% AUC improvement (p < 0.05)
%DIFDELCMD < - vs. Multi-feature combination: Comparable performance with XX\% fewer features
%DIFDELCMD < }\DIFaddend 
%DIFDELCMD < %%%
\DIFdelend \DIFaddbegin \DIFadd{PAC-based prediction outperformed conventional EEG features:
- vs. Spectral power: +XX\% AUC improvement (p < 0.01)
- vs. Line length: +XX\% AUC improvement (p < 0.05)
- vs. Multi-feature combination: Comparable performance with XX\% fewer features
}\DIFaddend 

\DIFdelbegin %DIFDELCMD < \DIFdelbegin %%%
%DIF < DIFDELCMD < \DIFdelbegin %%%
%DIF < DIF < DIFDELCMD < \DIFdelbegin %%%
%DIF < DIF < DIF < DIFDELCMD < \DIFadd{Combined models incorporating PAC with traditional features achieved marginal improvement (AUC increase: 0.XX, p = 0.XX).
%DIF < DIF < DIF < DIFDELCMD < }
%DIF < DIF < DIF < DIFDELCMD < %%%
%DIF < DIF < DIFDELCMD < \DIFdelend \DIFaddbegin \DIFadd{Combined models incorporating PAC with traditional features achieved marginal improvement (AUC increase: 0.XX, p = 0.XX).
%DIF < DIF < DIFDELCMD < }\DIFaddend 
%DIF < DIF < DIFDELCMD < %%%
%DIF < DIFDELCMD < \DIFdelend \DIFaddbegin \DIFadd{Combined models incorporating PAC with traditional features achieved marginal improvement (AUC increase: 0.XX, p = 0.XX).
%DIF < DIFDELCMD < }\DIFaddend 
%DIF < DIFDELCMD < %%%
%DIFDELCMD < \DIFdelend \DIFaddbegin \DIFadd{Combined models incorporating PAC with traditional features achieved marginal improvement (AUC increase: 0.XX, p = 0.XX).
%DIFDELCMD < }\DIFaddend 
%DIFDELCMD < %%%
\DIFdelend \DIFaddbegin \DIFadd{Combined models incorporating PAC with traditional features achieved marginal improvement (AUC increase: 0.XX, p = 0.XX).
}\DIFaddend 

\label{sec:results}
\DIFdelbegin %DIFDELCMD < \DIFdelbegin %%%
%DIF < DIFDELCMD < \DIFdelbegin %%%
%DIF < DIF < DIFDELCMD < \DIFdelbegin %%%
%DIF < DIF < DIF < DIFDELCMD < \DIFaddend %%%
%DIF < DIF < DIFDELCMD < \DIFdelend %%%
%DIF < DIFDELCMD < \DIFdelend %%%
%DIFDELCMD < \DIFdelend %%%
\DIFdelend % Figure and table placeholders with descriptions
% Figure 1: Patient demographics and seizure frequency distributions
% Figure 2: PAC pattern heterogeneity across patients and channels  
% Figure 3: Temporal stability of PAC features
% Figure 4: Pre-ictal PAC dynamics and evolution
% Figure 5: Classification performance and feature importance
% Figure 6: Computational performance metrics
% Figure 7: Cross-validation and generalization results
% Table 1: Patient demographics and recording characteristics
% Table 2: Detailed classification performance metrics
% Table 3: Computational performance benchmarks

\label{sec:results}

%%%% EOF


%% ----------------------------------------
%% DISCUSSION
%% ----------------------------------------

%% ========================================
%% ./01_manuscript/contents/discussion.tex
%% ========================================
%% -*- coding: utf-8 -*-
%DIF < DIF < DIF < DIF < DIF < % Timestamp: "2025-09-26 18:19:43 (ywatanabe)"
%DIF < DIF < DIF < DIF < DIF < % File: "/ssh:sp:/home/ywatanabe/proj/neurovista/paper/01_manuscript/src/discussion.tex"
%DIF < DIF < DIF < DIF < DIF > % Timestamp: "2025-09-27 20:22:58 (ywatanabe)"
%DIF < DIF < DIF < DIF < DIF > % File: "/ssh:sp:/home/ywatanabe/proj/neurovista/paper/01_manuscript/contents/discussion.tex"
%DIF < DIF < DIF < DIF > % Timestamp: "2025-09-27 20:22:58 (ywatanabe)"
%DIF < DIF < DIF < DIF > % File: "/ssh:sp:/home/ywatanabe/proj/neurovista/paper/01_manuscript/contents/discussion.tex"
%DIF < DIF < DIF > % Timestamp: "2025-09-27 20:22:58 (ywatanabe)"
%DIF < DIF < DIF > % File: "/ssh:sp:/home/ywatanabe/proj/neurovista/paper/01_manuscript/contents/discussion.tex"
%DIF < DIF > % Timestamp: "2025-09-27 20:22:58 (ywatanabe)"
%DIF < DIF > % File: "/ssh:sp:/home/ywatanabe/proj/neurovista/paper/01_manuscript/contents/discussion.tex"
%DIF > % Timestamp: "2025-09-27 20:22:58 (ywatanabe)"
%DIF > % File: "/ssh:sp:/home/ywatanabe/proj/neurovista/paper/01_manuscript/contents/discussion.tex"

\section{Discussion}
\DIFdelbegin %DIFDELCMD < \DIFdelbegin %%%
%DIF < DIFDELCMD < \DIFdelbegin %%%
%DIF < DIF < DIFDELCMD < \DIFdelbegin %%%
%DIF < DIF < DIF < DIFDELCMD < \DIFdelbegin \DIFdel{Discussion here.}\DIFdelend \DIFaddbegin 
%DIF < DIF < DIF < DIFDELCMD < %%%
%DIF < DIF < DIFDELCMD < \DIFdelend 
%DIF < DIF < DIFDELCMD < %%%
%DIF < DIFDELCMD < \DIFdelend 
%DIF < DIFDELCMD < %%%
%DIFDELCMD < \DIFdelend 
%DIFDELCMD < %%%
\DIFdelend 

\subsection{\DIFdelbegin %DIFDELCMD < \DIFdelbegin %%%
%DIF < DIFDELCMD < \DIFdelbegin %%%
%DIF < DIF < DIFDELCMD < \DIFdelbegin %%%
%DIF < DIF < DIF < DIFDELCMD < \DIFadd{Principal Findings}%%%
%DIF < DIF < DIFDELCMD < \DIFdelend \DIFaddbegin \DIFadd{Principal Findings}\DIFaddend %%%
%DIF < DIFDELCMD < \DIFdelend \DIFaddbegin \DIFadd{Principal Findings}\DIFaddend %%%
%DIFDELCMD < \DIFdelend \DIFaddbegin \DIFadd{Principal Findings}\DIFaddend %%%
\DIFdelend \DIFaddbegin \DIFadd{Principal Findings}\DIFaddend }
\DIFdelbegin %DIFDELCMD < \DIFdelbegin %%%
%DIF < DIFDELCMD < \DIFdelbegin %%%
%DIF < DIF < DIFDELCMD < \DIFdelbegin %%%
%DIF < DIF < DIF < DIFDELCMD < \DIFadd{...
%DIF < DIF < DIF < DIFDELCMD < %DIF >  \subsection{Principal Findings}
%DIF < DIF < DIF < DIFDELCMD < %DIF >  Our comprehensive analysis of phase-amplitude coupling in long-term iEEG recordings from the NeuroVista dataset reveals several key insights into seizure dynamics and prediction. The observation of progressive PAC enhancement during the pre-ictal period, particularly in theta/alpha-gamma coupling, supports the hypothesis that seizures arise from a gradual transition in network dynamics rather than sudden, unpredictable events. This finding has important implications for the development of seizure warning systems and our understanding of epileptogenesis.
%DIF < DIF < DIF < DIFDELCMD < }
%DIF < DIF < DIF < DIFDELCMD < %%%
%DIF < DIF < DIFDELCMD < \DIFdelend \DIFaddbegin \DIFadd{...
%DIF < DIF < DIFDELCMD < %DIF >  \subsection{Principal Findings}
%DIF < DIF < DIFDELCMD < %DIF >  Our comprehensive analysis of phase-amplitude coupling in long-term iEEG recordings from the NeuroVista dataset reveals several key insights into seizure dynamics and prediction. The observation of progressive PAC enhancement during the pre-ictal period, particularly in theta/alpha-gamma coupling, supports the hypothesis that seizures arise from a gradual transition in network dynamics rather than sudden, unpredictable events. This finding has important implications for the development of seizure warning systems and our understanding of epileptogenesis.
%DIF < DIF < DIFDELCMD < }\DIFaddend 
%DIF < DIF < DIFDELCMD < %%%
%DIF < DIFDELCMD < \DIFdelend \DIFaddbegin \DIFadd{...
%DIF < DIFDELCMD < %DIF >  \subsection{Principal Findings}
%DIF < DIFDELCMD < %DIF >  Our comprehensive analysis of phase-amplitude coupling in long-term iEEG recordings from the NeuroVista dataset reveals several key insights into seizure dynamics and prediction. The observation of progressive PAC enhancement during the pre-ictal period, particularly in theta/alpha-gamma coupling, supports the hypothesis that seizures arise from a gradual transition in network dynamics rather than sudden, unpredictable events. This finding has important implications for the development of seizure warning systems and our understanding of epileptogenesis.
%DIF < DIFDELCMD < }\DIFaddend 
%DIF < DIFDELCMD < %%%
%DIFDELCMD < \DIFdelend \DIFaddbegin \DIFadd{...
%DIFDELCMD < %DIF >  \subsection{Principal Findings}
%DIFDELCMD < %DIF >  Our comprehensive analysis of phase-amplitude coupling in long-term iEEG recordings from the NeuroVista dataset reveals several key insights into seizure dynamics and prediction. The observation of progressive PAC enhancement during the pre-ictal period, particularly in theta/alpha-gamma coupling, supports the hypothesis that seizures arise from a gradual transition in network dynamics rather than sudden, unpredictable events. This finding has important implications for the development of seizure warning systems and our understanding of epileptogenesis.
%DIFDELCMD < }\DIFaddend 
%DIFDELCMD < %%%
\DIFdelend \DIFaddbegin \DIFadd{...
%DIF >  \subsection{Principal Findings}
%DIF >  Our comprehensive analysis of phase-amplitude coupling in long-term iEEG recordings from the NeuroVista dataset reveals several key insights into seizure dynamics and prediction. The observation of progressive PAC enhancement during the pre-ictal period, particularly in theta/alpha-gamma coupling, supports the hypothesis that seizures arise from a gradual transition in network dynamics rather than sudden, unpredictable events. This finding has important implications for the development of seizure warning systems and our understanding of epileptogenesis.
}\DIFaddend 

%DIF < DIF < DIF < DIF < DIF >  \subsection{Mechanisms of PAC in Seizure Generation}
%DIF < DIF < DIF < DIF < DIF >  The dominant theta-gamma PAC pattern observed in our study aligns with established theories of cortical information processing and pathological synchronization in epilepsy. Low-frequency oscillations (theta/alpha) are thought to coordinate activity across distributed neural populations, while high-frequency gamma oscillations reflect local processing and neuronal firing. The increased coupling strength during pre-ictal periods suggests a pathological enhancement of this coordination mechanism, potentially reflecting:
%DIF < DIF < DIF < DIF >  \subsection{Mechanisms of PAC in Seizure Generation}
%DIF < DIF < DIF < DIF >  The dominant theta-gamma PAC pattern observed in our study aligns with established theories of cortical information processing and pathological synchronization in epilepsy. Low-frequency oscillations (theta/alpha) are thought to coordinate activity across distributed neural populations, while high-frequency gamma oscillations reflect local processing and neuronal firing. The increased coupling strength during pre-ictal periods suggests a pathological enhancement of this coordination mechanism, potentially reflecting:
%DIF < DIF < DIF >  \subsection{Mechanisms of PAC in Seizure Generation}
%DIF < DIF < DIF >  The dominant theta-gamma PAC pattern observed in our study aligns with established theories of cortical information processing and pathological synchronization in epilepsy. Low-frequency oscillations (theta/alpha) are thought to coordinate activity across distributed neural populations, while high-frequency gamma oscillations reflect local processing and neuronal firing. The increased coupling strength during pre-ictal periods suggests a pathological enhancement of this coordination mechanism, potentially reflecting:
%DIF < DIF >  \subsection{Mechanisms of PAC in Seizure Generation}
%DIF < DIF >  The dominant theta-gamma PAC pattern observed in our study aligns with established theories of cortical information processing and pathological synchronization in epilepsy. Low-frequency oscillations (theta/alpha) are thought to coordinate activity across distributed neural populations, while high-frequency gamma oscillations reflect local processing and neuronal firing. The increased coupling strength during pre-ictal periods suggests a pathological enhancement of this coordination mechanism, potentially reflecting:
%DIF >  \subsection{Mechanisms of PAC in Seizure Generation}
%DIF >  The dominant theta-gamma PAC pattern observed in our study aligns with established theories of cortical information processing and pathological synchronization in epilepsy. Low-frequency oscillations (theta/alpha) are thought to coordinate activity across distributed neural populations, while high-frequency gamma oscillations reflect local processing and neuronal firing. The increased coupling strength during pre-ictal periods suggests a pathological enhancement of this coordination mechanism, potentially reflecting:

%DIF < DIF < DIF < DIF < DIF >  1. \textbf{Network hypersynchronization}: Progressive recruitment of neuronal populations into synchronized states
%DIF < DIF < DIF < DIF < DIF >  2. \textbf{Excitation-inhibition imbalance}: Failure of inhibitory control mechanisms leading to runaway excitation
%DIF < DIF < DIF < DIF < DIF >  3. \textbf{Critical state transitions}: Approach to a bifurcation point in neural dynamics
%DIF < DIF < DIF < DIF >  1. \textbf{Network hypersynchronization}: Progressive recruitment of neuronal populations into synchronized states
%DIF < DIF < DIF < DIF >  2. \textbf{Excitation-inhibition imbalance}: Failure of inhibitory control mechanisms leading to runaway excitation
%DIF < DIF < DIF < DIF >  3. \textbf{Critical state transitions}: Approach to a bifurcation point in neural dynamics
%DIF < DIF < DIF >  1. \textbf{Network hypersynchronization}: Progressive recruitment of neuronal populations into synchronized states
%DIF < DIF < DIF >  2. \textbf{Excitation-inhibition imbalance}: Failure of inhibitory control mechanisms leading to runaway excitation
%DIF < DIF < DIF >  3. \textbf{Critical state transitions}: Approach to a bifurcation point in neural dynamics
%DIF < DIF >  1. \textbf{Network hypersynchronization}: Progressive recruitment of neuronal populations into synchronized states
%DIF < DIF >  2. \textbf{Excitation-inhibition imbalance}: Failure of inhibitory control mechanisms leading to runaway excitation
%DIF < DIF >  3. \textbf{Critical state transitions}: Approach to a bifurcation point in neural dynamics
%DIF >  1. \textbf{Network hypersynchronization}: Progressive recruitment of neuronal populations into synchronized states
%DIF >  2. \textbf{Excitation-inhibition imbalance}: Failure of inhibitory control mechanisms leading to runaway excitation
%DIF >  3. \textbf{Critical state transitions}: Approach to a bifurcation point in neural dynamics

%DIF < DIF < DIF < DIF < DIF >  The spatial spread of PAC elevation from focal to distributed patterns mirrors the clinical evolution from focal onset to secondary generalization, providing an electrophysiological correlate of seizure propagation.
%DIF < DIF < DIF < DIF >  The spatial spread of PAC elevation from focal to distributed patterns mirrors the clinical evolution from focal onset to secondary generalization, providing an electrophysiological correlate of seizure propagation.
%DIF < DIF < DIF >  The spatial spread of PAC elevation from focal to distributed patterns mirrors the clinical evolution from focal onset to secondary generalization, providing an electrophysiological correlate of seizure propagation.
%DIF < DIF >  The spatial spread of PAC elevation from focal to distributed patterns mirrors the clinical evolution from focal onset to secondary generalization, providing an electrophysiological correlate of seizure propagation.
%DIF >  The spatial spread of PAC elevation from focal to distributed patterns mirrors the clinical evolution from focal onset to secondary generalization, providing an electrophysiological correlate of seizure propagation.

%DIF < DIF < DIF < DIF < DIF >  \subsection{Clinical Translation and Implementation}
%DIF < DIF < DIF < DIF < DIF >  The robust discrimination between interictal and pre-ictal states using PAC features (mean AUC > 0.XX) demonstrates clinical potential for seizure forecasting. Key advantages of PAC-based prediction include:
%DIF < DIF < DIF < DIF >  \subsection{Clinical Translation and Implementation}
%DIF < DIF < DIF < DIF >  The robust discrimination between interictal and pre-ictal states using PAC features (mean AUC > 0.XX) demonstrates clinical potential for seizure forecasting. Key advantages of PAC-based prediction include:
%DIF < DIF < DIF >  \subsection{Clinical Translation and Implementation}
%DIF < DIF < DIF >  The robust discrimination between interictal and pre-ictal states using PAC features (mean AUC > 0.XX) demonstrates clinical potential for seizure forecasting. Key advantages of PAC-based prediction include:
%DIF < DIF >  \subsection{Clinical Translation and Implementation}
%DIF < DIF >  The robust discrimination between interictal and pre-ictal states using PAC features (mean AUC > 0.XX) demonstrates clinical potential for seizure forecasting. Key advantages of PAC-based prediction include:
%DIF >  \subsection{Clinical Translation and Implementation}
%DIF >  The robust discrimination between interictal and pre-ictal states using PAC features (mean AUC > 0.XX) demonstrates clinical potential for seizure forecasting. Key advantages of PAC-based prediction include:

%DIF < DIF < DIF < DIF < DIF >  \textbf{Computational efficiency}: Our PyTorch-accelerated implementation enables real-time processing suitable for implantable devices with limited computational resources. The modulation index calculation requires only basic signal processing operations implementable in low-power hardware.
%DIF < DIF < DIF < DIF >  \textbf{Computational efficiency}: Our PyTorch-accelerated implementation enables real-time processing suitable for implantable devices with limited computational resources. The modulation index calculation requires only basic signal processing operations implementable in low-power hardware.
%DIF < DIF < DIF >  \textbf{Computational efficiency}: Our PyTorch-accelerated implementation enables real-time processing suitable for implantable devices with limited computational resources. The modulation index calculation requires only basic signal processing operations implementable in low-power hardware.
%DIF < DIF >  \textbf{Computational efficiency}: Our PyTorch-accelerated implementation enables real-time processing suitable for implantable devices with limited computational resources. The modulation index calculation requires only basic signal processing operations implementable in low-power hardware.
%DIF >  \textbf{Computational efficiency}: Our PyTorch-accelerated implementation enables real-time processing suitable for implantable devices with limited computational resources. The modulation index calculation requires only basic signal processing operations implementable in low-power hardware.

%DIF < DIF < DIF < DIF < DIF >  \textbf{Interpretability}: Unlike black-box machine learning approaches, PAC provides physiologically meaningful features that clinicians can interpret in the context of known seizure mechanisms.
%DIF < DIF < DIF < DIF >  \textbf{Interpretability}: Unlike black-box machine learning approaches, PAC provides physiologically meaningful features that clinicians can interpret in the context of known seizure mechanisms.
%DIF < DIF < DIF >  \textbf{Interpretability}: Unlike black-box machine learning approaches, PAC provides physiologically meaningful features that clinicians can interpret in the context of known seizure mechanisms.
%DIF < DIF >  \textbf{Interpretability}: Unlike black-box machine learning approaches, PAC provides physiologically meaningful features that clinicians can interpret in the context of known seizure mechanisms.
%DIF >  \textbf{Interpretability}: Unlike black-box machine learning approaches, PAC provides physiologically meaningful features that clinicians can interpret in the context of known seizure mechanisms.

%DIF < DIF < DIF < DIF < DIF >  \textbf{Stability}: The use of z-scored PAC values normalized to patient-specific baselines accounts for inter-individual variability and electrode placement differences, improving generalizability.
%DIF < DIF < DIF < DIF >  \textbf{Stability}: The use of z-scored PAC values normalized to patient-specific baselines accounts for inter-individual variability and electrode placement differences, improving generalizability.
%DIF < DIF < DIF >  \textbf{Stability}: The use of z-scored PAC values normalized to patient-specific baselines accounts for inter-individual variability and electrode placement differences, improving generalizability.
%DIF < DIF >  \textbf{Stability}: The use of z-scored PAC values normalized to patient-specific baselines accounts for inter-individual variability and electrode placement differences, improving generalizability.
%DIF >  \textbf{Stability}: The use of z-scored PAC values normalized to patient-specific baselines accounts for inter-individual variability and electrode placement differences, improving generalizability.

%DIF < DIF < DIF < DIF < DIF >  \subsection{Limitations and Methodological Considerations}
%DIF < DIF < DIF < DIF >  \subsection{Limitations and Methodological Considerations}
%DIF < DIF < DIF >  \subsection{Limitations and Methodological Considerations}
%DIF < DIF >  \subsection{Limitations and Methodological Considerations}
%DIF >  \subsection{Limitations and Methodological Considerations}

%DIF < DIF < DIF < DIF < DIF >  Several limitations should be considered when interpreting our results:
%DIF < DIF < DIF < DIF >  Several limitations should be considered when interpreting our results:
%DIF < DIF < DIF >  Several limitations should be considered when interpreting our results:
%DIF < DIF >  Several limitations should be considered when interpreting our results:
%DIF >  Several limitations should be considered when interpreting our results:

%DIF < DIF < DIF < DIF < DIF >  \textbf{Patient selection bias}: The NeuroVista trial included patients with focal epilepsy suitable for implantation, potentially limiting generalizability to other epilepsy types. The requirement for >10 lead seizures during the training period may select for patients with more predictable seizure patterns.
%DIF < DIF < DIF < DIF >  \textbf{Patient selection bias}: The NeuroVista trial included patients with focal epilepsy suitable for implantation, potentially limiting generalizability to other epilepsy types. The requirement for >10 lead seizures during the training period may select for patients with more predictable seizure patterns.
%DIF < DIF < DIF >  \textbf{Patient selection bias}: The NeuroVista trial included patients with focal epilepsy suitable for implantation, potentially limiting generalizability to other epilepsy types. The requirement for >10 lead seizures during the training period may select for patients with more predictable seizure patterns.
%DIF < DIF >  \textbf{Patient selection bias}: The NeuroVista trial included patients with focal epilepsy suitable for implantation, potentially limiting generalizability to other epilepsy types. The requirement for >10 lead seizures during the training period may select for patients with more predictable seizure patterns.
%DIF >  \textbf{Patient selection bias}: The NeuroVista trial included patients with focal epilepsy suitable for implantation, potentially limiting generalizability to other epilepsy types. The requirement for >10 lead seizures during the training period may select for patients with more predictable seizure patterns.

%DIF < DIF < DIF < DIF < DIF >  \textbf{Electrode coverage}: The 16-channel arrays provide limited spatial sampling compared to high-density recordings. PAC patterns in regions distant from electrodes may be missed, potentially explaining performance variability across patients.
%DIF < DIF < DIF < DIF >  \textbf{Electrode coverage}: The 16-channel arrays provide limited spatial sampling compared to high-density recordings. PAC patterns in regions distant from electrodes may be missed, potentially explaining performance variability across patients.
%DIF < DIF < DIF >  \textbf{Electrode coverage}: The 16-channel arrays provide limited spatial sampling compared to high-density recordings. PAC patterns in regions distant from electrodes may be missed, potentially explaining performance variability across patients.
%DIF < DIF >  \textbf{Electrode coverage}: The 16-channel arrays provide limited spatial sampling compared to high-density recordings. PAC patterns in regions distant from electrodes may be missed, potentially explaining performance variability across patients.
%DIF >  \textbf{Electrode coverage}: The 16-channel arrays provide limited spatial sampling compared to high-density recordings. PAC patterns in regions distant from electrodes may be missed, potentially explaining performance variability across patients.

%DIF < DIF < DIF < DIF < DIF >  \textbf{Stationarity assumptions}: Our analysis assumes relative stationarity of PAC patterns over the recording period. However, factors such as medication changes, sleep deprivation, or device-tissue interface evolution could introduce non-stationarities affecting long-term performance.
%DIF < DIF < DIF < DIF >  \textbf{Stationarity assumptions}: Our analysis assumes relative stationarity of PAC patterns over the recording period. However, factors such as medication changes, sleep deprivation, or device-tissue interface evolution could introduce non-stationarities affecting long-term performance.
%DIF < DIF < DIF >  \textbf{Stationarity assumptions}: Our analysis assumes relative stationarity of PAC patterns over the recording period. However, factors such as medication changes, sleep deprivation, or device-tissue interface evolution could introduce non-stationarities affecting long-term performance.
%DIF < DIF >  \textbf{Stationarity assumptions}: Our analysis assumes relative stationarity of PAC patterns over the recording period. However, factors such as medication changes, sleep deprivation, or device-tissue interface evolution could introduce non-stationarities affecting long-term performance.
%DIF >  \textbf{Stationarity assumptions}: Our analysis assumes relative stationarity of PAC patterns over the recording period. However, factors such as medication changes, sleep deprivation, or device-tissue interface evolution could introduce non-stationarities affecting long-term performance.

%DIF < DIF < DIF < DIF < DIF >  \textbf{Multiple comparisons}: Despite Bonferroni correction, the analysis of multiple frequency band pairs and time windows increases false discovery risk. Future studies should consider false discovery rate (FDR) control methods.
%DIF < DIF < DIF < DIF >  \textbf{Multiple comparisons}: Despite Bonferroni correction, the analysis of multiple frequency band pairs and time windows increases false discovery risk. Future studies should consider false discovery rate (FDR) control methods.
%DIF < DIF < DIF >  \textbf{Multiple comparisons}: Despite Bonferroni correction, the analysis of multiple frequency band pairs and time windows increases false discovery risk. Future studies should consider false discovery rate (FDR) control methods.
%DIF < DIF >  \textbf{Multiple comparisons}: Despite Bonferroni correction, the analysis of multiple frequency band pairs and time windows increases false discovery risk. Future studies should consider false discovery rate (FDR) control methods.
%DIF >  \textbf{Multiple comparisons}: Despite Bonferroni correction, the analysis of multiple frequency band pairs and time windows increases false discovery risk. Future studies should consider false discovery rate (FDR) control methods.

%DIF < DIF < DIF < DIF < DIF >  \subsection{Comparison with Previous Studies}
%DIF < DIF < DIF < DIF < DIF >  Our findings extend previous PAC studies in epilepsy in several ways:
%DIF < DIF < DIF < DIF >  \subsection{Comparison with Previous Studies}
%DIF < DIF < DIF < DIF >  Our findings extend previous PAC studies in epilepsy in several ways:
%DIF < DIF < DIF >  \subsection{Comparison with Previous Studies}
%DIF < DIF < DIF >  Our findings extend previous PAC studies in epilepsy in several ways:
%DIF < DIF >  \subsection{Comparison with Previous Studies}
%DIF < DIF >  Our findings extend previous PAC studies in epilepsy in several ways:
%DIF >  \subsection{Comparison with Previous Studies}
%DIF >  Our findings extend previous PAC studies in epilepsy in several ways:

%DIF < DIF < DIF < DIF < DIF >  Compared to Edakawa et al. (2016), who reported high-frequency oscillation (HFO) coupling in shorter recordings, our long-term analysis reveals slower dynamics and circadian modulation not observable in brief recordings.
%DIF < DIF < DIF < DIF >  Compared to Edakawa et al. (2016), who reported high-frequency oscillation (HFO) coupling in shorter recordings, our long-term analysis reveals slower dynamics and circadian modulation not observable in brief recordings.
%DIF < DIF < DIF >  Compared to Edakawa et al. (2016), who reported high-frequency oscillation (HFO) coupling in shorter recordings, our long-term analysis reveals slower dynamics and circadian modulation not observable in brief recordings.
%DIF < DIF >  Compared to Edakawa et al. (2016), who reported high-frequency oscillation (HFO) coupling in shorter recordings, our long-term analysis reveals slower dynamics and circadian modulation not observable in brief recordings.
%DIF >  Compared to Edakawa et al. (2016), who reported high-frequency oscillation (HFO) coupling in shorter recordings, our long-term analysis reveals slower dynamics and circadian modulation not observable in brief recordings.

%DIF < DIF < DIF < DIF < DIF >  Unlike Amiri et al. (2016), who focused on single seizures, our large-scale analysis (>1,500 seizures) provides statistical power to identify consistent patterns and account for seizure-to-seizure variability.
%DIF < DIF < DIF < DIF >  Unlike Amiri et al. (2016), who focused on single seizures, our large-scale analysis (>1,500 seizures) provides statistical power to identify consistent patterns and account for seizure-to-seizure variability.
%DIF < DIF < DIF >  Unlike Amiri et al. (2016), who focused on single seizures, our large-scale analysis (>1,500 seizures) provides statistical power to identify consistent patterns and account for seizure-to-seizure variability.
%DIF < DIF >  Unlike Amiri et al. (2016), who focused on single seizures, our large-scale analysis (>1,500 seizures) provides statistical power to identify consistent patterns and account for seizure-to-seizure variability.
%DIF >  Unlike Amiri et al. (2016), who focused on single seizures, our large-scale analysis (>1,500 seizures) provides statistical power to identify consistent patterns and account for seizure-to-seizure variability.

%DIF < DIF < DIF < DIF < DIF >  Our improvement over traditional features (+XX\% AUC vs spectral power) is consistent with Jacobs et al. (2018), though our continuous monitoring approach offers advantages for real-world implementation.
%DIF < DIF < DIF < DIF >  Our improvement over traditional features (+XX\% AUC vs spectral power) is consistent with Jacobs et al. (2018), though our continuous monitoring approach offers advantages for real-world implementation.
%DIF < DIF < DIF >  Our improvement over traditional features (+XX\% AUC vs spectral power) is consistent with Jacobs et al. (2018), though our continuous monitoring approach offers advantages for real-world implementation.
%DIF < DIF >  Our improvement over traditional features (+XX\% AUC vs spectral power) is consistent with Jacobs et al. (2018), though our continuous monitoring approach offers advantages for real-world implementation.
%DIF >  Our improvement over traditional features (+XX\% AUC vs spectral power) is consistent with Jacobs et al. (2018), though our continuous monitoring approach offers advantages for real-world implementation.

%DIF < DIF < DIF < DIF < DIF >  \subsection{Future Directions}
%DIF < DIF < DIF < DIF >  \subsection{Future Directions}
%DIF < DIF < DIF >  \subsection{Future Directions}
%DIF < DIF >  \subsection{Future Directions}
%DIF >  \subsection{Future Directions}

%DIF < DIF < DIF < DIF < DIF >  Several avenues warrant further investigation:
%DIF < DIF < DIF < DIF >  Several avenues warrant further investigation:
%DIF < DIF < DIF >  Several avenues warrant further investigation:
%DIF < DIF >  Several avenues warrant further investigation:
%DIF >  Several avenues warrant further investigation:

%DIF < DIF < DIF < DIF < DIF >  \textbf{Adaptive algorithms}: Implementing online learning to adapt to non-stationary dynamics and improve long-term stability.
%DIF < DIF < DIF < DIF >  \textbf{Adaptive algorithms}: Implementing online learning to adapt to non-stationary dynamics and improve long-term stability.
%DIF < DIF < DIF >  \textbf{Adaptive algorithms}: Implementing online learning to adapt to non-stationary dynamics and improve long-term stability.
%DIF < DIF >  \textbf{Adaptive algorithms}: Implementing online learning to adapt to non-stationary dynamics and improve long-term stability.
%DIF >  \textbf{Adaptive algorithms}: Implementing online learning to adapt to non-stationary dynamics and improve long-term stability.

%DIF < DIF < DIF < DIF < DIF >  \textbf{Multi-modal integration}: Combining PAC with other biomarkers (HFOs, connectivity measures, behavioral sensors) may further improve prediction accuracy.
%DIF < DIF < DIF < DIF >  \textbf{Multi-modal integration}: Combining PAC with other biomarkers (HFOs, connectivity measures, behavioral sensors) may further improve prediction accuracy.
%DIF < DIF < DIF >  \textbf{Multi-modal integration}: Combining PAC with other biomarkers (HFOs, connectivity measures, behavioral sensors) may further improve prediction accuracy.
%DIF < DIF >  \textbf{Multi-modal integration}: Combining PAC with other biomarkers (HFOs, connectivity measures, behavioral sensors) may further improve prediction accuracy.
%DIF >  \textbf{Multi-modal integration}: Combining PAC with other biomarkers (HFOs, connectivity measures, behavioral sensors) may further improve prediction accuracy.

%DIF < DIF < DIF < DIF < DIF >  \textbf{Mechanism-based interventions}: Understanding PAC dynamics could inform closed-loop stimulation protocols that disrupt pre-ictal buildup.
%DIF < DIF < DIF < DIF >  \textbf{Mechanism-based interventions}: Understanding PAC dynamics could inform closed-loop stimulation protocols that disrupt pre-ictal buildup.
%DIF < DIF < DIF >  \textbf{Mechanism-based interventions}: Understanding PAC dynamics could inform closed-loop stimulation protocols that disrupt pre-ictal buildup.
%DIF < DIF >  \textbf{Mechanism-based interventions}: Understanding PAC dynamics could inform closed-loop stimulation protocols that disrupt pre-ictal buildup.
%DIF >  \textbf{Mechanism-based interventions}: Understanding PAC dynamics could inform closed-loop stimulation protocols that disrupt pre-ictal buildup.

%DIF < DIF < DIF < DIF < DIF >  \textbf{Generalization to other epilepsy types}: Extending analysis to generalized epilepsies and pediatric populations.
%DIF < DIF < DIF < DIF >  \textbf{Generalization to other epilepsy types}: Extending analysis to generalized epilepsies and pediatric populations.
%DIF < DIF < DIF >  \textbf{Generalization to other epilepsy types}: Extending analysis to generalized epilepsies and pediatric populations.
%DIF < DIF >  \textbf{Generalization to other epilepsy types}: Extending analysis to generalized epilepsies and pediatric populations.
%DIF >  \textbf{Generalization to other epilepsy types}: Extending analysis to generalized epilepsies and pediatric populations.

%DIF < DIF < DIF < DIF < DIF >  \textbf{Optimization of electrode placement}: Using PAC maps to guide surgical planning and optimize electrode positioning for monitoring and stimulation.
%DIF < DIF < DIF < DIF >  \textbf{Optimization of electrode placement}: Using PAC maps to guide surgical planning and optimize electrode positioning for monitoring and stimulation.
%DIF < DIF < DIF >  \textbf{Optimization of electrode placement}: Using PAC maps to guide surgical planning and optimize electrode positioning for monitoring and stimulation.
%DIF < DIF >  \textbf{Optimization of electrode placement}: Using PAC maps to guide surgical planning and optimize electrode positioning for monitoring and stimulation.
%DIF >  \textbf{Optimization of electrode placement}: Using PAC maps to guide surgical planning and optimize electrode positioning for monitoring and stimulation.

%DIF < DIF < DIF < DIF < DIF >  \subsection{Implications for Seizure Advisory Systems}
%DIF < DIF < DIF < DIF < DIF >  The successful pseudo-prospective evaluation demonstrates feasibility for clinical implementation. A PAC-based advisory system could provide:
%DIF < DIF < DIF < DIF < DIF >  - Early warnings (XX minutes) for XX\% of seizures
%DIF < DIF < DIF < DIF < DIF >  - Low false positive rates (XX per day) minimizing alarm fatigue
%DIF < DIF < DIF < DIF < DIF >  - Personalized risk assessments based on individual PAC patterns
%DIF < DIF < DIF < DIF >  \subsection{Implications for Seizure Advisory Systems}
%DIF < DIF < DIF < DIF >  The successful pseudo-prospective evaluation demonstrates feasibility for clinical implementation. A PAC-based advisory system could provide:
%DIF < DIF < DIF < DIF >  - Early warnings (XX minutes) for XX\% of seizures
%DIF < DIF < DIF < DIF >  - Low false positive rates (XX per day) minimizing alarm fatigue
%DIF < DIF < DIF < DIF >  - Personalized risk assessments based on individual PAC patterns
%DIF < DIF < DIF >  \subsection{Implications for Seizure Advisory Systems}
%DIF < DIF < DIF >  The successful pseudo-prospective evaluation demonstrates feasibility for clinical implementation. A PAC-based advisory system could provide:
%DIF < DIF < DIF >  - Early warnings (XX minutes) for XX\% of seizures
%DIF < DIF < DIF >  - Low false positive rates (XX per day) minimizing alarm fatigue
%DIF < DIF < DIF >  - Personalized risk assessments based on individual PAC patterns
%DIF < DIF >  \subsection{Implications for Seizure Advisory Systems}
%DIF < DIF >  The successful pseudo-prospective evaluation demonstrates feasibility for clinical implementation. A PAC-based advisory system could provide:
%DIF < DIF >  - Early warnings (XX minutes) for XX\% of seizures
%DIF < DIF >  - Low false positive rates (XX per day) minimizing alarm fatigue
%DIF < DIF >  - Personalized risk assessments based on individual PAC patterns
%DIF >  \subsection{Implications for Seizure Advisory Systems}
%DIF >  The successful pseudo-prospective evaluation demonstrates feasibility for clinical implementation. A PAC-based advisory system could provide:
%DIF >  - Early warnings (XX minutes) for XX\% of seizures
%DIF >  - Low false positive rates (XX per day) minimizing alarm fatigue
%DIF >  - Personalized risk assessments based on individual PAC patterns

%DIF < DIF < DIF < DIF < DIF >  Integration with existing seizure detection algorithms and consideration of circadian patterns could further enhance performance.
%DIF < DIF < DIF < DIF >  Integration with existing seizure detection algorithms and consideration of circadian patterns could further enhance performance.
%DIF < DIF < DIF >  Integration with existing seizure detection algorithms and consideration of circadian patterns could further enhance performance.
%DIF < DIF >  Integration with existing seizure detection algorithms and consideration of circadian patterns could further enhance performance.
%DIF >  Integration with existing seizure detection algorithms and consideration of circadian patterns could further enhance performance.

%DIF < DIF < DIF < DIF < DIF >  \subsection{Conclusion}
%DIF < DIF < DIF < DIF < DIF >  This comprehensive analysis of phase-amplitude coupling in long-term iEEG recordings provides robust evidence for PAC as a biomarker of seizure susceptibility. The progressive enhancement of theta-gamma coupling during pre-ictal periods offers both mechanistic insights and practical applications for seizure prediction. Our optimized computational pipeline and systematic database management approach enable scalable analysis applicable to next-generation implantable devices. While patient-specific variability remains a challenge, the consistent patterns across multiple seizures and patients support the clinical translation of PAC-based seizure forecasting systems. Future work should focus on real-time implementation, algorithm adaptation, and integration with therapeutic interventions to improve quality of life for patients with drug-resistant epilepsy.
%DIF < DIF < DIF < DIF >  \subsection{Conclusion}
%DIF < DIF < DIF < DIF >  This comprehensive analysis of phase-amplitude coupling in long-term iEEG recordings provides robust evidence for PAC as a biomarker of seizure susceptibility. The progressive enhancement of theta-gamma coupling during pre-ictal periods offers both mechanistic insights and practical applications for seizure prediction. Our optimized computational pipeline and systematic database management approach enable scalable analysis applicable to next-generation implantable devices. While patient-specific variability remains a challenge, the consistent patterns across multiple seizures and patients support the clinical translation of PAC-based seizure forecasting systems. Future work should focus on real-time implementation, algorithm adaptation, and integration with therapeutic interventions to improve quality of life for patients with drug-resistant epilepsy.
%DIF < DIF < DIF >  \subsection{Conclusion}
%DIF < DIF < DIF >  This comprehensive analysis of phase-amplitude coupling in long-term iEEG recordings provides robust evidence for PAC as a biomarker of seizure susceptibility. The progressive enhancement of theta-gamma coupling during pre-ictal periods offers both mechanistic insights and practical applications for seizure prediction. Our optimized computational pipeline and systematic database management approach enable scalable analysis applicable to next-generation implantable devices. While patient-specific variability remains a challenge, the consistent patterns across multiple seizures and patients support the clinical translation of PAC-based seizure forecasting systems. Future work should focus on real-time implementation, algorithm adaptation, and integration with therapeutic interventions to improve quality of life for patients with drug-resistant epilepsy.
%DIF < DIF >  \subsection{Conclusion}
%DIF < DIF >  This comprehensive analysis of phase-amplitude coupling in long-term iEEG recordings provides robust evidence for PAC as a biomarker of seizure susceptibility. The progressive enhancement of theta-gamma coupling during pre-ictal periods offers both mechanistic insights and practical applications for seizure prediction. Our optimized computational pipeline and systematic database management approach enable scalable analysis applicable to next-generation implantable devices. While patient-specific variability remains a challenge, the consistent patterns across multiple seizures and patients support the clinical translation of PAC-based seizure forecasting systems. Future work should focus on real-time implementation, algorithm adaptation, and integration with therapeutic interventions to improve quality of life for patients with drug-resistant epilepsy.
%DIF >  \subsection{Conclusion}
%DIF >  This comprehensive analysis of phase-amplitude coupling in long-term iEEG recordings provides robust evidence for PAC as a biomarker of seizure susceptibility. The progressive enhancement of theta-gamma coupling during pre-ictal periods offers both mechanistic insights and practical applications for seizure prediction. Our optimized computational pipeline and systematic database management approach enable scalable analysis applicable to next-generation implantable devices. While patient-specific variability remains a challenge, the consistent patterns across multiple seizures and patients support the clinical translation of PAC-based seizure forecasting systems. Future work should focus on real-time implementation, algorithm adaptation, and integration with therapeutic interventions to improve quality of life for patients with drug-resistant epilepsy.

\DIFdelbegin %DIFDELCMD < \DIFdelbegin %%%
%DIF < DIFDELCMD < \DIFdelbegin %%%
%DIF < DIF < DIFDELCMD < \DIFdelbegin %%%
%DIF < DIF < DIF < DIFDELCMD < \DIFaddend %%%
%DIF < DIF < DIFDELCMD < \DIFdelend %%%
%DIF < DIFDELCMD < \DIFdelend %%%
%DIFDELCMD < \DIFdelend %%%
\DIFdelend \label{sec:discussion}

%%%% EOF


%% ----------------------------------------
%% DATA AVAILABILITY
%% ----------------------------------------

%% ========================================
%% ./01_manuscript/contents/data_availability.tex
%% ========================================
%% -*- coding: utf-8 -*-
%DIF < DIF < DIF < DIF < DIF < % Timestamp: "2025-09-19 10:34:25 (ywatanabe)"
%DIF < DIF < DIF < DIF < DIF < % File: "/ssh:sp:/home/ywatanabe/proj/neurovista/paper/manuscript/src/data_availability_revised.tex"
%DIF < DIF < DIF < DIF < DIF > % Timestamp: "2025-09-27 20:21:43 (ywatanabe)"
%DIF < DIF < DIF < DIF < DIF > % File: "/ssh:sp:/home/ywatanabe/proj/neurovista/paper/01_manuscript/contents/data_availability.tex"
%DIF < DIF < DIF < DIF > % Timestamp: "2025-09-27 20:21:43 (ywatanabe)"
%DIF < DIF < DIF < DIF > % File: "/ssh:sp:/home/ywatanabe/proj/neurovista/paper/01_manuscript/contents/data_availability.tex"
%DIF < DIF < DIF > % Timestamp: "2025-09-27 20:21:43 (ywatanabe)"
%DIF < DIF < DIF > % File: "/ssh:sp:/home/ywatanabe/proj/neurovista/paper/01_manuscript/contents/data_availability.tex"
%DIF < DIF > % Timestamp: "2025-09-27 20:21:43 (ywatanabe)"
%DIF < DIF > % File: "/ssh:sp:/home/ywatanabe/proj/neurovista/paper/01_manuscript/contents/data_availability.tex"
%DIF > % Timestamp: "2025-09-27 20:21:43 (ywatanabe)"
%DIF > % File: "/ssh:sp:/home/ywatanabe/proj/neurovista/paper/01_manuscript/contents/data_availability.tex"
\pdfbookmark[1]{Data Availability Statement}{data_availability}

\section*{Data Availability Statement}

The NeuroVista dataset used in this study is publicly available through the International Epilepsy Electrophysiology Portal (IEEG.org) at \url{https://www.ieeg.org}. Access requires registration and approval for research purposes. 

The processed PAC databases and analysis code are available at \DIFdelbegin %DIFDELCMD < \DIFdelbegin %%%
%DIF < DIFDELCMD < \DIFdelbegin %%%
%DIF < DIF < DIFDELCMD < \DIFdelbegin %%%
%DIF < DIF < DIF < DIFDELCMD < \DIFdelbegin %%%
%DIF < DIF < DIF < DIF < DIFDELCMD < \url{https://github.com/ywatanabe1989/neurovista-pac} %%%
%DIF < DIF < DIF < DIFDELCMD < \DIFdel{(to be made public upon publication). The repository includes:
%DIF < DIF < DIF < DIFDELCMD < - PAC computation pipeline (PyTorch implementation)
%DIF < DIF < DIF < DIFDELCMD < - Database management utilities (SQLite3)
%DIF < DIF < DIF < DIFDELCMD < - Visualization scripts for comodulograms and temporal analysis
%DIF < DIF < DIF < DIFDELCMD < - Statistical analysis and seizure prediction algorithms
%DIF < DIF < DIF < DIFDELCMD < - Jupyter notebooks reproducing all figures and results
%DIF < DIF < DIF < DIFDELCMD < }%%%
%DIF < DIF < DIF < DIF < DIFDELCMD < 
%DIF < DIF < DIF < DIFDELCMD < 
%DIF < DIF < DIFDELCMD < 
%DIF < DIFDELCMD < 
%DIFDELCMD < 

%DIFDELCMD < %%%
%DIF < DIFDELCMD < %%%
%DIF < DIF < DIFDELCMD < %%%
%DIF < DIF < DIF < DIFDELCMD < %%%
%DIF < DIF < DIF < DIF < DIFDELCMD < %%%
%DIF < DIF < DIF < DIFDELCMD < \DIFdel{The SciTeX framework }\DIFdelend \DIFaddbegin %%%
%DIF < DIF < DIFDELCMD < \DIFdelend %%%
%DIF < DIFDELCMD < \DIFdelend %%%
%DIFDELCMD < \DIFdelend %%%
\DIFdelend \url{https://github.com/ywatanabe1989/neurovista}\DIFdelbegin %DIFDELCMD < \DIFdelbegin %%%
%DIF < DIFDELCMD < \DIFdelbegin %%%
%DIF < DIF < DIFDELCMD < \DIFdelbegin %%%
%DIF < DIF < DIF < DIFDELCMD < \DIFadd{. GPU-accerelated PAC calculation code is available as a standalone Python package `gpac` at }%%%
%DIF < DIF < DIFDELCMD < \DIFdelend \DIFaddbegin \DIFadd{. GPU-accerelated PAC calculation code is available as a standalone Python package `gpac` at }\DIFaddend %%%
%DIF < DIFDELCMD < \DIFdelend \DIFaddbegin \DIFadd{. GPU-accerelated PAC calculation code is available as a standalone Python package `gpac` at }\DIFaddend %%%
%DIFDELCMD < \DIFdelend \DIFaddbegin \DIFadd{. GPU-accerelated PAC calculation code is available as a standalone Python package `gpac` at }\DIFaddend %%%
\DIFdelend \DIFaddbegin \DIFadd{. GPU-accerelated PAC calculation code is available as a standalone Python package `gpac` at }\DIFaddend \url{https://github.com/ywatanabe1989/gPAC}\DIFdelbegin %DIFDELCMD < \DIFdelbegin %%%
%DIF < DIFDELCMD < \DIFdelbegin %%%
%DIF < DIF < DIFDELCMD < \DIFdelbegin %%%
%DIF < DIF < DIF < DIFDELCMD < \DIFadd{. The SciTeX Python utilities }\DIFaddend %%%
%DIF < DIF < DIFDELCMD < \DIFdelend \DIFaddbegin \DIFadd{. The SciTeX Python utilities }\DIFaddend %%%
%DIF < DIFDELCMD < \DIFdelend \DIFaddbegin \DIFadd{. The SciTeX Python utilities }\DIFaddend %%%
%DIFDELCMD < \DIFdelend \DIFaddbegin \DIFadd{. The SciTeX Python utilities }\DIFaddend %%%
\DIFdelend \DIFaddbegin \DIFadd{. The SciTeX Python utilities }\DIFaddend used for reproducible computing is available at \url{https://github.com/ywatanabe1989/SciTeX}.

For questions regarding data access or analysis procedures, please contact the corresponding author.

\label{data and code availability}

%%%% EOF


%% ----------------------------------------
%% REFERENCE STYLES
%% ----------------------------------------
\pdfbookmark[1]{References}{references}
\bibliography{./01_manuscript/contents/bibliography}

%% ========================================
%% ./01_manuscript/contents/latex_styles/bibliography.tex
%% ========================================
%% -*- coding: utf-8 -*-
%% Timestamp: "2025-09-26 18:18:03 (ywatanabe)"
%% File: "/ssh:sp:/home/ywatanabe/proj/neurovista/paper/01_manuscript/src/styles/bibliography.tex"

% Note Re-compile is required

% %% Numbering Style (sorted and listed)
% [1, 2, 3, 4]

%% Numbering Style (sorted)
\bibliographystyle{elsarticle-num}

% Author Style
% \bibliographystyle{plainnat}
% use \citet{}

% Numbering Style (not-sorted) 
% \bibliographystyle{plainnat}
% use \cite{}

%%%% EOF


%% ----------------------------------------
%% ADDITIONAL INFORMATION
%% ----------------------------------------

%% ========================================
%% ./01_manuscript/contents/additional_info.tex
%% ========================================
%% -*- coding: utf-8 -*-
%DIF < DIF < DIF < DIF < DIF < % Timestamp: "2025-09-22 08:08:53 (ywatanabe)"
%DIF < DIF < DIF < DIF < DIF < % File: "/ssh:sp:/home/ywatanabe/proj/neurovista/paper/manuscript/src/additional_info.tex"
%DIF < DIF < DIF < DIF < DIF > % Timestamp: "2025-09-27 20:17:53 (ywatanabe)"
%DIF < DIF < DIF < DIF < DIF > % File: "/ssh:sp:/home/ywatanabe/proj/neurovista/paper/01_manuscript/contents/additional_info.tex"
\DIFdelbegin %DIFDELCMD < \DIFdelbegin %%%
%DIF < DIFDELCMD < \DIFdelbegin %%%
%DIF < DIF < DIFDELCMD < \DIFdelbegin %%%
%DIF < DIF < DIF < DIFDELCMD < \DIFaddbegin 
%DIF < DIF < DIF < DIFDELCMD < %%%
%DIF < DIF < DIFDELCMD < \DIFdelend %%%
%DIF < DIF < DIF < DIF > % Timestamp: "2025-09-27 20:17:53 (ywatanabe)"
%DIF < DIF < DIF < DIF > % File: "/ssh:sp:/home/ywatanabe/proj/neurovista/paper/01_manuscript/contents/additional_info.tex"
%DIF < DIFDELCMD < \DIFdelend %%%
%DIF < DIF < DIF > % Timestamp: "2025-09-27 20:17:53 (ywatanabe)"
%DIF < DIF < DIF > % File: "/ssh:sp:/home/ywatanabe/proj/neurovista/paper/01_manuscript/contents/additional_info.tex"
%DIFDELCMD < \DIFdelend %%%
%DIF < DIF > % Timestamp: "2025-09-27 20:17:53 (ywatanabe)"
%DIF < DIF > % File: "/ssh:sp:/home/ywatanabe/proj/neurovista/paper/01_manuscript/contents/additional_info.tex"
\DIFdelend %DIF > % Timestamp: "2025-09-27 20:17:53 (ywatanabe)"
%DIF > % File: "/ssh:sp:/home/ywatanabe/proj/neurovista/paper/01_manuscript/contents/additional_info.tex"

\DIFdelbegin %DIFDELCMD < \DIFdelbegin %%%
%DIF < DIFDELCMD < \DIFdelbegin %%%
%DIF < DIF < DIFDELCMD < \DIFdelbegin %%%
%DIF < DIF < DIF < DIFDELCMD < \DIFaddend %%%
%DIF < DIF < DIFDELCMD < \DIFdelend %%%
%DIF < DIFDELCMD < \DIFdelend %%%
%DIFDELCMD < \DIFdelend %%%
\DIFdelend \pdfbookmark[1]{Additional Information}{additional_information}

\pdfbookmark[2]{Ethics Declarations}{ethics_declarations}                    
\section*{Ethics Declarations}
All study participants provided their written informed consent ...
\label{ethics declarations}

\pdfbookmark[2]{Contributors}{author_contributions}                    
\section*{Author Contributions}
Y.W., T.Y., and D.G. conceptualized the study ...
\label{author contributions}

\pdfbookmark[2]{Acknowledgments}{acknowledgments}                    
\section*{Acknowledgments}
This research was funded by \DIFdelbegin %DIFDELCMD < \DIFdelbegin %%%
%DIF < DIFDELCMD < \DIFdelbegin %%%
%DIF < DIF < DIFDELCMD < \DIFdelbegin %%%
%DIF < DIF < DIF < DIFDELCMD < \DIFdelbegin \DIFdel{...
%DIF < DIF < DIF < DIFDELCMD < }\DIFdelend \DIFaddbegin %%%
%DIF < DIF < DIFDELCMD < \DIFdelend %%%
%DIF < DIFDELCMD < \DIFdelend %%%
%DIFDELCMD < \DIFdelend %%%
\DIFdelend \hl{funding bodies here}
\DIFdelbegin %DIFDELCMD < \DIFdelbegin %%%
%DIF < DIFDELCMD < \DIFdelbegin %%%
%DIF < DIF < DIFDELCMD < \DIFdelbegin %%%
%DIF < DIF < DIF < DIFDELCMD < \DIFaddend %%%
%DIF < DIF < DIFDELCMD < \DIFdelend %%%
%DIF < DIFDELCMD < \DIFdelend %%%
%DIFDELCMD < \DIFdelend %%%
\DIFdelend \label{acknowledgments}

\pdfbookmark[2]{Declaration of Interests}{declaration_of_interest}           \section*{Declaration of Interests}
The authors declare that they have no competing interests.
\label{declaration of interests}

\DIFdelbegin %DIFDELCMD < \DIFdelbegin %%%
%DIF < DIFDELCMD < \DIFdelbegin %%%
%DIF < DIF < DIFDELCMD < \DIFdelbegin %%%
%DIF < DIF < DIF < DIFDELCMD < \DIFdelbegin %%%
%DIF < DIF < DIF < DIF < DIFDELCMD < \pdfbookmark[2]{Inclusion and Diversity Statement}{inclusion_and_diversity_statement}        
%DIF < DIF < DIF < DIF < DIFDELCMD < %%%
\section*{%DIF < DIF < DIF < DIFDELCMD < \DIFdel{Inclusion and Diversity Statement}%%%
}
%DIFAUXCMD
%DIF < DIFAUXCMD
%DIF < DIF < DIFAUXCMD
%DIF < DIF < DIF < DIFAUXCMD
%DIF < DIF < DIF < DIF < DIFAUXCMD
%DIF < DIF < DIF < DIFDELCMD < \DIFdel{We support inclusive, diverse, and equitable conduct of research.
%DIF < DIF < DIF < DIFDELCMD < }%%%
%DIF < DIF < DIF < DIF < DIFDELCMD < \label{inclusion and diversity statement}
%DIF < DIF < DIF < DIF < DIFDELCMD < 
%DIF < DIF < DIF < DIFDELCMD < 
%DIF < DIF < DIFDELCMD < 
%DIF < DIFDELCMD < 
%DIFDELCMD < 

%DIFDELCMD < %%%
%DIF < DIFDELCMD < %%%
%DIF < DIF < DIFDELCMD < %%%
%DIF < DIF < DIF < DIFDELCMD < %%%
%DIF < DIF < DIF < DIF < DIFDELCMD < %%%
%DIF < DIF < DIF < DIFDELCMD < \DIFdelend %%%
%DIF < DIF < DIFDELCMD < \DIFdelend %%%
%DIF < DIFDELCMD < \DIFdelend %%%
%DIFDELCMD < \DIFdelend %%%
\DIFdelend \pdfbookmark[2]{Declaration of Generative AI in Scientific Writing}{declaration_of_generative_ai}
\section*{Declaration of Generative AI in Scientific Writing}
The authors employed \DIFdelbegin %DIFDELCMD < \DIFdelbegin %%%
%DIF < DIFDELCMD < \DIFdelbegin %%%
%DIF < DIF < DIFDELCMD < \DIFdelbegin %%%
%DIF < DIF < DIF < DIFDELCMD < \DIFdelbegin \DIFdel{Claude Code (provided by Anthropic }\DIFdelend \DIFaddbegin \DIFadd{large language models such as Claude (Anthropic Inc.}\DIFaddend %%%
%DIF < DIF < DIFDELCMD < \DIFdelend \DIFaddbegin \DIFadd{large language models such as Claude (Anthropic Inc.}\DIFaddend %%%
%DIF < DIFDELCMD < \DIFdelend \DIFaddbegin \DIFadd{large language models such as Claude (Anthropic Inc.}\DIFaddend %%%
%DIFDELCMD < \DIFdelend \DIFaddbegin \DIFadd{large language models such as Claude (Anthropic Inc.}\DIFaddend %%%
\DIFdelend \DIFaddbegin \DIFadd{large language models such as Claude (Anthropic Inc.}\DIFaddend ) for code development and \DIFdelbegin %DIFDELCMD < \DIFdelbegin %%%
%DIF < DIFDELCMD < \DIFdelbegin %%%
%DIF < DIF < DIFDELCMD < \DIFdelbegin %%%
%DIF < DIF < DIF < DIFDELCMD < \DIFdelbegin \DIFdel{ChatGPT (provided by OpenAI) for enhancing the }\DIFdelend \DIFaddbegin \DIFadd{complementing }\DIFaddend %%%
%DIF < DIF < DIFDELCMD < \DIFdelend \DIFaddbegin \DIFadd{complementing }\DIFaddend %%%
%DIF < DIFDELCMD < \DIFdelend \DIFaddbegin \DIFadd{complementing }\DIFaddend %%%
%DIFDELCMD < \DIFdelend \DIFaddbegin \DIFadd{complementing }\DIFaddend %%%
\DIFdelend \DIFaddbegin \DIFadd{complementing }\DIFaddend manuscript's English language quality. After incorporating \DIFdelbegin %DIFDELCMD < \DIFdelbegin %%%
%DIF < DIFDELCMD < \DIFdelbegin %%%
%DIF < DIF < DIFDELCMD < \DIFdelbegin %%%
%DIF < DIF < DIF < DIFDELCMD < \DIFdelbegin \DIFdel{the }\DIFdelend %%%
%DIF < DIF < DIFDELCMD < \DIFdelend %%%
%DIF < DIFDELCMD < \DIFdelend %%%
%DIFDELCMD < \DIFdelend %%%
\DIFdelend suggested improvements, the authors meticulously revised the content. Ultimate responsibility for the final content of this publication rests entirely with the authors.
\label{declaration of generative ai in scientific writing}

%DIF < DIF < DIF < DIF < DIF < % \pdfbookmark[2]{Appendices}{appendices}                    
%DIF < DIF < DIF < DIF < DIF < % \appendix
%DIF < DIF < DIF < DIF < DIF < % \section{}
%DIF < DIF < DIF < DIF < DIF < % \label{}
\DIFdelbegin %DIFDELCMD < \DIFdelbegin %%%
%DIF < DIFDELCMD < \DIFdelbegin %%%
%DIF < DIF < DIFDELCMD < \DIFdelbegin %%%
%DIF < DIF < DIF < DIFDELCMD < \DIFdelbegin %%%
%DIF < DIF < DIF < DIF < DIFDELCMD < 
%DIF < DIF < DIF < DIFDELCMD < 
%DIF < DIF < DIFDELCMD < 
%DIF < DIFDELCMD < 
%DIFDELCMD < 

%DIFDELCMD < %%%
%DIF < DIFDELCMD < %%%
%DIF < DIF < DIFDELCMD < %%%
%DIF < DIF < DIF < DIFDELCMD < %%%
%DIF < DIF < DIF < DIF < DIFDELCMD < %%%
%DIF < DIF < DIF < DIFDELCMD < \DIFdelend %%%
%DIF < DIF < DIFDELCMD < \DIFdelend %%%
%DIF < DIFDELCMD < \DIFdelend %%%
%DIFDELCMD < \DIFdelend %%%
\DIFdelend %%%% EOF


%% ----------------------------------------
%% TABLES
%% ----------------------------------------
\clearpage
\section*{Tables}
\label{tables}
\pdfbookmark[1]{Tables}{tables}

%% ========================================
%% ./01_manuscript/contents/tables/compiled/FINAL.tex
%% ========================================
% Auto-generated file containing all table inputs



%% ----------------------------------------
%% FIGURES
%% ----------------------------------------
\clearpage
\section*{Figures}
\label{figures}
\pdfbookmark[1]{Figures}{figures}

%% ========================================
%% ./01_manuscript/contents/figures/compiled/FINAL.tex
%% ========================================
% Generated by compile_figure_tex_files()
% This file includes all figure files in order

%DIF < DIF < DIF < DIF < DIF <  Figure 00: Figure 00
%DIF < DIF < DIF < DIF < DIF >  Figure 01: Figure 01
%DIF < DIF < DIF < DIF >  Figure 01: Figure 01
%DIF < DIF < DIF >  Figure 01: Figure 01
%DIF < DIF >  Figure 01a_demographic_data: Figure 01a_demographic_data
%DIF >  Figure 01: Figure 01
\begin{figure*}[p]
    \DIFdelbeginFL %DIFDELCMD < \DIFdelbeginFL %%%
%DIF < DIFDELCMD < \DIFdelbeginFL %%%
%DIF < DIF < DIFDELCMD < \DIFdelbeginFL %%%
%DIF < DIF < DIF < DIFDELCMD < \DIFdelbeginFL %%%
%DIF < DIF < DIF < DIF < DIFDELCMD < \pdfbookmark[2]{Figure 00}{figure_id_00}
%DIF < DIF < DIF < DIF < DIFDELCMD <     %%%
%DIF < DIF < DIF < DIFDELCMD < \DIFdelendFL \DIFaddbeginFL %%%
%DIF < DIF < DIFDELCMD < \DIFdelendFL %%%
%DIF < DIFDELCMD < \DIFdelendFL \pdfbookmark[2]{Figure 01}{figure_id_01}
%DIF < DIFDELCMD <     \DIFdelbeginFL %%%
%DIF < DIF < DIFDELCMD < \DIFdelbeginFL %%%
%DIF < DIF < DIF < DIFDELCMD < \DIFaddendFL %%%
%DIF < DIF < DIFDELCMD < \DIFdelendFL %%%
%DIF < DIFDELCMD < \DIFdelendFL %%%
%DIFDELCMD < \DIFdelendFL \DIFaddbeginFL \pdfbookmark[2]{Figure 01a_demographic_data}{figure_id_01a_demographic_data}
%DIFDELCMD <     \DIFaddendFL \centering
%DIFDELCMD <     \DIFdelbeginFL %%%
%DIF < DIFDELCMD < \DIFdelbeginFL %%%
%DIF < DIF < DIFDELCMD < \DIFdelbeginFL %%%
%DIF < DIF < DIF < DIFDELCMD < \DIFdelbeginFL %%%
%DIF < DIF < DIF < DIF < DIFDELCMD < \includegraphics[width=0.95\textwidth]{./01_manuscript/contents/figures/caption_and_media/jpg_for_compilation/Figure_ID_00_template.jpg}
%DIF < DIF < DIF < DIF < DIFDELCMD <     %%%
%DIF < DIF < DIF < DIFDELCMD < \DIFdelendFL \DIFaddbeginFL %%%
%DIF < DIF < DIFDELCMD < \DIFdelendFL %%%
%DIF < DIFDELCMD < \DIFdelendFL \includegraphics[width=0.95\textwidth]{./01_manuscript/contents/figures/caption_and_media/jpg_for_compilation/Figure_ID_01_demographic_data_a.jpg}
%DIF < DIFDELCMD <     \DIFdelbeginFL %%%
%DIF < DIF < DIFDELCMD < \DIFdelbeginFL %%%
%DIF < DIF < DIF < DIFDELCMD < \DIFaddendFL %%%
%DIF < DIF < DIFDELCMD < \DIFdelendFL %%%
%DIF < DIFDELCMD < \DIFdelendFL %%%
%DIFDELCMD < \DIFdelendFL \DIFaddbeginFL \includegraphics[width=0.95\textwidth]{./01_manuscript/contents/figures/caption_and_media/jpg_for_compilation/Figure_ID_01a_demographic_data.jpg}
%DIFDELCMD <     \DIFaddendFL %%%
%DIFDELCMD < \caption{%
{%DIFAUXCMD
%DIFDELCMD < \DIFdelbeginFL %%%
%DIF < DIFDELCMD < \DIFdelbeginFL %%%
%DIF < DIF < DIFDELCMD < \DIFdelbeginFL %%%
%DIF < DIF < DIF < DIFDELCMD < \DIFdelbeginFL %%%
\textbf{%DIF < DIF < DIF < DIFDELCMD < \DIFdelFL{Figure 00
%DIF < DIF < DIF < DIFDELCMD < }%%%
}
%DIFAUXCMD
%DIF < DIFAUXCMD
%DIF < DIF < DIFAUXCMD
%DIF < DIF < DIF < DIFAUXCMD
%DIF < DIF < DIF < DIF < DIFAUXCMD
%DIF < DIF < DIF < DIFDELCMD < \DIFdelendFL \DIFaddbeginFL %%%
\textbf{%DIF < DIF < DIF < DIFDELCMD < \DIFaddFL{Figure 01
%DIF < DIF < DIF < DIFDELCMD < }%%%
}
%DIFAUXCMD
%DIF < DIFAUXCMD
%DIF < DIF < DIFAUXCMD
%DIF < DIF < DIF < DIFAUXCMD
%DIF < DIF < DIF < DIFDELCMD < \DIFaddendFL %%%
%DIF < DIF < DIFDELCMD < \DIFdelendFL \DIFaddbeginFL %%%
\textbf{%DIF < DIF < DIFDELCMD < \DIFaddFL{Figure 01
%DIF < DIF < DIFDELCMD < }%%%
}
%DIFAUXCMD
%DIF < DIFAUXCMD
%DIF < DIF < DIFAUXCMD
%DIF < DIF < DIFDELCMD < \DIFaddendFL %%%
%DIF < DIFDELCMD < \DIFdelendFL \DIFaddbeginFL %%%
\textbf{%DIF < DIFDELCMD < \DIFaddFL{Figure 01
%DIF < DIFDELCMD < }%%%
}
%DIFAUXCMD
%DIF < DIFAUXCMD
%DIF < DIFDELCMD < \DIFaddendFL %%%
%DIFDELCMD < \DIFdelendFL \DIFaddbeginFL %%%
\textbf{%DIFDELCMD < \DIFaddFL{Figure 01a_demographic_data
%DIFDELCMD < }%%%
}
%DIFAUXCMD
%DIFDELCMD < \DIFaddendFL \smallskip
%DIFDELCMD < \\
%DIFDELCMD < %%%
\DIFdelFL{Description for figure }%DIFDELCMD < \DIFdelbeginFL %%%
%DIF < DIFDELCMD < \DIFdelbeginFL %%%
%DIF < DIF < DIFDELCMD < \DIFdelbeginFL %%%
%DIF < DIF < DIF < DIFDELCMD < \DIFdelbeginFL \DIFdelFL{00.
%DIF < DIF < DIF < DIFDELCMD < }\DIFdelendFL \DIFaddbeginFL \DIFaddFL{01.
%DIF < DIF < DIF < DIFDELCMD < }\DIFaddendFL %%%
%DIF < DIF < DIFDELCMD < \DIFdelendFL \DIFaddbeginFL \DIFaddFL{01.
%DIF < DIF < DIFDELCMD < }\DIFaddendFL %%%
%DIF < DIFDELCMD < \DIFdelendFL \DIFaddbeginFL \DIFaddFL{01.
%DIF < DIFDELCMD < }\DIFaddendFL %%%
%DIFDELCMD < \DIFdelendFL \DIFaddbeginFL \DIFaddFL{01a_demographic_data.
%DIFDELCMD < }\DIFaddendFL %%%
}
    %DIFAUXCMD
%DIFDELCMD < \DIFdelbeginFL %%%
%DIF < DIFDELCMD < \DIFdelbeginFL %%%
%DIF < DIF < DIFDELCMD < \DIFdelbeginFL %%%
%DIF < DIF < DIF < DIFDELCMD < \DIFdelbeginFL %%%
%DIF < DIF < DIF < DIF < DIFDELCMD < \label{fig:00_template}
%DIF < DIF < DIF < DIF < DIFDELCMD < %%%
%DIF < DIF < DIF < DIFDELCMD < \DIFdelendFL \DIFaddbeginFL %%%
%DIF < DIF < DIFDELCMD < \DIFdelendFL %%%
%DIF < DIFDELCMD < \DIFdelendFL \label{fig:01_demographic_data_a}
%DIF < DIFDELCMD < \DIFdelbeginFL %%%
%DIF < DIF < DIFDELCMD < \DIFdelbeginFL %%%
%DIF < DIF < DIF < DIFDELCMD < \DIFaddendFL %%%
%DIF < DIF < DIFDELCMD < \DIFdelendFL %%%
%DIF < DIFDELCMD < \DIFdelendFL %%%
%DIFDELCMD < \DIFdelendFL \DIFaddbeginFL \label{fig:01a_demographic_data}
%DIFDELCMD < \DIFaddendFL \end{figure*}
%DIFDELCMD < 

%DIFDELCMD < %%%
%DIF < DIF <  Figure 01: Figure 01
%DIF < DIF >  Figure 01b_demographic_data: Figure 01b_demographic_data
%DIFDELCMD < \clearpage
%DIFDELCMD < \begin{figure*}[p]
%DIFDELCMD <     \DIFdelbeginFL %%%
%DIF < DIFDELCMD < \pdfbookmark[2]{Figure 01}{figure_id_01}
%DIF < DIFDELCMD <     %%%
%DIFDELCMD < \DIFdelendFL \DIFaddbeginFL \pdfbookmark[2]{Figure 01b_demographic_data}{figure_id_01b_demographic_data}
%DIFDELCMD <     \DIFaddendFL \centering
%DIFDELCMD <     \DIFdelbeginFL %%%
%DIF < DIFDELCMD < \DIFdelbeginFL %%%
%DIF < DIF < DIFDELCMD < \DIFdelbeginFL %%%
%DIF < DIF < DIF < DIFDELCMD < \DIFdelbeginFL %%%
%DIF < DIF < DIF < DIF < DIFDELCMD < \includegraphics[width=0.95\textwidth]{./01_manuscript/contents/figures/caption_and_media/jpg_for_compilation/Figure_ID_01_workflow.jpg}
%DIF < DIF < DIF < DIF < DIFDELCMD <     %%%
%DIF < DIF < DIF < DIFDELCMD < \DIFdelendFL \DIFaddbeginFL %%%
%DIF < DIF < DIFDELCMD < \DIFdelendFL %%%
%DIF < DIFDELCMD < \DIFdelendFL \includegraphics[width=0.95\textwidth]{./01_manuscript/contents/figures/caption_and_media/jpg_for_compilation/Figure_ID_01_demographic_data_b.jpg}
%DIF < DIFDELCMD <     \DIFdelbeginFL %%%
%DIF < DIF < DIFDELCMD < \DIFdelbeginFL %%%
%DIF < DIF < DIF < DIFDELCMD < \DIFaddendFL %%%
%DIF < DIF < DIFDELCMD < \DIFdelendFL %%%
%DIF < DIFDELCMD < \DIFdelendFL %%%
%DIFDELCMD < \DIFdelendFL \DIFaddbeginFL \includegraphics[width=0.95\textwidth]{./01_manuscript/contents/figures/caption_and_media/jpg_for_compilation/Figure_ID_01b_demographic_data.jpg}
%DIFDELCMD <     \DIFaddendFL %%%
%DIFDELCMD < \caption{%
{%DIFAUXCMD
%DIFDELCMD < \DIFdelbeginFL %%%
\textbf{%DIFDELCMD < \DIFdelFL{Figure 01
%DIFDELCMD < }%%%
}
%DIFAUXCMD
%DIF < DIFAUXCMD
%DIFDELCMD < \DIFdelendFL \DIFaddbeginFL %%%
\textbf{%DIFDELCMD < \DIFaddFL{Figure 01b_demographic_data
%DIFDELCMD < }%%%
}
%DIFAUXCMD
%DIFDELCMD < \DIFaddendFL \smallskip
%DIFDELCMD < \\
%DIFDELCMD < %%%
\DIFdelFL{Description for figure }%DIFDELCMD < \DIFdelbeginFL \DIFdelFL{01.
%DIFDELCMD < }\DIFdelendFL \DIFaddbeginFL \DIFaddFL{01b_demographic_data.
%DIFDELCMD < }\DIFaddendFL %%%
}
    %DIFAUXCMD
%DIFDELCMD < \DIFdelbeginFL %%%
%DIF < DIFDELCMD < \DIFdelbeginFL %%%
%DIF < DIF < DIFDELCMD < \DIFdelbeginFL %%%
%DIF < DIF < DIF < DIFDELCMD < \DIFdelbeginFL %%%
%DIF < DIF < DIF < DIF < DIFDELCMD < \label{fig:01_workflow}
%DIF < DIF < DIF < DIF < DIFDELCMD < %%%
%DIF < DIF < DIF < DIFDELCMD < \DIFdelendFL \DIFaddbeginFL %%%
%DIF < DIF < DIFDELCMD < \DIFdelendFL %%%
%DIF < DIFDELCMD < \DIFdelendFL \label{fig:01_demographic_data_b}
%DIF < DIFDELCMD < %%%
%DIFDELCMD < \DIFdelendFL \DIFaddbeginFL \label{fig:01b_demographic_data}
%DIFDELCMD < \DIFaddendFL \end{figure*}
%DIFDELCMD < 

%DIFDELCMD < %%%
%DIF < DIF < DIF < DIF < DIF >  Figure 01: Figure 01
%DIF < DIF < DIF < DIF >  Figure 01: Figure 01
%DIF < DIF < DIF >  Figure 01: Figure 01
%DIF < DIF >  Figure 01c_demographic_data: Figure 01c_demographic_data
%DIFDELCMD < \clearpage
%DIFDELCMD < \begin{figure*}[p]
%DIFDELCMD <     \DIFdelbeginFL %%%
%DIF < DIFDELCMD < \pdfbookmark[2]{Figure 01}{figure_id_01}
%DIF < DIFDELCMD <     %%%
%DIFDELCMD < \DIFdelendFL \DIFaddbeginFL \pdfbookmark[2]{Figure 01c_demographic_data}{figure_id_01c_demographic_data}
%DIFDELCMD <     \DIFaddendFL \centering
%DIFDELCMD <     \DIFdelbeginFL %%%
%DIF < DIFDELCMD < \includegraphics[width=0.95\textwidth]{./01_manuscript/contents/figures/caption_and_media/jpg_for_compilation/Figure_ID_01_demographic_data_c.jpg}
%DIF < DIFDELCMD <     %%%
%DIFDELCMD < \DIFdelendFL \DIFaddbeginFL \includegraphics[width=0.95\textwidth]{./01_manuscript/contents/figures/caption_and_media/jpg_for_compilation/Figure_ID_01c_demographic_data.jpg}
%DIFDELCMD <     \DIFaddendFL %%%
%DIFDELCMD < \caption{%
{%DIFAUXCMD
%DIFDELCMD < \DIFdelbeginFL %%%
%DIF < DIFDELCMD < \DIFdelbeginFL %%%
%DIF < DIF < DIFDELCMD < \DIFdelbeginFL %%%
\textbf{%DIF < DIF < DIF < DIFDELCMD < \DIFaddFL{Figure 01
%DIF < DIF < DIF < DIFDELCMD < }%%%
}
%DIFAUXCMD
%DIF < DIFAUXCMD
%DIF < DIF < DIFAUXCMD
%DIF < DIF < DIF < DIFAUXCMD
%DIF < DIF < DIFDELCMD < \DIFdelendFL \DIFaddbeginFL %%%
\textbf{%DIF < DIF < DIFDELCMD < \DIFaddFL{Figure 01
%DIF < DIF < DIFDELCMD < }%%%
}
%DIFAUXCMD
%DIF < DIFAUXCMD
%DIF < DIF < DIFAUXCMD
%DIF < DIF < DIFDELCMD < \DIFaddendFL %%%
%DIF < DIFDELCMD < \DIFdelendFL \DIFaddbeginFL %%%
\textbf{%DIF < DIFDELCMD < \DIFaddFL{Figure 01
%DIF < DIFDELCMD < }%%%
}
%DIFAUXCMD
%DIF < DIFAUXCMD
%DIF < DIFDELCMD < \DIFaddendFL %%%
%DIFDELCMD < \DIFdelendFL \DIFaddbeginFL %%%
\textbf{%DIFDELCMD < \DIFaddFL{Figure 01c_demographic_data
%DIFDELCMD < }%%%
}
%DIFAUXCMD
%DIFDELCMD < \DIFaddendFL \smallskip
%DIFDELCMD < \\
%DIFDELCMD < \DIFdelbeginFL %%%
%DIF < DIFDELCMD < \DIFdelbeginFL %%%
%DIF < DIF < DIFDELCMD < \DIFdelbeginFL %%%
%DIF < DIF < DIF < DIFDELCMD < \DIFaddFL{Description for figure 01.
%DIF < DIF < DIF < DIFDELCMD < }%%%
%DIF < DIF < DIFDELCMD < \DIFdelendFL \DIFaddbeginFL \DIFaddFL{Description for figure 01.
%DIF < DIF < DIFDELCMD < }\DIFaddendFL %%%
%DIF < DIFDELCMD < \DIFdelendFL \DIFaddbeginFL \DIFaddFL{Description for figure 01.
%DIF < DIFDELCMD < }\DIFaddendFL %%%
%DIFDELCMD < \DIFdelendFL \DIFaddbeginFL \DIFaddFL{Description for figure 01c_demographic_data.
%DIFDELCMD < }\DIFaddendFL %%%
}
    %DIFAUXCMD
%DIFDELCMD < \DIFdelbeginFL %%%
%DIF < DIFDELCMD < \label{fig:01_demographic_data_c}
%DIF < DIFDELCMD < %%%
%DIFDELCMD < \DIFdelendFL \DIFaddbeginFL \label{fig:01c_demographic_data}
%DIFDELCMD < \DIFaddendFL \end{figure*}
%DIFDELCMD < 

%DIFDELCMD < %%%
%DIF < DIF < DIF < DIF < DIF >  Figure 01: Figure 01
%DIF < DIF < DIF < DIF >  Figure 01: Figure 01
%DIF < DIF < DIF >  Figure 01: Figure 01
%DIF < DIF >  Figure 01d_demographic_data: Figure 01d_demographic_data
%DIFDELCMD < \clearpage
%DIFDELCMD < \begin{figure*}[p]
%DIFDELCMD <     \DIFaddbeginFL \pdfbookmark[2]{Figure 01d_demographic_data}{figure_id_01d_demographic_data}
%DIFDELCMD <     \centering
%DIFDELCMD <     \includegraphics[width=0.95\textwidth]{./01_manuscript/contents/figures/caption_and_media/jpg_for_compilation/Figure_ID_01d_demographic_data.jpg}
%DIFDELCMD <     %%%
%DIFDELCMD < \caption{%
{%DIFAUXCMD
\textbf{%DIFDELCMD < \DIFaddFL{Figure 01d_demographic_data
%DIFDELCMD < }%%%
}
%DIFAUXCMD
%DIFDELCMD < \smallskip
%DIFDELCMD < \\
%DIFDELCMD < \DIFaddFL{Description for figure 01d_demographic_data.
%DIFDELCMD < }%%%
}
    %DIFAUXCMD
%DIFDELCMD < \label{fig:01d_demographic_data}
%DIFDELCMD < \end{figure*}
%DIFDELCMD < 

%DIFDELCMD < %%%
%DIF < DIF >  Figure 01: Figure 01
%DIFDELCMD < \clearpage
%DIFDELCMD < \begin{figure*}[p]
%DIFDELCMD <     \DIFaddendFL %%%
\DIFdelendFL \pdfbookmark[2]{Figure 01}{figure_id_01}
    \centering
    \DIFdelbeginFL %DIFDELCMD < \DIFdelbeginFL %%%
%DIF < DIFDELCMD < \includegraphics[width=0.95\textwidth]{./01_manuscript/contents/figures/caption_and_media/jpg_for_compilation/Figure_ID_01_demographic_data_d.jpg}
%DIF < DIFDELCMD <     %%%
%DIFDELCMD < \DIFdelendFL \DIFaddbeginFL %%%
\DIFdelendFL \includegraphics[width=0.95\textwidth]{./01_manuscript/contents/figures/caption_and_media/jpg_for_compilation/Figure_ID_01_demographic_data.jpg}
    \DIFdelbeginFL %DIFDELCMD < \DIFaddendFL %%%
\DIFdelendFL \caption{
\DIFdelbeginFL %DIFDELCMD < \DIFdelbeginFL %%%
%DIF < DIFDELCMD < \DIFdelbeginFL %%%
%DIF < DIF < DIFDELCMD < \DIFdelbeginFL %%%
\textbf{%DIF < DIF < DIF < DIFDELCMD < \DIFaddFL{Figure 01
%DIF < DIF < DIF < DIFDELCMD < }%%%
}
%DIFAUXCMD
%DIF < DIFAUXCMD
%DIF < DIF < DIFAUXCMD
%DIF < DIF < DIF < DIFAUXCMD
%DIF < DIF < DIFDELCMD < \DIFdelendFL \DIFaddbeginFL %%%
\textbf{%DIF < DIF < DIFDELCMD < \DIFaddFL{Figure 01
%DIF < DIF < DIFDELCMD < }%%%
}
%DIFAUXCMD
%DIF < DIFAUXCMD
%DIF < DIF < DIFAUXCMD
%DIF < DIF < DIFDELCMD < \DIFaddendFL %%%
%DIF < DIFDELCMD < \DIFdelendFL \DIFaddbeginFL %%%
\textbf{%DIF < DIFDELCMD < \DIFaddFL{Figure 01
%DIF < DIFDELCMD < }%%%
}
%DIFAUXCMD
%DIF < DIFAUXCMD
%DIF < DIFDELCMD < \DIFaddendFL %%%
%DIFDELCMD < \DIFdelendFL \DIFaddbeginFL %%%
\textbf{%DIFDELCMD < \DIFaddFL{Figure 01
%DIFDELCMD < }%%%
}
%DIFAUXCMD
%DIFDELCMD < \DIFaddendFL %%%
\DIFdelendFL \DIFaddbeginFL \textbf{
\DIFaddFL{Figure 01
}}
\DIFaddendFL \smallskip
\\
\DIFdelbeginFL %DIFDELCMD < \DIFdelbeginFL %%%
%DIF < DIFDELCMD < \DIFdelbeginFL %%%
%DIF < DIF < DIFDELCMD < \DIFdelbeginFL %%%
%DIF < DIF < DIF < DIFDELCMD < \DIFaddFL{Description for figure 01.
%DIF < DIF < DIF < DIFDELCMD < }%%%
%DIF < DIF < DIFDELCMD < \DIFdelendFL \DIFaddbeginFL \DIFaddFL{Description for figure 01.
%DIF < DIF < DIFDELCMD < }\DIFaddendFL %%%
%DIF < DIFDELCMD < \DIFdelendFL \DIFaddbeginFL \DIFaddFL{Description for figure 01.
%DIF < DIFDELCMD < }\DIFaddendFL %%%
%DIFDELCMD < \DIFdelendFL \DIFaddbeginFL \DIFaddFL{Description for figure 01.
%DIFDELCMD < }\DIFaddendFL %%%
\DIFdelendFL \DIFaddbeginFL \DIFaddFL{Description for figure 01.
}\DIFaddendFL }
    \DIFdelbeginFL %DIFDELCMD < \DIFdelbeginFL %%%
%DIF < DIFDELCMD < \label{fig:01_demographic_data_d}
%DIF < DIFDELCMD < \DIFdelbeginFL %%%
%DIF < DIF < DIFDELCMD < \DIFdelbeginFL %%%
%DIF < DIF < DIF < DIFDELCMD < \DIFaddendFL %%%
%DIF < DIF < DIFDELCMD < \DIFdelendFL %%%
%DIF < DIFDELCMD < \DIFdelendFL %%%
%DIFDELCMD < \DIFdelendFL \DIFaddbeginFL %%%
\DIFdelendFL \label{fig:01_demographic_data}
\DIFdelbeginFL %DIFDELCMD < \DIFaddendFL %%%
\DIFdelendFL \end{figure*}




%% ----------------------------------------
%% END of DOCUMENT
%% ----------------------------------------
\end{document}

