%% -*- coding: utf-8 -*-

\section{Methods}

The SciTeX Writer framework implements a modular architecture designed around three core principles: reproducible compilation, content-structure separation, and automated asset management. The system organizes documents into three primary directories, each serving distinct purposes in the manuscript lifecycle while sharing common resources to maintain consistency.

\subsection{Repository Structure and Organization}

The framework employs a hierarchical directory structure where the \texttt{00\_shared/} directory serves as the single source of truth for metadata including title, author information, keywords, and bibliographic references. This centralized approach eliminates duplication and ensures consistency across all output documents. The \texttt{01\_manuscript/} directory contains the main manuscript with subdirectories for content sections, figures, and tables. Similarly, \texttt{02\_supplementary/} follows an identical structure for supplementary materials, while \texttt{03\_revision/} organizes revision letters by reviewer. Each content section exists as an independent LaTeX file, facilitating modular development and enabling multiple authors to work on different sections simultaneously without merge conflicts.

\subsection{Container-Based Compilation System}

To ensure reproducible builds across diverse computing environments, the framework leverages both Docker and Singularity containerization technologies~\citep{Smith2020_NeuralAnalysis}. The compilation environment encapsulates specific versions of TeX Live and all required packages, eliminating dependency on the host system's LaTeX installation. Users invoke compilation through a simple Makefile interface that abstracts the container complexity~\citep{Johnson2019_SpectralAnalysis}. The command \texttt{make manuscript} compiles the main document, while \texttt{make all} processes all three document types in parallel. This containerized approach guarantees that the same source files produce identical PDFs regardless of the underlying operating system, making the system equally functional on Linux, macOS, Windows, and high-performance computing clusters.

\subsection{Automated Asset Processing}

The system implements automatic format conversion for both figures and tables through preprocessing scripts that execute during compilation~\citep{Chen2021_MachineLearning}. For figures, the framework accepts common image formats including PNG, JPEG, SVG, and PDF, automatically converting them to formats optimized for LaTeX inclusion. Each figure resides in its own subdirectory within \texttt{01\_manuscript/contents/figures/caption\_and\_media/}, with the caption defined in a corresponding \texttt{.tex} file. During compilation, a preprocessing script scans these directories, generates figure inclusion code, and compiles all figures into \texttt{FINAL.tex} for inclusion in the main document. Tables follow an analogous structure, allowing authors to define complex table layouts separately from their incorporation into the document flow~\citep{Brown2018_DeepLearning}.

\subsection{Version Control and Difference Tracking}

The framework integrates with Git to provide systematic version tracking and automatic generation of difference documents. When authors create a new version through \texttt{make archive}, the system archives the current manuscript with a timestamp and version number. Subsequently, invoking \texttt{make diff} generates a PDF highlighting changes between versions using the latexdiff utility. This functionality proves particularly valuable during revision processes, where journals often require marked-up versions showing modifications. The revision directory structure accommodates multiple rounds of review, with separate subdirectories for editor and reviewer responses, each containing both the original comments and author responses in a structured format that ensures complete documentation of the revision process.

%%%% EOF
