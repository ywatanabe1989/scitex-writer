%% -*- coding: utf-8 -*-

\section{Results}

The SciTeX Writer framework successfully demonstrates comprehensive manuscript preparation capabilities through its modular design and automated workflows. This section presents the key features and functionalities that the system provides to researchers.

\subsection{Cross-Platform Reproducibility}

The containerized compilation system achieves complete reproducibility across different operating systems and computing environments. Testing across Linux distributions, macOS, and Windows Subsystem for Linux confirmed that identical source files produce byte-for-byte identical PDF outputs when compiled using the same container image. This reproducibility extends to high-performance computing environments where Singularity containers enable compilation on systems without Docker support. The elimination of environment-dependent compilation issues represents a significant improvement over traditional local LaTeX installations, where package version mismatches frequently cause inconsistent outputs or compilation failures.

\subsection{Automated Figure and Table Management}

The automatic asset processing system effectively handles diverse input formats and streamlines figure incorporation~\citep{YourName2023_NovelMethod}. Figure~\ref{fig:example_figure_01} demonstrates the framework's capability to include images with properly formatted captions, while Figure~\ref{fig:example_figure_02} shows how multiple figures can be managed systematically. The preprocessing pipeline converts source images to optimal formats, maintaining quality while ensuring compatibility with LaTeX compilation requirements~\citep{YourName2022_PreviousWork}. For tables, the system provides structured organization as shown in Table~\ref{tab:example_table_01}, where complex tabular data can be defined independently and automatically integrated into the document flow. This separation of content from presentation enables authors to focus on data rather than formatting syntax.

\subsection{Modular Content Organization}

The framework's modular structure facilitates collaborative writing by isolating different manuscript components into separate files. Each section, from the introduction through the discussion, exists as an independent LaTeX file that can be edited without affecting other sections. This organization minimizes merge conflicts in version control systems and allows multiple authors to work simultaneously on different parts of the manuscript. The shared metadata system ensures that changes to author lists, affiliations, or keywords propagate automatically across the main manuscript, supplementary materials, and revision documents without requiring manual updates in multiple locations.

\subsection{Version Tracking and Difference Generation}

The integrated version control system maintains a complete history of manuscript evolution through the archive mechanism. Each archived version receives a timestamp and sequential version number, creating a clear audit trail of document development. The automatic difference generation produces professionally formatted PDFs highlighting textual changes between versions, using color coding to indicate additions and deletions. This functionality proves particularly valuable during peer review, where revision letters must clearly document modifications made in response to reviewer comments. The system handles this process automatically, requiring only simple Makefile commands rather than manual execution of latexdiff with complex parameters.

%%%% EOF
