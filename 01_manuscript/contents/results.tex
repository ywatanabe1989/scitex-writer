%% -*- coding: utf-8 -*-

\section{Results}

Present your findings in a clear, logical sequence. Replace this placeholder text with your actual results.

\subsection{Overview of Dataset}

Begin with descriptive statistics about your dataset or study population. For example:
\begin{itemize}
    \item Sample size and characteristics
    \item Data quality metrics
    \item Descriptive statistics
\end{itemize}

\subsection{Primary Findings}

Present your main results, organized by research question or hypothesis. Use figures and tables to illustrate key findings. For example, Figure~\ref{fig:example_figure_01} shows an example result.

Describe statistical comparisons and their significance. Report effect sizes along with p-values. For instance: ``The treatment group showed significantly higher performance (mean = XX.X ± SD) compared to control (mean = YY.Y ± SD), t(df) = ZZ.Z, p < 0.001, Cohen's d = W.WW.''

\subsection{Secondary Analyses}

Present additional analyses that support or extend your primary findings. Include:
\begin{itemize}
    \item Subgroup analyses
    \item Sensitivity analyses
    \item Additional statistical tests
    \item Exploratory findings
\end{itemize}

Reference your figures (Figure~\ref{fig:example_figure_02}) and tables (Table~\ref{tab:example_table_01}) appropriately throughout the results section. Let the data speak for itself - save interpretation for the Discussion section.

%%%% EOF
