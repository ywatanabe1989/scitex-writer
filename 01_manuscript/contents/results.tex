%% -*- coding: utf-8 -*-

\section{Results}

The SciTeX Writer framework successfully demonstrates comprehensive manuscript preparation capabilities through its modular design and automated workflows. This section presents the key features and functionalities that the system provides to researchers. The framework's architecture, illustrated in Figure~\ref{fig:system_architecture}, implements a layered design from user interface to output generation, while Figure~\ref{fig:directory_structure} shows the detailed file organization that minimizes conflicts during collaborative editing. The compilation workflow (Figure~\ref{fig:compilation_workflow}) shows how the system automatically processes multiple asset types in parallel while maintaining reproducibility across platforms. Figure~\ref{fig:feature_overview} provides a comprehensive mind map of all major capabilities, from compilation engines to version control.

\subsection{Multi-Engine Compilation System}

SciTeX Writer supports three compilation engines optimized for different scenarios (Table~\ref{tab:compilation_engines}): latexmk for rapid iterative development ($\sim$3s), Tectonic for reproducible builds ($\sim$4--5s), and traditional 3-pass compilation for guaranteed compatibility ($\sim$6--7s). The engine selection logic (Figure~\ref{fig:engine_selection_logic}) automatically detects the best available option, prioritizing speed while maintaining broad compatibility. Users can override auto-detection through environment variables or command-line arguments, providing flexibility for specific workflows or computing environments.

The compilation system provides extensive customization through command-line options (Table~\ref{tab:compilation_options}). Quick compilation modes enable authors to iterate rapidly during writing: \texttt{--no\_figs} and \texttt{--no\_tables} skip asset processing, \texttt{--draft} uses single-pass compilation, and \texttt{--no\_diff} omits difference generation. These optimizations reduce compilation time from $\sim$15s for full processing to under 3s for ultra-fast draft mode, significantly improving the writing experience. Environmental variables (Table~\ref{tab:environment_variables}) provide system-level configuration for logging verbosity, engine priority, citation styles, and file paths.

\subsection{Cross-Platform Reproducibility}

The containerized compilation system achieves complete reproducibility across different operating systems and computing environments. Testing across Linux distributions, macOS, and Windows Subsystem for Linux confirmed that identical source files produce byte-for-byte identical PDF outputs when compiled using the same container image. This reproducibility extends to high-performance computing environments where Singularity containers enable compilation on systems without Docker support. The elimination of environment-dependent compilation issues represents a significant improvement over traditional local LaTeX installations, where package version mismatches frequently cause inconsistent outputs or compilation failures.

\subsection{Automated Figure and Table Management}

The automatic asset processing system effectively handles diverse input formats and streamlines figure incorporation~\cite{YourName2023_NovelMethod}. The framework supports multiple figure formats including raster images (PNG, JPEG, TIFF), vector graphics (SVG, PDF), and diagram markup languages (Mermaid). Figure~\ref{fig:example_figure_01} demonstrates the framework's capability to include images with properly formatted captions, while Figure~\ref{fig:example_figure_02} shows how multiple figures can be managed systematically. Complex workflow diagrams, such as the compilation pipeline shown in Figure~\ref{fig:compilation_workflow}, can be created using Mermaid syntax and automatically rendered during compilation. The directory structure visualization (Figure~\ref{fig:directory_structure}) exemplifies how technical diagrams integrate seamlessly with the manuscript preparation workflow.

The preprocessing pipeline converts source images to optimal formats, maintaining quality while ensuring compatibility with LaTeX compilation requirements~\cite{YourName2022_PreviousWork}. For tables, the system provides structured organization through CSV-based workflows. Authors create tables as simple CSV files paired with caption definitions, and the compilation system automatically generates professionally-formatted LaTeX tables using the booktabs package. Tables~\ref{tab:compilation_options}, \ref{tab:environment_variables}, and \ref{tab:compilation_engines} all demonstrate automatic CSV-to-LaTeX conversion, showcasing the system's capability to handle diverse table structures from simple configuration lists to categorized reference data. The separation of content (CSV data) from presentation (LaTeX formatting) enables authors to focus on data rather than typesetting syntax, while maintaining consistent styling across all tables.

\subsection{Multi-file Bibliography Management}

The bibliography system (Figure~\ref{fig:bibliography_deduplication}) enables researchers to organize references by topic across multiple .bib files in the \texttt{00\_shared/bib\_files/} directory. For example, authors might maintain separate files for methodological references (\texttt{methods\_refs.bib}), field background (\texttt{field\_background.bib}), and personal publications (\texttt{my\_papers.bib}). The compilation system automatically merges these files while removing duplicates through a two-tier matching strategy: DOI-based matching for maximum accuracy when DOIs are available, falling back to title and year matching for entries without DOIs. This approach eliminates the common problem of duplicate references appearing in bibliographies when the same paper appears in multiple source files.

\subsection{Modular Content Organization}

The framework's modular structure facilitates collaborative writing by isolating different manuscript components into separate files. Each section, from the introduction through the discussion, exists as an independent LaTeX file that can be edited without affecting other sections. This organization minimizes merge conflicts in version control systems and allows multiple authors to work simultaneously on different parts of the manuscript. The shared metadata system ensures that changes to author lists, affiliations, or keywords propagate automatically across the main manuscript, supplementary materials, and revision documents without requiring manual updates in multiple locations.

\subsection{Version Tracking and Difference Generation}

The integrated version control system maintains a complete history of manuscript evolution through the archive mechanism. Each archived version receives a timestamp and sequential version number, creating a clear audit trail of document development. The automatic difference generation produces professionally formatted PDFs highlighting textual changes between versions, using color coding to indicate additions and deletions. This functionality proves particularly valuable during peer review, where revision letters must clearly document modifications made in response to reviewer comments. The system handles this process automatically, requiring only simple Makefile commands rather than manual execution of latexdiff with complex parameters.

%%%% EOF
