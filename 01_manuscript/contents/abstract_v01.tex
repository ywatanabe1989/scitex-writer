%% -*- coding: utf-8 -*-
%% Timestamp: "2025-09-27 20:14:56 (ywatanabe)"
%% File: "/ssh:sp:/home/ywatanabe/proj/neurovista/paper/01_manuscript/contents/abstract.tex"
\begin{abstract}
  \pdfbookmark[1]{Abstract}{abstract}

Neural oscillations exhibit cross-frequency interactions that are fundamental to brain function and disrupted in neurological disorders \cite{Canolty2010TheFRC,Aru2014UntanglingCCD}. Phase-amplitude coupling (PAC), where the phase of low-frequency oscillations modulates the amplitude of high-frequency activity, serves as a biomarker for various brain states including epileptic seizures \cite{Tort2010MeasuringPCE,Hlsemann2019QuantificationOPA}. Previous studies have demonstrated PAC changes around seizure events \cite{Zhang2017TemporalspatialCOAG,Miao2021SeizureOZBG}, but characterization across extended timescales remains limited availability of long-term recording data and high computational requirements in PAC computation \cite{Kuhlmann2018SeizurePA,Combrisson2020TensorpacAOAH}. The challenge of processing continuous, long-term neural recordings has hindered the development of reliable seizure prediction systems \cite{Freestone2015SeizurePSBF,Natu2022ReviewOEB}. Here we show that the combination of the NeuroVista dataset and our GPU-accelerated PAC computation system enables ...

comprehensive analysis of 4.1 TB of continuous intracranial electroencephalogram data from 15 patients with focal epilepsy (NeuroVista dataset \cite{Kuhlmann2018SeizurePA}), encompassing 1,539 seizures over monitoring periods ranging from 6 months to 2 years. \hl{We found distinct PAC signatures between theta-to-beta phase (2-30 Hz, 25 bands) and gamma amplitude (60-180 Hz, 25 bands) that systematically modulated 5-60 minutes before seizure onset}, achieving balanced accuracy of \hl{0.55±0.04} and ROC-AUC of \hl{0.58±0.02} for discriminating pre-ictal from interictal states. Our GPU-accelerated implementation achieved \hl{100-fold} speed improvements compared to conventional CPU-based methods \cite{Combrisson2020TensorpacAOAH}, reducing computation time from years to months and potentially enabling real-time PAC monitoring with less than 2-minute processing latency. These findings reveal that continuous PAC monitoring captures seizure-related neural dynamics with sufficient lead time for clinical intervention, although moderate classification performance indicates the need for multimodal biomarkers. The computational framework and temporal PAC patterns identified here provide a foundation for next-generation implantable seizure advisory systems \cite{Kuhlmann2018SeizurePA}, potentially improving quality of life for millions with drug-resistant epilepsy through reliable seizure warnings integrated with patient-specific therapeutic interventions.

\end{abstract}

%%%% EOF