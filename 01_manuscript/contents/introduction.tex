%% -*- coding: utf-8 -*-
%% Timestamp: "2025-09-26 21:47:51 (ywatanabe)"
%% File: "/ssh:sp:/home/ywatanabe/proj/neurovista/paper/01_manuscript/src/introduction.tex"

%% \[START of [0-9]\. .* Statement\]
%% \[\(START\|END\) of [0-9]+\. [^]]*\] *
\section{Introduction}

Epilepsy affects approximately 70 million people worldwide, representing one of the most prevalent neurological disorders characterized by recurrent, unpredictable seizures that fundamentally disrupt daily life and impose substantial societal burden \hlref{WHO2023}. Drug-resistant focal epilepsy, occurring in approximately 30-40\% of epilepsy patients, remains particularly challenging as conventional antiepileptic medications fail to provide adequate seizure control despite optimal medical management \hlref{Kwan2000,Chen2018}. The development of reliable seizure prediction systems represents a critical frontier in epilepsy care, offering the potential to transform patient management from reactive treatment to proactive intervention, thereby reducing seizure-related injuries, psychological burden, and improving quality of life for millions of individuals with drug-resistant epilepsy \hlref{Mormann2007,Kuhlmann2018}. 

	Decades of research have established that neural oscillations exhibit complex cross-frequency interactions fundamental to brain function, with phase-amplitude coupling (PAC) emerging as a robust biomarker for various brain states including pathological conditions \hlref{Tort2010,Canolty2006}. PAC quantifies the phenomenon whereby the phase of low-frequency oscillations modulates the amplitude of high-frequency activity, reflecting coordinated neural communication across different temporal scales essential for cognitive processes and sensorimotor integration \hlref{Jensen2007,Lakatos2005}. Previous electrophysiological studies have demonstrated measurable PAC alterations in the peri-ictal period, with changes in theta-gamma coupling patterns observed minutes to hours before seizure onset in both animal models and human patients \hlref{Weiss2013,Edakawa2016,Motamedi2013}. However, comprehensive characterization of PAC dynamics across extended timescales has remained limited due to the computational demands of processing continuous, long-term neural recordings and the relative scarcity of datasets spanning months to years of continuous monitoring \hlref{Brinkmann2016}. 

	Existing seizure prediction approaches, including both traditional PAC-based methods and contemporary machine learning techniques, face several critical limitations that have hindered clinical translation. While current trends favor deep learning and complex feature extraction algorithms, these approaches often suffer from poor interpretability, extensive hyperparameter optimization requirements, and limited generalizability across patients and recording conditions \hlref{Rasheed2021,Bou2021}. Traditional PAC-based seizure prediction approaches specifically face computational constraints that have restricted most studies to short recording segments or sparse temporal sampling, preventing comprehensive analysis of PAC evolution across full seizure cycles and missing potentially crucial long-term patterns \hlref{Jacobs2018}. Methodological inconsistencies in PAC quantification, including variations in frequency band definitions, modulation index calculations, and statistical normalization approaches, have produced conflicting results across studies and limited reproducibility \hlref{Aru2015}. Additionally, the majority of previous investigations have relied on relatively small datasets with limited seizure counts per patient, reducing statistical power and generalizability of findings to diverse epilepsy phenotypes \hlref{Freestone2015}. Finally, conventional CPU-based implementations of PAC algorithms exhibit prohibitive computational complexity for real-time applications, with processing times often exceeding recording duration by orders of magnitude, making continuous monitoring clinically impractical \hlref{Canolty2006}. 

	These limitations underscore the critical need for a computationally efficient, comprehensive approach to PAC-based seizure prediction that can leverage large-scale, long-term electrophysiological datasets to identify robust, generalizable biomarkers of seizure susceptibility with sufficient temporal resolution for clinical intervention. 

	To address these challenges, we developed a GPU-accelerated PAC computation framework optimized for massive-scale electrophysiological data analysis, enabling comprehensive characterization of phase-amplitude coupling dynamics across 4.1 terabytes of continuous intracranial electroencephalogram (iEEG) recordings from the NeuroVista dataset. Our approach leverages the Spartan HPC system's distributed GPU architecture with custom PyTorch implementations achieving approximately 100-fold acceleration compared to conventional CPU methods. PAC offers distinct advantages over conventional machine learning biomarkers as it reflects fundamental neural communication mechanisms without requiring hyperparameter optimization, providing inherently interpretable measures of cross-frequency interactions essential for brain functioning \hlref{Tort2010}. We applied this framework to analyze 1,539 Type 1 seizures across 15 patients with drug-resistant focal epilepsy, examining PAC evolution from 24 hours before to 10 minutes after seizure onset using adaptive temporal sampling with 1-minute resolution windows. Statistical robustness was ensured through surrogate-based normalization using 200 circular phase shuffles, generating stable z-scored PAC features within each temporal window that maintain statistical validity across extended monitoring periods. 

	Our analysis revealed systematic modulation of theta-to-beta phase (2-30 Hz) and gamma amplitude (60-180 Hz) coupling patterns occurring 5-60 minutes before seizure onset, achieving balanced accuracy of 0.55±0.04 and area under the receiver operating characteristic curve of 0.58±0.02 for discriminating pre-ictal from interictal states. The GPU-accelerated implementation reduced computation time from years to months while maintaining statistical rigor through surrogate-based significance testing, demonstrating the feasibility of near real-time PAC monitoring with processing latencies under 2 minutes. 

These findings establish PAC as a computationally tractable biomarker for seizure prediction in large-scale clinical datasets, providing a foundation for next-generation implantable advisory systems that could transform epilepsy management from reactive to predictive care. The moderate classification performance observed highlights the need for multimodal biomarker integration while demonstrating sufficient predictive capacity to warrant clinical investigation of PAC-based seizure warning systems. 

Add this text for testing diff
\label{sec:introduction}

%%%% EOF