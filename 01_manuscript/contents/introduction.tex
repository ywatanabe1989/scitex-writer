%% -*- coding: utf-8 -*-

\section{Introduction}

The preparation of scientific manuscripts involves numerous technical challenges that extend beyond the intellectual task of communicating research findings~\cite{Anderson2017_ReviewNeuroscience}. Researchers must navigate complex typesetting systems, manage multiple document versions, coordinate figures and tables across formats, and ensure reproducible compilation environments~\cite{Wilson2015_Neuroscience}. These technical burdens can distract from the primary goal of clear scientific communication and often lead to inconsistencies, formatting errors, and wasted time troubleshooting environment-specific compilation issues.

Traditional approaches to manuscript preparation typically rely on local LaTeX installations, where the specific versions of packages and compilation tools can vary significantly across different machines and over time~\cite{Lee2016_BrainNetworks}. This variability creates reproducibility challenges, particularly in collaborative environments where multiple authors work on different systems~\cite{Garcia2019_CognitiveNeuroscience}. Furthermore, the proliferation of image formats and the need to convert between them for different submission requirements adds another layer of complexity. Researchers often resort to ad-hoc scripts or manual processes to handle these conversions, leading to potential errors and inconsistent results.

Existing solutions have addressed some aspects of this problem~\cite{Thompson2018_SystemsNeuroscience}. Overleaf and similar cloud-based platforms provide consistent compilation environments but require continuous internet connectivity and may not suit all research workflows. Version control systems like Git effectively track changes but require researchers to understand both LaTeX and version control simultaneously. Template repositories exist for various journals, but they typically focus on formatting requirements rather than workflow automation and often duplicate common elements across documents.

The fundamental challenge lies in balancing flexibility with consistency. Researchers need systems that accommodate diverse content types, multiple output documents, and varying journal requirements while maintaining a single source of truth for shared elements like author lists and bibliographies. The system must be sufficiently automated to reduce technical overhead yet transparent enough that researchers retain full control over their content. Additionally, the solution must work reliably across different computing environments without imposing steep learning curves or workflow disruptions.

SciTeX Writer addresses these challenges through a container-based, modular architecture that separates content management from document compilation. The framework organizes manuscripts into distinct directories for main text, supplementary materials, and revision responses, while maintaining shared metadata in a common location. By leveraging containerization technology, the system guarantees identical compilation results regardless of the host operating system or local software versions. Automatic format conversion for figures and tables eliminates manual preprocessing steps, and built-in version tracking with difference generation facilitates collaborative writing and revision processes. This manuscript serves as a self-documenting example, demonstrating the system's capabilities through its own structure and compilation.

%%%% EOF
