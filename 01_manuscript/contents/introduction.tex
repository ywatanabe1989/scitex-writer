%% -*- coding: utf-8 -*-
%% Timestamp: "2025-09-29 12:32:59 (ywatanabe)"
%% File: "/ssh:sp:/home/ywatanabe/proj/neurovista/paper/01_manuscript/contents/introduction.tex"

%% \[START of [0-9]\. .* Statement\]
%% \[\(START\|END\) of [0-9]+\. [^]]*\] *
\section{Introduction}

%% ============================================================
%% ORIGINAL VERSION (PRESERVED AS COMMENTS):
%% ============================================================
%% Epilepsy affects approximately 70 million people worldwide, representing one of the most prevalent neurological disorders characterized by recurrent, unpredictable seizures that fundamentally disrupt daily life and impose substantial societal burden \hlref{WHO2023}. Drug-resistant focal epilepsy, occurring in approximately 30-40\% of epilepsy patients, remains particularly challenging as conventional antiepileptic medications fail to provide adequate seizure control despite optimal medical management \hlref{Kwan2000,Chen2018}. The development of reliable seizure prediction systems represents a critical frontier in epilepsy care, offering the potential to transform patient management from reactive treatment to proactive intervention, thereby reducing seizure-related injuries, psychological burden, and improving quality of life for millions of individuals with drug-resistant epilepsy \hlref{Mormann2007,Kuhlmann2018}.
%%
%% Existing seizure prediction approaches face several critical limitations that have hindered clinical translation. Current machine learning methods achieve [ACCURACY RANGE]\% accuracy across studies, yet clinically successful prediction algorithms remain unavailable due to [SPECIFIC LIMITATIONS: overfitting, limited datasets, computational constraints]. Deep learning approaches, while showing promise in controlled settings, suffer from overfitting with limited patient samples and require extensive hyperparameter optimization that reduces generalizability across diverse epilepsy phenotypes \hlref{RecentMLReviews}. Most published studies analyze [TYPICAL DATASET SIZE] seizures per patient over [TYPICAL DURATION] monitoring periods, insufficient for establishing robust predictive models \hlref{LimitationsStudies}. These limitations underscore the critical need for computationally efficient, interpretable biomarkers requiring minimal hyperparameter tuning.
%%
%% Phase-amplitude coupling (PAC) represents a promising biomarker for seizure prediction, quantifying the modulation of high-frequency amplitude by low-frequency phase oscillations that reflects fundamental neural communication mechanisms \hlref{Tort2010}. Previous electrophysiological studies demonstrated PAC alterations [TIME RANGE] before seizure onset in [NUMBER] patients, achieving [SENSITIVITY VALUES]\% sensitivity and [SPECIFICITY VALUES]\% specificity \hlref{PreviousPACStudies}. However, comprehensive PAC analysis has been limited by computational bottlenecks - conventional CPU implementations require [PROCESSING TIME] per [DATA DURATION], making real-time applications impractical \hlref{ComputationalLimitations}. Long-term stability of PAC features over months-to-years monitoring periods remains unclear, limiting clinical applicability.
%%
%% In this study, we hypothesized that i) PAC computation is accerelated using GPU parallelization and applicable to real-time prediction, ii) PAC features can be reliable biomarkers for seizure prediction. To address these hypotheses, we developed a GPU-accelerated PAC computation framework and applied to one-of-a-kind long-term recording the NeuroVista dataset, and performed exploratory data analyses, seizure type classification tasks, and peseudo-prospective seizure prediction tasks.
%%
%% These findings establish PAC as a computationally tractable biomarker for seizure prediction in large-scale clinical datasets, providing a foundation for next-generation implantable advisory systems that could transform epilepsy management from reactive to predictive care and implications for underlying mechanisms of seizure occurrence.
%% ============================================================
%% END OF ORIGINAL VERSION
%% ============================================================

Epilepsy affects approximately 70 million people worldwide, representing one of the most prevalent neurological disorders characterized by recurrent, unpredictable seizures that fundamentally disrupt daily life and impose substantial societal burden. Drug-resistant focal epilepsy, occurring in approximately 30-40\% of epilepsy patients, remains particularly challenging as conventional antiepileptic medications fail to provide adequate seizure control despite optimal medical management. The development of reliable seizure prediction systems represents a critical frontier in epilepsy care, offering the potential to transform patient management from reactive treatment to proactive intervention, thereby reducing seizure-related injuries, psychological burden, and improving quality of life for millions of individuals with drug-resistant epilepsy \cite{Kuhlmann2018SeizurePA,Freestone2015SeizurePSBF}.

Despite decades of research, existing seizure prediction approaches face several critical limitations that have hindered clinical translation \cite{Natu2022ReviewOEB,Talukder2023ComparativeAOM}. Machine learning and deep learning methods have shown promise in controlled settings, achieving moderate prediction accuracies across diverse datasets \cite{Truong2021SeizureSPV,Dissanayake2020PatientindependentESY,KiralKornek2017EpilepticSPW,Messaoud2021RandomFCR}. However, these approaches often suffer from overfitting with limited patient samples, require extensive hyperparameter optimization, and lack physiological interpretability \cite{Usman2017EpilepticSPH,Hussein2022MultiChannelVTE}. Most published studies analyze tens to hundreds of seizures per patient over weeks to months, insufficient for characterizing long-term feature stability and establishing robust predictive models \cite{Kuhlmann2018SeizurePA,DAlessandro2003EpilepticSPQ}. Furthermore, patient-specific variability in seizure patterns necessitates individualized models that can adapt to evolving brain states \cite{Aldahr2023PatientSpecificPPL,Pinto2021APAP}. These challenges underscore the critical need for computationally efficient, interpretable, and physiologically meaningful biomarkers that require minimal hyperparameter tuning while maintaining stability across extended monitoring periods.

Phase-amplitude coupling (PAC), where the phase of low-frequency oscillations modulates the amplitude of high-frequency activity, represents a promising biomarker that addresses many of these limitations \cite{Tort2010MeasuringPCE,Canolty2010TheFRC}. PAC quantifies fundamental neural communication mechanisms across temporal and spatial scales, reflecting the coordination of distributed network dynamics through cross-frequency interactions \cite{Aru2014UntanglingCCD,Hlsemann2019QuantificationOPA}. Recent electrophysiological studies have demonstrated systematic PAC alterations minutes to hours before seizure onset, suggesting progressive preictal state transitions rather than sudden, unpredictable events \cite{Kapoor2022EpilepticSPJ,Detti2020EEGSAC,Zhang2017TemporalspatialCOAG,Miao2021SeizureOZBG}. Theta-gamma coupling, in particular, has emerged as a critical marker of epileptogenic network dynamics \cite{Ahn2022TheFIT,Radiske2020CrossFrequencyPCAR,Ponzi2023ThetagammaPAAT}. However, comprehensive PAC analysis of large-scale, long-term datasets has been limited by substantial computational bottlenecks—conventional CPU-based implementations require days to weeks for processing hours of multi-channel recordings with multiple frequency band combinations, making real-time clinical applications impractical \cite{Combrisson2020TensorpacAOAH,MartnezCancino2020ComputingPABK}. Additionally, the long-term stability of PAC features across months-to-years monitoring periods, critical for reliable clinical deployment, remains largely unexplored \cite{Rakowska2021LongTEQ}.

In this study, we address these computational and temporal challenges through two key hypotheses: (i) GPU-accelerated parallel computation can enable large-scale PAC analysis with sufficient speed for near-real-time seizure prediction applications, and (ii) PAC features derived from long-term continuous recordings provide reliable and interpretable biomarkers for seizure forecasting. To test these hypotheses, we developed a GPU-accelerated PAC computation framework (gPAC) and applied it to the NeuroVista dataset—one of the largest continuous intracranial EEG datasets available, comprising 4.1 TB of data from 15 patients with drug-resistant focal epilepsy monitored over 6 months to 2 years \cite{Kuhlmann2018SeizurePA}. We performed systematic exploratory analyses of PAC dynamics across multiple temporal scales (24 hours to minutes before seizure onset), characterized patient-specific patterns, and evaluated pseudo-prospective seizure prediction performance using PAC-derived statistical features.

Our findings demonstrate that GPU acceleration achieves approximately 100-fold speed improvements over conventional CPU methods, enabling comprehensive PAC analysis of this unprecedented dataset scale within practical timeframes. We identified distinct preictal PAC signatures in theta-to-beta phase and gamma amplitude coupling that systematically modulated 5-60 minutes before seizure onset, achieving balanced accuracy of 0.55±0.04 and ROC-AUC of 0.58±0.02 for discriminating preictal from interictal states. These results establish PAC as a computationally tractable and physiologically interpretable biomarker for seizure prediction in large-scale clinical datasets, providing a foundation for next-generation implantable seizure advisory systems that could transform epilepsy management from reactive to predictive care.

\label{sec:introduction}

%%%% EOF