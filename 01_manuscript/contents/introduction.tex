%% -*- coding: utf-8 -*-
%% Timestamp: "2025-09-29 12:32:59 (ywatanabe)"
%% File: "/ssh:sp:/home/ywatanabe/proj/neurovista/paper/01_manuscript/contents/introduction.tex"

%% \[START of [0-9]\. .* Statement\]
%% \[\(START\|END\) of [0-9]+\. [^]]*\] *
\section{Introduction}

Epilepsy affects approximately 70 million people worldwide, representing one of the most prevalent neurological disorders characterized by recurrent, unpredictable seizures that fundamentally disrupt daily life and impose substantial societal burden \hlref{WHO2023}. Drug-resistant focal epilepsy, occurring in approximately 30-40\% of epilepsy patients, remains particularly challenging as conventional antiepileptic medications fail to provide adequate seizure control despite optimal medical management \hlref{Kwan2000,Chen2018}. The development of reliable seizure prediction systems represents a critical frontier in epilepsy care, offering the potential to transform patient management from reactive treatment to proactive intervention, thereby reducing seizure-related injuries, psychological burden, and improving quality of life for millions of individuals with drug-resistant epilepsy \hlref{Mormann2007,Kuhlmann2018}.

Existing seizure prediction approaches face several critical limitations that have hindered clinical translation. Current machine learning methods achieve [ACCURACY RANGE]\% accuracy across studies, yet clinically successful prediction algorithms remain unavailable due to [SPECIFIC LIMITATIONS: overfitting, limited datasets, computational constraints]. Deep learning approaches, while showing promise in controlled settings, suffer from overfitting with limited patient samples and require extensive hyperparameter optimization that reduces generalizability across diverse epilepsy phenotypes \hlref{RecentMLReviews}. Most published studies analyze [TYPICAL DATASET SIZE] seizures per patient over [TYPICAL DURATION] monitoring periods, insufficient for establishing robust predictive models \hlref{LimitationsStudies}. These limitations underscore the critical need for computationally efficient, interpretable biomarkers requiring minimal hyperparameter tuning.

Phase-amplitude coupling (PAC) represents a promising biomarker for seizure prediction, quantifying the modulation of high-frequency amplitude by low-frequency phase oscillations that reflects fundamental neural communication mechanisms \hlref{Tort2010}. Previous electrophysiological studies demonstrated PAC alterations [TIME RANGE] before seizure onset in [NUMBER] patients, achieving [SENSITIVITY VALUES]\% sensitivity and [SPECIFICITY VALUES]\% specificity \hlref{PreviousPACStudies}. However, comprehensive PAC analysis has been limited by computational bottlenecks - conventional CPU implementations require [PROCESSING TIME] per [DATA DURATION], making real-time applications impractical \hlref{ComputationalLimitations}. Long-term stability of PAC features over months-to-years monitoring periods remains unclear, limiting clinical applicability.

In this study, we hypothesized that i) PAC computation is accerelated using GPU parallelization and applicable to real-time prediction, ii) PAC features can be reliable biomarkers for seizure prediction. To address these hypotheses, we developed a GPU-accelerated PAC computation framework and applied to one-of-a-kind long-term recording the NeuroVista dataset, and performed exploratory data analyses, seizure type classification tasks, and peseudo-prospective seizure prediction tasks.

These findings establish PAC as a computationally tractable biomarker for seizure prediction in large-scale clinical datasets, providing a foundation for next-generation implantable advisory systems that could transform epilepsy management from reactive to predictive care and implications for underlying mechanisms of seizure occurrence.

\label{sec:introduction}

%%%% EOF