%% -*- coding: utf-8 -*-
%% Timestamp: "2025-09-29 10:48:53 (ywatanabe)"
%% File: "/ssh:sp:/home/ywatanabe/proj/neurovista/paper/01_manuscript/contents/discussion.tex"

\section{Discussion}

\subsection{Principal Findings}
Our comprehensive analysis of phase-amplitude coupling in long-term iEEG recordings from the NeuroVista dataset reveals several key insights into seizure dynamics and prediction \cite{Kuhlmann2018SeizurePA,Freestone2015SeizurePSBF}. The observation of progressive PAC enhancement during the pre-ictal period, particularly in theta/alpha-gamma coupling \cite{Ahn2022TheFIT,Radiske2020CrossFrequencyPCAR,Ponzi2023ThetagammaPAAT}, supports the hypothesis that seizures arise from a gradual transition in network dynamics rather than sudden, unpredictable events. This finding has important implications for the development of seizure warning systems and our understanding of epileptogenesis \cite{Canolty2010TheFRC,Tort2010MeasuringPCE}.

\subsection{Mechanisms of PAC in Seizure Generation}
The dominant theta-gamma PAC pattern observed in our study aligns with established theories of cortical information processing and pathological synchronization in epilepsy \cite{Canolty2010TheFRC,Aru2014UntanglingCCD,Bergmann2018PhaseAmplitudeCAN}. Low-frequency oscillations (theta/alpha) are thought to coordinate activity across distributed neural populations, while high-frequency gamma oscillations reflect local processing and neuronal firing \cite{Tort2010MeasuringPCE,Hlsemann2019QuantificationOPA}. PAC serves as a fundamental mechanism linking memory processing, synaptic plasticity, and distributed neural communication \cite{Bergmann2018PhaseAmplitudeCAN}, with implications for understanding how pathological network states emerge in epilepsy. From a theoretical perspective, these cross-frequency interactions reflect hierarchical information processing and predictive coding mechanisms that become dysfunctional during seizure generation \cite{Friston2020GenerativeMLB}. The increased coupling strength during pre-ictal periods suggests a pathological enhancement of this coordination mechanism \cite{Zhang2017TemporalspatialCOAG,Miao2021SeizureOZBG}, potentially reflecting:

1. \textbf{Network hypersynchronization}: Progressive recruitment of neuronal populations into synchronized states \cite{Ahn2022TheFIT}
2. \textbf{Excitation-inhibition imbalance}: Failure of inhibitory control mechanisms leading to runaway excitation \cite{Radiske2020CrossFrequencyPCAR}
3. \textbf{Critical state transitions}: Approach to a bifurcation point in neural dynamics \cite{Ponzi2023ThetagammaPAAT}

The spatial spread of PAC elevation from focal to distributed patterns mirrors the clinical evolution from focal onset to secondary generalization, providing an electrophysiological correlate of seizure propagation.

\subsection{Clinical Translation and Implementation}
The robust discrimination between interictal and pre-ictal states using PAC features (mean AUC > 0.XX) demonstrates clinical potential for seizure forecasting \cite{Kuhlmann2018SeizurePA,Freestone2015SeizurePSBF}. Key advantages of PAC-based prediction include:

\textbf{Computational efficiency}: Our PyTorch-accelerated implementation \cite{Combrisson2020TensorpacAOAH} enables real-time processing suitable for implantable devices with limited computational resources. The modulation index calculation \cite{Tort2010MeasuringPCE,Hlsemann2019QuantificationOPA} requires only basic signal processing operations implementable in low-power hardware.

\textbf{Interpretability}: Unlike black-box machine learning approaches \cite{Natu2022ReviewOEB,Dissanayake2020PatientindependentESY}, PAC provides physiologically meaningful features that clinicians can interpret in the context of known seizure mechanisms \cite{Canolty2010TheFRC,Aru2014UntanglingCCD}.

\textbf{Stability}: The use of z-scored PAC values normalized to patient-specific baselines accounts for inter-individual variability and electrode placement differences \cite{Aldahr2023PatientSpecificPPL,Pinto2021APAP}, improving generalizability.

\subsection{Limitations and Methodological Considerations}

Several limitations should be considered when interpreting our results:

\textbf{Patient selection bias}: The NeuroVista trial \cite{Kuhlmann2018SeizurePA} included patients with focal epilepsy suitable for implantation, potentially limiting generalizability to other epilepsy types. The requirement for >10 lead seizures during the training period may select for patients with more predictable seizure patterns.

\textbf{Electrode coverage}: The 16-channel arrays provide limited spatial sampling compared to high-density recordings. PAC patterns in regions distant from electrodes may be missed, potentially explaining performance variability across patients \cite{Hussein2019HumanIEAQ}.

\textbf{Stationarity assumptions}: Our analysis assumes relative stationarity of PAC patterns over the recording period \cite{Rakowska2021LongTEQ}. However, factors such as medication changes, sleep deprivation, or device-tissue interface evolution could introduce non-stationarities affecting long-term performance.

\textbf{Multiple comparisons}: Despite Bonferroni correction, the analysis of multiple frequency band pairs and time windows increases false discovery risk \cite{Jensen2016DiscriminatingVFR,Aru2014UntanglingCCD}. Future studies should consider false discovery rate (FDR) control methods \cite{PintoOrellana2023StatisticalIFF}.

% \subsection{Comparison with Previous Studies}
% Our findings extend previous PAC studies in epilepsy in several ways:

% Compared to Edakawa et al. (2016), who reported high-frequency oscillation (HFO) coupling in shorter recordings, our long-term analysis reveals slower dynamics and circadian modulation not observable in brief recordings.

% Unlike Amiri et al. (2016), who focused on single seizures, our large-scale analysis (>1,500 seizures) provides statistical power to identify consistent patterns and account for seizure-to-seizure variability.

% Our improvement over traditional features (+XX\% AUC vs spectral power) is consistent with Jacobs et al. (2018), though our continuous monitoring approach offers advantages for real-world implementation.

% \subsection{Future Directions}

% Several avenues warrant further investigation:

% \textbf{Adaptive algorithms}: Implementing online learning to adapt to non-stationary dynamics and improve long-term stability.

% \textbf{Multi-modal integration}: Combining PAC with other biomarkers (HFOs, connectivity measures, behavioral sensors) may further improve prediction accuracy.

% \textbf{Mechanism-based interventions}: Understanding PAC dynamics could inform closed-loop stimulation protocols that disrupt pre-ictal buildup.

% \textbf{Generalization to other epilepsy types}: Extending analysis to generalized epilepsies and pediatric populations.

% \textbf{Optimization of electrode placement}: Using PAC maps to guide surgical planning and optimize electrode positioning for monitoring and stimulation.

% \subsection{Implications for Seizure Advisory Systems}
% The successful pseudo-prospective evaluation demonstrates feasibility for clinical implementation. A PAC-based advisory system could provide:
% - Early warnings (XX minutes) for XX\% of seizures
% - Low false positive rates (XX per day) minimizing alarm fatigue
% - Personalized risk assessments based on individual PAC patterns

\subsection{Computational Feasibility for Real-Time Implementation}
Our GPU-accelerated PAC computation framework \cite{Combrisson2020TensorpacAOAH} achieved approximately 100-fold speed improvements over conventional CPU-based implementations, reducing total processing time for the complete NeuroVista dataset from an estimated \hl{[XX]} years to \hl{[XX.X]} months on distributed GPU infrastructure. For real-time deployment scenarios, single-segment processing (1-minute window) demonstrated total latency of approximately 110 seconds, comprising:
\begin{itemize}
\item Data acquisition and transfer: \hl{[XX]} seconds
\item PAC computation (625 frequency pairs × 16 channels with 200 surrogates): \hl{[XX]} seconds
\item Feature extraction (17 statistical metrics): \hl{[XX]} seconds
\item Classification inference: \hl{[XX]} seconds
\item Result transmission and display: \hl{[XX]} seconds
\item Buffer time for computational variability: \hl{[XX]} seconds
\end{itemize}

This processing latency of <2 minutes per 1-minute data segment enables near-real-time seizure risk assessment suitable for implantable advisory systems \cite{Kuhlmann2018SeizurePA}, where warnings 5-60 minutes before seizure onset would provide actionable intervention windows. Further optimization through model pruning, reduced surrogate iterations for online analysis, and hardware-specific implementations could potentially reduce latency to <30 seconds, approaching true real-time performance requirements for closed-loop therapeutic applications.

% Integration with existing seizure detection algorithms and consideration of circadian patterns could further enhance performance.

% \subsection{Conclusion}
% This comprehensive analysis of phase-amplitude coupling in long-term iEEG recordings provides robust evidence for PAC as a biomarker of seizure susceptibility. The progressive enhancement of theta-gamma coupling during pre-ictal periods offers both mechanistic insights and practical applications for seizure prediction. Our optimized computational pipeline and systematic database management approach enable scalable analysis applicable to next-generation implantable devices. While patient-specific variability remains a challenge, the consistent patterns across multiple seizures and patients support the clinical translation of PAC-based seizure forecasting systems. Future work should focus on real-time implementation, algorithm adaptation, and integration with therapeutic interventions to improve quality of life for patients with drug-resistant epilepsy.


\label{sec:discussion}

%%%% EOF