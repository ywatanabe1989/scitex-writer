%% -*- coding: utf-8 -*-

\section{Discussion}

Interpret your findings and place them in the broader scientific context. Replace this placeholder text with your discussion.

\subsection{Principal Findings}

Begin by restating your main findings without simply repeating the Results section. Explain what your results mean and how they address your research questions or hypotheses. For example: ``Our study demonstrates that [main finding], which supports the hypothesis that [interpretation].''

\subsection{Comparison with Previous Work}

Compare your findings with existing literature:
\begin{itemize}
    \item How do your results confirm or contradict previous studies \cite{example_study_2021}?
    \item What novel contributions does your work provide?
    \item How do you reconcile any discrepancies with prior research?
\end{itemize}

\subsection{Mechanisms and Implications}

Discuss the underlying mechanisms or theoretical implications of your findings. Consider:
\begin{itemize}
    \item Biological, physical, or theoretical mechanisms
    \item Broader implications for the field
    \item Potential applications of your findings
    \item Future research directions
\end{itemize}

\subsection{Limitations}

Acknowledge the limitations of your study honestly:
\begin{itemize}
    \item Sample size or selection limitations
    \item Methodological constraints
    \item Alternative explanations for your findings
    \item Generalizability considerations
\end{itemize}

\subsection{Conclusions}

Conclude with a concise summary of your key findings and their significance. Avoid introducing new information or overstating your conclusions. End with a forward-looking statement about future research directions or practical implications.

%%%% EOF
