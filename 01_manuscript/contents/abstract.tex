%% -*- coding: utf-8 -*-
%% Timestamp: "2025-09-27 20:14:56 (ywatanabe)"
%% File: "/ssh:sp:/home/ywatanabe/proj/neurovista/paper/01_manuscript/contents/abstract.tex"
\begin{abstract}
  \pdfbookmark[1]{Abstract}{abstract}

%% ============================================================
%% ORIGINAL VERSION (PRESERVED AS COMMENTS):
%% ============================================================
%% Neural oscillations exhibit cross-frequency interactions that are fundamental to brain function and disrupted in neurological disorders. Phase-amplitude coupling (PAC), where the phase of low-frequency oscillations modulates the amplitude of high-frequency activity, serves as a biomarker for various brain states including epileptic seizures. Previous studies have demonstrated PAC changes around seizure events, but characterization across extended timescales remains limited availability of long-term recording data and high computational requirements in PAC computation. The challenge of processing continuous, long-term neural recordings has hindered the development of reliable seizure prediction systems. Here we show that the combination of the NeuroVista dataset and our GPU-accelerated PAC computation system enables ...
%% comprehensive analysis of 4.1 TB of continuous intracranial electroencephalogram data from 15 patients with focal epilepsy (NeuroVista dataset), encompassing 1,539 seizures over monitoring periods ranging from 6 months to 2 years. We found distinct PAC signatures between theta-to-beta phase (2-30 Hz, 25 bands) and gamma amplitude (60-180 Hz, 25 bands) that systematically modulated 5-60 minutes before seizure onset, achieving balanced accuracy of 0.55±0.04 and ROC-AUC of 0.58±0.02 for discriminating pre-ictal from interictal states. Our GPU-accelerated implementation achieved 100-fold speed improvements compared to conventional CPU-based methods, reducing computation time from years to months and potentially enabling real-time PAC monitoring with less than 2-minute processing latency. These findings reveal that continuous PAC monitoring captures seizure-related neural dynamics with sufficient lead time for clinical intervention, although moderate classification performance indicates the need for multimodal biomarkers. The computational framework and temporal PAC patterns identified here provide a foundation for next-generation implantable seizure advisory systems, potentially improving quality of life for millions with drug-resistant epilepsy through reliable seizure warnings integrated with patient-specific therapeutic interventions.
%% ============================================================
%% END OF ORIGINAL VERSION
%% ============================================================

Neural oscillations exhibit cross-frequency interactions that coordinate information processing across temporal and spatial scales, with disruptions implicated in neurological disorders including epilepsy. Phase-amplitude coupling (PAC), quantifying how low-frequency phase modulates high-frequency amplitude, has emerged as a promising biomarker for epileptic state transitions, reflecting fundamental cross-frequency neural communication mechanisms. While recent studies demonstrate systematic PAC alterations surrounding seizure events, comprehensive characterization across extended timescales has been limited by computational constraints and scarcity of long-term continuous recordings. The inability to efficiently process large-scale datasets has hindered development of reliable seizure prediction systems. Here we address these challenges through GPU-accelerated PAC computation applied to the NeuroVista dataset—comprising 4.1 TB of continuous intracranial electroencephalogram recordings from 15 patients with drug-resistant focal epilepsy monitored over 6 months to 2 years, encompassing 1,539 Type 1 clinical seizures. We computed PAC between 25 phase bands (2-30 Hz) and 25 amplitude bands (60-180 Hz) across 16 channels, extracting 17 statistical features from resulting PAC distributions at 127 temporal sampling points spanning 24 hours before to 10 minutes after seizure onset. \hl{We identified systematic preictal PAC modulation beginning 5-60 minutes before seizure onset, with theta-to-beta phase and gamma amplitude coupling showing the strongest discriminative power}. Pseudo-prospective seizure prediction achieved balanced accuracy of \hl{[XX.X±XX.X]\%} and ROC-AUC of \hl{[0.XX±0.XX]} for discriminating preictal from interictal states, with patient-specific variability reflecting individual seizure dynamics. Our GPU-accelerated implementation achieved approximately \hl{100-fold} speed improvements over conventional CPU methods, reducing processing time from years to months and enabling near-real-time analysis with \hl{<2-minute} latency per data segment. These findings establish PAC as a computationally tractable and physiologically interpretable biomarker for seizure prediction, providing a foundation for next-generation implantable seizure advisory systems that could transform epilepsy management from reactive to predictive care.

\end{abstract}

%%%% EOF