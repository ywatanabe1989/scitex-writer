% ==============================================================================
% SciTeX Writer 2.6.3 (https://scitex.ai)
% LaTeX compilation engine: auto
% Compiled: 2026-02-11 18:23:39
% Source: 01_manuscript/base.tex
% ==============================================================================

%% -*- coding: utf-8 -*-
%% Timestamp: "2025-09-27 22:21:42 (ywatanabe)"
%% File: "/ssh:sp:/home/ywatanabe/proj/neurovista/paper/01_manuscript/base.tex"
\UseRawInputEncoding

%% ----------------------------------------
%% SETTINGS
%% ----------------------------------------

% ======================================================================
% File: ./01_manuscript/contents/latex_styles/columns.tex
% ======================================================================
%% -*- coding: utf-8 -*-
%% File: 00_shared/latex_styles/columns.tex

%% --- Columns ---
%% \documentclass[final,3p,times,twocolumn]{elsarticle} %% Use it for submission
%% Use the options 1p,twocolumn; 3p; 3p,twocolumn; 5p; or 5p,twocolumn
%% for a journal layout:
%% \documentclass[final,1p,times]{elsarticle}
%% \documentclass[final,1p,times,twocolumn]{elsarticle}
%% \documentclass[final,3p,times]{elsarticle}
%% \documentclass[final,3p,times,twocolumn]{elsarticle}
%% \documentclass[final,5p,times]{elsarticle}
%% \documentclass[final,5p,times,twocolumn]{elsarticle}
\documentclass[preprint,review,12pt]{elsarticle}

%%%% EOF


% ======================================================================
% File: ./01_manuscript/contents/latex_styles/packages.tex
% ======================================================================
%% -*- coding: utf-8 -*-
%% File: 00_shared/latex_styles/packages.tex
%% -*- coding: utf-8 -*-
%% Timestamp: "2025-09-27 16:01:16 (ywatanabe)"

%% Language and encoding
\usepackage[english]{babel}
\usepackage[T1]{fontenc}
\usepackage[utf8]{inputenc}

%% Colors (load early to avoid option clashes with tikz, pgfplots, tcolorbox)
% Include all common color options: table (for colortbl), svgnames (for tcolorbox)
\usepackage[table,svgnames]{xcolor}

%% Mathematics
\usepackage{amsmath, amssymb, amsthm}
\usepackage{siunitx}
\sisetup{round-mode=figures,round-precision=3}

%% Graphics and figures
\usepackage{graphicx}
\usepackage{tikz}
\usepackage{pgfplots, pgfplotstable}
\usetikzlibrary{positioning,shapes,arrows,fit,calc,graphs,graphs.standard}

%% Tables
\usepackage{booktabs, colortbl, longtable, supertabular, tabularx, xltabular}
\usepackage{csvsimple, makecell}

%% Table formatting
\renewcommand\theadfont{\bfseries}
\renewcommand\theadalign{c}
\newcolumntype{C}[1]{>{\centering\arraybackslash}m{#1}}
\renewcommand{\arraystretch}{1.5}
\definecolor{lightgray}{gray}{0.95}

%% Layout and geometry
\usepackage[pass]{geometry}
\usepackage{pdflscape, indentfirst, calc}
\usepackage{titlesec}  % For custom section formatting

%% Captions and references
\usepackage[margin=10pt,font=small,labelfont=bf,labelsep=endash]{caption}
\usepackage[numbers]{natbib}  % numbers: numeric citations [1], [2]
\setcitestyle{sort=false}     % Preserve citation order as written
\usepackage{hyperref}

%% Document features
\usepackage{accsupp, lineno, bashful, lipsum}

%% Visual enhancements
\usepackage[most]{tcolorbox}

%% External references
\usepackage{xr-hyper}

%% EOF

%%%% EOF


% ======================================================================
% File: ./01_manuscript/contents/latex_styles/linker.tex
% ======================================================================
%% -*- coding: utf-8 -*-
%% File: 00_shared/latex_styles/linker.tex

%% --- Linker for supplemtal material ---
\usepackage{xr}
\makeatletter
\newcommand*{\addFileDependency}[1]{% argument=file name and extension
  \typeout{(#1)}
  \@addtofilelist{#1}
  \IfFileExists{#1}{}{\typeout{No file #1.}}
}
\makeatother

\newcommand*{\link}[2][]{%
    \externaldocument[#1]{#2}%
    \addFileDependency{#2.tex}%
    \addFileDependency{#2.aux}%
}

%%%% EOF


% ======================================================================
% File: ./01_manuscript/contents/latex_styles/formatting.tex
% ======================================================================
%% -*- coding: utf-8 -*-
%% File: 00_shared/latex_styles/formatting.tex

%% --- Journal name header (show journal name as-is, without "Preprint submitted to" prefix) ---
\makeatletter
\def\ps@pprintTitle{%
  \def\@oddhead{\reset@font\footnotesize\itshape\@journal\hfill}%
  \let\@evenhead\@oddhead
  \def\@oddfoot{\reset@font\hfil\thepage\hfil}%
  \let\@evenfoot\@oddfoot}
\makeatother

%% --- Image width ---
\newlength{\imagewidth}
\newlength{\imagescale}

%% --- Line numbers ---
\linespread{1.2}
\linenumbers

%% --- Colors ---
\definecolor{GreenBG}{rgb}{0,1,0}
\definecolor{RedBG}{rgb}{1,0,0}

%% --- Highlight boxes ---
\newtcbox{\greenhighlight}[1][]{on line,colframe=GreenBG,colback=GreenBG!50!white,boxrule=0pt,arc=0pt,boxsep=0pt,left=1pt,right=1pt,top=2pt,bottom=2pt,tcbox raise base}
\newtcbox{\redhighlight}[1][]{on line,colframe=RedBG,colback=RedBG!50!white,boxrule=0pt,arc=0pt,boxsep=0pt,left=1pt,right=1pt,top=2pt,bottom=2pt,tcbox raise base}

\newcommand{\REDSTARTS}{\color{red}}
\newcommand{\REDENDS}{\color{black}}
\newcommand{\GREENSTARTS}{\color{green}}
\newcommand{\GREENENDS}{\color{black}}

%% --- Word count ---
\newread\wordcount
\newcommand\readwordcount[1]{%
\openin\wordcount=#1
\read\wordcount to \thewordcount
\closein\wordcount
\begingroup\sisetup{round-mode=none}\num{\thewordcount}\endgroup%
}

%% --- Text highlighting ---
\usepackage{soul}
\sethlcolor{yellow}

%% --- Reference handling ---
\usepackage{refcount}

\let\oldref\ref
\newcommand{\hlref}[1]{%
  \ifnum\getrefnumber{#1}=0
    \colorbox{yellow}{\ref*{#1}}%  % Use colorbox for references (no line break needed)
  \else
    \ref{#1}%
  \fi
}

% To add an 'S' prefixes to a reference
\newcommand*\sref[1]{S\hlref{#1}}
\newcommand*\sfref[1]{Supplementary Figure S\hlref{#1}}
\newcommand*\stref[1]{Supplementary Table S\hlref{#1}}
\newcommand*\smref[1]{Supplementary Materials S\hlref{#1}}

%%%% EOF

\link[supple-]{./02_supplementary/supplementary}

%% ----------------------------------------
%% JOURNAL NAME
%% ----------------------------------------

% ======================================================================
% File: ./01_manuscript/contents/journal_name.tex
% ======================================================================
\journal{Journal Name Here}



%% ----------------------------------------
%% START of DOCUMENT
%% ----------------------------------------

% Dark mode styling (inlined at compile time)
%% -*- coding: utf-8 -*-
%% Timestamp: "2025-11-10 01:30:00 (ywatanabe)"
%% File: ./00_shared/latex_styles/dark_mode.tex
%% Description: Dark mode styling for scientific manuscripts

%% Dark mode configuration
%% Background: Monaco/VS Code editor (#1E1E1E)
%% Text: VS Code default foreground (#D4D4D4)
%% Figures: unchanged (preserve original colors)

% Set page and text colors - matches Monaco/VS Code dark theme
\definecolor{MonacoBg}{HTML}{1E1E1E}
\definecolor{MonacoFg}{HTML}{F0F0F0}
\pagecolor{MonacoBg}
\color{MonacoFg}

% Link text colors for dark mode (colorlinks=true)
\definecolor{DarkGreen}{HTML}{90C695}     % Internal links (Figure/Table): muted sage
\definecolor{DarkCyan}{HTML}{87CEEB}      % Citations/references: sky blue
\definecolor{DarkOrange}{HTML}{DEB887}    % External URLs: burlywood

% Hyperlink colors set via hypersetup (injected after \begin{document})

% Override hlref colorbox for dark mode (no yellow box on dark background)
\renewcommand{\hlref}[1]{%
  \ifnum\getrefnumber{#1}=0
    {\color{DarkGreen}\ref*{#1}}%
  \else
    \ref{#1}%
  \fi
}

% Section titles: same color as body text, just bold
\titleformat{\section}
  {\normalfont\Large\bfseries\color{MonacoFg}}{\thesection}{1em}{}
\titleformat{\subsection}
  {\normalfont\large\bfseries\color{MonacoFg}}{\thesubsection}{1em}{}
\titleformat{\subsubsection}
  {\normalfont\normalsize\bfseries\color{MonacoFg}}{\thesubsubsection}{1em}{}

% Caption labels: same as body text, left-aligned
\captionsetup{
  labelfont={bf,color=MonacoFg},
  textfont={color=MonacoFg},
  justification=raggedright,
  singlelinecheck=false
}

% Adjust table colors for dark mode
\definecolor{lightgray}{gray}{0.2}  % Darker "light" gray for table rows

% Override color commands to work well in dark mode
\renewcommand{\REDENDS}{\color{white}}    % Reset to white instead of black
\renewcommand{\GREENENDS}{\color{white}}  % Reset to white instead of black

% Adjust highlight boxes for dark mode
\newtcbox{\darkgreenhighlight}[1][]{on line,colframe=green!70!white,colback=green!20!black,boxrule=0pt,arc=0pt,boxsep=0pt,left=1pt,right=1pt,top=2pt,bottom=2pt,tcbox raise base}
\newtcbox{\darkredhighlight}[1][]{on line,colframe=red!70!white,colback=red!20!black,boxrule=0pt,arc=0pt,boxsep=0pt,left=1pt,right=1pt,top=2pt,bottom=2pt,tcbox raise base}

% Adjust soul highlighting for dark mode
\sethlcolor{yellow!30!black}  % Darker yellow highlight

%%%% EOF

\begin{document}

%% ----------------------------------------
%% Frontmatter
%% ----------------------------------------
\begin{frontmatter}

% ======================================================================
% File: ./01_manuscript/contents/highlights.tex
% ======================================================================
%% -*- coding: utf-8 -*-
%% \begin{highlights}
%% \pdfbookmark[1]{Highlights}{highlights}

%% \item Highlight \#1

%% \item Highlight \#2

%% \item Highlight \#3

%% \end{highlights}

%%%% EOF



% ======================================================================
% File: ./01_manuscript/contents/title.tex
% ======================================================================
%% -*- coding: utf-8 -*-
\title{
Your Manuscript Title Here
}

%%%% EOF



% ======================================================================
% File: ./01_manuscript/contents/authors.tex
% ======================================================================
%% -*- coding: utf-8 -*-
\author[1]{First Author\corref{cor1}}
\author[2]{Second Author}
\author[3]{Third Author}

\address[1]{First Institution, Department, City, Country}
\address[2]{Second Institution, Department, City, Country}
\address[3]{Third Institution, Department, City, Country}

\cortext[cor1]{Corresponding author. Email: your.email@institution.edu}

%%%% EOF



% ======================================================================
% File: ./01_manuscript/contents/graphical_abstract.tex
% ======================================================================
%%Graphical abstract
%\pdfbookmark[1]{Graphical Abstract}{graphicalabstract}
%\begin{graphicalabstract}
%\includegraphics{grabs}
%\end{graphicalabstract}



% ======================================================================
% File: ./01_manuscript/contents/abstract.tex
% ======================================================================
%% -*- coding: utf-8 -*-
\begin{abstract}
  \pdfbookmark[1]{Abstract}{abstract}

Replace this text with your manuscript abstract. Typically 150--250 words summarizing objectives, methods, key findings, and conclusions.

\vspace{1em}
\end{abstract}

%%%% EOF



% ======================================================================
% File: ./01_manuscript/contents/keywords.tex
% ======================================================================
% \pdfbookmark[1]{Keywords}{keywords}
\begin{keyword}
keyword one \sep keyword two \sep keyword three \sep keyword four \sep keyword five
\end{keyword}


\end{frontmatter}

%% ----------------------------------------
%% Word Counter
%% ----------------------------------------

% ======================================================================
% File: ./01_manuscript/contents/wordcount.tex
% ======================================================================
%% -*- coding: utf-8 -*-
%% File: 01_manuscript/contents/wordcount.tex
\noindent\small
\readwordcount{./01_manuscript/contents/wordcounts/figure_count.txt} figures,
\readwordcount{./01_manuscript/contents/wordcounts/table_count.txt} tables,
\readwordcount{./01_manuscript/contents/wordcounts/abstract_count.txt} words for abstract, and
\readwordcount{./01_manuscript/contents/wordcounts/imrd_count.txt} words for main text
(\readwordcount{./01_manuscript/contents/wordcounts/introduction_count.txt} for introduction,
\readwordcount{./01_manuscript/contents/wordcounts/methods_count.txt} for methods,
\readwordcount{./01_manuscript/contents/wordcounts/results_count.txt} for results,
\readwordcount{./01_manuscript/contents/wordcounts/discussion_count.txt} for discussion)
\normalsize

%%%% EOF


%% ----------------------------------------
%% INTRODUCTION
%% ----------------------------------------

% ======================================================================
% File: ./01_manuscript/contents/introduction.tex
% ======================================================================
%% -*- coding: utf-8 -*-

\section{Introduction}

Replace this with your introduction. Establish context, review relevant work \cite{example_reference_2020}, identify gaps, and state your objectives.

%%%% EOF



%% ----------------------------------------
%% METHODS
%% ----------------------------------------

% ======================================================================
% File: ./01_manuscript/contents/methods.tex
% ======================================================================
%% -*- coding: utf-8 -*-

\section{Methods}

Replace this with your methods. Describe study design, data collection, and analysis procedures \cite{example_method_2019}. Provide enough detail for reproducibility.

%%%% EOF



%% ----------------------------------------
%% RESULTS
%% ----------------------------------------

% ======================================================================
% File: ./01_manuscript/contents/results.tex
% ======================================================================
%% -*- coding: utf-8 -*-

\section{Results}

Replace this with your results. Present findings with references to figures (Figure~\ref{fig:01_example_figure}) and tables (Table~\ref{tab:01_example_table}).

%%%% EOF



%% ----------------------------------------
%% DISCUSSION
%% ----------------------------------------

% ======================================================================
% File: ./01_manuscript/contents/discussion.tex
% ======================================================================
%% -*- coding: utf-8 -*-

\section{Discussion}

Replace this with your discussion. Interpret findings, compare with previous work, discuss limitations, and state conclusions.

%%%% EOF



%% ----------------------------------------
%% REFERENCE STYLES
%% ----------------------------------------
\pdfbookmark[1]{References}{references}
% Note: Path without ./ prefix allows bibtex to use BIBINPUTS search path
\bibliography{01_manuscript/contents/bibliography}

% ======================================================================
% File: ./01_manuscript/contents/latex_styles/bibliography.tex
% ======================================================================
%% -*- coding: utf-8 -*-
%% File: 00_shared/latex_styles/bibliography.tex

%% ============================================================================
%% BIBLIOGRAPHY STYLE CONFIGURATION
%% ============================================================================

%% ----------------------------------------------------------------------------
%% OPTION 1: NUMBERED CITATIONS (Order of Appearance) - CURRENTLY ACTIVE
%% ----------------------------------------------------------------------------
%% Description: Citations numbered [1], [2], [3]... in the order they first
%%              appear in the manuscript
%% Sorting: By first citation order (NOT alphabetical)
%% Example: \cite{Tort2010,Canolty2010} → [1, 2] (if these are first citations)
%% Commands: \cite{key} → [1]
%%           \cite{key1,key2} → [1, 2]
%% Best for: Most scientific journals, clear citation tracking
%% Compatible with: natbib package
\bibliographystyle{unsrtnat}

%% ----------------------------------------------------------------------------
%% OPTION 2: NUMBERED CITATIONS (Alphabetical by Author)
%% ----------------------------------------------------------------------------
%% Description: Citations numbered [1], [2], [3]... sorted alphabetically by
%%              first author's last name
%% Sorting: Alphabetical by author (Canolty before Tort)
%% Example: \cite{Tort2010,Canolty2010} → [2, 1] (C before T alphabetically)
%% Commands: \cite{key} → [1]
%% Best for: When you want bibliography sorted alphabetically
%% Compatible with: elsarticle class
% \bibliographystyle{elsarticle-num}

%% Alternative alphabetical styles:
% \bibliographystyle{plain}      % Basic alphabetical, no natbib features
% \bibliographystyle{ieeetr}     % IEEE style, order of appearance
% \bibliographystyle{siam}       % SIAM style, alphabetical

%% ----------------------------------------------------------------------------
%% OPTION 3: AUTHOR-YEAR CITATIONS
%% ----------------------------------------------------------------------------
%% Description: Citations show author name and year (Smith, 2020) or (Smith 2020)
%% Format: (Author, Year) or Author (Year) depending on command
%% Example: \cite{Tort2010} → (Tort et al., 2010)
%%          \citet{Tort2010} → Tort et al. (2010) [textual]
%%          \cite{Tort2010} → (Tort et al., 2010) [parenthetical]
%% Commands:
%%   - \citet{key}  → Author (Year)  [for text: "As shown by Author (2020)..."]
%%   - \cite{key}  → (Author, Year) [for parentheses: "...as shown (Author, 2020)"]
%%   - \cite{key}   → Same as \cite{key}
%% Best for: Review papers, humanities, some social sciences
%% Requires: natbib package (already loaded)
% \bibliographystyle{plainnat}   % Author-year, alphabetical
% \bibliographystyle{abbrvnat}   % Author-year, abbreviated names
% \bibliographystyle{apalike}    % APA-like author-year style

%% ----------------------------------------------------------------------------
%% OPTION 4: JOURNAL-SPECIFIC STYLES
%% ----------------------------------------------------------------------------
%% Elsevier journals:
% \bibliographystyle{elsarticle-num}        % Numbered, alphabetical
% \bibliographystyle{elsarticle-num-names}  % Numbered, alphabetical, full names
% \bibliographystyle{elsarticle-harv}       % Author-year (Harvard style)

%% Nature family:
% \bibliographystyle{naturemag}             % Nature magazine style

%% IEEE:
% \bibliographystyle{IEEEtran}              % IEEE Transactions style

%% APA:
% \bibliographystyle{apalike}               % APA-like style

%% ----------------------------------------------------------------------------
%% OPTION 5: ADDITIONAL CITATION STYLES
%% ----------------------------------------------------------------------------
%% Note: Many of these styles require biblatex instead of natbib.
%% To use biblatex, you need to modify the preamble and use biber instead of bibtex.
%% Basic conversion: Replace natbib package with biblatex, and use \printbibliography
%% instead of \bibliographystyle + \bibliography commands.

%% ----------------------------------------
%% CHEMISTRY
%% ----------------------------------------
%% American Chemical Society (ACS):
%% Installation: Download achemso.bst or use biblatex with style=chem-acs
%% Format: Numbered, order of appearance, (1) Author, A. B. Title. Journal Year, Volume, Pages.
%% BibTeX method:
% \bibliographystyle{achemso}              % ACS style (requires achemso package)
%% Biblatex method (recommended):
% \usepackage[style=chem-acs]{biblatex}

%% ----------------------------------------
%% MEDICAL & HEALTH SCIENCES
%% ----------------------------------------
%% American Medical Association (AMA) 11th edition:
%% Format: Numbered, order of appearance, superscript numbers
%% Installation: Requires biblatex with biblatex-ama style
%% Method:
% \usepackage[style=ama]{biblatex}         % AMA 11th ed (requires biblatex-ama package)

%% Vancouver style (ICMJE):
%% Format: Numbered [1], order of appearance, commonly used in medical journals
%% Note: unsrtnat (currently active) is very similar to Vancouver
% \bibliographystyle{vancouver}            % Vancouver/ICMJE style (if .bst available)
% \bibliographystyle{unsrtnat}             % Similar to Vancouver (currently active)

%% ----------------------------------------
%% SOCIAL SCIENCES
%% ----------------------------------------
%% American Psychological Association (APA) 7th edition:
%% Format: Author-year, (Author, Year), alphabetical by author
%% BibTeX method (APA-like, not full APA 7th):
% \bibliographystyle{apalike}              % APA-like style (simplified)
% \bibliographystyle{apacite}              % APA 6th/7th (requires apacite package)
%% Biblatex method (recommended for full APA 7th compliance):
% \usepackage[style=apa]{biblatex}         % Full APA 7th edition (requires biblatex-apa)

%% American Sociological Association (ASA) 6th/7th edition:
%% Format: Author-year, (Author Year), alphabetical, similar to Chicago author-date
%% Method:
% \bibliographystyle{asaetr}               % ASA-like style (if .bst available)
%% Biblatex method:
% \usepackage[style=authoryear]{biblatex} % Generic author-year (customizable to ASA)

%% American Political Science Association (APSA):
%% Format: Author-year, similar to Chicago author-date
%% Method:
% \usepackage[style=authoryear-comp]{biblatex}  % Compressed author-year for APSA

%% ----------------------------------------
%% HUMANITIES
%% ----------------------------------------
%% Chicago Manual of Style 18th edition (author-date):
%% Format: Author-year, (Author Year), commonly used in social sciences and humanities
% \bibliographystyle{chicago}              % Chicago author-date (if .bst available)
%% Biblatex method (recommended):
% \usepackage[style=chicago-authordate]{biblatex}  % Chicago 18th ed author-date

%% Chicago Manual of Style 18th edition (notes and bibliography):
%% Format: Footnote/endnote citations with full bibliography
%% Method:
% \usepackage[style=chicago-notes]{biblatex}  % Chicago 18th ed notes style

%% Chicago Manual of Style 18th edition (shortened notes and bibliography):
%% Format: Shortened footnote citations after first full citation
%% Method:
% \usepackage[style=chicago-notes]{biblatex}  % Use with ibidtracker option

%% Modern Language Association (MLA) 9th edition:
%% Format: Author-page, (Author Page), works cited list
%% Method:
% \usepackage[style=mla]{biblatex}         % MLA 9th edition (requires biblatex-mla)

%% Modern Humanities Research Association (MHRA) 4th edition:
%% Format: Footnote citations with bibliography
%% Method:
% \usepackage[style=mhra]{biblatex}        % MHRA 4th edition (requires biblatex-mhra)

%% ----------------------------------------
%% HARVARD STYLES
%% ----------------------------------------
%% Cite Them Right 12th edition - Harvard:
%% Format: Author-year, (Author, Year), widely used in UK universities
% \bibliographystyle{agsm}                 % Harvard style (Australian)
% \bibliographystyle{dcu}                  % Harvard style (Dublin City University)
%% Biblatex method:
% \usepackage[style=authoryear]{biblatex} % Generic Harvard-style (author-year)

%% Elsevier - Harvard (with titles):
%% Format: Author-year with article titles included
% \bibliographystyle{elsarticle-harv}      % Elsevier Harvard style (already listed above)

%% ----------------------------------------
%% ENGINEERING & COMPUTER SCIENCE
%% ----------------------------------------
%% IEEE (Institute of Electrical and Electronics Engineers):
%% Format: Numbered [1], order of appearance, widely used in engineering
% \bibliographystyle{IEEEtran}             % IEEE Transactions style (already listed above)

%% ----------------------------------------
%% NATURAL SCIENCES
%% ----------------------------------------
%% Nature:
%% Format: Numbered, superscript, order of appearance
% \bibliographystyle{naturemag}            % Nature magazine style (already listed above)
% \bibliographystyle{naturemag-doi}        % Nature with DOIs

%% ----------------------------------------------------------------------------
%% BIBLATEX SETUP INSTRUCTIONS
%% ----------------------------------------------------------------------------
%% To switch from natbib to biblatex:
%%
%% 1. In packages.tex, replace:
%%    \usepackage[numbers]{natbib}
%%    with:
%%    \usepackage[style=STYLENAME,backend=biber]{biblatex}
%%    \addbibresource{path/to/bibliography.bib}
%%
%% 2. In this file (bibliography.tex), replace:
%%    \bibliographystyle{...}
%%    with:
%%    % No \bibliographystyle needed with biblatex
%%
%% 3. In your main .tex file, replace:
%%    \bibliography{path/to/bibliography}
%%    with:
%%    \printbibliography
%%
%% 4. Change compilation command:
%%    pdflatex → biber → pdflatex → pdflatex
%%    (instead of pdflatex → bibtex → pdflatex → pdflatex)
%%
%% Example biblatex styles:
%%   style=numeric-comp     → Compressed numeric [1-3,5]
%%   style=authoryear       → (Author, Year)
%%   style=authoryear-comp  → (Author1, 2020; Author2, 2021)
%%   style=apa              → APA 7th edition
%%   style=chicago-authordate → Chicago author-date
%%   style=ieee             → IEEE style
%%   style=nature           → Nature style
%%   style=mla              → MLA 9th edition

%% ----------------------------------------------------------------------------
%% CITATION COMMAND REFERENCE (with natbib)
%% ----------------------------------------------------------------------------
%% Basic commands:
%%   \cite{key}              → [1] or (Author, Year) depending on style
%%   \cite{key1,key2}        → [1, 2] or (Author1, Year1; Author2, Year2)
%%
%% Advanced natbib commands (only work with natbib-compatible styles):
%%   \citet{key}             → Author (Year)  [textual citation]
%%   \cite{key}             → (Author, Year) [parenthetical citation]
%%   \citet*{key}            → Full author list (Year)
%%   \cite*{key}            → (Full author list, Year)
%%   \citealt{key}           → Author Year [no parentheses]
%%   \citealp{key}           → Author, Year [no parentheses]
%%   \citeauthor{key}        → Author [name only]
%%   \citeyear{key}          → Year [year only]
%%   \citeyearpar{key}       → (Year) [year in parentheses]
%%
%% Pre/post notes:
%%   \cite[see][p.~10]{key} → (see Author, Year, p. 10)
%%   \cite[p.~10]{key}      → (Author, Year, p. 10)
%%
%% Multiple citations:
%%   \cite{key1,key2,key3}  → (Author1, Year1; Author2, Year2; Author3, Year3)
%%
%% Suppressing parts:
%%   \cite[e.g.,][]{key}    → (e.g., Author, Year)
%%   \cite[][see p.~10]{key}→ (Author, Year, see p. 10)
%%
%% ----------------------------------------------------------------------------
%% TROUBLESHOOTING
%% ----------------------------------------------------------------------------
%% Problem: Citations appear as [?] or undefined
%% Solution: Run compilation 3-4 times to resolve all references
%%
%% Problem: Citation numbers out of order [3, 1] instead of [1, 3]
%% Solution: Use unsrtnat (order of appearance) instead of elsarticle-num
%%
%% Problem: "Undefined control sequence \citet"
%% Solution: \citet only works with natbib-compatible styles (unsrtnat, plainnat)
%%           Use \cite{} with non-natbib styles
%%
%% Problem: Bibliography not appearing
%% Solution: Ensure \bibliography{path/to/bibfile} command exists in main file
%%           Run: pdflatex → bibtex → pdflatex → pdflatex

%%%% EOF


%% ----------------------------------------
%% DATA AVAILABILITY
%% ----------------------------------------

% ======================================================================
% File: ./01_manuscript/contents/data_availability.tex
% ======================================================================
%% -*- coding: utf-8 -*-
\pdfbookmark[1]{Data Availability Statement}{data_availability}

\section*{Data Availability Statement}

The data and code that support the findings of this study are available from the corresponding author upon reasonable request.

\label{data and code availability}

%%%% EOF



%% ----------------------------------------
%% ADDITIONAL INFORMATION
%% ----------------------------------------

% ======================================================================
% File: ./01_manuscript/contents/additional_info.tex
% ======================================================================
%% -*- coding: utf-8 -*-

\pdfbookmark[1]{Additional Information}{additional_information}

\pdfbookmark[2]{Ethics Declarations}{ethics_declarations}
\section*{Ethics Declarations}
Replace with your ethics statement.
\label{ethics declarations}

\pdfbookmark[2]{Contributors}{author_contributions}
\section*{Author Contributions}
Replace with author contributions.
\label{author contributions}

\pdfbookmark[2]{Acknowledgments}{acknowledgments}
\section*{Acknowledgments}
Replace with acknowledgments.
\label{acknowledgments}

\pdfbookmark[2]{Declaration of Interests}{declaration_of_interest}
\section*{Declaration of Interests}
The authors declare that they have no competing interests.
\label{declaration of interests}

\pdfbookmark[2]{Declaration of Generative AI in Scientific Writing}{declaration_of_generative_ai}
\section*{Declaration of Generative AI in Scientific Writing}
Disclose any use of generative AI tools here.
\label{declaration of generative ai in scientific writing}

%%%% EOF



%% ----------------------------------------
%% TABLES
%% ----------------------------------------
\clearpage
\section*{Tables}
\label{tables}
\pdfbookmark[1]{Tables}{tables}
\vspace{1cm}

% ======================================================================
% File: ./01_manuscript/contents/tables/compiled/FINAL.tex
% ======================================================================
% Auto-generated file containing all table inputs
% Generated by gather_table_tex_files()

% Table from: 01_example_table.tex

% ======================================================================
% File: ./01_manuscript/contents/tables/compiled/01_example_table.tex
% ======================================================================
\pdfbookmark[2]{Table 1_example_table}{table_01_example_table}
\begin{table}[htbp]
\centering
\footnotesize
\begin{tabular}{|l|l|l|}
\hline
Category&Value&Description\\
\hline
Example A&1.00&Replace with your data\\
\hline
Example B&2.00&Replace with your data\\
\hline
Example C&3.00&Replace with your data\\
\hline
\end{tabular}
%% Edit this file: 01_manuscript/contents/tables/caption_and_media/01_example_table.tex
\caption{\textbf{Replace with your table title.}\\
\smallskip
Replace with your table legend text. Edit this caption at \texttt{01\_manuscript/contents/tables/caption\_and\_media/01\_example\_table.tex}.
}
\label{tab:01_example_table}
\end{table}






%% ----------------------------------------
%% FIGURES
%% ----------------------------------------
\clearpage
\section*{Figures}
\label{figures}
\pdfbookmark[1]{Figures}{figures}
\vspace{1cm}

% ======================================================================
% File: ./01_manuscript/contents/figures/compiled/FINAL.tex
% ======================================================================
% Generated by compile_figure_tex_files()
% This file includes all figure files in order

% Figure 01
\begin{figure*}[htbp]
    \pdfbookmark[2]{Figure 01}{.01}
    \centering
    \includegraphics[width=0.9\textwidth]{./01_manuscript/contents/figures/caption_and_media/jpg_for_compilation/01_example_figure.jpg}
    \caption{\textbf{Replace with your figure title.}\\
\smallskip
Replace with your figure legend text. Edit this caption at \texttt{01\_manuscript/contents/figures/caption\_and\_media/01\_example\_figure.tex}.
}
\label{fig:01_example_figure}
\end{figure*}

% Figure 02
\begin{figure*}[htbp]
    \pdfbookmark[2]{Figure 02}{.02}
    \centering
    \includegraphics[width=0.9\textwidth]{./01_manuscript/contents/figures/caption_and_media/jpg_for_compilation/02_another_example.jpg}
    \caption{\textbf{Replace with your figure title.}\\
\smallskip
Replace with your figure legend text. Edit this caption at \texttt{01\_manuscript/contents/figures/caption\_and\_media/02\_another\_example.tex}.
}
\label{fig:02_another_example}
\end{figure*}




%% ----------------------------------------
%% END of DOCUMENT
%% ----------------------------------------
\end{document}

%%%% EOF