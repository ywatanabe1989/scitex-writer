%% Response to reviewer 2 comment 1
\subsection*{Response to Reviewer 2, Comment 1}

\authorText{We appreciate these thoughtful observations. We have substantially expanded the ``Comparison with Existing Solutions'' subsection in the Discussion to provide a more detailed analysis of how SciTeX Writer differs from Overleaf.}

\authorText{The key distinctions are: (1) SciTeX Writer operates entirely on local systems or institutional computing infrastructure, eliminating dependency on internet connectivity and addressing concerns about sensitive research data on cloud platforms; (2) the framework provides complete control over the compilation environment through transparent, modifiable container definitions rather than a proprietary compilation service; (3) the modular file structure and automated asset management go beyond what Overleaf provides, actively preventing merge conflicts and automating figure/table preprocessing; and (4) the system integrates seamlessly with existing Git workflows and institutional HPC resources that often prohibit cloud services.}

\authorText{Regarding accessibility, we acknowledge this is a valid limitation. We have added discussion of this concern in the ``Limitations and Considerations'' subsection, explicitly noting that the command-line interface may present a learning curve for some researchers. We have also proposed future development directions including optional graphical interfaces and expanded documentation for LaTeX newcomers. However, we note that our target audience includes researchers already using or willing to learn Git for version control, a group that increasingly represents the norm in computational research fields.}

\authorText{We have also clarified in the Introduction that SciTeX Writer is positioned as a complementary tool rather than a universal replacement for existing solutions. Different research workflows have different requirements, and we now better articulate the specific use cases where our framework provides the greatest value.}
