% Example results section demonstrating SciTex features

Our evaluation of the SciTex system revealed several significant findings regarding its effectiveness for scientific manuscript preparation. This section presents the key results from our user study and system performance analysis.

\subsection{Time Efficiency}

Participants using the SciTex system showed significant improvements in manuscript preparation time compared to traditional methods. \figref{05} illustrates the time spent on different aspects of the writing process.

The most substantial time savings were observed in:
\begin{itemize}
    \item Figure and table preparation (42\% reduction, $p < 0.001$)
    \item Formatting and style adjustments (65\% reduction, $p < 0.001$)
    \item Citation management (38\% reduction, $p < 0.01$)
\end{itemize}

Overall, participants completed their manuscripts in an average of 3.2 hours using SciTex, compared to 5.7 hours using traditional methods, representing a 44\% reduction in total time ($t(24) = 8.76, p < 0.001$).

\subsection{Document Quality}

The quality of manuscripts produced using SciTex was evaluated using a standardized rubric covering grammar, style, formatting consistency, and adherence to scientific writing principles. As shown in \tabref{01}, SciTex-produced manuscripts scored higher across all quality dimensions.

Particularly notable improvements were seen in:
\begin{itemize}
    \item Consistency of terminology (improved by 47\%)
    \item Formatting accuracy (improved by 62\%)
    \item Citation completeness (improved by 31\%)
\end{itemize}

Independent expert reviewers rated SciTex-produced manuscripts as significantly more professional and publication-ready than those created using traditional methods ($\chi^2(1) = 16.4, p < 0.001$).

\subsection{User Satisfaction}

User satisfaction was measured using a 7-point Likert scale across several dimensions. \figref{04} presents these results, showing consistently high satisfaction with the SciTex system. The highest-rated aspects were:

\begin{itemize}
    \item Figure and table handling (mean = 6.4/7)
    \item AI-assisted text revision (mean = 6.2/7)
    \item Modular document organization (mean = 6.0/7)
\end{itemize}

Qualitative feedback from participants highlighted several themes:

\begin{quote}
"The integration of AI-assisted writing with LaTeX's powerful formatting made my writing process significantly smoother." (Participant 12)
\end{quote}

\begin{quote}
"I especially appreciated the automated figure handling—converting my PowerPoint slides to publication-ready figures saved hours of work." (Participant 7)
\end{quote}

\subsection{Learning Curve}

Despite the advanced features, SciTex's learning curve was found to be manageable. New users required an average of 45 minutes to become proficient with the basic system, with an additional 1.2 hours to master advanced features. 

Participants with prior LaTeX experience ($n = 14$) became productive more quickly (mean = 28 minutes) than those without LaTeX experience ($n = 11$, mean = 67 minutes). However, by the end of the study, both groups achieved similar proficiency levels.

\subsection{System Performance}

Performance metrics for the SciTex system are summarized in \tabref{02}, showing processing times for various operations. The system demonstrated good performance even with complex documents, with compile times remaining under 15 seconds for manuscripts up to 50 pages with 20 figures and tables.

AI-assisted operations (text revision, terminology checking, and citation insertion) averaged 12.3 seconds per operation, with 94\% of suggestions rated as helpful by users. The false positive rate for terminology inconsistencies was 7\%, while the false negative rate was 4\%.

\subsection{Comparative Analysis}

In comparative analysis against other scientific writing tools, SciTex demonstrated several advantages:

\begin{itemize}
    \item More comprehensive figure handling than alternatives
    \item Better integration of AI assistance into the workflow
    \item Superior version control and collaborative features
    \item More flexible template customization options
\end{itemize}

These advantages were consistent across different scientific disciplines represented in our user sample, suggesting the system's versatility for various types of scientific writing.