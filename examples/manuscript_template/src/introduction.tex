% Example introduction demonstrating SciTex features

Scientific writing is a critical skill for researchers, yet the process of preparing manuscripts for publication can be time-consuming and challenging \cite{smith2010scientific}. Traditional approaches to manuscript preparation involve juggling multiple files, formatting requirements, and citation styles \cite{johnson2015academic}. Modern document preparation systems like LaTeX provide powerful tools for creating professional documents, but they can have steep learning curves \cite{lamport1994latex}.

The SciTex system addresses these challenges by providing an integrated framework that combines the power of LaTeX with modern AI-assisted writing tools \cite{watanabe2025scitex}. This example manuscript demonstrates the key features and benefits of the SciTex system.

\subsection{Challenges in Scientific Writing}

Scientific manuscripts must meet several key requirements:

\begin{itemize}
    \item Clear organization with standard sections (Introduction, Methods, Results, Discussion)
    \item Professional formatting according to journal guidelines
    \item Proper handling of mathematical equations and symbols
    \item Consistent citation management
    \item Effective integration of figures and tables
\end{itemize}

Meeting these requirements while focusing on content can be challenging for researchers, especially those new to academic publishing \cite{williams2018challenges}.

\subsection{AI-Assisted Document Preparation}

Recent advances in large language models (LLMs) have created new opportunities for AI-assisted writing \cite{brown2020language}. These models can help with tasks such as grammar checking, style improvement, terminology consistency, and even citation suggestions \cite{lee2022ai}. 

SciTex integrates these capabilities into a LaTeX-based workflow, allowing researchers to benefit from AI assistance while maintaining complete control over their document \cite{thompson2023integrated}. As shown in Figure~\ref{fig:01}, this integration creates a seamless workflow from initial drafting to final publication. The system architecture (Figure~\ref{fig:02}) is designed to be modular and extensible, accommodating various scientific disciplines and publication workflows.

\subsection{Figure and Table Management}

A key advantage of SciTex is its structured approach to handling figures and tables. Figure~\ref{fig:03} illustrates the figure processing pipeline, from source files to final document integration. This approach ensures consistent formatting and simplifies referencing throughout the document. The multi-panel capabilities (Figure~\ref{fig:06}) allow for complex data visualization while maintaining clear organization.

Performance evaluations demonstrate significant time savings compared to traditional methods. As shown in Figure~\ref{fig:05}, compilation times are reduced by up to 45\% when using SciTex's optimized figure processing. Table~\ref{tab:01} summarizes the different AI prompts used in the system and their effectiveness rates.

\subsection{Objectives of This Example}

This example manuscript demonstrates several key features of the SciTex system:

\begin{enumerate}
    \item Modular document organization using separate files for each section
    \item Structured handling of figures and tables with automated processing
    \item Integration with GPT models for text revision and enhancement
    \item Automated citation management and bibliography generation
    \item Version tracking and collaborative editing support
\end{enumerate}

In the following sections, we describe the methods used in the SciTex system, present results from example usage scenarios, and discuss implications for scientific writing workflows. Throughout this manuscript, we include examples of LaTeX features, AI-assisted content, and best practices for scientific document preparation.