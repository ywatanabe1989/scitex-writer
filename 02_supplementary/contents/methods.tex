%% -*- coding: utf-8 -*-

\section*{Supplementary Methods}

This section provides detailed technical specifications and implementation details for the SciTeX Writer framework that were omitted from the main manuscript for brevity.

\subsection*{Container Image Construction}

The Docker and Singularity container images are built from a base TeX Live distribution, specifically using the \texttt{texlive/texlive:latest} official image. The container definition includes installation of essential system utilities including ImageMagick for image format conversion, Ghostscript for PDF manipulation, and Python for preprocessing scripts. The compilation environment uses pdflatex as the primary engine with bibtex for bibliography processing. The container image size is approximately 3.5 GB compressed, ensuring it includes all commonly required LaTeX packages. Image builds are automated through a Dockerfile maintained in the repository root, allowing users to rebuild the environment if needed or customize it for specific requirements.

\subsection*{Makefile Command Reference}

The Makefile provides a comprehensive set of targets for document compilation and management. The \texttt{make manuscript} command compiles the main manuscript by first executing preprocessing scripts, then running pdflatex twice, followed by bibtex, and finally pdflatex twice more to resolve all cross-references. The \texttt{make clean} target removes auxiliary files while preserving source content and compiled PDFs. The \texttt{make archive} command creates a timestamped copy of the current manuscript in the archive directory using the format \texttt{manuscript\_vXXX.tex} where XXX is an automatically incremented version number. The \texttt{make diff} target executes latexdiff between the current version and the most recent archived version, producing a PDF with color-coded additions and deletions.

\subsection*{Preprocessing Pipeline Implementation}

Figure preprocessing involves scanning the \texttt{01\_manuscript/contents/figures/caption\_and\_media/} directory for subdirectories containing image files and corresponding \texttt{.tex} caption files. The script extracts the caption text, determines the appropriate image file based on priority ordering (PDF, then PNG, then JPEG), and generates LaTeX figure inclusion code using the \texttt{graphicx} package. The generated code maintains aspect ratios, sets maximum widths to column width, and includes proper labeling for cross-referencing. All generated figure code is concatenated into \texttt{FINAL.tex} which is included by the main document. The table preprocessing follows an analogous workflow but handles the additional complexity of tabular environments and allows for both simple and complex multi-column layouts.

\subsection*{Version Control Integration}

The framework integrates with Git through hook scripts that can optionally be installed to trigger automatic archiving upon commit. The \texttt{.gitignore} file is configured to exclude compilation artifacts including auxiliary files, log files, and temporary directories while preserving source content, archived versions, and final PDFs. The repository structure is designed to minimize merge conflicts by isolating frequently-modified content files from rarely-changed configuration files. Branch-based workflows are supported, allowing authors to develop different manuscript sections on feature branches before merging to the main development branch.

\subsection*{Cross-Reference Management}

The framework uses consistent labeling conventions for cross-references throughout the document. Figures use the prefix \texttt{fig:}, tables use \texttt{tab:}, sections use \texttt{sec:}, and equations use \texttt{eq:}. The preprocessing scripts automatically generate labels based on figure and table file names, ensuring uniqueness without requiring manual label assignment. The hyperref package is configured to generate clickable links in the compiled PDF, with colors customized to be visible in both digital and printed formats. Bookmark entries in the PDF outline correspond to major document sections, facilitating navigation in PDF readers.
