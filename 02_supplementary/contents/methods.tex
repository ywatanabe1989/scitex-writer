%% -*- coding: utf-8 -*-
%% Timestamp: "2025-09-30 10:19:20 (ywatanabe)"
%% File: "/ssh:sp:/home/ywatanabe/proj/neurovista/paper/02_supplementary/contents/methods.tex"
%% -*- coding: utf-8 -*-
%% File: "/ssh:sp:/home/ywatanabe/proj/neurovista/paper/02_supplementary/contents/methods.tex"

%% ============================================================
%% ORIGINAL VERSION (PRESERVED AS COMMENTS):
%% ============================================================
%% This section provides additional methodological details not included in the main manuscript.
%% These supplementary methods describe the extended analytical procedures and validation techniques used in our study.
%% ============================================================
%% END OF ORIGINAL VERSION
%% ============================================================

\section{Supplementary Methods}

\subsection{Hardware and Computational Infrastructure}
All GPU-accelerated PAC computations were performed on the Spartan HPC system at The University of Melbourne. The distributed computing infrastructure comprised \hl{[XX]} GPU nodes, each equipped with \hl{[GPU MODEL]} GPUs (VRAM: \hl{[XX]} GB per GPU), providing total available VRAM of approximately 320 GB. CPU nodes featured \hl{[XX-core PROCESSOR]} processors with \hl{[XXX]} GB RAM per node. Data storage utilized high-performance parallel file systems with \hl{[XXX]} TB capacity and \hl{[XXX]} GB/s read/write throughput.

\subsection{Detailed PAC Computation Parameters}
\subsubsection{Frequency Band Specification}
Phase frequency bands (25 bands, 2.0-30.0 Hz range) were generated using adaptive bandwidth approach \cite{Tort2010MeasuringPCE}: bandwidth = $f_c/2$, where $f_c$ represents center frequency. This yielded bands with centers at \hl{[LIST OF FREQUENCIES]} Hz and bandwidths ranging from 0.5 Hz (lowest band) to 11.9 Hz (highest band). Amplitude frequency bands (25 bands, 60.0-180.0 Hz range) employed bandwidth = $f_c/4$, generating centers at \hl{[LIST OF FREQUENCIES]} Hz with bandwidths from 7.5 Hz to 40.0 Hz. These specifications ensured adequate frequency resolution while maintaining sufficient temporal precision for 1-minute analysis windows \cite{Hlsemann2019QuantificationOPA,Munia2019TimeFrequencyBPK}.

\subsubsection{Surrogate Data Generation}
Statistical significance testing utilized 200 surrogate datasets generated via circular phase shuffling \cite{Aru2014UntanglingCCD}. For each frequency pair, amplitude time series were circularly shifted by random offsets (uniformly distributed between 1 and signal length-1 samples) while preserving phase time series, thereby destroying genuine phase-amplitude relationships while maintaining individual signal statistics. PAC values were z-score normalized: $z = (MI_{observed} - \mu_{surrogate})/\sigma_{surrogate}$, where $\mu_{surrogate}$ and $\sigma_{surrogate}$ represent mean and standard deviation across 200 surrogates \cite{Jensen2016DiscriminatingVFR}.

\subsection{Statistical Feature Extraction Details}
\subsubsection{Distribution Statistics}
From 10,000 PAC z-values per time window (25 phase × 25 amplitude × 16 channels), we computed: minimum ($\min$), maximum ($\max$), mean ($\mu$), standard deviation ($\sigma$), median ($Q_{50}$), 25th percentile ($Q_{25}$), 75th percentile ($Q_{75}$), kurtosis ($\kappa$, computed as Fisher's excess kurtosis), and skewness ($\gamma$, using adjusted Fisher-Pearson coefficient) \cite{Hlsemann2019QuantificationOPA,Scherer2022DirectMIM}.

\subsubsection{Bimodality Analysis via Gaussian Mixture Models}
Bimodality characteristics were assessed by fitting 2-component Gaussian Mixture Models using expectation-maximization algorithm with \hl{[XXX]} iterations and convergence threshold of \hl{[XX.XX]}. Four metrics quantified distribution bimodality: (1) Ashman's D statistic: $D = \sqrt{2}|\mu_1 - \mu_2|/\sqrt{\sigma_1^2 + \sigma_2^2}$ (D > 2 indicates distinct modes), (2) weight ratio: $w_{ratio} = \min(w_1, w_2)/\max(w_1, w_2)$ where $w_i$ are component weights, (3) Bhattacharyya coefficient measuring overlap between Gaussian components, and (4) bimodality coefficient: $BC = (\gamma^2 + 1)/({\kappa + 3(n-1)^2/((n-2)(n-3))})$ where n = sample size.

\subsubsection{Circular Statistics for Phase Preferences}
Preferred coupling phases were analyzed using circular statistics \cite{PintoOrellana2023StatisticalIFF}: circular mean ($\mu_{circ}$) computed as $\arctan2(\sum \sin \theta_i, \sum \cos \theta_i)$, concentration parameter ($\kappa_{circ}$, inverse of circular variance) estimated via maximum likelihood, circular skewness ($\gamma_{circ}$), and circular kurtosis ($\kappa_{4,circ}$).

% \subsection{Machine Learning Implementation Details}
% \subsubsection{Model Architecture and Training}
% Classification models employed \hl{[ALGORITHM NAME]} with hyperparameters: \hl{[PARAMETER LIST]}. Training utilized \hl{[XX]}\%-\hl{[XX]}\% train-validation split with \hl{[XX]}-fold cross-validation within training set \cite{Messaoud2021RandomFCR,Hussein2022MultiChannelVTE}. Optimization employed \hl{[OPTIMIZER NAME]} with learning rate \hl{[XX.XXXX]}, batch size \hl{[XXX]}, and \hl{[XXX]} epochs with early stopping (patience: \hl{[XX]} epochs). Class imbalance was addressed using \hl{[BALANCING METHOD]}.

% \subsubsection{Patient-Specific Model Development}
% Individual patient models were trained using patient-specific data partitioning that respected temporal ordering (pseudo-prospective design) \cite{Kuhlmann2018SeizurePA,Aldahr2023PatientSpecificPPL}. Training data comprised first \hl{[XX]}\% of temporal sequence (lead-in period), while testing utilized remaining \hl{[XX]}\% (testing period), simulating real-world deployment where models predict future seizures based on historical data \cite{Freestone2015SeizurePSBF}.

\subsection{Statistical Testing Procedures}
\subsubsection{Group Comparisons}
Brunner-Munzel tests assessed differences between preictal and interictal feature distributions, chosen for robustness to non-normality and variance heterogeneity. Multiple comparison correction employed \hl{Bonferroni} adjustment with family-wise error rate \hl{$\alpha = 0.05$}, yielding adjusted significance threshold \hl{$\alpha_{adj} = 0.05/(17 \text{ features} \times 7 \text{ time bins}) = $} \hl{[XX.XXXXX]}.

\subsubsection{Temporal Trend Analysis}
Linear regression quantified temporal accumulation of preictal changes, modeling effect size (Brunner-Munzel statistic) as function of time-to-seizure across seven logarithmically-spaced preictal bins \cite{Kuhlmann2018SeizurePA}. Model fit was assessed using coefficient of determination ($R^2$) and slope significance ($p < 0.05$).

\subsection{Data Management and Reproducibility}
All analyses employed fixed random seeds (seed = 42) for reproducibility. Processed data were stored in patient-specific SQLite3 databases with zlib compression (compression level 9), achieving \hl{[XX]}\%-\hl{[XX]}\% size reduction. Database schema included metadata tables (patient demographics, seizure annotations, processing parameters), PAC data tables (compressed BLOBs), and quality assurance tables (computation timestamps, software versions). All analysis scripts were version-controlled using git with tagged releases corresponding to manuscript revisions.

%%%% EOF