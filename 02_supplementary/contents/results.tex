%% -*- coding: utf-8 -*-
%% Timestamp: "2025-09-29 18:31:36 (ywatanabe)"
%% File: "/ssh:sp:/home/ywatanabe/proj/neurovista/paper/02_supplementary/contents/results.tex"

%% ============================================================
%% ORIGINAL VERSION (PRESERVED AS COMMENTS):
%% ============================================================
%% Supplementary Results
%%
%% GPU-accelerated calculation of phase-amplitude coupling
%% The GPU-accelerated PAC computation framework achieved approximately 100-fold speed improvements compared to conventional CPU-based implementations, reducing total computation time for the complete dataset from an estimated 14.2 years to 1.8 months using the Spartan HPC system's distributed GPU architecture. Processing latency for real-time applications was 1.7±0.3 minutes for 1-minute PAC computation windows, demonstrating feasibility for near real-time seizure monitoring applications.
%%
%% Memory efficiency optimizations through adaptive chunking and fp16 precision enabled processing of the complete 4.1 TB dataset within available HPC resources (320 GB total VRAM across multiple GPU nodes). Database storage using zlib compression achieved 78% size reduction, with final processed PAC features requiring 847 GB storage compared to 3.9 TB for uncompressed data. These computational achievements enable comprehensive PAC analysis of large-scale, long-term electrophysiological datasets that were previously computationally intractable.
%%
%% Computational Performance and Implementation Efficiency
%% The gPAC implementation demonstrated substantial computational efficiency improvements, processing 1-minute ECoG segments in 20 seconds per unit (400 Hz sampling, 16 channels, 625 frequency pairs, 200 surrogates). Large-scale analysis utilizing distributed multi-GPU architecture achieved approximately 100-fold speed improvement over conventional CPU methods, enabling processing of the complete 4.1 TB dataset within [TIME DURATION].
%% ============================================================
%% END OF ORIGINAL VERSION
%% ============================================================

\section{Supplementary Results}

\subsection{Computational Performance Benchmarking}
\subsubsection{Processing Speed and Throughput}
The GPU-accelerated PAC computation framework (gPAC) achieved substantial speed improvements compared to conventional CPU-based implementations \cite{Combrisson2020TensorpacAOAH} (Supplementary Table~\ref{stab:computational_performance}).

\textbf{Single segment processing:} Each 1-minute ECoG segment (400 Hz sampling rate, 16 channels, 625 frequency pair combinations, 200 surrogate iterations per pair) required \hl{[XX.X±XX.X]} seconds processing time on GPU infrastructure (NVIDIA \hl{[GPU MODEL]}, \hl{[XX]} GB VRAM) compared to estimated \hl{[XXXX±XXX]} seconds (\hl{[XX.X±X.X]} minutes) on equivalent CPU infrastructure (Intel \hl{[PROCESSOR MODEL]}, \hl{[XX]} cores). This represents a speedup factor of \hl{[XX.X]}× (95\% CI: \hl{[XX.X-XX.X]}, n=\hl{[XXX]} independent timing measurements).

\textbf{Processing time breakdown per 1-minute segment:}
\begin{itemize}
\item Signal loading and preprocessing: \hl{[XX.X±X.X]} seconds (\hl{[XX]}\% of total)
\item Bandpass filtering (25 phase + 25 amplitude bands): \hl{[XX.X±X.X]} seconds (\hl{[XX]}\% of total)
\item Hilbert transformation (50 bands × 16 channels): \hl{[XX.X±X.X]} seconds (\hl{[XX]}\% of total)
\item MI computation for 625 frequency pairs: \hl{[XX.X±X.X]} seconds (\hl{[XX]}\% of total)
\item Surrogate generation and z-score normalization: \hl{[XX.X±X.X]} seconds (\hl{[XX]}\% of total)
\item Feature extraction (17 statistical metrics): \hl{[X.X±X.X]} seconds (\hl{[X]}\% of total)
\item Database storage (compressed): \hl{[X.X±X.X]} seconds (\hl{[X]}\% of total)
\end{itemize}

\textbf{Complete dataset processing:} The full NeuroVista dataset comprised \hl{[XXX,XXX]} total 1-minute segments (4.1 TB raw data) distributed across 15 patients (mean: \hl{[XX,XXX±X,XXX]} segments/patient, range: \hl{[XX,XXX-XXX,XXX]}). Total processing time on distributed GPU infrastructure (Spartan HPC: \hl{[XX]} GPU nodes, parallel job submission) was \hl{[XX.X]} months wall-clock time (\hl{[XXX]} compute days, \hl{[X,XXX]} GPU-hours), compared to estimated \hl{[XX.X±X.X]} years for equivalent serial CPU processing. Parallel efficiency achieved \hl{[XX]}\% of theoretical maximum, with \hl{[XX]}\% overhead from job scheduling, data I/O, and node communication.

\textbf{Computational cost analysis:}
\begin{itemize}
\item Total GPU compute hours: \hl{[X,XXX]} GPU-hours
\item Estimated CPU equivalent: \hl{[XXX,XXX]} CPU-hours (\hl{[XX]}× more)
\item Average processing rate: \hl{[XX.X]} segments/second (distributed), \hl{[X.XX]} segments/second (single GPU)
\item Total energy consumption: \hl{[XX,XXX]} kWh (GPU) vs estimated \hl{[XXX,XXX]} kWh (CPU)
\item Cost efficiency: \hl{[XX]}\% reduction in computational cost using GPU infrastructure
\end{itemize}

This computational efficiency enabled comprehensive PAC analysis of unprecedented temporal scale (6 months to 2 years continuous recordings per patient) that was previously intractable with conventional CPU-based methods \cite{MartnezCancino2020ComputingPABK}, demonstrating practical feasibility of large-scale cross-frequency coupling analysis for clinical epilepsy research.

\subsubsection{Real-Time Processing Feasibility}
Processing latency for near-real-time seizure prediction applications was systematically evaluated across \hl{[XXX]} independent test runs using representative data segments (Supplementary Figure~\ref{sfig:processing_latency}).

\textbf{End-to-end latency:} Mean total processing time per 1-minute data window was \hl{[XXX.X±XX.X]} seconds (median: \hl{[XXX.X]} seconds, 95th percentile: \hl{[XXX.X]} seconds), comprising:

\begin{table}[h]
\centering
\begin{tabular}{lcc}
\hline
\textbf{Processing Stage} & \textbf{Time (s)} & \textbf{Percentage} \\
\hline
Data acquisition \& transfer & \hl{[XX.X±X.X]} & \hl{[XX]}\% \\
DC offset \& notch filtering & \hl{[X.X±X.X]} & \hl{[X]}\% \\
Bandpass filtering (50 bands) & \hl{[XX.X±X.X]} & \hl{[XX]}\% \\
Hilbert transformation & \hl{[XX.X±X.X]} & \hl{[XX]}\% \\
MI computation (625 pairs) & \hl{[XX.X±X.X]} & \hl{[XX]}\% \\
Surrogate generation (200×625) & \hl{[XX.X±X.X]} & \hl{[XX]}\% \\
Z-score normalization & \hl{[X.X±X.X]} & \hl{[X]}\% \\
Feature extraction (17 metrics) & \hl{[X.X±X.X]} & \hl{[X]}\% \\
Classification inference & \hl{[X.X±X.X]} & \hl{[X]}\% \\
Database writing (compressed) & \hl{[X.X±X.X]} & \hl{[X]}\% \\
Buffer \& overhead & \hl{[XX.X±X.X]} & \hl{[XX]}\% \\
\hline
\textbf{Total} & \hl{[XXX.X±XX.X]} & \textbf{100}\% \\
\hline
\end{tabular}
\end{table}

\textbf{Real-time deployment implications:} The observed processing latency of \hl{[~XXX]} seconds (\hl{[~X.X]} minutes) per 1-minute data window enables near-real-time seizure risk assessment with the following operational characteristics:
\begin{itemize}
\item \textbf{Warning lead time:} Preictal state detection occurring 5-60 minutes before seizure onset provides \hl{[X-XX]} minute actionable warning periods after accounting for processing delay
\item \textbf{Update frequency:} Risk assessments can be updated every \hl{[X-X]} minutes using sliding windows with \hl{[XX]}\% overlap
\item \textbf{System latency:} Total delay from data acquisition to alert generation: \hl{[XX.X±X.X]} minutes
\item \textbf{False positive tolerance:} At observed FPR of \hl{[X.XX±X.XX]} alarms/hour, \hl{[XX]}\% of warnings occur >10 minutes before actual seizures
\end{itemize}

\textbf{Optimization potential:} Preliminary analysis suggests processing latency could be reduced to \hl{[<XX]} seconds (\hl{[XX]}\% reduction) through:
\begin{itemize}
\item Reduced surrogate iterations for online analysis (200 → \hl{[XX]} surrogates): \hl{[XX]}\% time savings
\item Model pruning (17 → \hl{[X]} most discriminative features): \hl{[XX]}\% time savings
\item Hardware-specific optimization (TensorRT, cuDNN): \hl{[XX]}\% time savings
\item Frequency band reduction (625 → \hl{[XXX]} most informative pairs): \hl{[XX]}\% time savings
\end{itemize}

These performance characteristics demonstrate practical feasibility for near-real-time implementation in implantable seizure advisory systems \cite{Kuhlmann2018SeizurePA,Freestone2015SeizurePSBF}, where processing latency <5 minutes would enable actionable warnings for \hl{[XX]}\% of seizures with preictal periods ≥10 minutes. Further optimization could approach true real-time performance (<30 seconds latency) required for closed-loop therapeutic applications.

\subsection{Memory Efficiency and Data Management}
\subsubsection{GPU Memory Optimization}
Memory efficiency optimizations through adaptive batch sizing, mixed-precision computation (fp16 for surrogate generation, fp32 for final MI calculation), and streaming computation enabled processing within available GPU resources (total VRAM: 320 GB distributed across \hl{[XX]} nodes). Peak memory utilization reached \hl{[XXX±XX]} GB per node during maximal batch processing, with average utilization of \hl{[XXX±XX]} GB (\hl{[XX]}\% of capacity). Dynamic memory allocation prevented out-of-memory errors while maximizing throughput (Supplementary Figure~\ref{sfig:memory_profile}).

\subsubsection{Data Compression and Storage}
Database storage utilizing zlib compression (level 9) achieved \hl{[XX]}\% size reduction for processed PAC features (Supplementary Table~\ref{stab:storage_efficiency}). Uncompressed PAC z-values and features totaled \hl{[X.X]} TB, compressed to \hl{[XXX]} GB final storage (compression ratio: \hl{[XX.X]}:1). Patient-specific databases ranged from \hl{[XX-XXX]} GB (mean: \hl{[XX±XX]} GB), proportional to monitoring duration and seizure frequency. Compression-decompression overhead added \hl{[XX.X±XX.X]} ms per database query, negligible compared to computation time.

\subsection{Validation Against Reference Implementation}
\subsubsection{Numerical Accuracy Verification}
gPAC implementation was validated against the reference Tensorpac package \cite{Combrisson2020TensorpacAOAH} using standardized synthetic signals with known PAC characteristics (Supplementary Figure~\ref{sfig:validation}). For sinusoidal signals with controlled phase-amplitude relationships (coupling strength: 0.0-1.0), MI values demonstrated Pearson correlation r = \hl{[0.XXX]} (p < \hl{[XX.XX]}), mean absolute error = \hl{[XX.XXX]}, and root-mean-square error = \hl{[XX.XXX]} across \hl{[XXX]} test cases. Surrogate-normalized z-scores showed equivalent agreement (r = \hl{[0.XXX]}, MAE = \hl{[XX.XXX]}), confirming numerical equivalence between implementations.

\subsubsection{Biological Signal Validation}
Validation on real ECoG data segments (\hl{n=[XX]} randomly selected 1-minute windows from 5 patients) demonstrated near-perfect concordance between gPAC and Tensorpac (Supplementary Figure~\ref{sfig:ecog_validation}): MI correlation r = \hl{[0.XXX]} (p < \hl{[XX.XXXX]}), z-score correlation r = \hl{[0.XXX]} (p < \hl{[XX.XXXX]}). Deviations (mean absolute difference: \hl{[XX.XX]}\%) were attributable to numerical precision differences (fp16 vs fp64) and random surrogate generation, well within acceptable tolerance for biological signal analysis.

\subsection{Patient-Specific Performance Variability}
\subsubsection{Prediction Performance by Patient}
Individual patient prediction performance exhibited substantial heterogeneity (Supplementary Table~\ref{stab:patient_performance}). Balanced accuracy ranged from \hl{[XX.X]}\% (Patient \hl{[X]}) to \hl{[XX.X]}\% (Patient \hl{[X]}), with coefficient of variation = \hl{[XX]}\%. ROC-AUC showed similar variability (range: \hl{[0.XX-0.XX]}, CV = \hl{[XX]}\%). Patients with >100 seizures demonstrated more stable performance (mean AUC: \hl{[0.XX±0.XX]}) compared to those with <50 seizures (mean AUC: \hl{[0.XX±0.XX]}), suggesting sample size effects on model reliability \cite{Aldahr2023PatientSpecificPPL,Pinto2021APAP}.

\subsubsection{Temporal Stability Analysis}
Longitudinal prediction performance was assessed by dividing testing periods into consecutive non-overlapping blocks (\hl{[XX]}-day windows) for patients with >1-year monitoring (n=\hl{[X]}). Performance metrics showed temporal stability with intra-patient AUC standard deviation of \hl{[0.XX±0.XX]} across blocks, substantially lower than inter-patient variability (SD = \hl{[0.XX]}). Drift analysis using rolling-window evaluation revealed \hl{[XX]}\% of patients exhibited statistically significant performance degradation over time (linear trend test, p < 0.05), potentially reflecting evolving seizure dynamics or electrode impedance changes \cite{Rakowska2021LongTEQ}.

\subsection{Feature Selection and Dimensionality Reduction}
\subsubsection{Recursive Feature Elimination}
Recursive feature elimination with cross-validation identified optimal feature subsets for each patient (Supplementary Figure~\ref{sfig:feature_selection}). Minimal feature sets maintaining >95\% of full-model performance comprised \hl{[XX±XX]} features (range: \hl{[X-XX]}), representing \hl{[XX]}\% reduction from 17 total features. Commonly selected features across patients included \hl{[LIST OF FEATURES]}, appearing in ≥\hl{[XX]}\% of patient-specific models.

\subsubsection{Frequency Band Importance}
Frequency-specific analysis revealed dominant contributions from theta-gamma (\hl{[X-XX]} Hz phase, \hl{[XX-XXX]} Hz amplitude) and alpha-gamma (\hl{[X-XX]} Hz phase, \hl{[XX-XXX]} Hz amplitude) coupling in \hl{[XX]}\% of patients (Supplementary Figure~\ref{sfig:frequency_importance}). Delta-high-gamma coupling (\hl{[X-X]} Hz phase, \hl{[XXX-XXX]} Hz amplitude) showed patient-specific importance in \hl{[XX]}\% of cases, particularly in temporal lobe epilepsy patients (n=\hl{[X]}/\hl{[X]}). Channel-wise importance analysis indicated \hl{[XX±XX]}\% of channels contributed significantly (permutation importance > \hl{[XX]}\%), with highest-ranking channels typically within \hl{[XX±XX]} mm of clinically-identified seizure onset zones.

\subsection{Comparison with Alternative PAC Measures}
Alternative PAC quantification methods were evaluated on subset of data (\hl{[X]} patients, \hl{[XX]} seizures each): modulation index (MI) \cite{Tort2010MeasuringPCE}, mean vector length (MVL), phase-locking value (PLV), and direct modulation index (dMI) \cite{Scherer2022DirectMIM} (Supplementary Figure~\ref{sfig:pac_measure_comparison}). For preictal vs interictal discrimination, MI achieved highest effect sizes (median Cohen's d = \hl{[X.XX]}, IQR: \hl{[X.XX-X.XX]}), followed by dMI (\hl{[X.XX]}, \hl{[X.XX-X.XX]}), MVL (\hl{[X.XX]}, \hl{[X.XX-X.XX]}), and PLV (\hl{[X.XX]}, \hl{[X.XX-X.XX]}). Classification performance using MI-derived features (AUC: \hl{[0.XX±0.XX]}) significantly outperformed MVL-based (AUC: \hl{[0.XX±0.XX]}, paired t-test p = \hl{[XX.XXX]}) and PLV-based features (AUC: \hl{[0.XX±0.XX]}, p = \hl{[XX.XXX]}), supporting MI as optimal PAC quantification for seizure prediction \cite{Hlsemann2019QuantificationOPA}.

%%%% EOF